\paragraph{UC06: Xem Báo cáo AI Tổng hợp}
\mbox{}

% UC06.1 - Xem báo cáo tổng hợp
\textbf{UC06.1 - Xem báo cáo tổng hợp}

\begin{figure}[H]
    \centering
    \includegraphics[width=0.7\textwidth]{images/uc06_1_view_ai_report.png}
    \caption{Sơ đồ use case chức năng xem báo cáo tổng hợp}
    \label{fig:uc06_1_view_ai_report}
\end{figure}

\begin{longtblr}[
    caption = {Đặc tả use case UC06.1 - Xem báo cáo tổng hợp},
    label = {tab:uc06_1},
]{
    colspec={|l|p{.7\linewidth}|}
}
\hline
\textbf{Tên chức năng} & \textbf{Xem báo cáo tổng hợp} \\\hline
ID & UC06.1 \\\hline
Người sử dụng & Workspace Owner, Member \\\hline
Mức độ cần thiết & Bắt buộc \\\hline
Phân loại & Cao \\\hline
Các thành phần tham gia &
\begin{minipage}{\linewidth}
    \vskip 4pt
    + \textbf{User:} Muốn xem báo cáo được AI tổng hợp từ các nguồn dữ liệu. \\
    + \textbf{LLM Provider:} Cung cấp khả năng phân tích và tổng hợp nội dung.
    \vskip 1pt
\end{minipage}
\\\hline
Mô tả tóm tắt & Chức năng hiển thị báo cáo được AI tổng hợp từ meetings, chats, và tệp tin. \\\hline
Trigger & User truy cập trang AI Reports. \\\hline
Kiểu sự kiện & External. \\\hline
Luồng xử lý bình thường &
\begin{minipage}{\linewidth}
    \vskip 4pt
    \begin{enumerate}
        \item User truy cập AI Reports.
        \item Hệ thống hiển thị danh sách báo cáo (daily, weekly...).
        \item User chọn loại báo cáo.
        \item Hệ thống tổng hợp từ các nguồn (meetings, chats, files).
        \item Hệ thống gọi AI phân tích (Include UC06.4).
        \item Hệ thống hiển thị báo cáo với insights.
    \end{enumerate}
    \vskip 1pt
\end{minipage}
\\\hline
Các luồng sự kiện con & Include UC06.4 (Truy vấn LLM Provider). \\\hline
Luồng luân phiên/đặc biệt &
\begin{minipage}{\linewidth}
    \vskip 4pt
    \textcolor{red}{+ E1: Chưa cấu hình LLM Provider → Hiển thị hướng dẫn cấu hình.} \\
    \textcolor{red}{+ E2: LLM Provider không phản hồi → Hiển thị lỗi và cho phép thử lại.}
    \vskip 1pt
\end{minipage}
\\\hline
Kết quả & User xem được báo cáo AI tổng hợp với insights hữu ích. \\\hline
\end{longtblr}

\vspace{1em}

% UC06.2 - Tạo báo cáo tùy chỉnh
\textbf{UC06.2 - Tạo báo cáo tùy chỉnh}

\begin{figure}[H]
    \centering
    \includegraphics[width=0.7\textwidth]{images/uc06_2_create_custom_report.png}
    \caption{Sơ đồ use case chức năng tạo báo cáo tùy chỉnh}
    \label{fig:uc06_2_create_custom_report}
\end{figure}

\begin{longtblr}[
    caption = {Đặc tả use case UC06.2 - Tạo báo cáo tùy chỉnh},
    label = {tab:uc06_2},
]{
    colspec={|l|p{.7\linewidth}|}
}
\hline
\textbf{Tên chức năng} & \textbf{Tạo báo cáo tùy chỉnh} \\\hline
ID & UC06.2 \\\hline
Người sử dụng & Workspace Owner \\\hline
Mức độ cần thiết & Bắt buộc \\\hline
Phân loại & Cao \\\hline
Các thành phần tham gia &
\begin{minipage}{\linewidth}
    \vskip 4pt
    + \textbf{Workspace Owner:} Muốn tạo báo cáo AI theo yêu cầu cụ thể. \\
    + \textbf{LLM Provider:} Xử lý yêu cầu phân tích tùy chỉnh.
    \vskip 1pt
\end{minipage}
\\\hline
Mô tả tóm tắt & Owner tạo báo cáo AI tùy chỉnh với prompt và nguồn dữ liệu. \\\hline
Trigger & Owner chọn "Tạo báo cáo mới" trong AI Reports. \\\hline
Kiểu sự kiện & External. \\\hline
Luồng xử lý bình thường &
\begin{minipage}{\linewidth}
    \vskip 4pt
    \begin{enumerate}
        \item Owner chọn "Tạo báo cáo mới".
        \item Hệ thống hiển thị wizard.
        \item Owner chọn nguồn (meetings, chats, files).
        \item Owner chọn khoảng thời gian.
        \item Owner nhập prompt cụ thể.
        \item Owner nhấn "Tạo báo cáo".
        \item Hệ thống collect dữ liệu.
        \item Hệ thống gọi AI phân tích (Include UC06.4).
        \item Hệ thống lưu và hiển thị báo cáo.
    \end{enumerate}
    \vskip 1pt
\end{minipage}
\\\hline
Các luồng sự kiện con & Include UC06.4 (Truy vấn LLM Provider). \\\hline
Luồng luân phiên/đặc biệt &
\begin{minipage}{\linewidth}
    \vskip 4pt
    \textcolor{red}{+ E1: Không đủ dữ liệu → Gợi ý mở rộng phạm vi.} \\
    \textcolor{red}{+ E2: Prompt phức tạp → Gợi ý chia nhỏ.}
    \vskip 1pt
\end{minipage}
\\\hline
Kết quả & Báo cáo tùy chỉnh được tạo và lưu để xem lại sau. \\\hline
\end{longtblr}

\vspace{1em}

% UC06.3 - Xuất báo cáo
\textbf{UC06.3 - Xuất báo cáo}

\begin{figure}[H]
    \centering
    \includegraphics[width=0.7\textwidth]{images/uc06_3_export_report.png}
    \caption{Sơ đồ use case chức năng xuất báo cáo}
    \label{fig:uc06_3_export_report}
\end{figure}

\begin{longtblr}[
    caption = {Đặc tả use case UC06.3 - Xuất báo cáo},
    label = {tab:uc06_3},
]{
    colspec={|l|p{.7\linewidth}|}
}
\hline
\textbf{Tên chức năng} & \textbf{Xuất báo cáo} \\\hline
ID & UC06.3 \\\hline
Người sử dụng & Workspace Owner, Member \\\hline
Mức độ cần thiết & Bắt buộc \\\hline
Phân loại & Trung bình \\\hline
Các thành phần tham gia & + \textbf{User:} Muốn xuất báo cáo để chia sẻ hoặc lưu trữ. \\\hline
Mô tả tóm tắt & Chức năng export báo cáo AI sang các định dạng phổ biến. \\\hline
Trigger & User chọn "Xuất báo cáo" cho một báo cáo đã tạo. \\\hline
Kiểu sự kiện & External. \\\hline
Luồng xử lý bình thường &
\begin{minipage}{\linewidth}
    \vskip 4pt
    \begin{enumerate}
        \item User chọn "Xuất báo cáo".
        \item Hệ thống hiển thị options định dạng (PDF, DOCX, MD, HTML).
        \item User chọn định dạng mong muốn.
        \item User nhấn "Xuất".
        \item Hệ thống render báo cáo theo định dạng.
        \item Hệ thống tạo file và trigger download.
    \end{enumerate}
    \vskip 1pt
\end{minipage}
\\\hline
Các luồng sự kiện con & N/A \\\hline
Luồng luân phiên/đặc biệt &
\begin{minipage}{\linewidth}
    \vskip 4pt
    \textcolor{red}{+ A1: Chia sẻ qua email → Gửi báo cáo đến email.}
    \vskip 1pt
\end{minipage}
\\\hline
Kết quả & Báo cáo được xuất và download về máy user. \\\hline
\end{longtblr}

\vspace{1em}

% UC06.4 - Truy vấn LLM Provider
\textbf{UC06.4 - Truy vấn LLM Provider}

\begin{figure}[H]
    \centering
    \includegraphics[width=0.7\textwidth]{images/uc06_4_query_llm.png}
    \caption{Sơ đồ use case chức năng truy vấn LLM Provider}
    \label{fig:uc06_4_query_llm}
\end{figure}

\begin{longtblr}[
    caption = {Đặc tả use case UC06.4 - Truy vấn LLM Provider},
    label = {tab:uc06_4},
]{
    colspec={|l|p{.7\linewidth}|}
}
\hline
\textbf{Tên chức năng} & \textbf{Truy vấn LLM Provider} \\\hline
ID & UC06.4 \\\hline
Người sử dụng & LLM Provider \\\hline
Mức độ cần thiết & Bắt buộc \\\hline
Phân loại & Cao \\\hline
Các thành phần tham gia &
\begin{minipage}{\linewidth}
    \vskip 4pt
    + \textbf{System:} Cần gửi request đến AI để phân tích dữ liệu. \\
    + \textbf{LLM Provider:} Xử lý request và trả về kết quả phân tích.
    \vskip 1pt
\end{minipage}
\\\hline
Mô tả tóm tắt & Hệ thống giao tiếp với LLM Provider (OpenAI, Anthropic) để xử lý AI. \\\hline
Trigger & Được gọi từ UC06.1, UC06.2. \\\hline
Kiểu sự kiện & Internal. \\\hline
Luồng xử lý bình thường &
\begin{minipage}{\linewidth}
    \vskip 4pt
    \begin{enumerate}
        \item Hệ thống chuẩn bị prompt.
        \item Hệ thống gửi request đến LLM API (dùng API key).
        \item LLM Provider xử lý request.
        \item LLM Provider trả response.
        \item Hệ thống parse response.
        \item Hệ thống ghi log.
        \item Hệ thống trả kết quả về UC gọi.
    \end{enumerate}
    \vskip 1pt
\end{minipage}
\\\hline
Các luồng sự kiện con & Include bởi UC06.1, UC06.2. \\\hline
Luồng luân phiên/đặc biệt &
\begin{minipage}{\linewidth}
    \vskip 4pt
    \textcolor{red}{+ E1: Timeout → Retry tối đa 3 lần.} \\
    \textcolor{red}{+ E2: Rate limit → Queue và retry sau.} \\
    \textcolor{red}{+ E3: Invalid key → Thông báo Owner.}
    \vskip 1pt
\end{minipage}
\\\hline
Kết quả & Hệ thống nhận được response từ AI để sử dụng trong báo cáo. \\\hline
\end{longtblr}
