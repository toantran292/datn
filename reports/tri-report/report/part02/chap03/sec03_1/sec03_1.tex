\subsection{Tổng quan phân hệ họp trực tuyến}

Phân hệ họp trực tuyến (Meeting Subsystem) là một thành phần quan trọng trong hệ thống SaaS Platform, cung cấp khả năng video conferencing, recording, transcription và AI-powered analysis cho workspace collaboration. Phân hệ này được thiết kế để đáp ứng nhu cầu họp trực tuyến chuyên nghiệp với các tính năng vượt trội so với các giải pháp video conferencing thông thường như Zoom hay Google Meet.

\textbf{Bối cảnh và nhu cầu:}

Trong thời đại làm việc từ xa và hybrid work ngày càng phổ biến, các cuộc họp trực tuyến đã trở thành phương thức giao tiếp chính trong doanh nghiệp. Tuy nhiên, các vấn đề thường gặp với các giải pháp hiện có bao gồm: thiếu tích hợp sâu với workflow và công cụ quản lý dự án, không có khả năng tự động transcribe và tóm tắt nội dung họp, recordings bị phân tán và khó quản lý, không có insights và action items tracking sau họp, và chi phí cao cho enterprise features. Phân hệ Họp trực tuyến (Meeting) của hệ thống được phát triển để giải quyết toàn diện các vấn đề trên với tight integration vào workspace ecosystem và AI-powered post-meeting processing.

\textbf{Các thành phần chính:}

Phân hệ Họp trực tuyến (Meeting) bao gồm ba thành phần chính tương tác với nhau để tạo thành một giải pháp hoàn chỉnh. Meeting Service (NestJS backend) quản lý meeting lifecycle, JWT authentication, Jibri allocation, và orchestrate post-meeting processing workflows. Jitsi Meet Infrastructure cung cấp real-time video conferencing với Videobridge (SFU) để forward media streams, Jicofo quản lý conference focus, Prosody (XMPP) cho signaling, và Jibri cho recording sử dụng Chrome headless và ffmpeg. AI Processing Pipeline tự động hóa post-meeting analysis với Speech-to-Text conversion từ audio recordings (Google Cloud, AWS Transcribe, Whisper), LLM-based summarization (OpenAI GPT-4, Anthropic Claude, Google Gemini) generate executive summaries, action items, key decisions, và discussion highlights với timestamps.

\textbf{Quy trình làm việc end-to-end:}

Một quy trình meeting điển hình bao gồm các bước sau. Pre-meeting: Member tạo meeting với title và description, Meeting Service generate unique room name và JWT token, system gửi invitations qua email và in-app notifications. During meeting: Participants join meeting room qua Jitsi Meet embedded interface, Moderator có thể start/stop recording, Jibri capture audio/video streams và upload to MinIO/S3, real-time notifications về participants join/leave được broadcast qua WebSocket. Post-meeting: Background job được trigger khi recording upload complete, ffmpeg extract audio track từ video file, audio được gửi đến STT API để convert thành transcript với timestamps, transcript được process bởi LLM để generate summary với action items và decisions, users nhận notification khi summary ready và có thể view trong meeting replay interface với synchronized video, transcript và AI insights.

\textbf{Tính năng nổi bật:}

Phân hệ Họp trực tuyến (Meeting) cung cấp các tính năng vượt trội khiến nó khác biệt với các giải pháp thương mại. Workspace Integration đảm bảo meetings được tổ chức theo workspace với access control dựa trên workspace membership, recording và transcripts được lưu trữ trong workspace storage quota, meeting analytics được tích hợp vào workspace dashboard. AI-Powered Insights tự động generate executive summary highlighting key topics discussed, extract action items với assigned persons và due dates, identify important decisions made during meeting, provide discussion highlights với clickable timestamps để jump to relevant video sections. Smart Search cho phép full-text search trong transcript để tìm specific topics, filter meetings by date range, participants, recording status, và search across multiple meetings để tìm recurring themes. Enterprise Features bao gồm customizable recording quality (720p, 1080p, 1440p), flexible retention policies (30, 60, 90 days) với automatic cleanup, storage quota management per workspace, participant limits based on workspace plan, và external guest access với approval workflow.

\textbf{Kiến trúc kỹ thuật:}

Phân hệ được thiết kế theo kiến trúc scalable và highly available. Frontend Integration sử dụng Next.js với Jitsi Meet External API (lib-jitsi-meet) embedded trong React components, WebSocket connection cho real-time updates về meeting status và participants, responsive design support desktop và mobile browsers. Backend Architecture với Meeting Service (NestJS) expose RESTful APIs cho meeting CRUD operations, implement JWT token generation với Jitsi-compatible format, manage Jibri instance pool với load balancing và health monitoring, orchestrate background jobs cho transcription và AI processing. Infrastructure Layer triển khai multiple Videobridge instances cho horizontal scaling, separate Jibri instances pool với auto-scaling capability, Prosody XMPP server với high availability configuration, Nginx reverse proxy cho SSL termination và load balancing. Data Storage sử dụng PostgreSQL cho meeting metadata, participants, recordings, transcripts và summaries với proper indexing cho performance, MinIO/S3 cho video files với lifecycle policies, và Redis cho caching meeting tokens và temporary session data.

\textbf{Khả năng mở rộng và hiệu năng:}

Hệ thống được thiết kế để scale từ small teams đến large enterprises. Small workspaces (10-25 participants) chạy trên single Videobridge instance với shared Jibri pool, recordings processed sequentially với acceptable latency. Medium workspaces (25-100 participants) sử dụng multiple Videobridge instances với load balancing, dedicated Jibri instances để avoid queue waiting, parallel processing cho multiple recordings. Large enterprises (100+ participants) deploy geographically distributed Videobridge instances, high-capacity Jibri pool với priority queuing, distributed processing với multiple AI Service workers, và CDN integration cho efficient video delivery. Performance metrics được monitor continuously: average latency < 150ms for media streams, packet loss rate < 1\% under normal conditions, recording processing time < 5 minutes cho 1-hour meeting, transcript accuracy > 95\% với proper audio quality.

\textbf{Bảo mật và quyền riêng tư:}

Phân hệ Họp trực tuyến (Meeting) implement multiple security layers để protect sensitive business communications. Authentication và Authorization sử dụng JWT tokens với workspace-specific claims và short expiration (2 hours), role-based permissions cho recording controls (chỉ moderators có thể start/stop recording), và workspace membership verification trước khi allow join meeting. Media Encryption với DTLS-SRTP mandatory cho all media streams, end-to-end encryption (E2EE) optional sử dụng Insertable Streams API, và encrypted storage cho recordings trong S3 với AES-256. Access Control cho phép workspace owners configure external guest policies (allow, deny, require approval), set recording permissions per role, và define transcript visibility (workspace-only hoặc public link sharing). Compliance features bao gồm audit logs cho all meeting actions (create, join, record, delete), retention policies tuân theo legal requirements, và data export capabilities cho compliance reporting.

\textbf{Tích hợp với AI và automation:}

Phân hệ Họp trực tuyến (Meeting) tận dụng AI để tạo giá trị gia tăng từ meeting data. Automatic Summarization với LLM-powered analysis of full transcript, multi-language support (Vietnamese và English), customizable summary styles (brief, detailed, technical), và consistent quality regardless of meeting length. Action Items Extraction tự động identify tasks mentioned during meeting, assign to participants based on context, suggest due dates based on urgency keywords, và integrate với task management features của workspace. Meeting Analytics tracking participation patterns (who speaks most, engagement metrics), topic trends across multiple meetings, sentiment analysis để gauge meeting effectiveness, và recommendations để improve future meetings. Smart Scheduling suggestions dựa trên past meeting patterns, automatic meeting notes distribution, và follow-up reminders cho action items.

Phân hệ họp trực tuyến không chỉ là một video conferencing tool đơn thuần mà là một complete meeting management solution được thiết kế để maximize productivity và extract maximum value từ mỗi cuộc họp thông qua AI-powered automation và tight workspace integration.
