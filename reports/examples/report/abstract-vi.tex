\thispagestyle{empty}
\setsection{TÓM TẮT}

\sloppy

\noindent Bối cảnh: Trong môi trường phát triển phần mềm theo phương pháp Agile, các doanh nghiệp thường sử dụng nhiều công cụ riêng biệt để quản lý dự án, giao tiếp nội bộ và tổ chức họp trực tuyến. Điều này dẫn đến tình trạng thông tin người dùng, dữ liệu dự án và tài liệu bị phân tán, khiến việc theo dõi tiến độ, quản lý tập trung và khai thác hiệu quả dữ liệu trở nên khó khăn. Do đó, cần có một nền tảng trung tâm đóng vai trò kết nối và thống nhất toàn bộ hệ thống, giúp doanh nghiệp quản lý mọi hoạt động một cách mạch lạc và hiệu quả hơn.

\noindent Mục tiêu: Đề tài xây dựng một nền tảng trung tâm cho hệ thống quản lý dự án Agile tích hợp AI. Nền tảng này cung cấp khả năng quản lý tài khoản người dùng và không gian làm việc (workspace), một giao diện quản trị dành cho người quản lý workspace, khu vực lưu trữ và chia sẻ tài liệu chung, hệ thống thông báo tập trung, cùng với trợ lý AI thông minh có khả năng tự động tổng hợp thông tin từ các phân hệ quản lý dự án, truyền thông và họp trực tuyến để tạo ra các báo cáo hỗ trợ việc giám sát và ra quyết định.

\noindent Phương pháp: Nền tảng được xây dựng theo kiến trúc microservice với các dịch vụ độc lập giao tiếp qua API, bao gồm: cổng API Gateway đảm nhận xác thực, phân quyền và định tuyến yêu cầu; giao diện quản trị được phát triển bằng React/Next.js mang đến trải nghiệm người dùng hiện đại; dịch vụ lưu trữ tệp tập trung; hệ thống thông báo tự động thu thập và phân phối thông tin từ các phân hệ; và dịch vụ AI tích hợp với các nhà cung cấp bên thứ ba để phân tích dữ liệu. Hệ thống sử dụng cơ sở dữ liệu quan hệ để quản lý thông tin tài khoản và workspace, kết hợp với giải pháp lưu trữ đám mây cho tệp tin. Toàn bộ kiến trúc được thiết kế để dễ dàng mở rộng và bảo trì.

\noindent Kết quả: Sản phẩm hoàn thiện cho phép người dùng đăng ký, đăng nhập và truy cập toàn bộ các tính năng của hệ thống thông qua một cổng trung tâm. Người quản lý workspace có thể dễ dàng quản lý thành viên, cấu hình không gian làm việc, theo dõi thông báo và xem các báo cáo tổng hợp thông qua giao diện quản trị trực quan. Hệ thống thông báo giúp các thông tin quan trọng từ quản lý dự án, trò chuyện nhóm và cuộc họp được gửi đến đúng người vào đúng thời điểm. Đặc biệt, trợ lý AI được tích hợp sẵn có khả năng thu thập dữ liệu từ các hoạt động của dự án và tạo ra các bản tóm tắt, báo cáo ngắn gọn, giúp người quản lý nắm bắt nhanh tình hình mà không cần phải kiểm tra từng hệ thống riêng lẻ.

\noindent Kết luận: Nền tảng trung tâm đã được xây dựng thành công, đóng vai trò xương sống kết nối hệ thống quản lý dự án Agile. Sản phẩm giúp doanh nghiệp giảm thiểu sự phức tạp khi sử dụng nhiều công cụ riêng lẻ và tạo ra môi trường làm việc thống nhất. Với khả năng mở rộng linh hoạt và trợ lý AI tích hợp, nền tảng tạo nền tảng vững chắc để phát triển thêm các tính năng hỗ trợ ra quyết định và phân tích dữ liệu trong tương lai.

\fussy
