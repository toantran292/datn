\thispagestyle{empty}
\setsection{TÓM TẮT}

\sloppy

\noindent Bối cảnh: Trong môi trường phát triển phần mềm theo phương pháp Agile/Scrum, việc quản lý dự án hiệu quả đòi hỏi khả năng theo dõi sprints, issues, và tiến độ công việc một cách trực quan. Các công cụ quản lý dự án hiện tại thường thiếu tính linh hoạt trong việc tùy chỉnh workflow theo nhu cầu đặc thù của từng team, và chưa tận dụng được sức mạnh của trí tuệ nhân tạo (AI) để tự động hóa các tác vụ tốn thời gian như tổng hợp sprint retrospective và tạo danh sách công việc từ mô tả tổng quan. Do đó, cần có một nền tảng SaaS quản lý dự án tích hợp AI giúp teams làm việc hiệu quả hơn với custom workflow và AI-assisted features.

\noindent Mục tiêu: Đề tài xây dựng một nền tảng SaaS quản lý dự án Agile tích hợp AI với các chức năng cốt lõi: quản lý projects với custom status workflow, sprint lifecycle management (FUTURE-ACTIVE-CLOSED), Board/Backlog views với drag-and-drop, issue tracking (STORY/TASK/BUG/EPIC), project-level access control (Project Lead/Team Member/Viewer), và hai tính năng AI chính: (1) AI Sprint Summary tự động generate retrospective khi complete sprint với structured format (overview, achievements, issues found, recommendations, metrics), và (2) Tự động tạo công việc từ mô tả cho phép users nhập mô tả tổng quan hoặc user story, sau đó AI phân tích và tạo danh sách công việc với thông tin chi tiết (name, description, type, priority, story points). Hệ thống hỗ trợ real-time collaboration qua WebSocket và tích hợp multiple LLM providers (OpenAI GPT-4, Anthropic Claude, Google Gemini).

\noindent Phương pháp: Hệ thống được xây dựng theo kiến trúc microservices với Account Service (Spring Boot) xử lý authentication qua JWT và Google OAuth 2.0, và PM Service (NestJS + Prisma ORM) quản lý projects, sprints, issues, custom statuses với WebSocket Gateway cho real-time updates. Frontend sử dụng Next.js 14 với App Router, MobX cho state management, và DnD Kit cho drag-and-drop functionality trên Board view. Database PostgreSQL với composite indexes (project\_id, sprint\_id, status\_id) optimize Board queries, Redis cache frequently accessed data và hỗ trợ WebSocket Pub/Sub. AI integration thông qua LangChain.js với structured output dùng Zod schemas. Issue attachments lưu trên MinIO (development) và S3 (production) với path format org-id/project-id/issue-id/filename. Nginx API Gateway đảm nhận SSL termination, load balancing, và routing requests.

\noindent Kết quả: Sản phẩm hoàn thiện cho phép Organization Owners tạo và quản lý multiple projects, Project Leads quản lý sprints với lifecycle FUTURE → ACTIVE → CLOSED (chỉ một ACTIVE sprint per project), configure custom statuses (TODO/IN\_PROGRESS/DONE categories), manage project members với roles. Team Members có thể tạo issues với types (STORY, TASK, BUG, EPIC), drag-and-drop issues giữa custom status columns trên Board view với real-time synchronization, assign issues, add comments với mentions support, upload attachments (screenshots, documents, logs). Hai tính năng AI chính đã được tích hợp thành công: (1) AI Sprint Summary generate structured retrospective (overview, achievements, issues found, recommendations, metrics) khi Project Lead complete sprint, giúp team nhanh chóng review sprint retrospective mà không cần tổng hợp thủ công; (2) Tự động tạo công việc từ mô tả cho phép users nhập mô tả tổng quan (requirements, user story, feature description), sau đó AI phân tích và tạo danh sách công việc với preview screen để users review, edit, uncheck các công việc không mong muốn trước khi confirm tạo vào project. Board view với drag-and-drop và Backlog view giúp teams visualize workflow progress và prioritize issues hiệu quả.

\noindent Kết luận: Nền tảng SaaS quản lý dự án Agile tích hợp AI đã được xây dựng thành công với đầy đủ các chức năng cốt lõi và hai tính năng AI nổi bật (AI Sprint Summary và tự động tạo công việc từ mô tả). Hệ thống giúp software development teams, product teams, startups, và SMEs áp dụng Agile/Scrum hiệu quả với custom workflow flexibility, real-time collaboration, và AI automation cho time-consuming tasks như sprint retrospective và task creation. Với kiến trúc microservices scalable và multi-tenant architecture, nền tảng có thể serve multiple organizations với data isolation hoàn toàn. Hướng phát triển tương lai bao gồm: Git repository integration (GitHub, GitLab, Bitbucket) để link commits với issues, advanced Analytics dashboard với burndown charts và velocity trends, mobile apps (React Native/Flutter) cho on-the-go management, integration với Slack/Teams/Discord cho notifications, và mở rộng AI capabilities với Issue Description Generation từ title, Task Breakdown cho EPICs, Code Review Integration, Risk Prediction, và Story Points Estimation.

\fussy
