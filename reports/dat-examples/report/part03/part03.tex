% !TEX root = ../main.tex

% Reset numbering for Part 3
\renewcommand{\thesubsection}{\arabic{section}.\arabic{subsection}}
\setcounter{section}{0}
\setcounter{subsection}{0}

\phantomsection
\addcontentsline{toc}{section}{Phần 3: KẾT LUẬN}
\section*{Phần 3: KẾT LUẬN}

\phantomsection
\addcontentsline{toc}{subsection}{1. Kết quả đạt được}
\subsection*{1. Kết quả đạt được}

\subsubsection*{1.1. Về lý thuyết}

\begin{itemize}
    \item Rèn luyện được kỹ năng giải quyết vấn đề, đọc hiểu tài liệu và tra cứu thông tin kỹ thuật trong lĩnh vực quản lý dự án Agile.
    \item Nâng cao được khả năng tư duy, phân tích và thiết kế cơ sở dữ liệu quan hệ cho hệ thống quản lý dự án phức tạp với nhiều mối quan hệ giữa projects, sprints, issues và custom statuses.
    \item Học hỏi thêm kiến thức về các công nghệ hiện đại như NestJS với Prisma ORM, Next.js 14 với App Router, Spring Boot cho authentication service, PostgreSQL với composite indexes, Redis cho caching và WebSocket support, MobX cho client state management, DnD Kit cho drag-and-drop functionality, và các dịch vụ cloud như MinIO/S3 cho issue attachments.
    \item Hiểu được quy trình phát triển ứng dụng web theo mô hình SaaS (Software as a Service) với kiến trúc microservices và multi-tenant isolation ở organization-level với project-level access control.
    \item Tìm hiểu và áp dụng được các API của các nhà cung cấp LLM như OpenAI GPT-4, Anthropic Claude, và Google Gemini thông qua LangChain.js vào việc tạo AI Sprint Summary khi complete sprint, Issue Description Generation từ title, và Task Breakdown cho EPIC issues thành sub-tasks.
    \item Nghiên cứu và hiểu rõ hơn về các phương pháp xác thực và phân quyền trong hệ thống web, bao gồm JWT, Google OAuth 2.0 và project-level RBAC với các roles: Project Lead, Team Member, Viewer và custom permissions override.
    \item Nắm vững kiến trúc real-time collaborative editing với WebSocket rooms theo project, optimistic UI updates cho drag-and-drop operations, và broadcast mechanisms cho Board view synchronization giữa multiple users.
\end{itemize}

\subsubsection*{1.2. Về chương trình}

\begin{itemize}
    \item Phát triển thành công một nền tảng SaaS quản lý dự án Agile hoàn chỉnh với kiến trúc microservices trên nền tảng công nghệ Spring Boot (Account Service), NestJS (PM Service) và Next.js 14, đáp ứng đầy đủ nhu cầu quản lý projects, sprints, issues với Board/Backlog views intuitive, custom status workflow, và AI-assisted features. Giao diện được thiết kế trực quan lấy cảm hứng từ các nền tảng quản lý dự án hiện đại như Linear, Jira, ClickUp, tối ưu cho collaborative project management với drag-and-drop, mang lại trải nghiệm mượt mà cho người dùng.
    \item Về phía người dùng, hệ thống cho phép thực hiện các thao tác từ đăng ký, đăng nhập qua Google OAuth đến tạo và quản lý projects với custom workflow. Team Members có thể tạo issues (STORY, TASK, BUG, EPIC), drag-and-drop issues giữa các custom status columns trên Board view, assign issues, add comments với mentions, upload attachments (screenshots, documents, logs), và track progress qua sprint velocity. Project Leads có thể tạo sprints với lifecycle management (FUTURE, ACTIVE, CLOSED), start và complete sprints, configure custom statuses cho project, manage project members với roles, và request AI-generated Sprint Summary khi complete sprint. Đặc biệt, người dùng có thể tận dụng sức mạnh của các mô hình ngôn ngữ lớn (LLM) để tự động generate issue descriptions từ title, break down EPIC thành sub-tasks, và tổng hợp sprint retrospective với insights và recommendations.
    \item Về phía quản trị, Project Leads có đầy đủ quyền quản lý project settings, custom status configuration, member roles (PROJECT\_LEAD, TEAM\_MEMBER, VIEWER), và AI settings (preferred LLM provider, custom prompts). Organization Owners có thể quản lý multiple projects, organization members, storage usage, và view analytics dashboard với sprint velocity trends, issue completion rates across projects. Hệ thống ghi lại đầy đủ audit log và activity feed cho mọi actions như issue\_created, status\_changed, sprint\_started, comment\_added để track project progress và compliance.
    \item Hệ thống được tích hợp lưu trữ issue attachments qua MinIO (development) và S3 (production) với path format org-id/project-id/issue-id/filename, giúp quản lý và phân phối files an toàn và hiệu quả. Một điểm nổi bật đặc biệt là ba tính năng AI: (1) AI Sprint Summary tự động generate retrospective khi Project Lead complete sprint với structured format (overview, achievements, issues found, recommendations, metrics), (2) AI Issue Description Generation cho phép users request LLM tạo detailed description từ issue title, và (3) AI Task Breakdown decompose EPIC issues thành sub-tasks với story points estimates. Users có thể chọn LLM Provider (OpenAI GPT-4, Anthropic Claude, Google Gemini) và customize prompts theo nhu cầu. Nhìn chung, ứng dụng không chỉ đáp ứng đầy đủ các chức năng quản lý dự án Agile cốt lõi mà còn mang lại những trải nghiệm hiện đại với real-time collaboration, AI assistance, và custom workflow flexibility, tạo sự khác biệt so với các nền tảng quản lý dự án truyền thống.
\end{itemize}

\subsubsection*{1.3. Về vận dụng thực tế}

\begin{itemize}
    \item Kết quả của quá trình kiểm thử và đánh giá cũng cho thấy, ứng dụng web có khả năng hoạt động một cách ổn định và hiệu quả với real-time collaboration qua WebSocket, drag-and-drop smoothness, và optimistic UI updates mang lại responsive experience. Nhìn chung, "Xây dựng Nền tảng SaaS tích hợp AI nhằm Thống nhất Quản lý Dự án Agile" đã được xây dựng thành công và đáp ứng đầy đủ các chức năng cốt lõi của một hệ thống quản lý dự án Agile, bao gồm quản lý projects với custom workflow, sprint lifecycle management (FUTURE-ACTIVE-CLOSED), Board/Backlog views với drag-and-drop, issue tracking (STORY/TASK/BUG/EPIC), project-level access control, và ba tính năng AI (Sprint Summary, Issue Description Generation, Task Breakdown). Bên cạnh đó, giao diện được thiết kế hiện đại, trực quan theo phong cách Linear/Jira với Board columns cho custom statuses, giúp teams dễ dàng visualize workflow và manage sprints hiệu quả.
    \item Điểm nổi bật của ứng dụng là tính năng tích hợp AI để tự động hóa các tác vụ time-consuming: AI Sprint Summary cho phép Project Leads nhanh chóng có được sprint retrospective với insights, achievements, issues found và recommendations mà không cần tốn nhiều thời gian tổng hợp thủ công; AI Issue Description Generation giúp teams tạo detailed descriptions với acceptance criteria từ issue title; AI Task Breakdown decomposes EPICs thành actionable sub-tasks với story points estimates. Ngoài ra, ứng dụng còn được tích hợp các tính năng nâng cao trải nghiệm collaborative editing như WebSocket real-time updates broadcast changes đến tất cả project members ngay lập tức, activity feed để track project timeline với before/after changes, và phân quyền linh hoạt theo vai trò project-level (Project Lead, Team Member, Viewer) với custom permissions override.
    \item Với những ưu điểm này, hệ thống hoàn toàn có thể triển khai trong thực tế để phục vụ các software development teams, product teams, startup, doanh nghiệp vừa và nhỏ áp dụng phương pháp Agile/Scrum trong việc quản lý sprints, track issues, và collaborate real-time. Board view với drag-and-drop giúp teams visualize workflow progress, Backlog view giúp prioritize issues, và AI features giúp automate retrospective analysis và task planning. Hệ thống với custom status workflow có thể adapt với different team processes (Kanban-style, Scrum-style, custom workflows), và với multi-tenant architecture có thể serve multiple organizations với data isolation hoàn toàn, tạo ra trải nghiệm chuyên nghiệp và scalable cho users.
\end{itemize}

\phantomsection
\addcontentsline{toc}{subsection}{2. Hướng phát triển}
\subsection*{2. Hướng phát triển}

\begin{itemize}
    \item Mở rộng tính năng AI với thêm các capabilities như AI Code Review Integration để analyze code changes trong issues, AI Risk Prediction dự đoán sprint risks dựa trên historical data và velocity trends, AI Story Points Estimation tự động suggest story points cho issues dựa trên similar past issues, và AI Sprint Planning Assistant recommend optimal issue allocation cho upcoming sprints. Tích hợp chatbot AI vào platform để users có thể query project data, ask questions về sprint progress, và get recommendations thông qua natural language interface.
    \item Phát triển advanced Board customization với multiple Board views per project (Kanban board không giới hạn WIP, Sprint board với swimlanes theo assignee/priority, Roadmap view với timeline visualization), saved filters và quick filters cho power users, Board templates cho common workflows, và Board sharing với external stakeholders qua public links với read-only access.
    \item Tích hợp Git repository connections (GitHub, GitLab, Bitbucket) để link commits và pull requests với issues, auto-update issue status khi PR merged, track code changes related đến issues, và display deployment status trên Board view. Trong tương lai, có thể tích hợp các công cụ CI/CD như Jenkins, CircleCI, GitHub Actions để auto-create issues khi build fails và track deployment pipeline status.
    \item Xây dựng ứng dụng mobile native cho iOS và Android sử dụng React Native hoặc Flutter để users có thể view Board, update issue status, add comments, upload attachments, và receive push notifications cho issue assignments và mentions mọi lúc mọi nơi. Mobile app sẽ sync real-time với web app qua WebSocket.
    \item Tích hợp với các công cụ communication phổ biến như Slack (post sprint summaries, issue updates), Microsoft Teams (notifications cho mentions), Discord (webhooks cho project activities) để keep teams informed without switching apps. Tích hợp với time tracking tools như Toggl, Harvest để track actual time spent on issues và compare với story points estimates.
    \item Phát triển comprehensive Analytics dashboard với advanced metrics: burndown charts cho sprints, velocity trends across sprints, cycle time analysis (average time từ To Do đến Done), cumulative flow diagrams cho workflow bottlenecks, sprint retrospective analytics comparing planned vs actual completion, issue type distribution, assignee workload balance, và custom metrics định nghĩa bởi Project Leads. Export analytics reports dưới dạng PDF hoặc CSV cho stakeholder presentations.
    \item Implement advanced collaboration features như real-time co-editing cho issue descriptions sử dụng operational transformation hoặc CRDTs, video/voice calls integration trực tiếp trong issue detail panels, whiteboard functionality cho sprint planning sessions, và @mentions notifications với in-app và email delivery options.
\end{itemize}
