\phantomsection
\subsection*{6. Những đóng góp chính của đề tài}
\addcontentsline{toc}{subsection}{6. Những đóng góp chính của đề tài}
\setcounter{subsection}{6}
\setcounter{subsubsection}{0}
\setcounter{figure}{0}

\begin{itemize}
    \item \textbf{Về mặt kinh tế -- tổ chức:}

    Đề tài xây dựng một phân hệ quản lý dự án theo phương pháp Agile/Scrum với khả năng tùy biến cao và tích hợp AI, giúp các nhóm phát triển phần mềm và doanh nghiệp quản lý công việc hiệu quả hơn. Phân hệ cung cấp đầy đủ các tính năng cần thiết cho quy trình Scrum như Sprint Planning (Backlog view), Sprint Execution (Board view với drag-and-drop), Sprint Review (AI-generated Sprint Summary) và Retrospective (Activity Feed). Việc tích hợp AI để tự động hóa các tác vụ lặp đi lặp lại như viết issue description, breakdown Epic thành subtasks, và tổng hợp sprint summary giúp tiết kiệm thời gian cho Project Lead và Team Members, cho phép họ tập trung vào công việc có giá trị hơn. Khi được triển khai thực tế, phân hệ có thể góp phần tăng năng suất làm việc, cải thiện khả năng theo dõi tiến độ theo thời gian thực, và hỗ trợ ra quyết định dựa trên dữ liệu về hiệu suất sprint.

    \item \textbf{Về mặt xã hội -- cộng tác:}

    Đề tài cung cấp một giải pháp quản lý dự án hỗ trợ làm việc nhóm trong bối cảnh làm việc từ xa và hybrid ngày càng phổ biến. Phân hệ quản lý dự án cho phép nhiều thành viên cùng làm việc đồng thời trên cùng một project với các vai trò khác nhau (Project Lead, Team Member, Viewer/Stakeholder), mỗi vai trò có quyền hạn phù hợp để đảm bảo tính bảo mật và hiệu quả công việc. Tính năng real-time collaboration thông qua WebSocket giúp các thành viên thấy ngay lập tức khi có người khác kéo thả issue, thay đổi trạng thái, hoặc thêm comment, tạo cảm giác làm việc cùng nhau dù ở xa. Activity Feed ghi lại toàn bộ lịch sử thay đổi giúp các thành viên dễ dàng nắm bắt diễn biến dự án, ai đã làm gì và khi nào, tăng tính minh bạch và trách nhiệm. Board view và Backlog view với drag-and-drop trực quan giúp team dễ dàng thảo luận và điều chỉnh công việc trong daily standup và sprint planning mà không cần công cụ phức tạp.

    \item \textbf{Về mặt giáo dục -- nghiên cứu:}

    Đề tài là một ví dụ cụ thể về việc áp dụng các công nghệ và kỹ thuật hiện đại trong phát triển ứng dụng web full-stack: kiến trúc microservice với PM Service (NestJS + Prisma ORM + PostgreSQL) xử lý backend logic, pm-web (Next.js 14 App Router + React Query + dnd-kit) xây dựng giao diện người dùng interactive, WebSocket Gateway cho real-time communication, fractional indexing để xử lý drag-and-drop ordering, và tích hợp LLM API (OpenAI GPT-4o-mini) để thực hiện các tính năng AI. Sản phẩm và quá trình thực hiện giúp sinh viên hiểu sâu về: (1) thiết kế database schema với relationships phức tạp và cross-service references trong môi trường microservice, (2) xây dựng RESTful API với validation, error handling và transaction management, (3) state management và optimistic updates trong React application, (4) drag-and-drop implementation với collision detection và fractional indexing algorithm, (5) real-time broadcasting với WebSocket và room-based messaging, (6) prompt engineering và LLM integration để tạo các tính năng AI thực tế. Kết quả đề tài có thể được sử dụng làm tài liệu tham khảo cho các khóa sau khi nghiên cứu về phát triển ứng dụng Agile Project Management với TypeScript full-stack và AI integration.
\end{itemize}
