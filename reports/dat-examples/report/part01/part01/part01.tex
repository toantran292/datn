\phantomsection
\subsection*{1. Đặt vấn đề}
\addcontentsline{toc}{subsection}{1. Đặt vấn đề}
\setcounter{subsection}{1}
\setcounter{subsubsection}{0}
\setcounter{figure}{0}

Trong bối cảnh chuyển đổi số diễn ra mạnh mẽ, ngày càng nhiều doanh nghiệp và nhóm phát triển phần mềm lựa chọn phương pháp Agile nhằm rút ngắn vòng đời phát triển, tăng khả năng thích ứng với thay đổi và nâng cao chất lượng sản phẩm. Các báo cáo tổng hợp gần đây cho thấy mức độ áp dụng Agile trong các nhóm phát triển phần mềm đã tăng rất nhanh: theo báo cáo State of Agile lần thứ 15, tỷ lệ đội ngũ phát triển phần mềm áp dụng Agile đã tăng từ 37\% năm 2020 lên 86\% năm 2021, phản ánh việc Agile đã trở thành phương pháp chủ đạo trong nhiều tổ chức thay vì chỉ là một xu hướng thử nghiệm~\cite{digitalai_state_of_agile_15}. Song song đó, thị trường phần mềm quản lý dự án cũng tăng trưởng mạnh mẽ; một số nghiên cứu thị trường ước tính quy mô thị trường phần mềm quản lý dự án toàn cầu đạt khoảng 9{,}76 tỷ USD vào năm 2025 và có thể tăng lên hơn 20 tỷ USD vào năm 2030 với tốc độ tăng trưởng trung bình trên 15\% mỗi năm~\cite{mordor_pm_software_market_2025}. Những con số này cho thấy việc sử dụng các nền tảng SaaS để hỗ trợ quản lý dự án và cộng tác đã trở thành xu thế tất yếu trong ngành công nghệ.

Tuy nhiên, sự phát triển nhanh của các công cụ lại kéo theo hiện tượng ``bùng nổ ứng dụng'' trong môi trường làm việc. Một nhóm Agile điển hình thường phải đồng thời sử dụng công cụ quản lý công việc (như Jira, Linear), công cụ ghi chú và tài liệu (Notion, Confluence), nền tảng nhắn tin (Slack, Microsoft Teams) và các hệ thống họp trực tuyến (Google Meet, Zoom). Các nghiên cứu về năng suất lao động cho thấy việc liên tục chuyển đổi giữa các ứng dụng này gây tổn thất đáng kể về thời gian và sự tập trung. Một khảo sát do RingCentral công bố cho thấy có tới 69\% người lao động lãng phí đến 60 phút mỗi ngày chỉ để điều hướng giữa các ứng dụng làm việc khác nhau, tương đương khoảng 32 ngày làm việc mỗi năm~\cite{ringcentral_connected_workplace}. Một nghiên cứu khác, dựa trên dữ liệu từ Asana, cho biết nhân viên văn phòng phải sử dụng 10 ứng dụng hoặc hơn mỗi ngày và có thể mất trung bình khoảng 3{,}6 giờ mỗi tuần do bị gián đoạn bởi việc chuyển đổi qua lại giữa các công cụ~\cite{asana_context_switching_basicops}. Thời gian bị bào mòn bởi việc chuyển ngữ cảnh không chỉ làm giảm năng suất mà còn gia tăng gánh nặng tinh thần cho người sử dụng.

Trong thực tế triển khai các dự án Agile, dữ liệu cốt lõi của hệ thống -- từ thông tin tài khoản người dùng, cấu hình workspace, phân quyền truy cập, cho đến tệp tài liệu dùng chung, lịch sử thông báo và các sự kiện hệ thống -- thường bị phân tán giữa nhiều dịch vụ khác nhau. Người quản lý muốn nắm được bức tranh tổng thể về một dự án phải truy cập nhiều hệ thống và tự ghép nối thông tin từ nhiều nguồn; trong khi đó, các thành viên dự án tốn không ít thời gian để tìm lại tài liệu, kiểm tra thông báo hoặc đăng nhập qua nhiều lớp hệ thống khác nhau. Việc thiếu vắng một lớp \textit{nền tảng và thông tin} thống nhất làm xương sống khiến cho các phân hệ chức năng khó chia sẻ dữ liệu với nhau một cách ổn định, đồng thời làm tăng độ phức tạp trong vận hành khi quy mô dự án và số lượng người dùng ngày càng lớn.

Song song với sự phát triển của Agile và các nền tảng SaaS, trí tuệ nhân tạo (AI) cũng đang được tích hợp ngày càng sâu vào quy trình phát triển phần mềm và quản lý dự án. Các báo cáo gần đây về State of Agile cho thấy AI đang chuyển từ vai trò công cụ hỗ trợ sang vị trí một thành phần trung tâm trong chu trình cung cấp phần mềm, với tỷ lệ đội ngũ sử dụng AI cho các hoạt động liên quan đến Agile tăng từ khoảng 64\% lên 84\% chỉ trong một năm~\cite{digitalai_state_of_agile_18_ai_itpro}. AI được kỳ vọng có thể hỗ trợ tóm tắt nội dung, tổng hợp thông tin, phát hiện rủi ro và cung cấp góc nhìn thời gian thực về tiến độ dự án. Tuy nhiên, phần lớn các tính năng AI hiện tại vẫn gắn với từng công cụ riêng lẻ và chỉ ``nhìn thấy'' phần dữ liệu thuộc về công cụ đó. Khi không có một lớp nền tảng đứng giữa để gom, chuẩn hoá và phân phối dữ liệu từ các phân hệ như quản lý dự án, truyền thông nội bộ, họp trực tuyến, AI khó có thể hình thành được bức tranh toàn diện về dự án để đưa ra các báo cáo và gợi ý có chiều sâu.

Thực tế trên đặt ra nhu cầu về một phân hệ \textit{nền tảng và thông tin} đóng vai trò xương sống trong hệ sinh thái SaaS tích hợp AI phục vụ quản lý dự án Agile. Phân hệ này cần đảm nhiệm đồng thời nhiều chức năng: quản lý tập trung tài khoản và workspace; cung cấp một lớp \textit{edge} hoạt động như API Gateway để xử lý xác thực, phân quyền và định tuyến yêu cầu đến các dịch vụ phía sau; quản lý lưu trữ và truy xuất tệp dùng chung giữa các phân hệ; tập trung hoá và phân phối thông báo từ các hệ thống con; đồng thời là điểm hội tụ dữ liệu để dịch vụ AI có thể thu thập và tổng hợp thông tin từ các phân hệ khác như quản lý dự án, truyền thông và họp trực tuyến, qua đó tạo ra các bản tóm tắt và báo cáo phục vụ công tác giám sát và điều hành. Khi lớp nền tảng và thông tin được thiết kế tốt, hệ thống không chỉ giảm bớt sự rời rạc trong trải nghiệm người dùng mà còn tạo nền tảng kỹ thuật vững chắc để mở rộng các tính năng phân tích và hỗ trợ ra quyết định trong tương lai.

Với ý nghĩa đó, đề tài \textit{``Xây dựng Nền tảng SaaS tích hợp AI nhằm Thống nhất Quản lý Dự án Agile -- Phân hệ quản lý dự án''} được đề xuất và thực hiện, nhằm góp phần giải quyết bài toán tập trung hóa dữ liệu, giảm thiểu tình trạng phân mảnh công cụ và xây dựng một lớp nền tảng dùng chung cho các hệ thống con trong môi trường quản lý dự án Agile hiện đại.
