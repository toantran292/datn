\phantomsection
\subsection*{3. Mục tiêu đề tài}
\addcontentsline{toc}{subsection}{3. Mục tiêu đề tài}
\setcounter{subsection}{3}
\setcounter{subsubsection}{0}
\setcounter{figure}{0}

\subsubsection{Mục tiêu tổng quát}

Phát triển phân hệ quản lý dự án cho nền tảng SaaS tích hợp AI nhằm thống nhất quản lý dự án Agile, cung cấp đầy đủ các tính năng cốt lõi để lập kế hoạch sprint, quản lý công việc, theo dõi tiến độ và phân tích hiệu suất dự án. Phân hệ này xây dựng trên kiến trúc microservice với giao diện trực quan, hỗ trợ quy trình làm việc linh hoạt theo phương pháp Scrum, cho phép các nhóm dự án tổ chức backlog, quản lý sprint, theo dõi issue với nhiều loại công việc (Story, Task, Bug, Epic), tùy chỉnh luồng trạng thái và phân tích dữ liệu dự án một cách hiệu quả.

\subsubsection{Mục tiêu cụ thể}

\textbf{Mục tiêu chuyên môn:}

\begin{itemize}
    \item Rèn luyện kỹ năng thiết kế và triển khai hệ thống quản lý dự án theo kiến trúc microservice, áp dụng phương pháp Agile/Scrum vào sản phẩm thực tế. Xây dựng backend service sử dụng NestJS với Prisma ORM kết nối PostgreSQL, thiết kế API RESTful tuân thủ chuẩn OpenAPI/Swagger, đảm bảo tính mở rộng, bảo trì và khả năng tích hợp với các phân hệ khác trong nền tảng.

    \item Phát triển module quản lý dự án (Project Management) cung cấp các chức năng tạo, cấu hình dự án với identifier duy nhất, gán project lead và default assignee, quản lý sequence ID cho từng issue, thiết lập các thông số cơ bản phục vụ cho việc tổ chức và vận hành dự án theo workspace.

    \item Xây dựng module quản lý sprint với đầy đủ vòng đời: tạo mới sprint với mục tiêu và khoảng thời gian, chuyển trạng thái từ FUTURE sang ACTIVE khi bắt đầu, và chuyển sang CLOSED khi hoàn thành; hỗ trợ hiển thị tổng quan sprint với số lượng issue theo trạng thái (TODO, IN\_PROGRESS, DONE) và theo dõi tiến độ thực hiện.

    \item Phát triển module quản lý issue với các loại công việc đa dạng (Story, Task, Bug, Epic), hỗ trợ phân loại theo mức độ ưu tiên (Low, Medium, High, Critical), story points cho việc ước lượng, thiết lập ngày bắt đầu/mục tiêu, gán nhiều người thực hiện, thiết lập quan hệ cha-con giữa các issue và theo dõi lịch sử thay đổi chi tiết.

    \item Thiết kế và triển khai hệ thống trạng thái tùy chỉnh (Custom Issue Status) cho phép mỗi dự án định nghĩa luồng công việc riêng với các trạng thái, màu sắc và thứ tự hiển thị phù hợp, hỗ trợ drag-and-drop để sắp xếp lại trạng thái và áp dụng linh hoạt cho nhiều quy trình làm việc khác nhau.

    \item Xây dựng module bình luận (Comments) và theo dõi hoạt động (Activity Log) để ghi nhận tất cả các thay đổi trên issue bao gồm thông tin trước/sau khi chỉnh sửa, người thực hiện và thời gian, tạo audit trail đầy đủ cho việc truy vết và đánh giá tiến độ công việc.

    \item Phát triển module phân tích và thống kê (Analytics) cung cấp biểu đồ theo dõi số lượng issue được tạo và hoàn thành theo thời gian, thống kê phân bố issue theo loại, mức độ ưu tiên và trạng thái, hỗ trợ người quản lý đánh giá hiệu suất và xu hướng của dự án.

    \item Xây dựng giao diện người dùng với NextJS 14 (App Router) kết hợp MobX để quản lý state phản ứng, tích hợp thư viện Pragmatic Drag and Drop của Atlassian cho tính năng kéo thả issue trên bảng Kanban và backlog, sử dụng design system tùy chỉnh để đảm bảo trải nghiệm người dùng nhất quán và trực quan.

    \item Phát triển hai giao diện chính là Board View (bảng Kanban theo cột trạng thái) và Backlog View (danh sách sprint và backlog), tích hợp panel chi tiết issue với khả năng chỉnh sửa nhanh, hiển thị comments, activity log và các thuộc tính liên quan, hỗ trợ thao tác drag-and-drop để di chuyển và sắp xếp lại issue giữa các sprint và trạng thái.

    \item Nâng cao kỹ năng làm việc với cơ sở dữ liệu quan hệ PostgreSQL thông qua Prisma ORM, thiết kế schema với các ràng buộc và quan hệ phức tạp, viết migration và tối ưu hóa truy vấn; đồng thời rèn luyện kỹ năng viết tài liệu kỹ thuật, kiểm thử và quản lý tiến độ thực hiện đồ án.
\end{itemize}

\textbf{Mục tiêu về sản phẩm:}

\begin{itemize}
    \item Xây dựng phân hệ quản lý dự án hoàn chỉnh với đầy đủ tính năng để triển khai và sử dụng độc lập, đồng thời có khả năng tích hợp với các phân hệ khác trong nền tảng (truyền thông, họp trực tuyến, nền tảng và thông tin). Hệ thống cho phép tạo và quản lý nhiều dự án trong workspace, cấu hình thông tin dự án với identifier riêng biệt, gán người phụ trách và thiết lập các tham số vận hành phù hợp với từng nhóm làm việc.

    \item Cung cấp công cụ lập kế hoạch sprint theo phương pháp Scrum, cho phép tạo sprint với mục tiêu rõ ràng, khoảng thời gian xác định, chuyển đổi trạng thái sprint từ kế hoạch sang thực hiện và hoàn thành; hiển thị tổng quan sprint với số lượng công việc theo từng trạng thái và tiến độ hoàn thành để người quản lý dễ dàng theo dõi và điều phối.

    \item Xây dựng hệ thống quản lý issue linh hoạt với nhiều loại công việc (Story cho tính năng nghiệp vụ, Task cho công việc kỹ thuật, Bug cho lỗi cần sửa, Epic cho nhóm công việc lớn), hỗ trợ phân loại mức độ ưu tiên, ước lượng điểm story points, gán nhiều người thực hiện, thiết lập mối quan hệ cha-con và theo dõi đầy đủ lịch sử thay đổi của từng issue.

    \item Cung cấp giao diện Board View dạng bảng Kanban trực quan với các cột tương ứng với trạng thái công việc, hỗ trợ kéo thả issue giữa các cột để thay đổi trạng thái nhanh chóng, hiển thị thông tin quan trọng trên thẻ issue (loại công việc, mức độ ưu tiên, người thực hiện, thời hạn) giúp nhóm dự án dễ dàng nắm bắt và cập nhật tiến độ công việc theo thời gian thực.

    \item Phát triển giao diện Backlog View cho phép tổ chức danh sách công việc theo sprint và backlog, hỗ trợ kéo thả issue giữa các sprint để sắp xếp lại kế hoạch, hiển thị tổng hợp số lượng issue theo trạng thái cho từng sprint, tích hợp panel chi tiết issue bên cạnh để xem và chỉnh sửa thông tin công việc mà không cần chuyển trang.

    \item Tích hợp hệ thống bình luận và theo dõi hoạt động cho từng issue, ghi nhận đầy đủ các thay đổi về tiêu đề, mô tả, trạng thái, người thực hiện, mức độ ưu tiên và các thuộc tính khác kèm thông tin người thực hiện và thời gian, tạo audit trail chi tiết để theo dõi quá trình phát triển và giải quyết vấn đề.

    \item Cung cấp module phân tích và báo cáo với biểu đồ theo dõi số lượng issue được tạo và hoàn thành theo thời gian (Created vs Resolved), thống kê phân bố issue theo loại công việc, mức độ ưu tiên và trạng thái hiện tại, giúp người quản lý đánh giá năng suất làm việc, phát hiện điểm nghẽn và dự báo tiến độ dự án.

    \item Đảm bảo hệ thống vận hành ổn định với kiến trúc microservice, API RESTful được thiết kế rõ ràng theo chuẩn OpenAPI, database được tối ưu hóa với PostgreSQL và Prisma ORM, giao diện phản hồi nhanh nhờ MobX và React Query; đồng thời cung cấp tài liệu kỹ thuật đầy đủ về thiết kế, API và hướng dẫn triển khai để hỗ trợ bảo trì và phát triển tiếp trong tương lai.
\end{itemize}