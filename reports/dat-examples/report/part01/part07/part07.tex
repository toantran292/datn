\phantomsection
\subsection*{7. Bố cục quyển báo cáo}
\addcontentsline{toc}{subsection}{7. Bố cục quyển báo cáo}
\setcounter{subsection}{7}
\setcounter{subsubsection}{0}
\setcounter{figure}{0}

Bố cục luận văn dự kiến gồm các phần: mục lục, danh mục hình, danh mục bảng, danh mục từ viết tắt, tóm tắt, abstract và ba phần chính: giới thiệu, nội dung, kết luận.

\begin{itemize}
    \item Phần \textbf{Giới thiệu} trình bày bối cảnh và lý do chọn đề tài về quản lý dự án Agile, tổng quan các công cụ quản lý dự án hiện có (Jira, Linear, ClickUp) trong và ngoài nước, xác định khoảng trống nghiên cứu về tích hợp AI trong quy trình Scrum, mục tiêu xây dựng phân hệ quản lý dự án với Board/Backlog views và AI features, đối tượng -- phạm vi nghiên cứu (PM Service và pm-web), quy trình nghiên cứu từ Agile/Scrum theory đến implementation và testing, công nghệ sử dụng (NestJS, Next.js, Prisma, PostgreSQL, WebSocket, OpenAI API), và những đóng góp dự kiến của đề tài về mặt kinh tế, xã hội và giáo dục.

    \item Phần \textbf{Nội dung} là trọng tâm của luận văn, bao gồm: \textbf{Chương 1} mô tả chi tiết bài toán quản lý dự án Agile/Scrum, yêu cầu chức năng thông qua use case diagrams cho 3 roles (Project Lead, Team Member, Viewer/Stakeholder), yêu cầu phi chức năng về hiệu năng và real-time collaboration; \textbf{Chương 2} phân tích và đánh giá các giải pháp hiện có (Jira Software với Atlassian Intelligence, Linear với AI features, ClickUp với automation); \textbf{Chương 3} thiết kế hệ thống bao gồm kiến trúc microservice (PM Service + pm-web + Account Service integration), thiết kế database schema (Project, Sprint, Issue, IssueStatus, IssueComment, IssueActivity, ProjectMember) với Prisma ORM, class diagram với entities và relationships, thiết kế UI wireframes (Projects list, Board view, Backlog view, Calendar view, Timeline view, Issue detail), thiết kế chức năng (Project management, Sprint lifecycle, Issue CRUD, Board/Backlog drag-and-drop, Members management, Activity tracking, AI features), và activity diagrams cho 16 use cases; \textbf{Chương 4} kiểm thử với 12 test scenarios và 46 test cases covering tất cả chức năng chính, đạt 100\% success rate.

    \item Phần \textbf{Kết luận} tổng hợp các kết quả chính của đề tài về phân hệ quản lý dự án Agile/Scrum với tích hợp AI, nêu rõ những ưu điểm (drag-and-drop mượt mà với fractional indexing, real-time collaboration với WebSocket, AI automation tiết kiệm thời gian), hạn chế (AI features phụ thuộc vào LLM API external, chưa hỗ trợ offline mode), khó khăn gặp phải trong quá trình thực hiện (xử lý concurrent updates với optimistic locking, tối ưu WebSocket broadcasting performance), đồng thời đề xuất một số hướng phát triển tiếp theo như: mở rộng AI features với smart sprint planning và issue assignment recommendations, tích hợp CI/CD automation để link issues với pull requests và deployments, cải thiện Analytics dashboard với advanced metrics (velocity charts, burndown charts, cycle time), hỗ trợ custom fields và automation rules cho workflow flexibility, và tối ưu hiệu năng database queries khi scale lên thousands of issues per project.
\end{itemize}

Ngoài ra, luận văn còn có phần tài liệu tham khảo liệt kê các nguồn tài liệu đã sử dụng (Scrum Guide, Agile methodology books, technical documentation của NestJS/Next.js/Prisma, OpenAI API docs) và phụ lục trình bày một số nội dung hỗ trợ như: mô hình database schema chi tiết với indexes và constraints, các đoạn code quan trọng (fractional indexing algorithm, WebSocket event handlers, LLM prompt templates), cấu hình triển khai với Docker Compose, giao diện minh họa (12 UI wireframes và screenshots thực tế), và các kết quả kiểm thử tiêu biểu với test cases details.
