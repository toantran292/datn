\phantomsection
\subsection*{5. Quy trình nghiên cứu và Công nghệ sử dụng}
\addcontentsline{toc}{subsection}{5. Quy trình nghiên cứu và Công nghệ sử dụng}
\setcounter{subsection}{5}
\setcounter{subsubsection}{0}
\setcounter{figure}{0}

\subsubsection{Quy trình nghiên cứu}

\begin{itemize}
    \item Tìm hiểu lý thuyết về quản lý dự án theo phương pháp Agile/Scrum, các khái niệm cốt lõi như Sprint, Backlog, Issue Types (Story/Task/Bug/Epic), Story Points, Sprint Planning, Sprint Review và nhu cầu theo dõi tiến độ công việc theo thời gian thực trong môi trường làm việc nhóm.

    \item Khảo sát và phân tích một số công cụ quản lý dự án tiêu biểu như Jira Software, Linear, ClickUp, Asana, Monday.com để rút ra ưu điểm (Board view, Backlog management, Sprint tracking), hạn chế (chi phí cao, độ phức tạp, thiếu tùy biến) và xác định khoảng trống về khả năng tích hợp AI để tự động hóa các tác vụ lặp đi lặp lại.

    \item Xác định yêu cầu và mục tiêu cho phân hệ quản lý dự án: quản lý Project với custom workflow, quản lý Sprint lifecycle (FUTURE/ACTIVE/CLOSED), quản lý Issue với nhiều loại (Story/Task/Bug/Epic), Board view với drag-and-drop, Backlog view với sprint planning, quản lý Project Members với phân quyền theo vai trò (Project Lead/Team Member/Viewer), Activity Tracking theo thời gian thực và tích hợp AI để hỗ trợ generate Sprint Summary, Issue Description và Epic Breakdown.

    \item Phân tích, lựa chọn mô hình kiến trúc phù hợp: kiến trúc microservice với PM Service (backend) xử lý nghiệp vụ quản lý dự án, pm-web (frontend) cung cấp giao diện người dùng, tích hợp với Account Service để xác thực và phân quyền, và tích hợp với LLM Provider để thực hiện các tính năng AI.

    \item Thiết kế kiến trúc hệ thống chi tiết cho phân hệ quản lý dự án:
    \begin{itemize}
        \item PM Service dùng NestJS với TypeScript, Prisma ORM và PostgreSQL 16 để xử lý toàn bộ nghiệp vụ: CRUD operations cho Project/Sprint/Issue, custom status management, drag-and-drop với fractional indexing, activity logging, và WebSocket để broadcast real-time updates.
        \item pm-web dùng Next.js (App Router) với TypeScript, React Query cho data fetching, dnd-kit cho drag-and-drop, và WebSocket client để nhận real-time updates từ backend.
        \item Tích hợp với LLM API (OpenAI GPT-4o-mini) thông qua AI Service hoặc trực tiếp từ PM Service để thực hiện: AI-generated Sprint Summary khi complete sprint, AI Issue Description generation từ title, và AI Epic Breakdown để tách Epic thành các Story/Task con.
    \end{itemize}

    \item Thiết kế cơ sở dữ liệu cho phân hệ quản lý dự án bao gồm các entity chính: Project (lưu thông tin dự án, identifier, project lead), Sprint (quản lý sprint với status, goal, startDate, endDate), IssueStatus (custom statuses cho workflow), Issue (lưu công việc với type, priority, status, assignees, story points, fractional index), IssueComment (bình luận với markdown và HTML), IssueActivity (activity log cho audit trail), và ProjectMember (quản lý members với roles). Database được thiết kế theo mô hình microservice với cross-service references (userId, orgId) không dùng foreign key constraints.

    \item Thiết kế giao diện người dùng cho pm-web bằng Next.js: xác định luồng thao tác từ Projects list → Project detail → Board/Backlog/Sprint views → Issue detail modal, bố cục màn hình với sidebar navigation, main content area và right panel cho issue detail, cách trình bày Board view (Kanban với drag-and-drop), Backlog view (sprint sections với drag-and-drop), Calendar view, Timeline view, Activity Feed, Sprint Summary report và các trang cấu hình Project Settings.

    \item Xây dựng và hoàn thiện các chức năng của phân hệ theo từng giai đoạn:
    \begin{itemize}
        \item Giai đoạn 1: Xây dựng core entities (Project, Sprint, Issue, IssueStatus) với CRUD operations, database schema với Prisma, và REST API endpoints.
        \item Giai đoạn 2: Phát triển Board view và Backlog view trên pm-web với drag-and-drop functionality, sử dụng fractional indexing (DECIMAL 20,10) để sắp xếp thứ tự issue, tích hợp dnd-kit cho smooth UX.
        \item Giai đoạn 3: Xây dựng Sprint lifecycle management (create, start, complete sprint), Activity Tracking với IssueActivity entity, và WebSocket integration cho real-time updates khi drag issue hoặc update status.
        \item Giai đoạn 4: Tích hợp AI features với LLM API để generate Sprint Summary (tổng hợp công việc đã hoàn thành, chưa hoàn thành, challenges), AI Issue Description từ title và type, và AI Epic Breakdown để tách Epic thành các subtasks.
        \item Giai đoạn 5: Xây dựng Project Members management với roles (Project Lead, Team Member, Viewer), Custom Status Management cho workflow configuration, Calendar/Timeline views và Analytics dashboard.
    \end{itemize}

    \item Tích hợp AI với API của nhà cung cấp mô hình ngôn ngữ lớn (LLM) để thực hiện các tính năng tự động hóa: AI-generated Sprint Summary khi Project Lead complete sprint (tóm tắt các issue đã done, issue chưa xong, lessons learned), AI Issue Enhancement để generate mô tả chi tiết từ title (bao gồm acceptance criteria, technical notes), và AI Epic Breakdown để tự động tách Epic thành các Story/Task con với story points estimation.

    \item Thực hiện kiểm thử chức năng, kiểm thử tích hợp và đánh giá phân hệ quản lý dự án dựa trên các tiêu chí: đáp ứng yêu cầu chức năng (Project/Sprint/Issue management), tính ổn định (drag-and-drop hoạt động mượt mà), khả năng real-time (WebSocket broadcast events chính xác), thời gian phản hồi của AI features (Sprint Summary generation < 10s), và chất lượng output của AI (Sprint Summary có đủ thông tin, Issue Description hợp lý).
\end{itemize}

\subsubsection{Công nghệ sử dụng}

\begin{itemize}
    \item \textbf{NestJS (Node.js)}: framework backend chính để xây dựng PM Service, xử lý toàn bộ nghiệp vụ quản lý dự án bao gồm Project CRUD, Sprint lifecycle (create, start, complete), Issue management (create, update, drag-and-drop reordering), Custom Status Management, Activity Logging, và WebSocket Gateway để broadcast real-time events. NestJS được chọn nhờ kiến trúc module rõ ràng, hỗ trợ TypeScript mạnh mẽ, dependency injection, và khả năng tích hợp dễ dàng với Prisma ORM và WebSocket.

    \item \textbf{TypeScript}: ngôn ngữ lập trình chính cho cả backend (PM Service) và frontend (pm-web), đảm bảo type safety, giảm lỗi runtime, hỗ trợ IntelliSense tốt và dễ dàng refactor code. TypeScript giúp định nghĩa rõ ràng các entity types (Project, Sprint, Issue, IssueStatus), DTO (Data Transfer Objects) và API contracts.

    \item \textbf{Prisma ORM}: Object-Relational Mapping tool để tương tác với PostgreSQL database, cung cấp type-safe database client, schema migration, và query builder. Prisma schema định nghĩa các models: Project, Sprint, IssueStatus, Issue, IssueComment, IssueActivity, ProjectMember với các relationships và indexes. Prisma Client được sử dụng trong PM Service để thực hiện CRUD operations, complex queries (filter issues by sprint, get activity logs) và transactions.

    \item \textbf{PostgreSQL 16}: cơ sở dữ liệu quan hệ chính cho phân hệ quản lý dự án, lưu trữ toàn bộ dữ liệu về Project, Sprint, Issue, Comments, Activities và Members. PostgreSQL hỗ trợ các kiểu dữ liệu phức tạp như JSONB (cho assignees array, metadata), DECIMAL (20,10) cho fractional indexing trong drag-and-drop, và full-text search cho tìm kiếm issue. Database được thiết kế theo mô hình microservice với cross-service references (userId, orgId từ Account Service) không dùng foreign key constraints để đảm bảo loose coupling.

    \item \textbf{Redis}: cache layer để tăng tốc độ truy vấn dữ liệu thường xuyên như danh sách issues trong active sprint, project members, custom statuses. Redis cũng được sử dụng làm session store cho WebSocket connections, lưu mapping giữa userId/projectId và socket connectionId để hỗ trợ real-time broadcasting khi có update (drag issue, change status, add comment).

    \item \textbf{WebSocket (Socket.io hoặc ws)}: giao thức real-time communication để broadcast events từ PM Service đến tất cả clients đang online trong cùng project. Khi một user thực hiện action (drag issue, update status, add comment), backend emit WebSocket event với payload chứa changes, tất cả clients khác nhận event và update UI ngay lập tức mà không cần refresh hoặc polling. WebSocket Gateway trong NestJS quản lý connections, rooms (theo projectId) và event handlers.

    \item \textbf{OpenAI API (GPT-4o-mini)}: API của mô hình ngôn ngữ lớn để thực hiện các tính năng AI trong phân hệ quản lý dự án: (1) AI-generated Sprint Summary khi Project Lead complete sprint, LLM nhận context về các issues đã done/chưa xong và generate báo cáo markdown với summary, achievements, challenges, (2) AI Issue Description generation từ issue title và type, LLM generate mô tả chi tiết bao gồm acceptance criteria và technical notes, (3) AI Epic Breakdown để tự động tách Epic thành các Story/Task con với story points estimation. PM Service gọi OpenAI API thông qua HTTP request, truyền prompt đã được chuẩn bị với context và nhận response dạng JSON/Markdown.

    \item \textbf{Next.js 14 (App Router)}: framework React để xây dựng giao diện pm-web, sử dụng App Router với server components và client components, hỗ trợ SSR (Server-Side Rendering) cho trang Projects list, ISR (Incremental Static Regeneration) cho các trang ít thay đổi, và CSR (Client-Side Rendering) cho các trang interactive như Board view, Backlog view. Next.js cung cấp routing, data fetching với React Server Components, API routes (nếu cần BFF layer), và optimized production build.

    \item \textbf{React Query (TanStack Query)}: thư viện data fetching và state management cho pm-web, quản lý server state, caching, background refetching, optimistic updates và error handling. React Query được sử dụng để fetch danh sách issues, project details, sprint data từ PM Service API, tự động cache kết quả, invalidate cache khi có mutation (create/update/delete), và sync với WebSocket events để update UI real-time.

    \item \textbf{dnd-kit}: thư viện drag-and-drop cho React, được sử dụng trong Board view và Backlog view để kéo thả issue giữa các cột status hoặc giữa các sprint. dnd-kit cung cấp hooks (useDraggable, useDroppable, useSortable), sensors (mouse, touch, keyboard), và collision detection algorithms. Khi user drag issue, frontend tính toán fractional index mới (giữa 2 issues liền kề) và gọi API update orderIndex trong database.

    \item \textbf{Tailwind CSS}: utility-first CSS framework để styling giao diện pm-web, cung cấp các utility classes cho layout (flexbox, grid), spacing, colors, typography. Tailwind giúp xây dựng responsive UI nhanh chóng, consistent design system và dễ dàng customize theme (colors, fonts, breakpoints).

    \item \textbf{shadcn/ui}: component library dựa trên Radix UI và styled với Tailwind CSS, cung cấp các accessible UI components như Dialog (cho Issue detail modal), DropdownMenu (cho actions), Select (cho filters), Button, Input, Textarea. shadcn/ui components được copy vào project và có thể customize hoàn toàn, không phải install npm package.

    \item \textbf{Markdown và HTML}: Issue description, comments và Sprint Summary đều hỗ trợ Markdown format để user viết nội dung rich text dễ dàng. Frontend sử dụng markdown editor (ví dụ: react-markdown hoặc tiptap) để input, backend lưu cả markdown (raw text) và HTML (rendered) vào database, frontend hiển thị HTML đã sanitized để tránh XSS attacks.
\end{itemize}

\subsubsection{Công cụ hỗ trợ xây dựng và phát triển}

\begin{itemize}
    \item Công cụ lập trình: Visual Studio Code, Cursor.
    \item Công cụ quản lý và truy vấn cơ sở dữ liệu: DataGrip.
    \item Công cụ kiểm thử API: Postman.
    \item Công cụ container hóa và triển khai: Docker, Docker Compose.
    \item Công cụ quản lý mã nguồn: Git, GitHub.
\end{itemize}