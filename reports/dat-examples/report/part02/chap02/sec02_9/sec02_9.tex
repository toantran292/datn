\subsection{Giới thiệu về Nginx}

\subsubsection{Khái quát}

Nginx (đọc là engine-x) là một web server mã nguồn mở hiệu suất cao được phát triển bởi Igor Sysoev vào năm 2004, ban đầu được thiết kế để giải quyết vấn đề C10K về xử lý 10,000 concurrent connections và đã trở thành một trong những web servers phổ biến nhất thế giới. Nginx có khả năng hoạt động với nhiều vai trò khác nhau: Web Server để serve static files như HTML, CSS, JavaScript, và images, Reverse Proxy để forward requests từ clients đến backend servers, Load Balancer để phân phối traffic across multiple servers, HTTP Cache để cache responses giảm load trên backend, SSL/TLS Termination để xử lý HTTPS encryption và decryption, và API Gateway cho routing, rate limiting, authentication cho APIs. Tính linh hoạt và hiệu suất cao này làm cho Nginx trở thành lựa chọn ưu tiên trong các kiến trúc microservices và cloud-native applications.

\subsubsection{Kiến trúc và ưu điểm}

Nginx sử dụng kiến trúc event-driven, asynchronous, và non-blocking với Master Process đọc configuration, bind ports và quản lý worker processes, Worker Processes xử lý actual requests với mỗi worker có thể handle hàng nghìn concurrent connections nhờ event-driven model, và Cache Loader/Manager quản lý cache data trên disk. Kiến trúc này mang lại nhiều ưu điểm: high concurrency xử lý hàng chục nghìn concurrent connections với minimal memory, low memory footprint vì mỗi connection chỉ cần vài KB memory, high performance do event-driven model hiệu quả hơn thread-per-connection model traditional, và stability cao vì worker processes hoạt động độc lập nên một worker crash không ảnh hưởng đến workers khác.

\subsubsection{Tính năng và khả năng}

Nginx cung cấp Reverse Proxy đứng trước backend servers nhận requests từ clients và forward đến appropriate backend, ẩn backend infrastructure khỏi clients tăng cường bảo mật, SSL termination decrypt HTTPS tại Nginx và forward HTTP đến backend giảm computational load, request/response modification với headers, và WebSocket proxying support cho real-time applications. Load Balancing với nhiều algorithms: Round Robin phân phối requests tuần tự đến servers (default), Least Connections gửi request đến server có ít active connections nhất, IP Hash route requests từ cùng một IP đến cùng server cho session persistence, Weighted cho phép gán weights dựa trên server capacity, và Health Checks tự động remove unhealthy servers khỏi pool.

Caching capabilities cho phép cache static content và API responses, configurable cache expiration và invalidation, proxy cache cho backend responses, và FastCGI cache cho PHP/dynamic content. Security features bao gồm SSL/TLS support cho TLS 1.2/1.3 và HTTP/2 với OCSP stapling, Rate Limiting giới hạn requests per second/minute từ một IP chống abuse, Access Control allow/deny based on IP addresses, Request Filtering block malicious requests dựa trên patterns, và DDoS Mitigation với connection limits và request rate limits.

\subsubsection{Vận dụng vào đề tài}

Nginx đóng vai trò critical làm entry point duy nhất cho toàn hệ thống, hoạt động như Reverse Proxy và API Gateway routing requests đến appropriate services: /api/auth/* đến Account Service (Spring Boot), /api/pm/* đến PM Service (NestJS) bao gồm /api/pm/projects/*, /api/pm/sprints/*, /api/pm/issues/*, /api/pm/custom-statuses/*, /api/pm/ai/*, root path / đến Frontend (Next.js), WebSocket proxy cho real-time updates (/socket.io) routing đến PM Service WebSocketGateway, và SSL/TLS termination với Let's Encrypt certificates cho HTTPS connections.

Load Balancer functionality distribute traffic across multiple instances của PM Service để đảm bảo high availability và horizontal scalability khi nhiều users đồng thời làm việc trên Board/Backlog views, health checks tự động detect và remove failed PM Service instances khỏi rotation, và sticky sessions cho WebSocket connections đảm bảo clients luôn connect đến cùng một PM Service instance maintain state cho collaborative editing và real-time updates. Static File Serving được optimize với Nginx serve Next.js static assets như JavaScript bundles, CSS files, Board/Backlog view components với appropriate caching headers để improve performance, và gzip compression cho text-based files giảm bandwidth usage.

Security được enforce ở Nginx layer với rate limiting giới hạn 100 requests/minute cho API endpoints chống abuse và DDoS, request size limits prevent large file upload attacks, security headers như X-Frame-Options, X-Content-Type-Options, Content Secu\-rity Policy được inject vào responses, và CORS configuration cho phép controlled API access từ allowed origins. Nginx configuration được containerized với Docker và manag\-ed bằng infrastructure as code tools như Terraform và Kubernetes ConfigMaps cho easy deployment và version control.
