\subsection{Yêu cầu chức năng}

\subsubsection{Sơ đồ trường hợp sử dụng}

Dựa trên mô tả bài toán ở mục 1.1, phân hệ quản lý dự án có 3 tác nhân chính (Project Lead/Manager, Team Member, Viewer/Stakeholder) và các hệ thống bên ngoài tương tác để hỗ trợ quy trình Scrum. Các yêu cầu chức năng được phân loại theo từng tác nhân và được biểu diễn thông qua các sơ đồ trường hợp sử dụng (use case) dưới đây.

\begin{itemize}
    \item \textbf{Sơ đồ trường hợp sử dụng tổng quan:} Sơ đồ này mô tả toàn bộ các chức năng chính của phân hệ quản lý dự án và mối quan hệ giữa các tác nhân với hệ thống. Phân hệ quản lý dự án cung cấp các nhóm chức năng chính bao gồm: quản lý dự án (project configuration), quản lý sprint (sprint lifecycle), quản lý issue (create, update, delete, reorder issues), quản lý trạng thái tùy chỉnh (custom issue status), bảng Kanban (board view), quản lý backlog (backlog view), bình luận và theo dõi hoạt động (comments \& activity log), phân tích và thống kê (analytics), và tích hợp AI (issue enhancement, story point estimation, sprint summary). Mỗi tác nhân có quyền truy cập vào các chức năng tương ứng với vai trò của mình: Project Lead có toàn quyền, Team Member có quyền thao tác công việc, Viewer chỉ có quyền xem.

    \begin{figure}[H]
        \centering
        \includegraphics[width=0.5\textwidth]{images/usecase_tongquan.png}
        \caption{Sơ đồ trường hợp sử dụng tổng quan của phân hệ quản lý dự án}
        \label{fig:usecase_tongquan}
    \end{figure}

    \item \textbf{Sơ đồ trường hợp sử dụng của Project Lead/Manager:} Project Lead hoặc Project Manager là tác nhân có quyền hạn cao nhất trong dự án, chịu trách nhiệm quản lý toàn bộ các khía cạnh của dự án. Các chức năng chính của Project Lead bao gồm: quản lý dự án (tạo dự án mới, cấu hình thông tin dự án như identifier, project lead, default assignee, cập nhật và xóa dự án), quản lý sprint (tạo sprint mới với sprint goal và timeline, bắt đầu sprint chuyển từ FUTURE sang ACTIVE, hoàn thành sprint chuyển sang CLOSED, xem tổng quan sprint với số lượng issue theo trạng thái), quản lý trạng thái tùy chỉnh (tạo, chỉnh sửa, xóa custom issue status, sắp xếp lại thứ tự status trên board bằng drag-and-drop, cấu hình màu sắc cho từng status), quản lý issue với toàn quyền (tạo, xem, cập nhật, xóa bất kỳ issue nào trong dự án, gán assignees, thiết lập parent-child relationships, sắp xếp lại thứ tự issue), quản lý board và backlog (xem board view với các cột status, xem backlog view với danh sách sprint, kéo thả issue giữa các cột và sprint), quản lý comments và activity (thêm, xem, chỉnh sửa, xóa comments, xem đầy đủ activity log của mọi issue), xem analytics và báo cáo (biểu đồ Created vs Resolved, thống kê phân bố issue theo type/priority/status, đánh giá hiệu suất sprint), và sử dụng các tính năng AI (AI Issue Enhancement để cải thiện mô tả issue, Smart Story Point Estimation để ước lượng story points, Sprint Summary để tạo báo cáo tổng kết sprint tự động).

    \begin{figure}[H]
        \centering
        \includegraphics[width=0.3\textwidth]{images/usecase_project_lead.png}
        \caption{Sơ đồ trường hợp sử dụng của Project Lead/Manager}
        \label{fig:usecase_project_lead}
    \end{figure}

    \item \textbf{Sơ đồ trường hợp sử dụng của Team Member:} Team Member là thành viên tham gia trực tiếp vào việc thực hiện các công việc trong dự án, bao gồm Developer, Tester, Designer và các vai trò kỹ thuật khác. Các chức năng chính của Team Member bao gồm: quản lý issue (tạo issue mới, xem danh sách issue, cập nhật issue do mình tạo hoặc được gán, thay đổi trạng thái issue bằng cách kéo thả trên board hoặc backlog, cập nhật các thuộc tính như mô tả, story points, ngày bắt đầu/mục tiêu, gán assignees, thiết lập parent-child relationships, không có quyền xóa issue của người khác), xem sprint (xem danh sách sprint, xem chi tiết sprint với tổng quan số lượng issue, không có quyền tạo, bắt đầu hoặc hoàn thành sprint), sử dụng board view (xem bảng Kanban với các cột status, kéo thả issue giữa các cột để thay đổi trạng thái, lọc issue theo sprint, sắp xếp lại thứ tự issue trong cột), sử dụng backlog view (xem danh sách sprint và backlog, kéo thả issue giữa các sprint để di chuyển, xem panel chi tiết issue ở bên phải), quản lý comments (thêm comment vào issue, chỉnh sửa và xóa comment của mình, xem comment của người khác), xem activity log (theo dõi lịch sử thay đổi của issue để hiểu rõ quá trình phát triển), xem analytics (xem các biểu đồ và thống kê về dự án, nhưng không có quyền cấu hình hoặc xuất báo cáo), và sử dụng tính năng AI (AI Issue Enhancement để cải thiện mô tả issue khi tạo hoặc chỉnh sửa, Smart Story Point Estimation để nhận gợi ý ước lượng story points cho issue của mình, không có quyền tạo Sprint Summary).

    \begin{figure}[H]
        \centering
        \includegraphics[width=5.5\textwidth]{images/usecase_team_member.png}
        \caption{Sơ đồ trường hợp sử dụng của Team Member}
        \label{fig:usecase_team_member}
    \end{figure}

    \item \textbf{Sơ đồ trường hợp sử dụng của Viewer/Stakeholder:} Viewer hoặc Stakeholder là người quan tâm đến tiến độ dự án nhưng không tham gia trực tiếp vào việc thực hiện công việc, chẳng hạn như Product Owner, khách hàng, quản lý cấp cao hoặc các bên liên quan khác. Vai trò này được thiết kế để đảm bảo tính minh bạch của dự án trong khi vẫn bảo vệ dữ liệu khỏi các thay đổi không mong muốn. Các chức năng chính của Viewer bao gồm: xem thông tin dự án (xem thông tin cơ bản của dự án như tên, identifier, mô tả, project lead, danh sách thành viên), xem sprint (xem danh sách sprint, xem chi tiết sprint với tổng quan số lượng issue theo trạng thái, theo dõi tiến độ sprint), xem issue (xem danh sách issue, xem chi tiết issue bao gồm title, description, type, priority, status, story points, dates, assignees, parent-child relationships, không có quyền tạo hoặc chỉnh sửa issue), xem board view (xem bảng Kanban với các cột status và danh sách issue, lọc issue theo sprint, không có quyền kéo thả hoặc thay đổi trạng thái issue), xem backlog view (xem danh sách sprint và backlog với tất cả issue, không có quyền kéo thả hoặc di chuyển issue), quản lý comments (thêm comment vào issue để góp ý hoặc yêu cầu làm rõ, xem comment của người khác, chỉnh sửa và xóa comment của mình, không có quyền xóa comment của người khác), xem activity log (theo dõi đầy đủ lịch sử thay đổi của issue để hiểu quá trình phát triển), xem analytics (xem tất cả các biểu đồ và thống kê về dự án như Created vs Resolved, phân bố issue theo type/priority/status, đánh giá hiệu suất sprint), và không có quyền sử dụng các tính năng AI (không được phép sử dụng AI Issue Enhancement, Smart Story Point Estimation hoặc Sprint Summary).

    \begin{figure}[H]
        \centering
        \includegraphics[width=0.4\textwidth]{images/usecase_viewer.png}
        \caption{Sơ đồ trường hợp sử dụng của Viewer/Stakeholder}
        \label{fig:usecase_viewer}
    \end{figure}

    \item \textbf{Tích hợp với các hệ thống bên ngoài:} Ngoài ba tác nhân chính, phân hệ quản lý dự án còn tương tác với các hệ thống bên ngoài để cung cấp các chức năng mở rộng và hoàn chỉnh. Các hệ thống bên ngoài bao gồm: dịch vụ xác thực và phân quyền từ phân hệ nền tảng (Platform Services) để quản lý JWT token, xác thực người dùng, trích xuất thông tin organization/workspace, và đảm bảo tenant isolation khi truy cập dữ liệu; dịch vụ AI bên ngoài (OpenAI GPT API hoặc Claude API) để cung cấp các tính năng AI Issue Enhancement, Smart Story Point Estimation và Sprint Summary, với cơ chế retry logic và error handling để đảm bảo tính ổn định; cơ sở dữ liệu PostgreSQL để lưu trữ và truy vấn dữ liệu dự án (projects, sprints, issues, issue statuses, comments, activity log) thông qua Prisma ORM; và các phân hệ chức năng khác trong nền tảng như communication (để gửi thông báo khi có issue mới hoặc thay đổi quan trọng), meeting (để liên kết issue với các cuộc họp), file-storage (để đính kèm file vào issue), và notification (để thông báo cho các thành viên về các sự kiện liên quan đến dự án). Việc tích hợp với các hệ thống bên ngoài giúp phân hệ quản lý dự án tập trung vào core functionality (quản lý sprint và issue) trong khi tận dụng các dịch vụ chung từ nền tảng và các API bên ngoài, đảm bảo hệ thống linh hoạt, dễ mở rộng và có khả năng tích hợp cao với các phân hệ khác.
\end{itemize}
