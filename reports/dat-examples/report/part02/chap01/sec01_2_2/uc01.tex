\paragraph{UC01: Quản lý Dự án}
\mbox{}

\textbf{UC01.1 - Tạo dự án mới}

\begin{figure}[H]
    \centering
    \includegraphics[width=1\textwidth]{images/uc01_1_create_project.png}
    \caption{Sơ đồ use case chức năng tạo dự án}
    \label{fig:uc01_1_create_project}
\end{figure}

\begin{longtblr}[
    caption = {Đặc tả use case UC01.1 - Tạo dự án mới},
    label = {tab:uc01_1},
]{
    colspec={|l|p{.7\linewidth}|}
}
\hline
\textbf{Tên chức năng} & \textbf{Tạo dự án mới} \\\hline
ID & UC01.1 \\\hline
Người sử dụng & Project Lead, Team Member (có quyền) \\\hline
Mức độ cần thiết & Bắt buộc \\\hline
Phân loại & Cao \\\hline
Các thành phần tham gia &
\begin{minipage}{\linewidth}
    \vskip 4pt
    + \textbf{Project Lead:} Người tạo và quản lý dự án. \\
    + \textbf{PM Service (Backend):} Xử lý logic tạo dự án, kiểm tra identifier, lưu database. \\
    + \textbf{PostgreSQL Database:} Lưu trữ thông tin dự án.
    \vskip 1pt
\end{minipage}
\\\hline
Mô tả tóm tắt & Cho phép Project Lead tạo dự án mới với identifier duy nhất, tên dự án, project lead và default assignee. \\\hline
Trigger & User nhấn nút "Create Project" trong giao diện workspace. \\\hline
Kiểu sự kiện & External. \\\hline
Luồng xử lý bình thường &
\begin{minipage}{\linewidth}
    \vskip 4pt
    \begin{enumerate}
        \item User nhấn nút "Create Project" hoặc "New Project".
        \item Hệ thống hiển thị modal/form tạo dự án với các trường: Project Name (tên dự án), Identifier (mã dự án viết tắt), Project Lead (người chịu trách nhiệm), Default Assignee (người được gán mặc định cho issue mới).
        \item User nhập thông tin dự án, trong đó Identifier được tự động suggest dựa trên Project Name (ví dụ: "Project Management" → "PM").
        \item User có thể chỉnh sửa Identifier nếu muốn.
        \item Hệ thống tự động kiểm tra tính khả dụng của Identifier (gọi API check-identifier).
        \item Nếu Identifier đã tồn tại trong organization, hiển thị cảnh báo real-time.
        \item User chọn Project Lead từ danh sách thành viên workspace.
        \item User tùy chọn chọn Default Assignee (có thể để trống).
        \item User nhấn "Create" để tạo dự án.
        \item Hệ thống validate dữ liệu (Project Name không trống, Identifier hợp lệ và duy nhất).
        \item Backend tạo bản ghi Project mới trong database với orgId từ JWT token.
        \item Backend tự động tạo các Issue Status mặc định cho dự án (TODO, IN\_PROGRESS, DONE).
        \item Hệ thống hiển thị thông báo "Project created successfully".
        \item Hệ thống chuyển hướng user đến trang chi tiết dự án hoặc backlog view.
    \end{enumerate}
    \vskip 1pt
\end{minipage}
\\\hline
Các luồng sự kiện con & N/A \\\hline
Luồng thay thế/ngoại lệ &
\begin{minipage}{\linewidth}
    \vskip 4pt
    \textbf{\textcolor{red}{E1}} -- Identifier đã tồn tại: Hiển thị lỗi "Project identifier 'PM' is already used in this workspace. Please choose another identifier."

    \textbf{\textcolor{red}{E2}} -- Project Name trống: Hiển thị lỗi "Project name is required."

    \textbf{\textcolor{red}{E3}} -- Identifier không hợp lệ: Hiển thị lỗi "Identifier must be 2-10 uppercase letters (A-Z) only."

    \textbf{\textcolor{red}{E4}} -- Không có quyền tạo dự án: Hiển thị lỗi "You don't have permission to create projects in this workspace."
    \vskip 1pt
\end{minipage}
\\\hline
Kết quả & Dự án mới được tạo thành công với identifier duy nhất, các issue status mặc định được khởi tạo, user có thể bắt đầu tạo sprint và issue. \\\hline
\end{longtblr}

\vspace{1em}

\textbf{UC01.2 - Cấu hình dự án}

\begin{figure}[H]
    \centering
    \includegraphics[width=1\textwidth]{images/uc01_2_configure_project.png}
    \caption{Sơ đồ use case chức năng cấu hình dự án}
    \label{fig:uc01_2_configure_project}
\end{figure}

\begin{longtblr}[
    caption = {Đặc tả use case UC01.2 - Cấu hình dự án},
    label = {tab:uc01_2},
]{
    colspec={|l|p{.7\linewidth}|}
}
\hline
\textbf{Tên chức năng} & \textbf{Cấu hình dự án} \\\hline
ID & UC01.2 \\\hline
Người sử dụng & Project Lead \\\hline
Mức độ cần thiết & Bắt buộc \\\hline
Phân loại & Cao \\\hline
Các thành phần tham gia &
\begin{minipage}{\linewidth}
    \vskip 4pt
    + \textbf{Project Lead:} Người có quyền chỉnh sửa cấu hình dự án. \\
    + \textbf{PM Service:} Xử lý cập nhật thông tin dự án. \\
    + \textbf{Database:} Lưu trữ thay đổi.
    \vskip 1pt
\end{minipage}
\\\hline
Mô tả tóm tắt & Cho phép Project Lead cập nhật thông tin dự án như tên, project lead, default assignee. \\\hline
Trigger & User truy cập trang Project Settings hoặc nhấn "Edit Project". \\\hline
Kiểu sự kiện & External. \\\hline
Luồng xử lý bình thường &
\begin{minipage}{\linewidth}
    \vskip 4pt
    \begin{enumerate}
        \item User truy cập trang Project Settings hoặc nhấn icon "Edit" ở header dự án.
        \item Hệ thống hiển thị form cấu hình với thông tin hiện tại của dự án.
        \item Form hiển thị các trường: Project Name (có thể chỉnh sửa), Identifier (read-only, không được phép thay đổi sau khi tạo), Project Lead (dropdown chọn member), Default Assignee (dropdown chọn member).
        \item User chỉnh sửa các thông tin cần thiết.
        \item User nhấn "Save Changes".
        \item Hệ thống validate dữ liệu (Project Name không trống).
        \item Backend gọi ProjectService.update() để cập nhật thông tin.
        \item Backend ghi log activity về việc thay đổi cấu hình dự án.
        \item Hệ thống cập nhật thông tin vào database.
        \item Hệ thống hiển thị thông báo "Project settings updated successfully".
        \item Giao diện tự động cập nhật với thông tin mới.
    \end{enumerate}
    \vskip 1pt
\end{minipage}
\\\hline
Các luồng sự kiện con & N/A \\\hline
Luồng thay thế/ngoại lệ &
\begin{minipage}{\linewidth}
    \vskip 4pt
    \textbf{\textcolor{red}{E1}} -- Project Name trống: Hiển thị lỗi "Project name cannot be empty."

    \textbf{\textcolor{red}{E2}} -- Không có quyền chỉnh sửa: Hiển thị lỗi "Only Project Lead can modify project settings."

    \textbf{\textcolor{red}{E3}} -- Project Lead được chọn không tồn tại trong workspace: Hiển thị lỗi "Selected user is not a member of this workspace."
    \vskip 1pt
\end{minipage}
\\\hline
Kết quả & Thông tin dự án được cập nhật thành công, các thay đổi được phản ánh ngay lập tức trong giao diện. \\\hline
\end{longtblr}

\vspace{1em}

\textbf{UC01.3 - Xem danh sách và chi tiết dự án}

\begin{figure}[H]
    \centering
    \includegraphics[width=0.8\textwidth]{images/uc01_3_view_projects.png}
    \caption{Sơ đồ use case chức năng xem dự án}
    \label{fig:uc01_3_view_projects}
\end{figure}

\begin{longtblr}[
    caption = {Đặc tả use case UC01.3 - Xem danh sách và chi tiết dự án},
    label = {tab:uc01_3},
]{
    colspec={|l|p{.7\linewidth}|}
}
\hline
\textbf{Tên chức năng} & \textbf{Xem danh sách và chi tiết dự án} \\\hline
ID & UC01.3 \\\hline
Người sử dụng & Project Lead, Team Member, Viewer \\\hline
Mức độ cần thiết & Bắt buộc \\\hline
Phân loại & Cao \\\hline
Các thành phần tham gia &
\begin{minipage}{\linewidth}
    \vskip 4pt
    + \textbf{User:} Muốn xem danh sách dự án hoặc chi tiết một dự án. \\
    + \textbf{PM Service:} Cung cấp API để lấy danh sách và chi tiết dự án. \\
    + \textbf{Database:} Truy vấn dữ liệu dự án.
    \vskip 1pt
\end{minipage}
\\\hline
Mô tả tóm tắt & Cho phép user xem danh sách tất cả dự án trong workspace và xem chi tiết từng dự án. \\\hline
Trigger & User truy cập workspace hoặc nhấn vào một dự án cụ thể. \\\hline
Kiểu sự kiện & External. \\\hline
Luồng xử lý bình thường &
\begin{minipage}{\linewidth}
    \vskip 4pt
    \begin{enumerate}
        \item User đăng nhập và chọn workspace.
        \item Hệ thống gọi API GET /api/projects để lấy danh sách dự án trong workspace (filtered by orgId).
        \item Hệ thống hiển thị danh sách dự án ở sidebar với: Project Identifier, Project Name, số lượng issue đang active.
        \item User nhấn vào một dự án trong danh sách.
        \item Hệ thống gọi API GET /api/projects/:id để lấy chi tiết dự án.
        \item Hệ thống hiển thị trang chi tiết dự án với các tab: Backlog, Board, Summary, Settings.
        \item User có thể xem các thông tin: tên dự án, identifier, project lead, default assignee, ngày tạo, danh sách sprint, tổng số issue.
        \item User có thể chuyển giữa các tab để xem board, backlog, hoặc thống kê dự án.
    \end{enumerate}
    \vskip 1pt
\end{minipage}
\\\hline
Các luồng sự kiện con & Include UC05 (Board View), Include UC05 (Backlog View). \\\hline
Luồng thay thế/ngoại lệ &
\begin{minipage}{\linewidth}
    \vskip 4pt
    \textbf{\textcolor{red}{A1}} -- Không có dự án nào: Hiển thị empty state với nút "Create your first project".

    \textbf{\textcolor{red}{E1}} -- Project không tồn tại: Hiển thị lỗi 404 "Project not found".

    \textbf{\textcolor{red}{E2}} -- Không có quyền truy cập dự án: Hiển thị lỗi 403 "You don't have permission to access this project".
    \vskip 1pt
\end{minipage}
\\\hline
Kết quả & User xem được danh sách dự án và chi tiết từng dự án, có thể điều hướng đến các view khác nhau (board, backlog, summary). \\\hline
\end{longtblr}

\vspace{1em}

\textbf{UC01.4 - Xóa dự án}

\begin{figure}[H]
    \centering
    \includegraphics[width=1\textwidth]{images/uc01_4_delete_project.png}
    \caption{Sơ đồ use case chức năng xóa dự án}
    \label{fig:uc01_4_delete_project}
\end{figure}

\begin{longtblr}[
    caption = {Đặc tả use case UC01.4 - Xóa dự án},
    label = {tab:uc01_4},
]{
    colspec={|l|p{.7\linewidth}|}
}
\hline
\textbf{Tên chức năng} & \textbf{Xóa dự án} \\\hline
ID & UC01.4 \\\hline
Người sử dụng & Project Lead (chỉ có Project Lead) \\\hline
Mức độ cần thiết & Bắt buộc \\\hline
Phân loại & Cao \\\hline
Các thành phần tham gia &
\begin{minipage}{\linewidth}
    \vskip 4pt
    + \textbf{Project Lead:} Người có quyền xóa dự án. \\
    + \textbf{PM Service:} Xử lý xóa dự án và cascade delete các dữ liệu liên quan. \\
    + \textbf{Database:} Xóa dữ liệu dự án và các entity con.
    \vskip 1pt
\end{minipage}
\\\hline
Mô tả tóm tắt & Cho phép Project Lead xóa vĩnh viễn dự án và tất cả dữ liệu liên quan (sprints, issues, comments, activities). \\\hline
Trigger & User nhấn nút "Delete Project" trong Project Settings. \\\hline
Kiểu sự kiện & External. \\\hline
Luồng xử lý bình thường &
\begin{minipage}{\linewidth}
    \vskip 4pt
    \begin{enumerate}
        \item User truy cập trang Project Settings.
        \item User cuộn xuống phần "Danger Zone".
        \item User nhấn nút "Delete Project" (màu đỏ).
        \item Hệ thống hiển thị modal xác nhận với cảnh báo: "This action cannot be undone. This will permanently delete the project, all sprints, issues, comments and activities."
        \item User phải nhập tên dự án hoặc identifier để xác nhận (ví dụ: "Type 'PM' to confirm").
        \item User nhập chính xác identifier và nhấn "Delete Project".
        \item Hệ thống kiểm tra user có quyền Project Lead không.
        \item Backend gọi ProjectService.remove() để xóa dự án.
        \item Database thực hiện cascade delete: xóa sprints, issues, issue statuses, comments, activities theo foreign key constraints.
        \item Hệ thống ghi log về việc xóa dự án.
        \item Hệ thống hiển thị thông báo "Project deleted successfully".
        \item Hệ thống chuyển hướng user về trang danh sách dự án hoặc workspace dashboard.
    \end{enumerate}
    \vskip 1pt
\end{minipage}
\\\hline
Các luồng sự kiện con & N/A \\\hline
Luồng thay thế/ngoại lệ &
\begin{minipage}{\linewidth}
    \vskip 4pt
    \textbf{\textcolor{red}{A1}} -- User hủy xác nhận: Đóng modal, không xóa gì cả.

    \textbf{\textcolor{red}{E1}} -- Identifier nhập không khớp: Hiển thị lỗi "Identifier does not match. Please type exactly 'PM' to confirm."

    \textbf{\textcolor{red}{E2}} -- Không có quyền xóa: Hiển thị lỗi "Only Project Lead can delete this project."

    \textbf{\textcolor{red}{E3}} -- Lỗi database khi xóa: Hiển thị lỗi "Failed to delete project. Please try again later."
    \vskip 1pt
\end{minipage}
\\\hline
Kết quả & Dự án và tất cả dữ liệu liên quan (sprints, issues, comments, activities) được xóa vĩnh viễn khỏi database. User được chuyển về trang danh sách dự án. \\\hline
\end{longtblr}
