\paragraph{UC04: Quản lý trạng thái công việc}
\mbox{}

\textbf{UC04.1 - Tạo trạng thái công việc mới}

\begin{figure}[H]
    \centering
    \includegraphics[width=0.7\textwidth]{images/uc04_1_create_status.png}
    \caption{Sơ đồ use case chức năng tạo trạng thái công việc}
    \label{fig:uc04_1_create_status}
\end{figure}

\begin{longtblr}[
    caption = {Đặc tả use case UC04.1 - Tạo trạng thái công việc mới},
    label = {tab:uc04_1},
]{
    colspec={|l|p{.7\linewidth}|}
}
\hline
\textbf{Tên chức năng} & \textbf{Tạo trạng thái công việc mới} \\\hline
ID & UC04.1 \\\hline
Người sử dụng & Project Lead \\\hline
Mức độ cần thiết & Bắt buộc \\\hline
Phân loại & Cao \\\hline
Các thành phần tham gia & + \textbf{Project Lead:} Muốn tạo trạng thái công việc để phù hợp với workflow của dự án (ví dụ: TO DO, IN REVIEW, TESTING, DONE). \\\hline
Mô tả tóm tắt & Cho phép Project Lead tạo trạng thái công việc với name, description, color (hex). Status được tự động gán order tiếp theo (hiển thị ở cuối board). \\\hline
Trigger & Project Lead nhấn "Add Status" trong project settings. \\\hline
Kiểu sự kiện & External. \\\hline
Luồng xử lý bình thường &
\begin{minipage}{\linewidth}
    \vskip 4pt
    \begin{enumerate}
        \item Project Lead vào Settings > Issue Statuses
        \item Nhấn "Add Status"
        \item Form hiển thị: Status Name (required), Description (optional), Color (color picker)
        \item Project Lead nhập thông tin
        \item Backend validate: name không trống
        \item Backend tự động tính order = max(order) + 1 (đặt ở cuối)
        \item Backend tạo IssueStatus record
        \item Board view tự động thêm column mới ở cuối
        \item Hiển thị thông báo "Status created successfully"
    \end{enumerate}
    \vskip 1pt
\end{minipage}
\\\hline
Các luồng sự kiện con & N/A \\\hline
Luồng thay thế/ngoại lệ &
\begin{minipage}{\linewidth}
    \vskip 4pt
    \textbf{\textcolor{red}{E1}} -- Name trống: Hiển thị lỗi "Status name is required".

    \textbf{\textcolor{red}{A1}} -- Chọn color: Hỗ trợ color picker hoặc nhập hex code (ví dụ: \#3B82F6).

    \textbf{\textcolor{red}{A2}} -- Tạo nhiều status: Project Lead có thể tạo nhiều status liên tiếp.
    \vskip 1pt
\end{minipage}
\\\hline
Kết quả & Trạng thái công việc mới được tạo, hiển thị ở cuối board view, có thể reorder sau. \\\hline
\end{longtblr}

\vspace{1em}

\textbf{UC04.2 - Cập nhật trạng thái công việc}

\begin{figure}[H]
    \centering
    \includegraphics[width=0.7\textwidth]{images/uc04_2_update_status.png}
    \caption{Sơ đồ use case chức năng cập nhật trạng thái công việc}
    \label{fig:uc04_2_update_status}
\end{figure}

\begin{longtblr}[
    caption = {Đặc tả use case UC04.2 - Cập nhật trạng thái công việc},
    label = {tab:uc04_2},
]{
    colspec={|l|p{.7\linewidth}|}
}
\hline
\textbf{Tên chức năng} & \textbf{Cập nhật trạng thái công việc} \\\hline
ID & UC04.2 \\\hline
Người sử dụng & Project Lead \\\hline
Mức độ cần thiết & Bắt buộc \\\hline
Phân loại & Trung bình \\\hline
Các thành phần tham gia & + \textbf{Project Lead:} Muốn thay đổi tên, mô tả hoặc màu sắc của status. \\\hline
Mô tả tóm tắt & Cho phép Project Lead cập nhật thông tin của status: name, description, color. \\\hline
Trigger & Project Lead click vào status trong settings và chọn "Edit". \\\hline
Kiểu sự kiện & External. \\\hline
Luồng xử lý bình thường &
\begin{minipage}{\linewidth}
    \vskip 4pt
    \begin{enumerate}
        \item Project Lead vào Settings > Issue Statuses
        \item Click vào status cần edit
        \item Form hiển thị với giá trị hiện tại
        \item Project Lead thay đổi name, description hoặc color
        \item Backend validate: name không trống
        \item Backend cập nhật IssueStatus record
        \item Board view cập nhật tên và màu column real-time
        \item Tất cả issue với status này hiển thị màu badge mới
        \item Hiển thị thông báo "Status updated"
    \end{enumerate}
    \vskip 1pt
\end{minipage}
\\\hline
Các luồng sự kiện con & N/A \\\hline
Luồng thay thế/ngoại lệ &
\begin{minipage}{\linewidth}
    \vskip 4pt
    \textbf{\textcolor{red}{E1}} -- Name trống: Hiển thị lỗi "Status name is required".

    \textbf{\textcolor{red}{A1}} -- Inline edit: Project Lead có thể edit trực tiếp từ board view (click vào column header).
    \vskip 1pt
\end{minipage}
\\\hline
Kết quả & Status được cập nhật, tất cả issue với status này hiển thị thông tin mới (màu, tên). \\\hline
\end{longtblr}

\vspace{1em}

\textbf{UC04.3 - Reorder status columns}

\begin{figure}[H]
    \centering
    \includegraphics[width=0.7\textwidth]{images/uc04_3_reorder_status.png}
    \caption{Sơ đồ use case chức năng reorder status columns}
    \label{fig:uc04_3_reorder_status}
\end{figure}

\begin{longtblr}[
    caption = {Đặc tả use case UC04.3 - Reorder status columns},
    label = {tab:uc04_3},
]{
    colspec={|l|p{.7\linewidth}|}
}
\hline
\textbf{Tên chức năng} & \textbf{Reorder status columns} \\\hline
ID & UC04.3 \\\hline
Người sử dụng & Project Lead \\\hline
Mức độ cần thiết & Bắt buộc \\\hline
Phân loại & Trung bình \\\hline
Các thành phần tham gia & + \textbf{Project Lead:} Muốn thay đổi thứ tự hiển thị của status columns trong board view (ví dụ: TO DO → IN PROGRESS → IN REVIEW → DONE). \\\hline
Mô tả tóm tắt & Cho phép Project Lead kéo thả status columns để reorder trong settings hoặc board view. \\\hline
Trigger & Project Lead kéo thả status column trong settings hoặc board view. \\\hline
Kiểu sự kiện & External. \\\hline
Luồng xử lý bình thường &
\begin{minipage}{\linewidth}
    \vskip 4pt
    \begin{enumerate}
        \item Project Lead vào Settings > Issue Statuses hoặc board view
        \item Kéo status column từ vị trí cũ sang vị trí mới
        \item Frontend hiển thị preview drop zone
        \item Project Lead thả column
        \item Backend tính lại order cho tất cả status affected
        \item Backend update order trong database (batch update)
        \item Board view reorder columns real-time
        \item Hiển thị thông báo "Status order updated"
    \end{enumerate}
    \vskip 1pt
\end{minipage}
\\\hline
Các luồng sự kiện con & N/A \\\hline
Luồng thay thế/ngoại lệ &
\begin{minipage}{\linewidth}
    \vskip 4pt
    \textbf{\textcolor{red}{A1}} -- Reorder từ settings: Project Lead có thể dùng up/down arrows để reorder thay vì kéo thả.

    \textbf{\textcolor{red}{A2}} -- Batch reorder: Nếu nhiều status bị affect, backend tính lại order cho tất cả (0, 1, 2, ...).
    \vskip 1pt
\end{minipage}
\\\hline
Kết quả & Status columns hiển thị theo thứ tự mới trong board view, order được lưu vào database. \\\hline
\end{longtblr}

\vspace{1em}

\textbf{UC04.4 - Xóa trạng thái công việc}

\begin{figure}[H]
    \centering
    \includegraphics[width=0.7\textwidth]{images/uc04_4_delete_status.png}
    \caption{Sơ đồ use case chức năng xóa trạng thái công việc}
    \label{fig:uc04_4_delete_status}
\end{figure}

\begin{longtblr}[
    caption = {Đặc tả use case UC04.4 - Xóa trạng thái công việc},
    label = {tab:uc04_4},
]{
    colspec={|l|p{.7\linewidth}|}
}
\hline
\textbf{Tên chức năng} & \textbf{Xóa trạng thái công việc} \\\hline
ID & UC04.4 \\\hline
Người sử dụng & Project Lead \\\hline
Mức độ cần thiết & Bắt buộc \\\hline
Phân loại & Trung bình \\\hline
Các thành phần tham gia & + \textbf{Project Lead:} Muốn xóa status không còn sử dụng hoặc thừa. \\\hline
Mô tả tóm tắt & Cho phép Project Lead xóa trạng thái công việc. Nếu status đang có issue, phải di chuyển issue sang status khác trước. \\\hline
Trigger & Project Lead click "Delete" trên status trong settings. \\\hline
Kiểu sự kiện & External. \\\hline
Luồng xử lý bình thường &
\begin{minipage}{\linewidth}
    \vskip 4pt
    \begin{enumerate}
        \item Project Lead vào Settings > Issue Statuses
        \item Click "Delete" trên status
        \item Backend kiểm tra số lượng issue với status này
        \item Nếu có issue: Hiển thị dialog "Move N issues to:" với dropdown chọn status khác
        \item Project Lead chọn target status
        \item Project Lead xác nhận xóa
        \item Backend move tất cả issue sang status mới (batch update)
        \item Backend xóa IssueStatus record
        \item Board view remove column
        \item Hiển thị thông báo "Status deleted, N issues moved"
    \end{enumerate}
    \vskip 1pt
\end{minipage}
\\\hline
Các luồng sự kiện con & N/A \\\hline
Luồng thay thế/ngoại lệ &
\begin{minipage}{\linewidth}
    \vskip 4pt
    \textbf{\textcolor{red}{E1}} -- Xóa status cuối cùng: Không cho phép xóa nếu chỉ còn 1 status, hiển thị lỗi.

    \textbf{\textcolor{red}{A1}} -- Không có issue: Nếu status không có issue nào, xóa trực tiếp không cần chọn target status.

    \textbf{\textcolor{red}{A2}} -- Cancel: Project Lead có thể cancel nếu không muốn move issue.
    \vskip 1pt
\end{minipage}
\\\hline
Kết quả & Status bị xóa, tất cả issue được move sang status mới, board view cập nhật columns. \\\hline
\end{longtblr}
