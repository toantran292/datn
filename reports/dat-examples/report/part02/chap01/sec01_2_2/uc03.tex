\paragraph{UC03: Quản lý Công việc}
\mbox{}

\textbf{UC03.1 - Tạo công việc mới}

\begin{figure}[H]
    \centering
    \includegraphics[width=\textwidth]{images/uc03_1_create_issue.png}
    \caption{Sơ đồ use case chức năng tạo công việc}
    \label{fig:uc03_1_create_issue}
\end{figure}

\begin{longtblr}[
    caption = {Đặc tả use case UC03.1 - Tạo công việc mới},
    label = {tab:uc03_1},
]{
    colspec={|l|p{.7\linewidth}|}
}
\hline
\textbf{Tên chức năng} & \textbf{Tạo công việc mới} \\\hline
ID & UC03.1 \\\hline
Người sử dụng & Project Lead, Team Member \\\hline
Mức độ cần thiết & Bắt buộc \\\hline
Phân loại & Cao \\\hline
Các thành phần tham gia & + \textbf{User:} Muốn tạo công việc mới để theo dõi task, bug hoặc user story trong dự án. \\\hline
Mô tả tóm tắt & Cho phép user tạo công việc mới với thông tin: name, type (STORY/TASK/BUG/EPIC), priority (LOW/MEDIUM/HIGH/CRITICAL), description, sprint, assignees, story points. Công việc được tự động gán identifier theo dự án (ví dụ: PM-1, PM-2). \\\hline
Trigger & User nhấn "Tạo công việc" trong board view hoặc backlog view. \\\hline
Kiểu sự kiện & External. \\\hline
Luồng xử lý bình thường &
\begin{minipage}{\linewidth}
    \vskip 4pt
    \begin{enumerate}
        \item User nhấn "Tạo công việc"
        \item Form hiển thị với: Tên công việc (required), Type, Priority, Description, Sprint (optional), Assignees (optional), Story Points (optional), Start Date, Target Date
        \item User nhập thông tin công việc
        \item Backend validate: name không trống
        \item Backend tự động tạo sequenceId tiếp theo (ví dụ: 1, 2, 3...)
        \item Identifier = projectIdentifier + "-" + sequenceId (PM-1)
        \item Nếu không chọn status, gán status đầu tiên (thường là TODO)
        \item Nếu chọn sprint, gán vào sprint đó (hiển thị trong board nếu sprint đang ACTIVE)
        \item Backend tính sortOrder để đặt công việc ở cuối danh sách
        \item Tạo IssueActivity record ghi log "created"
        \item Hiển thị công việc mới trong board/backlog
    \end{enumerate}
    \vskip 1pt
\end{minipage}
\\\hline
Các luồng sự kiện con & N/A \\\hline
Luồng thay thế/ngoại lệ &
\begin{minipage}{\linewidth}
    \vskip 4pt
    \textbf{\textcolor{red}{E1}} -- Name trống: Hiển thị lỗi "Tên công việc là bắt buộc".

    \textbf{\textcolor{red}{E2}} -- Sprint không thuộc project: Hiển thị lỗi "Sprint không thuộc dự án".

    \textbf{\textcolor{red}{A1}} -- Tạo công việc con: User có thể chọn công việc cha để tạo công việc con (parentId).

    \textbf{\textcolor{red}{A2}} -- Tạo nhanh: User có thể tạo nhanh chỉ với name, các field khác dùng giá trị mặc định.
    \vskip 1pt
\end{minipage}
\\\hline
Kết quả & Công việc mới được tạo với identifier duy nhất (PM-1), hiển thị trong board/backlog view, có thể assign, comment và update sau. \\\hline
\end{longtblr}

\vspace{1em}

\textbf{UC03.2 - Cập nhật công việc}

\begin{figure}[H]
    \centering
    \includegraphics[width=\textwidth]{images/uc03_2_update_issue.png}
    \caption{Sơ đồ use case chức năng cập nhật công việc}
    \label{fig:uc03_2_update_issue}
\end{figure}

\begin{longtblr}[
    caption = {Đặc tả use case UC03.2 - Cập nhật công việc},
    label = {tab:uc03_2},
]{
    colspec={|l|p{.7\linewidth}|}
}
\hline
\textbf{Tên chức năng} & \textbf{Cập nhật công việc} \\\hline
ID & UC03.2 \\\hline
Người sử dụng & Project Lead, Team Member, Viewer (read-only) \\\hline
Mức độ cần thiết & Bắt buộc \\\hline
Phân loại & Cao \\\hline
Các thành phần tham gia & + \textbf{User:} Muốn cập nhật thông tin công việc như status, priority, assignees, description, sprint. \\\hline
Mô tả tóm tắt & Cho phép user cập nhật các thuộc tính của công việc. Mỗi thay đổi được ghi log vào IssueActivity. \\\hline
Trigger & User thay đổi field trong panel chi tiết công việc hoặc kéo thả trong board view. \\\hline
Kiểu sự kiện & External. \\\hline
Luồng xử lý bình thường &
\begin{minipage}{\linewidth}
    \vskip 4pt
    \begin{enumerate}
        \item User mở panel chi tiết công việc hoặc inline edit
        \item User thay đổi field: status, priority, assignees, description, sprint, story points, dates
        \item Backend validate thay đổi (ví dụ: status phải thuộc project)
        \item Backend cập nhật công việc
        \item Backend tạo IssueActivity record ghi log thay đổi (field nào, giá trị cũ, giá trị mới)
        \item Gửi notification cho assignees nếu có thay đổi về assignees
        \item UI cập nhật real-time (hiển thị thay đổi ngay)
        \item Hiển thị activity log trong tab Activity
    \end{enumerate}
    \vskip 1pt
\end{minipage}
\\\hline
Các luồng sự kiện con & N/A \\\hline
Luồng thay thế/ngoại lệ &
\begin{minipage}{\linewidth}
    \vskip 4pt
    \textbf{\textcolor{red}{A1}} -- Kéo thả công việc trong board: User kéo công việc sang column khác (status mới), backend cập nhật statusId và sortOrder.

    \textbf{\textcolor{red}{A2}} -- Di chuyển công việc vào sprint: User kéo công việc từ backlog vào sprint hoặc giữa các sprint.

    \textbf{\textcolor{red}{A3}} -- Sắp xếp lại công việc: User kéo thả để thay đổi thứ tự, backend tính lại sortOrder giữa 2 công việc lân cận.
    \vskip 1pt
\end{minipage}
\\\hline
Kết quả & Công việc được cập nhật thành công, activity log ghi lại thay đổi, các member liên quan nhận notification. \\\hline
\end{longtblr}

\vspace{1em}

\textbf{UC03.3 - Xem chi tiết công việc}

\begin{figure}[H]
    \centering
    \includegraphics[width=\textwidth]{images/uc03_3_view_issue.png}
    \caption{Sơ đồ use case chức năng xem chi tiết công việc}
    \label{fig:uc03_3_view_issue}
\end{figure}

\begin{longtblr}[
    caption = {Đặc tả use case UC03.3 - Xem chi tiết công việc},
    label = {tab:uc03_3},
]{
    colspec={|l|p{.7\linewidth}|}
}
\hline
\textbf{Tên chức năng} & \textbf{Xem chi tiết công việc} \\\hline
ID & UC03.3 \\\hline
Người sử dụng & Project Lead, Team Member, Viewer \\\hline
Mức độ cần thiết & Bắt buộc \\\hline
Phân loại & Trung bình \\\hline
Các thành phần tham gia & + \textbf{User:} Muốn xem thông tin đầy đủ của công việc, bao gồm description, comments, activity log, công việc con. \\\hline
Mô tả tóm tắt & Hiển thị chi tiết công việc trong panel bên phải với các tab: Details, Activity, Comments. \\\hline
Trigger & User click vào card công việc trong board view hoặc backlog view. \\\hline
Kiểu sự kiện & External. \\\hline
Luồng xử lý bình thường &
\begin{minipage}{\linewidth}
    \vskip 4pt
    \begin{enumerate}
        \item User click vào card công việc
        \item Backend query công việc với relations: status, sprint, assignees, comments, activities, subIssues
        \item Panel hiển thị:
        \begin{itemize}
            \item Header: Identifier (PM-1), Name, Close button
            \item Properties: Type, Priority, Status, Sprint, Assignees, Story Points, Dates
            \item Description (rich text editor với markdown support)
            \item Tab Activity: Log tất cả thay đổi (created, status changed, assigned, etc.)
            \item Tab Comments: Danh sách comments (include UC03.5)
            \item Công việc con (nếu có): Danh sách công việc con
        \end{itemize}
        \item User có thể inline edit các field
        \item User có thể thêm comment (UC03.5)
    \end{enumerate}
    \vskip 1pt
\end{minipage}
\\\hline
Các luồng sự kiện con &
\begin{minipage}{\linewidth}
    \vskip 4pt
    + Include UC03.5 (Thêm comment).
    \vskip 1pt
\end{minipage}
\\\hline
Luồng thay thế/ngoại lệ &
\begin{minipage}{\linewidth}
    \vskip 4pt
    \textbf{\textcolor{red}{A1}} -- Xem công việc con: Click vào công việc con để mở panel chi tiết của công việc đó.

    \textbf{\textcolor{red}{A2}} -- Navigate bằng keyboard: User có thể dùng phím mũi tên để di chuyển giữa các công việc.
    \vskip 1pt
\end{minipage}
\\\hline
Kết quả & User xem được đầy đủ thông tin công việc, có thể edit inline, thêm comment, xem activity history. \\\hline
\end{longtblr}

\vspace{1em}

\textbf{UC03.4 - Xóa công việc}

\begin{figure}[H]
    \centering
    \includegraphics[width=\textwidth]{images/uc03_4_delete_issue.png}
    \caption{Sơ đồ use case chức năng xóa công việc}
    \label{fig:uc03_4_delete_issue}
\end{figure}

\begin{longtblr}[
    caption = {Đặc tả use case UC03.4 - Xóa công việc},
    label = {tab:uc03_4},
]{
    colspec={|l|p{.7\linewidth}|}
}
\hline
\textbf{Tên chức năng} & \textbf{Xóa công việc} \\\hline
ID & UC03.4 \\\hline
Người sử dụng & Project Lead \\\hline
Mức độ cần thiết & Bắt buộc \\\hline
Phân loại & Trung bình \\\hline
Các thành phần tham gia & + \textbf{Project Lead:} Muốn xóa công việc không còn cần thiết hoặc tạo nhầm. \\\hline
Mô tả tóm tắt & Cho phép Project Lead xóa công việc. Nếu có công việc con, phải xử lý công việc con trước. \\\hline
Trigger & Project Lead nhấn "Xóa công việc" trong panel chi tiết. \\\hline
Kiểu sự kiện & External. \\\hline
Luồng xử lý bình thường &
\begin{minipage}{\linewidth}
    \vskip 4pt
    \begin{enumerate}
        \item Project Lead mở panel chi tiết công việc
        \item Nhấn button "Xóa công việc"
        \item Backend kiểm tra quyền (chỉ Project Lead)
        \item Backend kiểm tra công việc con
        \item Hiển thị dialog xác nhận với warning về công việc con (nếu có)
        \item Project Lead xác nhận
        \item Backend cascade delete: comments, activities
        \item Backend set parentId = NULL cho công việc con (hoặc xóa nếu user chọn)
        \item Backend xóa công việc
        \item UI remove công việc khỏi board/backlog
        \item Hiển thị toast "Đã xóa công việc"
    \end{enumerate}
    \vskip 1pt
\end{minipage}
\\\hline
Các luồng sự kiện con & N/A \\\hline
Luồng thay thế/ngoại lệ &
\begin{minipage}{\linewidth}
    \vskip 4pt
    \textbf{\textcolor{red}{E1}} -- Không có quyền: Team Member và Viewer không thể xóa công việc.

    \textbf{\textcolor{red}{A1}} -- Có công việc con: Hỏi user chọn: (a) Xóa cả công việc con, (b) Chuyển công việc con thành công việc độc lập.
    \vskip 1pt
\end{minipage}
\\\hline
Kết quả & Công việc và tất cả dữ liệu liên quan (comments, activities) bị xóa, công việc con được xử lý theo lựa chọn. \\\hline
\end{longtblr}

\vspace{1em}

\textbf{UC03.5 - Thêm comment vào công việc}

\begin{figure}[H]
    \centering
    \includegraphics[width=\textwidth]{images/uc03_5_add_comment.png}
    \caption{Sơ đồ use case chức năng thêm comment}
    \label{fig:uc03_5_add_comment}
\end{figure}

\begin{longtblr}[
    caption = {Đặc tả use case UC03.5 - Thêm comment vào công việc},
    label = {tab:uc03_5},
]{
    colspec={|l|p{.7\linewidth}|}
}
\hline
\textbf{Tên chức năng} & \textbf{Thêm comment vào công việc} \\\hline
ID & UC03.5 \\\hline
Người sử dụng & Project Lead, Team Member \\\hline
Mức độ cần thiết & Bắt buộc \\\hline
Phân loại & Trung bình \\\hline
Các thành phần tham gia & + \textbf{User:} Muốn thêm comment để thảo luận, đặt câu hỏi, hoặc update thông tin về công việc. \\\hline
Mô tả tóm tắt & Cho phép user thêm comment vào công việc với markdown support. Comment có thể mention user khác (@username). \\\hline
Trigger & User nhập text vào comment box và nhấn "Thêm comment" trong panel chi tiết công việc. \\\hline
Kiểu sự kiện & External. \\\hline
Luồng xử lý bình thường &
\begin{minipage}{\linewidth}
    \vskip 4pt
    \begin{enumerate}
        \item User mở tab Comments trong panel chi tiết công việc
        \item User nhập comment text (hỗ trợ markdown)
        \item User có thể mention user khác bằng @username
        \item User nhấn "Thêm comment"
        \item Backend validate comment không trống
        \item Backend tạo IssueComment record với: issueId, comment, commentHtml (rendered markdown), createdBy
        \item Backend parse mentions (@username) và gửi notification cho user được mention
        \item Hiển thị comment mới ở cuối danh sách với: avatar, username, timestamp, comment content
        \item Gửi notification cho assignees và followers của công việc
    \end{enumerate}
    \vskip 1pt
\end{minipage}
\\\hline
Các luồng sự kiện con & Include bởi UC03.3 (Xem chi tiết công việc). \\\hline
Luồng thay thế/ngoại lệ &
\begin{minipage}{\linewidth}
    \vskip 4pt
    \textbf{\textcolor{red}{E1}} -- Comment trống: Hiển thị lỗi "Comment không được để trống".

    \textbf{\textcolor{red}{A1}} -- Edit comment: User có thể edit comment của mình (hiển thị "Edited" badge).

    \textbf{\textcolor{red}{A2}} -- Delete comment: User có thể xóa comment của mình hoặc Project Lead xóa bất kỳ comment nào.
    \vskip 1pt
\end{minipage}
\\\hline
Kết quả & Comment được thêm vào công việc, user được mention nhận notification, activity log ghi lại "commented". \\\hline
\end{longtblr}

\vspace{1em}

\textbf{UC03.6 - Tự động tạo công việc từ mô tả}

\begin{figure}[H]
    \centering
    \includegraphics[width=\textwidth]{images/uc03_6_ai_generate_issues.png}
    \caption{Sơ đồ use case chức năng tự động tạo công việc bằng AI}
    \label{fig:uc03_6_ai_generate_issues}
\end{figure}

\begin{longtblr}[
    caption = {Đặc tả use case UC03.6 - Tự động tạo công việc từ mô tả},
    label = {tab:uc03_6},
]{
    colspec={|l|p{.7\linewidth}|}
}
\hline
\textbf{Tên chức năng} & \textbf{Tự động tạo công việc từ mô tả} \\\hline
ID & UC03.6 \\\hline
Người sử dụng & Project Lead, Team Member \\\hline
Mức độ cần thiết & Mở rộng \\\hline
Phân loại & Cao \\\hline
Các thành phần tham gia & + \textbf{User:} Muốn tạo nhiều công việc nhanh chóng từ mô tả tổng quan hoặc user story.

+ \textbf{LLM Service:} Phân tích mô tả đầu vào và tạo danh sách công việc với thông tin chi tiết. \\\hline
Mô tả tóm tắt & User nhập mô tả tổng quan (ví dụ: feature requirements, user story), hệ thống sử dụng LLM để tự động phân tích và tạo ra danh sách công việc (issues) với đầy đủ thông tin: name, description, type, priority, estimated story points. \\\hline
Trigger & User nhấn "Tạo công việc bằng AI" trong backlog view hoặc board view. \\\hline
Kiểu sự kiện & External. \\\hline
Luồng xử lý bình thường &
\begin{minipage}{\linewidth}
    \vskip 4pt
    \begin{enumerate}
        \item User nhấn "Tạo công việc bằng AI"
        \item Dialog hiển thị với textarea để nhập mô tả (requirements, user story, hoặc feature description)
        \item User nhập mô tả chi tiết (ví dụ: "Xây dựng trang đăng nhập với Google OAuth, validation form, remember me, forgot password")
        \item User nhấn "Tạo tự động"
        \item Frontend gửi mô tả đến Backend API
        \item Backend gọi LLM service với prompt template:
        \begin{itemize}
            \item System prompt: "Phân tích mô tả và tạo danh sách công việc chi tiết"
            \item User input: Mô tả từ user
            \item Expected output: JSON array với các công việc
        \end{itemize}
        \item LLM phân tích và trả về danh sách công việc với:
        \begin{itemize}
            \item name: Tên công việc ngắn gọn
            \item description: Mô tả chi tiết (markdown)
            \item type: STORY/TASK/BUG/EPIC
            \item priority: LOW/MEDIUM/HIGH/CRITICAL
            \item storyPoints: Ước lượng (1-13)
        \end{itemize}
        \item Backend validate kết quả từ LLM
        \item Hiển thị preview danh sách công việc được tạo
        \item User review và có thể edit từng công việc trước khi confirm
        \item User nhấn "Tạo tất cả"
        \item Backend tạo từng công việc với sequenceId tăng dần
        \item Ghi log vào AIGenerationLog với: prompt, response, tokensUsed, model
        \item Hiển thị công việc mới trong board/backlog
        \item Toast thông báo "Đã tạo X công việc từ mô tả"
    \end{enumerate}
    \vskip 1pt
\end{minipage}
\\\hline
Các luồng sự kiện con & N/A \\\hline
Luồng thay thế/ngoại lệ &
\begin{minipage}{\linewidth}
    \vskip 4pt
    \textbf{\textcolor{red}{E1}} -- Mô tả trống hoặc quá ngắn: Hiển thị lỗi "Vui lòng nhập mô tả chi tiết hơn (tối thiểu 50 ký tự)".

    \textbf{\textcolor{red}{E2}} -- LLM service lỗi: Hiển thị lỗi "Không thể kết nối AI service, vui lòng thử lại".

    \textbf{\textcolor{red}{E3}} -- LLM trả về kết quả không hợp lệ: Hiển thị lỗi "AI không thể phân tích mô tả, vui lòng làm rõ hơn".

    \textbf{\textcolor{red}{A1}} -- User chỉnh sửa trước khi tạo: User có thể edit name, description, type, priority của từng công việc trong preview.

    \textbf{\textcolor{red}{A2}} -- User loại bỏ một số công việc: User có thể uncheck công việc nào không muốn tạo.

    \textbf{\textcolor{red}{A3}} -- Tạo từ template: User có thể chọn template có sẵn (ví dụ: "Login feature", "REST API endpoint") để AI tạo nhanh hơn.
    \vskip 1pt
\end{minipage}
\\\hline
Kết quả & Danh sách công việc được tạo tự động từ mô tả, hiển thị trong board/backlog, user tiết kiệm thời gian so với tạo thủ công từng công việc. Log AI generation được lưu để tracking chi phí và performance. \\\hline
\end{longtblr}
