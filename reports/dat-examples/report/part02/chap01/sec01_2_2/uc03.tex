\paragraph{UC03: Quản lý Issue}
\mbox{}

\textbf{UC03.1 - Tạo issue mới}

\begin{figure}[H]
    \centering
    \includegraphics[width=0.7\textwidth]{images/uc03_1_create_issue.png}
    \caption{Sơ đồ use case chức năng tạo issue}
    \label{fig:uc03_1_create_issue}
\end{figure}

\begin{longtblr}[
    caption = {Đặc tả use case UC03.1 - Tạo issue mới},
    label = {tab:uc03_1},
]{
    colspec={|l|p{.7\linewidth}|}
}
\hline
\textbf{Tên chức năng} & \textbf{Tạo issue mới} \\\hline
ID & UC03.1 \\\hline
Người sử dụng & Project Lead, Team Member \\\hline
Mức độ cần thiết & Bắt buộc \\\hline
Phân loại & Cao \\\hline
Các thành phần tham gia & + \textbf{User:} Muốn tạo issue mới để theo dõi công việc, bug hoặc user story trong dự án. \\\hline
Mô tả tóm tắt & Cho phép user tạo issue mới với thông tin: name, type (STORY/TASK/BUG/EPIC), priority (LOW/MEDIUM/HIGH/CRITICAL), description, sprint, assignees, story points. Issue được tự động gán identifier theo dự án (ví dụ: PM-1, PM-2). \\\hline
Trigger & User nhấn "Create Issue" trong board view hoặc backlog view. \\\hline
Kiểu sự kiện & External. \\\hline
Luồng xử lý bình thường &
\begin{minipage}{\linewidth}
    \vskip 4pt
    \begin{enumerate}
        \item User nhấn "Create Issue"
        \item Form hiển thị với: Issue Name (required), Type, Priority, Description, Sprint (optional), Assignees (optional), Story Points (optional), Start Date, Target Date
        \item User nhập thông tin issue
        \item Backend validate: name không trống
        \item Backend tự động tạo sequenceId tiếp theo (ví dụ: 1, 2, 3...)
        \item Issue identifier = projectIdentifier + "-" + sequenceId (PM-1)
        \item Nếu không chọn status, gán status đầu tiên (thường là TODO)
        \item Nếu chọn sprint, gán vào sprint đó (hiển thị trong board nếu sprint đang ACTIVE)
        \item Backend tính sortOrder để đặt issue ở cuối danh sách
        \item Tạo IssueActivity record ghi log "created"
        \item Hiển thị issue mới trong board/backlog
    \end{enumerate}
    \vskip 1pt
\end{minipage}
\\\hline
Các luồng sự kiện con & N/A \\\hline
Luồng thay thế/ngoại lệ &
\begin{minipage}{\linewidth}
    \vskip 4pt
    \textbf{\textcolor{red}{E1}} -- Name trống: Hiển thị lỗi "Issue name is required".

    \textbf{\textcolor{red}{E2}} -- Sprint không thuộc project: Hiển thị lỗi "Sprint does not belong to project".

    \textbf{\textcolor{red}{A1}} -- Tạo sub-issue: User có thể chọn parent issue để tạo sub-issue (parentId).

    \textbf{\textcolor{red}{A2}} -- Tạo nhanh: User có thể tạo nhanh chỉ với name, các field khác dùng giá trị mặc định.
    \vskip 1pt
\end{minipage}
\\\hline
Kết quả & Issue mới được tạo với identifier duy nhất (PM-1), hiển thị trong board/backlog view, có thể assign, comment và update sau. \\\hline
\end{longtblr}

\vspace{1em}

\textbf{UC03.2 - Cập nhật issue}

\begin{figure}[H]
    \centering
    \includegraphics[width=0.7\textwidth]{images/uc03_2_update_issue.png}
    \caption{Sơ đồ use case chức năng cập nhật issue}
    \label{fig:uc03_2_update_issue}
\end{figure}

\begin{longtblr}[
    caption = {Đặc tả use case UC03.2 - Cập nhật issue},
    label = {tab:uc03_2},
]{
    colspec={|l|p{.7\linewidth}|}
}
\hline
\textbf{Tên chức năng} & \textbf{Cập nhật issue} \\\hline
ID & UC03.2 \\\hline
Người sử dụng & Project Lead, Team Member, Viewer (read-only) \\\hline
Mức độ cần thiết & Bắt buộc \\\hline
Phân loại & Cao \\\hline
Các thành phần tham gia & + \textbf{User:} Muốn cập nhật thông tin issue như status, priority, assignees, description, sprint. \\\hline
Mô tả tóm tắt & Cho phép user cập nhật các thuộc tính của issue. Mỗi thay đổi được ghi log vào IssueActivity. \\\hline
Trigger & User thay đổi field trong issue detail panel hoặc kéo thả issue trong board view. \\\hline
Kiểu sự kiện & External. \\\hline
Luồng xử lý bình thường &
\begin{minipage}{\linewidth}
    \vskip 4pt
    \begin{enumerate}
        \item User mở issue detail panel hoặc inline edit
        \item User thay đổi field: status, priority, assignees, description, sprint, story points, dates
        \item Backend validate thay đổi (ví dụ: status phải thuộc project)
        \item Backend cập nhật issue
        \item Backend tạo IssueActivity record ghi log thay đổi (field nào, giá trị cũ, giá trị mới)
        \item Gửi notification cho assignees nếu có thay đổi về assignees
        \item UI cập nhật real-time (hiển thị thay đổi ngay)
        \item Hiển thị activity log trong tab Activity
    \end{enumerate}
    \vskip 1pt
\end{minipage}
\\\hline
Các luồng sự kiện con & N/A \\\hline
Luồng thay thế/ngoại lệ &
\begin{minipage}{\linewidth}
    \vskip 4pt
    \textbf{\textcolor{red}{A1}} -- Kéo thả issue trong board: User kéo issue sang column khác (status mới), backend cập nhật statusId và sortOrder.

    \textbf{\textcolor{red}{A2}} -- Move issue to sprint: User kéo issue từ backlog vào sprint hoặc giữa các sprint.

    \textbf{\textcolor{red}{A3}} -- Reorder issue: User kéo thả để thay đổi thứ tự, backend tính lại sortOrder giữa 2 issue lân cận.
    \vskip 1pt
\end{minipage}
\\\hline
Kết quả & Issue được cập nhật thành công, activity log ghi lại thay đổi, các member liên quan nhận notification. \\\hline
\end{longtblr}

\vspace{1em}

\textbf{UC03.3 - Xem chi tiết issue}

\begin{figure}[H]
    \centering
    \includegraphics[width=0.7\textwidth]{images/uc03_3_view_issue.png}
    \caption{Sơ đồ use case chức năng xem chi tiết issue}
    \label{fig:uc03_3_view_issue}
\end{figure}

\begin{longtblr}[
    caption = {Đặc tả use case UC03.3 - Xem chi tiết issue},
    label = {tab:uc03_3},
]{
    colspec={|l|p{.7\linewidth}|}
}
\hline
\textbf{Tên chức năng} & \textbf{Xem chi tiết issue} \\\hline
ID & UC03.3 \\\hline
Người sử dụng & Project Lead, Team Member, Viewer \\\hline
Mức độ cần thiết & Bắt buộc \\\hline
Phân loại & Trung bình \\\hline
Các thành phần tham gia & + \textbf{User:} Muốn xem thông tin đầy đủ của issue, bao gồm description, comments, activity log, sub-issues. \\\hline
Mô tả tóm tắt & Hiển thị chi tiết issue trong panel bên phải với các tab: Details, Activity, Comments. \\\hline
Trigger & User click vào issue card trong board view hoặc backlog view. \\\hline
Kiểu sự kiện & External. \\\hline
Luồng xử lý bình thường &
\begin{minipage}{\linewidth}
    \vskip 4pt
    \begin{enumerate}
        \item User click vào issue card
        \item Backend query issue với relations: status, sprint, assignees, comments, activities, subIssues
        \item Panel hiển thị:
        \begin{itemize}
            \item Header: Identifier (PM-1), Name, Close button
            \item Properties: Type, Priority, Status, Sprint, Assignees, Story Points, Dates
            \item Description (rich text editor với markdown support)
            \item Tab Activity: Log tất cả thay đổi (created, status changed, assigned, etc.)
            \item Tab Comments: Danh sách comments (include UC03.5)
            \item Sub-issues (nếu có): Danh sách sub-issues
        \end{itemize}
        \item User có thể inline edit các field
        \item User có thể thêm comment (UC03.5)
    \end{enumerate}
    \vskip 1pt
\end{minipage}
\\\hline
Các luồng sự kiện con &
\begin{minipage}{\linewidth}
    \vskip 4pt
    + Include UC03.5 (Thêm comment).
    \vskip 1pt
\end{minipage}
\\\hline
Luồng thay thế/ngoại lệ &
\begin{minipage}{\linewidth}
    \vskip 4pt
    \textbf{\textcolor{red}{A1}} -- Xem sub-issues: Click vào sub-issue để mở detail panel của sub-issue đó.

    \textbf{\textcolor{red}{A2}} -- Navigate bằng keyboard: User có thể dùng phím mũi tên để di chuyển giữa các issue.
    \vskip 1pt
\end{minipage}
\\\hline
Kết quả & User xem được đầy đủ thông tin issue, có thể edit inline, thêm comment, xem activity history. \\\hline
\end{longtblr}

\vspace{1em}

\textbf{UC03.4 - Xóa issue}

\begin{figure}[H]
    \centering
    \includegraphics[width=0.7\textwidth]{images/uc03_4_delete_issue.png}
    \caption{Sơ đồ use case chức năng xóa issue}
    \label{fig:uc03_4_delete_issue}
\end{figure}

\begin{longtblr}[
    caption = {Đặc tả use case UC03.4 - Xóa issue},
    label = {tab:uc03_4},
]{
    colspec={|l|p{.7\linewidth}|}
}
\hline
\textbf{Tên chức năng} & \textbf{Xóa issue} \\\hline
ID & UC03.4 \\\hline
Người sử dụng & Project Lead \\\hline
Mức độ cần thiết & Bắt buộc \\\hline
Phân loại & Trung bình \\\hline
Các thành phần tham gia & + \textbf{Project Lead:} Muốn xóa issue không còn cần thiết hoặc tạo nhầm. \\\hline
Mô tả tóm tắt & Cho phép Project Lead xóa issue. Nếu issue có sub-issues, phải xử lý sub-issues trước. \\\hline
Trigger & Project Lead nhấn "Delete" trong issue detail panel. \\\hline
Kiểu sự kiện & External. \\\hline
Luồng xử lý bình thường &
\begin{minipage}{\linewidth}
    \vskip 4pt
    \begin{enumerate}
        \item Project Lead mở issue detail panel
        \item Nhấn button "Delete Issue"
        \item Backend kiểm tra quyền (chỉ Project Lead)
        \item Backend kiểm tra sub-issues
        \item Hiển thị dialog xác nhận với warning về sub-issues (nếu có)
        \item Project Lead xác nhận
        \item Backend cascade delete: comments, activities
        \item Backend set parentId = NULL cho sub-issues (hoặc xóa nếu user chọn)
        \item Backend xóa issue
        \item UI remove issue khỏi board/backlog
        \item Hiển thị toast "Issue deleted"
    \end{enumerate}
    \vskip 1pt
\end{minipage}
\\\hline
Các luồng sự kiện con & N/A \\\hline
Luồng thay thế/ngoại lệ &
\begin{minipage}{\linewidth}
    \vskip 4pt
    \textbf{\textcolor{red}{E1}} -- Không có quyền: Team Member và Viewer không thể xóa issue.

    \textbf{\textcolor{red}{A1}} -- Issue có sub-issues: Hỏi user chọn: (a) Xóa cả sub-issues, (b) Chuyển sub-issues thành issue độc lập.
    \vskip 1pt
\end{minipage}
\\\hline
Kết quả & Issue và tất cả dữ liệu liên quan (comments, activities) bị xóa, sub-issues được xử lý theo lựa chọn. \\\hline
\end{longtblr}

\vspace{1em}

\textbf{UC03.5 - Thêm comment vào issue}

\begin{figure}[H]
    \centering
    \includegraphics[width=0.7\textwidth]{images/uc03_5_add_comment.png}
    \caption{Sơ đồ use case chức năng thêm comment}
    \label{fig:uc03_5_add_comment}
\end{figure}

\begin{longtblr}[
    caption = {Đặc tả use case UC03.5 - Thêm comment vào issue},
    label = {tab:uc03_5},
]{
    colspec={|l|p{.7\linewidth}|}
}
\hline
\textbf{Tên chức năng} & \textbf{Thêm comment vào issue} \\\hline
ID & UC03.5 \\\hline
Người sử dụng & Project Lead, Team Member \\\hline
Mức độ cần thiết & Bắt buộc \\\hline
Phân loại & Trung bình \\\hline
Các thành phần tham gia & + \textbf{User:} Muốn thêm comment để thảo luận, đặt câu hỏi, hoặc update thông tin về issue. \\\hline
Mô tả tóm tắt & Cho phép user thêm comment vào issue với markdown support. Comment có thể mention user khác (@username). \\\hline
Trigger & User nhập text vào comment box và nhấn "Add Comment" trong issue detail panel. \\\hline
Kiểu sự kiện & External. \\\hline
Luồng xử lý bình thường &
\begin{minipage}{\linewidth}
    \vskip 4pt
    \begin{enumerate}
        \item User mở tab Comments trong issue detail panel
        \item User nhập comment text (hỗ trợ markdown)
        \item User có thể mention user khác bằng @username
        \item User nhấn "Add Comment"
        \item Backend validate comment không trống
        \item Backend tạo IssueComment record với: issueId, comment, commentHtml (rendered markdown), createdBy
        \item Backend parse mentions (@username) và gửi notification cho user được mention
        \item Hiển thị comment mới ở cuối danh sách với: avatar, username, timestamp, comment content
        \item Gửi notification cho assignees và followers của issue
    \end{enumerate}
    \vskip 1pt
\end{minipage}
\\\hline
Các luồng sự kiện con & Include bởi UC03.3 (Xem chi tiết issue). \\\hline
Luồng thay thế/ngoại lệ &
\begin{minipage}{\linewidth}
    \vskip 4pt
    \textbf{\textcolor{red}{E1}} -- Comment trống: Hiển thị lỗi "Comment cannot be empty".

    \textbf{\textcolor{red}{A1}} -- Edit comment: User có thể edit comment của mình (hiển thị "Edited" badge).

    \textbf{\textcolor{red}{A2}} -- Delete comment: User có thể xóa comment của mình hoặc Project Lead xóa bất kỳ comment nào.
    \vskip 1pt
\end{minipage}
\\\hline
Kết quả & Comment được thêm vào issue, user được mention nhận notification, activity log ghi lại "commented". \\\hline
\end{longtblr}
