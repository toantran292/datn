\paragraph{UC06: Tính năng AI}
\mbox{}

\textbf{UC06.1 - AI Sprint Summary}

\begin{figure}[H]
    \centering
    \includegraphics[width=1\textwidth]{images/uc06_1_ai_sprint_summary.png}
    \caption{Sơ đồ use case chức năng AI Sprint Summary}
    \label{fig:uc06_1_ai_sprint_summary}
\end{figure}

\begin{longtblr}[
    caption = {Đặc tả use case UC06.1 - AI Sprint Summary},
    label = {tab:uc06_1},
]{
    colspec={|l|p{.7\linewidth}|}
}
\hline
\textbf{Tên chức năng} & \textbf{AI Sprint Summary} \\\hline
ID & UC06.1 \\\hline
Người sử dụng & Project Lead \\\hline
Mức độ cần thiết & Tùy chọn \\\hline
Phân loại & Trung bình \\\hline
Các thành phần tham gia &
\begin{minipage}{\linewidth}
    \vskip 4pt
    + \textbf{Project Lead:} Muốn có summary tự động của sprint vừa hoàn thành. \\
    + \textbf{LLM Provider:} Phân tích dữ liệu sprint và tạo summary.
    \vskip 1pt
\end{minipage}
\\\hline
Mô tả tóm tắt & Khi complete sprint (UC02.3), Project Lead có option để AI tự động tạo sprint summary dựa trên issues completed, comments, activities. \\\hline
Trigger & Project Lead chọn "Generate AI Summary" khi complete sprint (UC02.3). \\\hline
Kiểu sự kiện & External. \\\hline
Luồng xử lý bình thường &
\begin{minipage}{\linewidth}
    \vskip 4pt
    \begin{enumerate}
        \item Project Lead complete sprint và check "Generate AI Summary"
        \item Backend collect sprint data:
        \begin{itemize}
            \item Tất cả issues của sprint (completed \& incomplete)
            \item Sprint goal, start date, end date
            \item Issue comments, activities
            \item Velocity, story points completed
        \end{itemize}
        \item Backend format data thành prompt cho AI
        \item Backend gọi LLM API (UC06.4)
        \item AI analyze và generate summary với sections:
        \begin{itemize}
            \item Overview: Sprint goal achievement
            \item Completed: Danh sách issues hoàn thành
            \item Incomplete: Issues chưa hoàn thành và lý do
            \item Insights: Patterns, blockers, recommendations
            \item Velocity: So sánh với sprint trước
        \end{itemize}
        \item Backend lưu summary vào Sprint record
        \item Hiển thị summary trong sprint detail page
        \item Team members có thể xem và export summary
    \end{enumerate}
    \vskip 1pt
\end{minipage}
\\\hline
Các luồng sự kiện con &
\begin{minipage}{\linewidth}
    \vskip 4pt
    + Include UC06.4 (LLM API Integration).
    \vskip 1pt
\end{minipage}
\\\hline
Luồng thay thế/ngoại lệ &
\begin{minipage}{\linewidth}
    \vskip 4pt
    \textbf{\textcolor{red}{E1}} -- LLM API unavailable: Hiển thị lỗi "AI service temporarily unavailable, try again later".

    \textbf{\textcolor{red}{A1}} -- Edit summary: Project Lead có thể edit summary sau khi AI generate.

    \textbf{\textcolor{red}{A2}} -- Regenerate: Project Lead có thể regenerate summary nếu không hài lòng.
    \vskip 1pt
\end{minipage}
\\\hline
Kết quả & Sprint summary được tạo tự động, giúp team review sprint hiệu quả hơn. \\\hline
\end{longtblr}

\vspace{1em}

\textbf{UC06.2 - AI tự động tinh chỉnh mô tả công việc}

\begin{figure}[H]
    \centering
    \includegraphics[width=1\textwidth]{images/uc06_2_ai_description.png}
    \caption{Sơ đồ use case chức năng AI tự động tinh chỉnh mô tả công việc}
    \label{fig:uc06_2_ai_description}
\end{figure}

\begin{longtblr}[
    caption = {Đặc tả use case UC06.2 - AI tự động tinh chỉnh mô tả công việc},
    label = {tab:uc06_2},
]{
    colspec={|l|p{.7\linewidth}|}
}
\hline
\textbf{Tên chức năng} & \textbf{AI tự động tinh chỉnh mô tả công việc} \\\hline
ID & UC06.2 \\\hline
Người sử dụng & Project Lead, Team Member \\\hline
Mức độ cần thiết & Tùy chọn \\\hline
Phân loại & Trung bình \\\hline
Các thành phần tham gia &
\begin{minipage}{\linewidth}
    \vskip 4pt
    + \textbf{User:} Muốn AI giúp refine và format lại issue description cho đẹp và đầy đủ hơn. \\
    + \textbf{LLM Provider:} Phân tích description hiện tại và refine theo best practices.
    \vskip 1pt
\end{minipage}
\\\hline
Mô tả tóm tắt & AI tự động phát hiện và tinh chỉnh mô tả công việc để format đẹp hơn, bổ sung thông tin thiếu, làm rõ acceptance criteria và thêm technical notes nếu cần. \\\hline
Trigger & User nhấn "Refine Description with AI" button trong issue detail hoặc issue edit form. \\\hline
Kiểu sự kiện & External. \\\hline
Luồng xử lý bình thường &
\begin{minipage}{\linewidth}
    \vskip 4pt
    \begin{enumerate}
        \item User mở issue có description chưa đầy đủ hoặc chưa format tốt
        \item User nhấn "Refine Description with AI"
        \item Backend detect mô tả cần cải thiện (thiếu sections, format không chuẩn)
        \item Backend tạo prompt với:
        \begin{itemize}
            \item Description hiện tại
            \item Issue type (STORY, TASK, BUG, EPIC)
            \item Project context, related issues
        \end{itemize}
        \item Backend gọi LLM API (UC06.4)
        \item AI refine description theo best practices:
        \begin{itemize}
            \item Format theo template chuẩn (User Story/Acceptance Criteria/Technical Notes)
            \item Bổ sung thông tin thiếu (edge cases, error handling)
            \item Làm rõ acceptance criteria thành bullet points cụ thể
            \item Thêm technical notes về implementation approach
        \end{itemize}
        \item Frontend hiển thị side-by-side comparison (current vs refined)
        \item User review changes, có thể accept/reject từng section
        \item User apply changes hoặc edit thêm trước khi save
        \item Backend log AI refinement activity
    \end{enumerate}
    \vskip 1pt
\end{minipage}
\\\hline
Các luồng sự kiện con &
\begin{minipage}{\linewidth}
    \vskip 4pt
    + Include UC06.4 (LLM API Integration).
    \vskip 1pt
\end{minipage}
\\\hline
Luồng thay thế/ngoại lệ &
\begin{minipage}{\linewidth}
    \vskip 4pt
    \textbf{\textcolor{red}{E1}} -- Description already good: AI suggest "Description looks good, no refinement needed".

    \textbf{\textcolor{red}{A1}} -- Regenerate: User có thể nhấn "Regenerate" để AI refine lại với approach khác.

    \textbf{\textcolor{red}{A2}} -- Partial accept: User có thể chọn apply chỉ một số sections (ví dụ: chỉ lấy acceptance criteria).
    \vskip 1pt
\end{minipage}
\\\hline
Kết quả & Issue description được tinh chỉnh tự động với format đẹp, đầy đủ thông tin, giúp team hiểu rõ requirements. \\\hline
\end{longtblr}


\vspace{1em}

\textbf{UC06.3 - AI dịch mô tả công việc}

\begin{figure}[H]
    \centering
    \includegraphics[width=1\textwidth]{images/uc06_3_ai_translate.png}
    \caption{Sơ đồ use case chức năng AI dịch mô tả công việc}
    \label{fig:uc06_3_ai_translate}
\end{figure}

\begin{longtblr}[
    caption = {Đặc tả use case UC06.3 - AI dịch mô tả công việc},
    label = {tab:uc06_3},
]{
    colspec={|l|p{.7\linewidth}|}
}
\hline
\textbf{Tên chức năng} & \textbf{AI dịch mô tả công việc} \\\hline
ID & UC06.3 \\\hline
Người sử dụng & Project Lead, Team Member \\\hline
Mức độ cần thiết & Tùy chọn \\\hline
Phân loại & Trung bình \\\hline
Các thành phần tham gia &
\begin{minipage}{\linewidth}
    \vskip 4pt
    + \textbf{User:} Muốn dịch mô tả công việc sang ngôn ngữ khác. \\
    + \textbf{LLM Provider:} Dịch văn bản với context-aware translation.
    \vskip 1pt
\end{minipage}
\\\hline
Mô tả tóm tắt & AI tự động dịch mô tả công việc sang ngôn ngữ đích (English, Korean, Japanese, Chinese, Vietnamese) với streaming real-time, giúp team đa quốc gia làm việc hiệu quả. \\\hline
Trigger & User nhấn "Translate" button và chọn ngôn ngữ đích trong issue detail. \\\hline
Kiểu sự kiện & External. \\\hline
Luồng xử lý bình thường &
\begin{minipage}{\linewidth}
    \vskip 4pt
    \begin{enumerate}
        \item User mở issue và nhấn "Translate" button
        \item Hệ thống hiển thị language picker với các options:
        \begin{itemize}
            \item \textbf{English} (en)
            \item \textbf{Korean} (ko)
            \item \textbf{Japanese} (ja)
            \item \textbf{Chinese} (zh)
            \item \textbf{Vietnamese} (vi)
        \end{itemize}
        \item User chọn ngôn ngữ đích
        \item Backend tạo prompt với:
        \begin{itemize}
            \item Issue description hiện tại
            \item Issue type và context
            \item Target language
        \end{itemize}
        \item Backend gọi LLM API với translate-description-stream endpoint (UC06.4)
        \item AI dịch văn bản với context-aware translation:
        \begin{itemize}
            \item Giữ nguyên technical terms (variable names, code snippets)
            \item Dịch user story và acceptance criteria phù hợp văn hóa
            \item Maintain formatting (markdown, lists, tables)
        \end{itemize}
        \item Frontend hiển thị translated text real-time với streaming
        \item User review và có thể copy translated text
        \item User có option để apply translated text vào description
        \item Backend log translation activity
    \end{enumerate}
    \vskip 1pt
\end{minipage}
\\\hline
Các luồng sự kiện con &
\begin{minipage}{\linewidth}
    \vskip 4pt
    + Include UC06.4 (LLM API Integration).
    \vskip 1pt
\end{minipage}
\\\hline
Luồng thay thế/ngoại lệ &
\begin{minipage}{\linewidth}
    \vskip 4pt
    \textbf{\textcolor{red}{E1}} -- Same language: Hiển thị warning "Description is already in target language".

    \textbf{\textcolor{red}{A1}} -- Multiple translations: User có thể dịch sang nhiều ngôn ngữ khác nhau.

    \textbf{\textcolor{red}{A2}} -- Copy to clipboard: User có thể copy translated text để paste elsewhere.
    \vskip 1pt
\end{minipage}
\\\hline
Kết quả & Issue description được dịch sang ngôn ngữ đích với context-aware translation, giúp team đa quốc gia làm việc hiệu quả. \\\hline
\end{longtblr}

\vspace{1em}

\textbf{UC06.4 - LLM API Integration}

\begin{figure}[H]
    \centering
    \includegraphics[width=1\textwidth]{images/uc06_4_llm_integration.png}
    \caption{Sơ đồ use case chức năng LLM API Integration}
    \label{fig:uc06_4_llm_integration}
\end{figure}

\begin{longtblr}[
    caption = {Đặc tả use case UC06.4 - LLM API Integration},
    label = {tab:uc06_4},
]{
    colspec={|l|p{.7\linewidth}|}
}
\hline
\textbf{Tên chức năng} & \textbf{LLM API Integration} \\\hline
ID & UC06.4 \\\hline
Người sử dụng & System \\\hline
Mức độ cần thiết & Bắt buộc (cho AI features) \\\hline
Phân loại & Cao \\\hline
Các thành phần tham gia &
\begin{minipage}{\linewidth}
    \vskip 4pt
    + \textbf{System:} Cần gọi LLM API để xử lý AI requests. \\
    + \textbf{LLM Provider:} Xử lý prompts và trả về responses.
    \vskip 1pt
\end{minipage}
\\\hline
Mô tả tóm tắt & Backend service tích hợp với LLM providers (OpenAI, Anthropic, etc.) để xử lý các AI features. \\\hline
Trigger & Được gọi từ UC06.1, UC06.2, UC03.6 khi cần AI processing. \\\hline
Kiểu sự kiện & Internal. \\\hline
Luồng xử lý bình thường &
\begin{minipage}{\linewidth}
    \vskip 4pt
    \begin{enumerate}
        \item Backend nhận AI request từ UC gọi
        \item Backend load LLM configuration:
        \begin{itemize}
            \item API key từ environment variables
            \item Model selection (gpt-4, claude-3, etc.)
            \item Max tokens, temperature settings
        \end{itemize}
        \item Backend format prompt với:
        \begin{itemize}
            \item System prompt (role definition)
            \item User prompt (actual request)
            \item Context data (project info, issue details)
        \end{itemize}
        \item Backend gửi HTTP request đến LLM API
        \item LLM Provider xử lý request
        \item Backend nhận response (stream hoặc complete)
        \item Backend parse và validate response
        \item Backend ghi log request/response cho debugging
        \item Backend trả parsed result về UC gọi
    \end{enumerate}
    \vskip 1pt
\end{minipage}
\\\hline
Các luồng sự kiện con &
\begin{minipage}{\linewidth}
    \vskip 4pt
    + Include bởi UC06.1, UC06.2, UC03.6.
    \vskip 1pt
\end{minipage}
\\\hline
Luồng thay thế/ngoại lệ &
\begin{minipage}{\linewidth}
    \vskip 4pt
    \textbf{\textcolor{red}{E1}} -- API timeout: Retry với exponential backoff (max 3 lần).

    \textbf{\textcolor{red}{E2}} -- Rate limit exceeded: Queue request và retry sau.

    \textbf{\textcolor{red}{E3}} -- Invalid API key: Log error và notify admin.

    \textbf{\textcolor{red}{E4}} -- Malformed response: Fallback to default message.

    \textbf{\textcolor{red}{A1}} -- Streaming response: Stream từng token về frontend real-time.
    \vskip 1pt
\end{minipage}
\\\hline
Kết quả & LLM response được xử lý thành công và trả về cho feature sử dụng. \\\hline
\end{longtblr}

\vspace{1em}

\textbf{UC06.5 - AI phát hiện rủi ro Sprint}

\begin{figure}[H]
    \centering
    \includegraphics[width=1\textwidth]{images/uc06_5_ai_risk_detector.png}
    \caption{Sơ đồ use case chức năng AI phát hiện rủi ro Sprint}
    \label{fig:uc06_5_ai_risk_detector}
\end{figure}

\begin{longtblr}[
    caption = {Đặc tả use case UC06.5 - AI phát hiện rủi ro Sprint},
    label = {tab:uc06_5},
]{
    colspec={|l|p{.7\linewidth}|}
}
\hline
\textbf{Tên chức năng} & \textbf{AI phát hiện rủi ro Sprint} \\\hline
ID & UC06.5 \\\hline
Người sử dụng & Project Lead \\\hline
Mức độ cần thiết & Tùy chọn \\\hline
Phân loại & Cao \\\hline
Các thành phần tham gia &
\begin{minipage}{\linewidth}
    \vskip 4pt
    + \textbf{Project Lead:} Muốn phát hiện sớm các rủi ro trong sprint. \\
    + \textbf{Risk Detector Service:} Phát hiện risks với rule-based detection. \\
    + \textbf{AI Risk Analyzer:} Phân tích risks và đưa ra recommendations.
    \vskip 1pt
\end{minipage}
\\\hline
Mô tả tóm tắt & Hệ thống tự động phát hiện rủi ro trong sprint (overcommitment, blocked issues, unestimated work) và sử dụng AI để phân tích nguyên nhân, đưa ra recommendations và hành động tự động. \\\hline
Trigger & Project Lead nhấn "Detect Risks" hoặc hệ thống tự động chạy định kỳ mỗi ngày. \\\hline
Kiểu sự kiện & External hoặc Internal (scheduled). \\\hline
Luồng xử lý bình thường &
\begin{minipage}{\linewidth}
    \vskip 4pt
    \begin{enumerate}
        \item Hệ thống hoặc Project Lead trigger risk detection cho sprint ACTIVE
        \item Risk Detector Service phân tích sprint với rule-based detection:
        \begin{itemize}
            \item \textbf{Overcommitment:} Story points vượt quá team velocity
            \item \textbf{Blocked Issues:} Issues bị blocked quá lâu (> 3 days)
            \item \textbf{Unestimated Work:} Issues chưa có story points
        \end{itemize}
        \item Với mỗi risk detected, hệ thống tạo RiskAlert record với:
        \begin{itemize}
            \item Risk type, severity (LOW, MEDIUM, HIGH, CRITICAL)
            \item Affected issues, impact analysis
            \item Detection time, status (OPEN, ACKNOWLEDGED, RESOLVED, DISMISSED)
        \end{itemize}
        \item AI Risk Analyzer Service phân tích risks detected:
        \begin{itemize}
            \item Tạo prompt với sprint data, issues data, risk details
            \item Gọi LLM API (UC06.4) để analyze root causes
            \item AI đưa ra detailed analysis, recommendations, priority order
        \end{itemize}
        \item Hệ thống lưu AI analysis vào RiskAlert.aiAnalysis
        \item Hệ thống tạo actionable recommendations:
        \begin{itemize}
            \item Move issues to next sprint
            \item Re-estimate story points
            \item Assign blocked issues to different team members
        \end{itemize}
        \item Frontend hiển thị Risk Dashboard với:
        \begin{itemize}
            \item Risk summary (count by severity)
            \item Risk alerts list với AI analysis
            \item Recommended actions với quick apply buttons
        \end{itemize}
        \item Project Lead review risks, có thể:
        \begin{itemize}
            \item Acknowledge risk với note
            \item Resolve risk với resolution note
            \item Dismiss risk (false positive)
            \item Apply recommended action (auto-execute)
        \end{itemize}
    \end{enumerate}
    \vskip 1pt
\end{minipage}
\\\hline
Các luồng sự kiện con &
\begin{minipage}{\linewidth}
    \vskip 4pt
    + Include UC06.4 (LLM API Integration).
    \vskip 1pt
\end{minipage}
\\\hline
Luồng thay thế/ngoại lệ &
\begin{minipage}{\linewidth}
    \vskip 4pt
    \textbf{\textcolor{red}{A1}} -- No risks detected: Hiển thị "No risks detected, sprint is on track".

    \textbf{\textcolor{red}{A2}} -- Auto-apply recommendation: Project Lead có thể apply recommendation với one click (e.g., move issues to next sprint automatically).

    \textbf{\textcolor{red}{E1}} -- AI analysis failed: Hiển thị basic risk info without AI insights.
    \vskip 1pt
\end{minipage}
\\\hline
Kết quả & Risks được phát hiện sớm với AI analysis và recommendations, giúp Project Lead xử lý proactive và đảm bảo sprint success. \\\hline
\end{longtblr}

\vspace{1em}

\textbf{UC06.6 - AI tìm kiếm thông minh (Semantic Search)}

\begin{figure}[H]
    \centering
    \includegraphics[width=1\textwidth]{images/uc06_6_ai_semantic_search.png}
    \caption{Sơ đồ use case chức năng AI tìm kiếm thông minh}
    \label{fig:uc06_6_ai_semantic_search}
\end{figure}

\begin{longtblr}[
    caption = {Đặc tả use case UC06.6 - AI tìm kiếm thông minh (Semantic Search)},
    label = {tab:uc06_6},
]{
    colspec={|l|p{.7\linewidth}|}
}
\hline
\textbf{Tên chức năng} & \textbf{AI tìm kiếm thông minh (Semantic Search)} \\\hline
ID & UC06.6 \\\hline
Người sử dụng & Project Lead, Team Member, Viewer \\\hline
Mức độ cần thiết & Tùy chọn \\\hline
Phân loại & Trung bình \\\hline
Các thành phần tham gia &
\begin{minipage}{\linewidth}
    \vskip 4pt
    + \textbf{User:} Muốn tìm issues theo ý nghĩa thay vì exact keywords. \\
    + \textbf{RAG Service:} Tạo embeddings và thực hiện vector search. \\
    + \textbf{Embedding Provider:} Generate embeddings từ text (OpenAI, sentence-transformers).
    \vskip 1pt
\end{minipage}
\\\hline
Mô tả tóm tắt & Hệ thống sử dụng AI embeddings và vector search để tìm kiếm issues dựa trên ý nghĩa ngữ nghĩa (semantic meaning) thay vì chỉ exact keyword matching, giúp tìm được issues liên quan ngay cả khi từ khóa khác nhau. \\\hline
Trigger & User nhập search query và toggle "Use AI Search" trong search modal. \\\hline
Kiểu sự kiện & External. \\\hline
Luồng xử lý bình thường &
\begin{minipage}{\linewidth}
    \vskip 4pt
    \begin{enumerate}
        \item User mở Search Modal (Cmd+K hoặc Ctrl+K)
        \item User nhập query (ví dụ: "login issues", "payment bugs")
        \item User toggle "Use AI Search" ON
        \item Frontend gửi search request với useAI=true
        \item Backend RAG Service:
        \begin{itemize}
            \item Generate embedding vector cho search query
            \item Thực hiện vector similarity search trong database
            \item Tìm top N issues có embedding gần nhất (cosine similarity)
        \end{itemize}
        \item Backend trả về results với similarity score
        \item Frontend hiển thị results sorted by relevance:
        \begin{itemize}
            \item Similarity score (0-100\%) cho mỗi result
            \item Highlight relevant sections trong description
            \item Show why this issue matches (AI explanation)
        \end{itemize}
        \item User click vào issue để xem detail
        \item User có thể refine search hoặc toggle AI search OFF để dùng basic search
    \end{enumerate}
    \vskip 1pt
\end{minipage}
\\\hline
Các luồng sự kiện con &
\begin{minipage}{\linewidth}
    \vskip 4pt
    + Background process: Hệ thống tự động generate embeddings cho mọi issues khi created/updated (cron job chạy mỗi 5 phút).
    \vskip 1pt
\end{minipage}
\\\hline
Luồng thay thế/ngoại lệ &
\begin{minipage}{\linewidth}
    \vskip 4pt
    \textbf{\textcolor{red}{A1}} -- Fallback to basic search: Nếu AI search fail, tự động fallback về basic keyword search.

    \textbf{\textcolor{red}{A2}} -- Filter + AI search: User có thể combine AI search với filters (priority, type, status).

    \textbf{\textcolor{red}{E1}} -- Embeddings not ready: Hiển thị "AI search not available yet, indexing in progress".
    \vskip 1pt
\end{minipage}
\\\hline
Kết quả & User tìm được issues relevant dựa trên semantic meaning thay vì exact keywords, cải thiện search accuracy đáng kể. \\\hline
\end{longtblr}
