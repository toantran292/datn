\paragraph{UC06: Tính năng AI}
\mbox{}

\textbf{UC06.1 - AI Sprint Summary}

\begin{figure}[H]
    \centering
    \includegraphics[width=0.7\textwidth]{images/uc06_1_ai_sprint_summary.png}
    \caption{Sơ đồ use case chức năng AI Sprint Summary}
    \label{fig:uc06_1_ai_sprint_summary}
\end{figure}

\begin{longtblr}[
    caption = {Đặc tả use case UC06.1 - AI Sprint Summary},
    label = {tab:uc06_1},
]{
    colspec={|l|p{.7\linewidth}|}
}
\hline
\textbf{Tên chức năng} & \textbf{AI Sprint Summary} \\\hline
ID & UC06.1 \\\hline
Người sử dụng & Project Lead \\\hline
Mức độ cần thiết & Tùy chọn \\\hline
Phân loại & Trung bình \\\hline
Các thành phần tham gia &
\begin{minipage}{\linewidth}
    \vskip 4pt
    + \textbf{Project Lead:} Muốn có summary tự động của sprint vừa hoàn thành. \\
    + \textbf{LLM Provider:} Phân tích dữ liệu sprint và tạo summary.
    \vskip 1pt
\end{minipage}
\\\hline
Mô tả tóm tắt & Khi complete sprint (UC02.3), Project Lead có option để AI tự động tạo sprint summary dựa trên issues completed, comments, activities. \\\hline
Trigger & Project Lead chọn "Generate AI Summary" khi complete sprint (UC02.3). \\\hline
Kiểu sự kiện & External. \\\hline
Luồng xử lý bình thường &
\begin{minipage}{\linewidth}
    \vskip 4pt
    \begin{enumerate}
        \item Project Lead complete sprint và check "Generate AI Summary"
        \item Backend collect sprint data:
        \begin{itemize}
            \item Tất cả issues của sprint (completed \& incomplete)
            \item Sprint goal, start date, end date
            \item Issue comments, activities
            \item Velocity, story points completed
        \end{itemize}
        \item Backend format data thành prompt cho AI
        \item Backend gọi LLM API (UC06.4)
        \item AI analyze và generate summary với sections:
        \begin{itemize}
            \item Overview: Sprint goal achievement
            \item Completed: Danh sách issues hoàn thành
            \item Incomplete: Issues chưa hoàn thành và lý do
            \item Insights: Patterns, blockers, recommendations
            \item Velocity: So sánh với sprint trước
        \end{itemize}
        \item Backend lưu summary vào Sprint record
        \item Hiển thị summary trong sprint detail page
        \item Team members có thể xem và export summary
    \end{enumerate}
    \vskip 1pt
\end{minipage}
\\\hline
Các luồng sự kiện con &
\begin{minipage}{\linewidth}
    \vskip 4pt
    + Include UC06.4 (LLM API Integration).
    \vskip 1pt
\end{minipage}
\\\hline
Luồng thay thế/ngoại lệ &
\begin{minipage}{\linewidth}
    \vskip 4pt
    \textbf{\textcolor{red}{E1}} -- LLM API unavailable: Hiển thị lỗi "AI service temporarily unavailable, try again later".

    \textbf{\textcolor{red}{A1}} -- Edit summary: Project Lead có thể edit summary sau khi AI generate.

    \textbf{\textcolor{red}{A2}} -- Regenerate: Project Lead có thể regenerate summary nếu không hài lòng.
    \vskip 1pt
\end{minipage}
\\\hline
Kết quả & Sprint summary được tạo tự động, giúp team review sprint hiệu quả hơn. \\\hline
\end{longtblr}

\vspace{1em}

\textbf{UC06.2 - AI Issue Description Generation}

\begin{figure}[H]
    \centering
    \includegraphics[width=0.7\textwidth]{images/uc06_2_ai_description.png}
    \caption{Sơ đồ use case chức năng AI Issue Description Generation}
    \label{fig:uc06_2_ai_description}
\end{figure}

\begin{longtblr}[
    caption = {Đặc tả use case UC06.2 - AI Issue Description Generation},
    label = {tab:uc06_2},
]{
    colspec={|l|p{.7\linewidth}|}
}
\hline
\textbf{Tên chức năng} & \textbf{AI Issue Description Generation} \\\hline
ID & UC06.2 \\\hline
Người sử dụng & Project Lead, Team Member \\\hline
Mức độ cần thiết & Tùy chọn \\\hline
Phân loại & Thấp \\\hline
Các thành phần tham gia &
\begin{minipage}{\linewidth}
    \vskip 4pt
    + \textbf{User:} Muốn AI giúp generate issue description chi tiết từ title. \\
    + \textbf{LLM Provider:} Expand title thành description đầy đủ.
    \vskip 1pt
\end{minipage}
\\\hline
Mô tả tóm tắt & User có thể dùng AI để generate issue description từ issue title. AI sẽ suggest format với user story, acceptance criteria, technical notes. \\\hline
Trigger & User nhấn "Generate Description with AI" button trong issue creation form. \\\hline
Kiểu sự kiện & External. \\\hline
Luồng xử lý bình thường &
\begin{minipage}{\linewidth}
    \vskip 4pt
    \begin{enumerate}
        \item User nhập issue title (ví dụ: "Add user authentication")
        \item User nhấn "Generate Description with AI"
        \item Backend tạo prompt với:
        \begin{itemize}
            \item Issue title
            \item Issue type (STORY, TASK, BUG, EPIC)
            \item Project context (tên project, tech stack nếu có)
        \end{itemize}
        \item Backend gọi LLM API (UC06.4)
        \item AI generate description với format:
        \begin{itemize}
            \item User Story: "As a [user], I want [goal] so that [benefit]"
            \item Acceptance Criteria: Bullet points
            \item Technical Notes: Implementation hints
        \end{itemize}
        \item Frontend hiển thị generated description trong editor
        \item User có thể edit description trước khi save
        \item User save issue với description
    \end{enumerate}
    \vskip 1pt
\end{minipage}
\\\hline
Các luồng sự kiện con &
\begin{minipage}{\linewidth}
    \vskip 4pt
    + Include UC06.4 (LLM API Integration).
    \vskip 1pt
\end{minipage}
\\\hline
Luồng thay thế/ngoại lệ &
\begin{minipage}{\linewidth}
    \vskip 4pt
    \textbf{\textcolor{red}{E1}} -- Title too vague: AI suggest user provide more context.

    \textbf{\textcolor{red}{A1}} -- Regenerate: User có thể nhấn "Regenerate" để AI tạo lại với approach khác.

    \textbf{\textcolor{red}{A2}} -- Use template: User có thể chọn template có sẵn thay vì AI generate.
    \vskip 1pt
\end{minipage}
\\\hline
Kết quả & Issue description được generate tự động, tiết kiệm thời gian và đảm bảo format consistent. \\\hline
\end{longtblr}

\vspace{1em}

\textbf{UC06.3 - AI Task Breakdown}

\begin{figure}[H]
    \centering
    \includegraphics[width=0.7\textwidth]{images/uc06_3_ai_task_breakdown.png}
    \caption{Sơ đồ use case chức năng AI Task Breakdown}
    \label{fig:uc06_3_ai_task_breakdown}
\end{figure}

\begin{longtblr}[
    caption = {Đặc tả use case UC06.3 - AI Task Breakdown},
    label = {tab:uc06_3},
]{
    colspec={|l|p{.7\linewidth}|}
}
\hline
\textbf{Tên chức năng} & \textbf{AI Task Breakdown} \\\hline
ID & UC06.3 \\\hline
Người sử dụng & Project Lead, Team Member \\\hline
Mức độ cần thiết & Tùy chọn \\\hline
Phân loại & Thấp \\\hline
Các thành phần tham gia &
\begin{minipage}{\linewidth}
    \vskip 4pt
    + \textbf{User:} Muốn breakdown EPIC/STORY thành sub-tasks. \\
    + \textbf{LLM Provider:} Phân tích và suggest sub-tasks.
    \vskip 1pt
\end{minipage}
\\\hline
Mô tả tóm tắt & AI giúp breakdown issue lớn (EPIC hoặc STORY) thành các sub-tasks nhỏ hơn để dễ estimate và execute. \\\hline
Trigger & User nhấn "AI Breakdown" trong issue detail panel. \\\hline
Kiểu sự kiện & External. \\\hline
Luồng xử lý bình thường &
\begin{minipage}{\linewidth}
    \vskip 4pt
    \begin{enumerate}
        \item User mở issue detail (type = EPIC hoặc STORY)
        \item User nhấn "AI Breakdown" button
        \item Backend tạo prompt với:
        \begin{itemize}
            \item Issue title, description
            \item Issue type (EPIC or STORY)
            \item Existing sub-issues (nếu có)
        \end{itemize}
        \item Backend gọi LLM API (UC06.4)
        \item AI analyze và suggest breakdown:
        \begin{itemize}
            \item Danh sách sub-tasks với title và description
            \item Estimated story points cho mỗi sub-task
            \item Dependencies giữa các sub-tasks
        \end{itemize}
        \item Frontend hiển thị suggested sub-tasks trong modal
        \item User có thể:
        \begin{itemize}
            \item Select/deselect sub-tasks
            \item Edit title, description của mỗi sub-task
            \item Adjust story points
        \end{itemize}
        \item User nhấn "Create Sub-tasks"
        \item Backend tạo sub-issues với parentId = issue hiện tại
        \item Sub-issues hiển thị trong issue detail panel
    \end{enumerate}
    \vskip 1pt
\end{minipage}
\\\hline
Các luồng sự kiện con &
\begin{minipage}{\linewidth}
    \vskip 4pt
    + Include UC06.4 (LLM API Integration).
    \vskip 1pt
\end{minipage}
\\\hline
Luồng thay thế/ngoại lệ &
\begin{minipage}{\linewidth}
    \vskip 4pt
    \textbf{\textcolor{red}{E1}} -- Issue too small: AI suggest không cần breakdown, issue đã đủ nhỏ.

    \textbf{\textcolor{red}{A1}} -- Manual breakdown: User có thể create sub-issues manually thay vì dùng AI.

    \textbf{\textcolor{red}{A2}} -- Adjust breakdown: User có thể adjust suggestions trước khi create.
    \vskip 1pt
\end{minipage}
\\\hline
Kết quả & Issue được breakdown thành sub-tasks rõ ràng, dễ estimate và distribute cho team. \\\hline
\end{longtblr}

\vspace{1em}

\textbf{UC06.4 - LLM API Integration}

\begin{figure}[H]
    \centering
    \includegraphics[width=0.7\textwidth]{images/uc06_4_llm_integration.png}
    \caption{Sơ đồ use case chức năng LLM API Integration}
    \label{fig:uc06_4_llm_integration}
\end{figure}

\begin{longtblr}[
    caption = {Đặc tả use case UC06.4 - LLM API Integration},
    label = {tab:uc06_4},
]{
    colspec={|l|p{.7\linewidth}|}
}
\hline
\textbf{Tên chức năng} & \textbf{LLM API Integration} \\\hline
ID & UC06.4 \\\hline
Người sử dụng & System \\\hline
Mức độ cần thiết & Bắt buộc (cho AI features) \\\hline
Phân loại & Cao \\\hline
Các thành phần tham gia &
\begin{minipage}{\linewidth}
    \vskip 4pt
    + \textbf{System:} Cần gọi LLM API để xử lý AI requests. \\
    + \textbf{LLM Provider:} Xử lý prompts và trả về responses.
    \vskip 1pt
\end{minipage}
\\\hline
Mô tả tóm tắt & Backend service tích hợp với LLM providers (OpenAI, Anthropic, etc.) để xử lý các AI features. \\\hline
Trigger & Được gọi từ UC06.1, UC06.2, UC06.3 khi cần AI processing. \\\hline
Kiểu sự kiện & Internal. \\\hline
Luồng xử lý bình thường &
\begin{minipage}{\linewidth}
    \vskip 4pt
    \begin{enumerate}
        \item Backend nhận AI request từ UC gọi
        \item Backend load LLM configuration:
        \begin{itemize}
            \item API key từ environment variables
            \item Model selection (gpt-4, claude-3, etc.)
            \item Max tokens, temperature settings
        \end{itemize}
        \item Backend format prompt với:
        \begin{itemize}
            \item System prompt (role definition)
            \item User prompt (actual request)
            \item Context data (project info, issue details)
        \end{itemize}
        \item Backend gửi HTTP request đến LLM API
        \item LLM Provider xử lý request
        \item Backend nhận response (stream hoặc complete)
        \item Backend parse và validate response
        \item Backend ghi log request/response cho debugging
        \item Backend trả parsed result về UC gọi
    \end{enumerate}
    \vskip 1pt
\end{minipage}
\\\hline
Các luồng sự kiện con &
\begin{minipage}{\linewidth}
    \vskip 4pt
    + Include bởi UC06.1, UC06.2, UC06.3.
    \vskip 1pt
\end{minipage}
\\\hline
Luồng thay thế/ngoại lệ &
\begin{minipage}{\linewidth}
    \vskip 4pt
    \textbf{\textcolor{red}{E1}} -- API timeout: Retry với exponential backoff (max 3 lần).

    \textbf{\textcolor{red}{E2}} -- Rate limit exceeded: Queue request và retry sau.

    \textbf{\textcolor{red}{E3}} -- Invalid API key: Log error và notify admin.

    \textbf{\textcolor{red}{E4}} -- Malformed response: Fallback to default message.

    \textbf{\textcolor{red}{A1}} -- Streaming response: Stream từng token về frontend real-time.
    \vskip 1pt
\end{minipage}
\\\hline
Kết quả & LLM response được xử lý thành công và trả về cho feature sử dụng. \\\hline
\end{longtblr}
