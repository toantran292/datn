\paragraph{UC06: Tính năng AI}
\mbox{}

\textbf{UC06.1 - AI Sprint Summary}

\begin{figure}[H]
    \centering
    \includegraphics[width=0.7\textwidth]{images/uc06_1_ai_sprint_summary.png}
    \caption{Sơ đồ use case chức năng AI Sprint Summary}
    \label{fig:uc06_1_ai_sprint_summary}
\end{figure}

\begin{longtblr}[
    caption = {Đặc tả use case UC06.1 - AI Sprint Summary},
    label = {tab:uc06_1},
]{
    colspec={|l|p{.7\linewidth}|}
}
\hline
\textbf{Tên chức năng} & \textbf{AI Sprint Summary} \\\hline
ID & UC06.1 \\\hline
Người sử dụng & Project Lead \\\hline
Mức độ cần thiết & Tùy chọn \\\hline
Phân loại & Trung bình \\\hline
Các thành phần tham gia &
\begin{minipage}{\linewidth}
    \vskip 4pt
    + \textbf{Project Lead:} Muốn có summary tự động của sprint vừa hoàn thành. \\
    + \textbf{LLM Provider:} Phân tích dữ liệu sprint và tạo summary.
    \vskip 1pt
\end{minipage}
\\\hline
Mô tả tóm tắt & Khi complete sprint (UC02.3), Project Lead có option để AI tự động tạo sprint summary dựa trên issues completed, comments, activities. \\\hline
Trigger & Project Lead chọn "Generate AI Summary" khi complete sprint (UC02.3). \\\hline
Kiểu sự kiện & External. \\\hline
Luồng xử lý bình thường &
\begin{minipage}{\linewidth}
    \vskip 4pt
    \begin{enumerate}
        \item Project Lead complete sprint và check "Generate AI Summary"
        \item Backend collect sprint data:
        \begin{itemize}
            \item Tất cả issues của sprint (completed \& incomplete)
            \item Sprint goal, start date, end date
            \item Issue comments, activities
            \item Velocity, story points completed
        \end{itemize}
        \item Backend format data thành prompt cho AI
        \item Backend gọi LLM API (UC06.4)
        \item AI analyze và generate summary với sections:
        \begin{itemize}
            \item Overview: Sprint goal achievement
            \item Completed: Danh sách issues hoàn thành
            \item Incomplete: Issues chưa hoàn thành và lý do
            \item Insights: Patterns, blockers, recommendations
            \item Velocity: So sánh với sprint trước
        \end{itemize}
        \item Backend lưu summary vào Sprint record
        \item Hiển thị summary trong sprint detail page
        \item Team members có thể xem và export summary
    \end{enumerate}
    \vskip 1pt
\end{minipage}
\\\hline
Các luồng sự kiện con &
\begin{minipage}{\linewidth}
    \vskip 4pt
    + Include UC06.4 (LLM API Integration).
    \vskip 1pt
\end{minipage}
\\\hline
Luồng thay thế/ngoại lệ &
\begin{minipage}{\linewidth}
    \vskip 4pt
    \textbf{\textcolor{red}{E1}} -- LLM API unavailable: Hiển thị lỗi "AI service temporarily unavailable, try again later".

    \textbf{\textcolor{red}{A1}} -- Edit summary: Project Lead có thể edit summary sau khi AI generate.

    \textbf{\textcolor{red}{A2}} -- Regenerate: Project Lead có thể regenerate summary nếu không hài lòng.
    \vskip 1pt
\end{minipage}
\\\hline
Kết quả & Sprint summary được tạo tự động, giúp team review sprint hiệu quả hơn. \\\hline
\end{longtblr}

\vspace{1em}

\textbf{UC06.2 - AI tự động tinh chỉnh mô tả công việc}

\begin{figure}[H]
    \centering
    \includegraphics[width=0.7\textwidth]{images/uc06_2_ai_description.png}
    \caption{Sơ đồ use case chức năng AI tự động tinh chỉnh mô tả công việc}
    \label{fig:uc06_2_ai_description}
\end{figure}

\begin{longtblr}[
    caption = {Đặc tả use case UC06.2 - AI tự động tinh chỉnh mô tả công việc},
    label = {tab:uc06_2},
]{
    colspec={|l|p{.7\linewidth}|}
}
\hline
\textbf{Tên chức năng} & \textbf{AI tự động tinh chỉnh mô tả công việc} \\\hline
ID & UC06.2 \\\hline
Người sử dụng & Project Lead, Team Member \\\hline
Mức độ cần thiết & Tùy chọn \\\hline
Phân loại & Trung bình \\\hline
Các thành phần tham gia &
\begin{minipage}{\linewidth}
    \vskip 4pt
    + \textbf{User:} Muốn AI giúp refine và format lại issue description cho đẹp và đầy đủ hơn. \\
    + \textbf{LLM Provider:} Phân tích description hiện tại và refine theo best practices.
    \vskip 1pt
\end{minipage}
\\\hline
Mô tả tóm tắt & AI tự động phát hiện và tinh chỉnh mô tả công việc để format đẹp hơn, bổ sung thông tin thiếu, làm rõ acceptance criteria và thêm technical notes nếu cần. \\\hline
Trigger & User nhấn "Refine Description with AI" button trong issue detail hoặc issue edit form. \\\hline
Kiểu sự kiện & External. \\\hline
Luồng xử lý bình thường &
\begin{minipage}{\linewidth}
    \vskip 4pt
    \begin{enumerate}
        \item User mở issue có description chưa đầy đủ hoặc chưa format tốt
        \item User nhấn "Refine Description with AI"
        \item Backend detect mô tả cần cải thiện (thiếu sections, format không chuẩn)
        \item Backend tạo prompt với:
        \begin{itemize}
            \item Description hiện tại
            \item Issue type (STORY, TASK, BUG, EPIC)
            \item Project context, related issues
        \end{itemize}
        \item Backend gọi LLM API (UC06.4)
        \item AI refine description theo best practices:
        \begin{itemize}
            \item Format theo template chuẩn (User Story/Acceptance Criteria/Technical Notes)
            \item Bổ sung thông tin thiếu (edge cases, error handling)
            \item Làm rõ acceptance criteria thành bullet points cụ thể
            \item Thêm technical notes về implementation approach
        \end{itemize}
        \item Frontend hiển thị side-by-side comparison (current vs refined)
        \item User review changes, có thể accept/reject từng section
        \item User apply changes hoặc edit thêm trước khi save
        \item Backend log AI refinement activity
    \end{enumerate}
    \vskip 1pt
\end{minipage}
\\\hline
Các luồng sự kiện con &
\begin{minipage}{\linewidth}
    \vskip 4pt
    + Include UC06.4 (LLM API Integration).
    \vskip 1pt
\end{minipage}
\\\hline
Luồng thay thế/ngoại lệ &
\begin{minipage}{\linewidth}
    \vskip 4pt
    \textbf{\textcolor{red}{E1}} -- Description already good: AI suggest "Description looks good, no refinement needed".

    \textbf{\textcolor{red}{A1}} -- Regenerate: User có thể nhấn "Regenerate" để AI refine lại với approach khác.

    \textbf{\textcolor{red}{A2}} -- Partial accept: User có thể chọn apply chỉ một số sections (ví dụ: chỉ lấy acceptance criteria).
    \vskip 1pt
\end{minipage}
\\\hline
Kết quả & Issue description được tinh chỉnh tự động với format đẹp, đầy đủ thông tin, giúp team hiểu rõ requirements. \\\hline
\end{longtblr}


\textbf{UC06.4 - LLM API Integration}

\begin{figure}[H]
    \centering
    \includegraphics[width=0.7\textwidth]{images/uc06_4_llm_integration.png}
    \caption{Sơ đồ use case chức năng LLM API Integration}
    \label{fig:uc06_4_llm_integration}
\end{figure}

\begin{longtblr}[
    caption = {Đặc tả use case UC06.4 - LLM API Integration},
    label = {tab:uc06_4},
]{
    colspec={|l|p{.7\linewidth}|}
}
\hline
\textbf{Tên chức năng} & \textbf{LLM API Integration} \\\hline
ID & UC06.4 \\\hline
Người sử dụng & System \\\hline
Mức độ cần thiết & Bắt buộc (cho AI features) \\\hline
Phân loại & Cao \\\hline
Các thành phần tham gia &
\begin{minipage}{\linewidth}
    \vskip 4pt
    + \textbf{System:} Cần gọi LLM API để xử lý AI requests. \\
    + \textbf{LLM Provider:} Xử lý prompts và trả về responses.
    \vskip 1pt
\end{minipage}
\\\hline
Mô tả tóm tắt & Backend service tích hợp với LLM providers (OpenAI, Anthropic, etc.) để xử lý các AI features. \\\hline
Trigger & Được gọi từ UC06.1, UC06.2, UC03.6 khi cần AI processing. \\\hline
Kiểu sự kiện & Internal. \\\hline
Luồng xử lý bình thường &
\begin{minipage}{\linewidth}
    \vskip 4pt
    \begin{enumerate}
        \item Backend nhận AI request từ UC gọi
        \item Backend load LLM configuration:
        \begin{itemize}
            \item API key từ environment variables
            \item Model selection (gpt-4, claude-3, etc.)
            \item Max tokens, temperature settings
        \end{itemize}
        \item Backend format prompt với:
        \begin{itemize}
            \item System prompt (role definition)
            \item User prompt (actual request)
            \item Context data (project info, issue details)
        \end{itemize}
        \item Backend gửi HTTP request đến LLM API
        \item LLM Provider xử lý request
        \item Backend nhận response (stream hoặc complete)
        \item Backend parse và validate response
        \item Backend ghi log request/response cho debugging
        \item Backend trả parsed result về UC gọi
    \end{enumerate}
    \vskip 1pt
\end{minipage}
\\\hline
Các luồng sự kiện con &
\begin{minipage}{\linewidth}
    \vskip 4pt
    + Include bởi UC06.1, UC06.2, UC03.6.
    \vskip 1pt
\end{minipage}
\\\hline
Luồng thay thế/ngoại lệ &
\begin{minipage}{\linewidth}
    \vskip 4pt
    \textbf{\textcolor{red}{E1}} -- API timeout: Retry với exponential backoff (max 3 lần).

    \textbf{\textcolor{red}{E2}} -- Rate limit exceeded: Queue request và retry sau.

    \textbf{\textcolor{red}{E3}} -- Invalid API key: Log error và notify admin.

    \textbf{\textcolor{red}{E4}} -- Malformed response: Fallback to default message.

    \textbf{\textcolor{red}{A1}} -- Streaming response: Stream từng token về frontend real-time.
    \vskip 1pt
\end{minipage}
\\\hline
Kết quả & LLM response được xử lý thành công và trả về cho feature sử dụng. \\\hline
\end{longtblr}
