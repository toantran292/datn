\subsection{Mô tả chi tiết bài toán}

Trong bối cảnh phát triển phần mềm hiện đại, Agile và cụ thể là Scrum trở thành phương pháp quản lý dự án phổ biến nhờ khả năng linh hoạt, phản hồi nhanh với thay đổi và tăng cường cộng tác giữa các thành viên. Tuy nhiên, để triển khai Scrum hiệu quả, các nhóm phát triển cần một công cụ quản lý dự án chuyên biệt hỗ trợ toàn bộ quy trình từ lập kế hoạch sprint, quản lý backlog, theo dõi công việc (issue/task), đến phân tích hiệu suất và tạo báo cáo. Hiện nay trên thị trường có nhiều công cụ quản lý dự án như Jira, Trello, Asana, nhưng hầu hết đều có những hạn chế: chi phí cao cho các tổ chức nhỏ và vừa, giao diện phức tạp khó sử dụng cho người mới, thiếu tính linh hoạt trong tùy chỉnh quy trình làm việc, và đặc biệt là chưa tận dụng triệt để công nghệ AI để hỗ trợ người dùng trong các tác vụ lặp đi lặp lại như viết mô tả công việc, ước lượng story points hay tạo báo cáo sprint.

Để giải quyết vấn đề này, cần xây dựng một hệ thống quản lý dự án Agile hiện đại, có giao diện trực quan dễ sử dụng, hỗ trợ đầy đủ quy trình Scrum, cho phép tùy chỉnh linh hoạt theo nhu cầu của từng nhóm, và tích hợp các tính năng AI thông minh để tăng năng suất làm việc. Thách thức không chỉ nằm ở việc xây dựng các tính năng quản lý cơ bản, mà còn phải đảm bảo hệ thống có hiệu năng cao khi xử lý nhiều dự án và issue đồng thời, hỗ trợ tương tác thời gian thực (real-time updates) khi nhiều người cùng làm việc trên một sprint, cung cấp trải nghiệm người dùng mượt mà với các thao tác drag-and-drop trên board và backlog, và đặc biệt là tích hợp AI một cách tự nhiên vào quy trình làm việc mà không làm gián đoạn người dùng.

Trong phạm vi đề tài này, tác giả tập trung vào việc xây dựng \textbf{phân hệ quản lý dự án} cho hệ thống SaaS quản lý dự án Agile tích hợp AI. Phân hệ này đóng vai trò cốt lõi trong việc hỗ trợ các nhóm phát triển thực hiện quy trình Scrum, cung cấp các chức năng chính bao gồm: quản lý dự án (project configuration), quản lý sprint với vòng đời đầy đủ, quản lý issue với nhiều loại công việc (Story, Task, Bug, Epic), hệ thống trạng thái tùy chỉnh (custom workflow), bảng Kanban (board view), quản lý backlog (backlog view), bình luận và theo dõi hoạt động (comments \& activity log), phân tích và thống kê (analytics), và đặc biệt là tích hợp AI để hỗ trợ cải thiện mô tả issue, ước lượng story points và tạo tóm tắt sprint tự động. Phân hệ này được xây dựng độc lập nhưng có khả năng tích hợp với các phân hệ khác trong nền tảng như truyền thông (communication), họp trực tuyến (meeting) và nền tảng thông tin (platform services).

Về đối tượng người dùng, phân hệ quản lý dự án phục vụ các vai trò chính trong một nhóm Scrum. Nhóm đầu tiên là Project Lead hoặc Project Manager, người chịu trách nhiệm chính về dự án, có quyền tạo và cấu hình dự án với thông tin cơ bản (tên dự án, identifier, mô tả), thiết lập các thông số như default assignee, quản lý danh sách thành viên dự án và phân quyền. Project Lead có toàn quyền quản lý sprint (tạo, bắt đầu, hoàn thành sprint), tạo và cấu hình các trạng thái tùy chỉnh (custom issue status) phù hợp với quy trình làm việc của nhóm, sắp xếp lại thứ tự các trạng thái trên board, và có quyền xóa hoặc chỉnh sửa bất kỳ issue nào trong dự án. Project Lead cũng có thể xem tất cả các báo cáo và thống kê về dự án (analytics), sử dụng các tính năng AI để tạo sprint summary và đánh giá hiệu suất của nhóm.

Nhóm thứ hai là Team Member (Thành viên nhóm), bao gồm Developer, Tester, Designer và các vai trò kỹ thuật khác, người tham gia trực tiếp vào việc thực hiện các công việc trong sprint. Team Member có quyền tạo issue mới, chỉnh sửa issue do mình tạo hoặc được gán, thay đổi trạng thái issue bằng cách kéo thả trên board hoặc backlog, cập nhật các thuộc tính như mô tả, story points, ngày bắt đầu/mục tiêu, gán người thực hiện và thiết lập quan hệ cha-con giữa các issue. Team Member có thể thêm bình luận (comments) vào issue để thảo luận, đặt câu hỏi hoặc báo cáo tiến độ, xem lịch sử hoạt động (activity log) để theo dõi các thay đổi trên issue. Team Member cũng được phép sử dụng các tính năng AI như AI Issue Enhancement để cải thiện mô tả công việc, Smart Story Point Estimation để nhận gợi ý ước lượng story points, nhưng không có quyền quản lý sprint hoặc cấu hình dự án.

Nhóm thứ ba là Viewer hoặc Stakeholder, người quan tâm đến tiến độ dự án nhưng không tham gia trực tiếp vào việc thực hiện công việc (ví dụ: Product Owner, khách hàng, quản lý cấp cao). Viewer có quyền xem toàn bộ thông tin dự án bao gồm danh sách sprint, backlog, board, chi tiết issue, comments và các báo cáo phân tích, nhưng không có quyền tạo, chỉnh sửa hoặc xóa issue. Viewer có thể thêm bình luận vào issue để góp ý hoặc yêu cầu làm rõ, nhưng không thể thay đổi trạng thái hoặc các thuộc tính quan trọng của issue. Vai trò này giúp đảm bảo tính minh bạch của dự án trong khi vẫn bảo vệ dữ liệu khỏi các thay đổi không mong muốn.

Về chức năng, phân hệ quản lý dự án cung cấp các module cốt lõi sau đây. Thứ nhất là quản lý dự án (Project Management), cho phép tạo dự án mới với các thông tin cấu hình cơ bản bao gồm tên dự án, identifier duy nhất (ví dụ: "PM" cho dự án Project Management, dùng để tạo issue ID như PM-1, PM-2), mô tả dự án, project lead (người chịu trách nhiệm chính), default assignee (người được gán mặc định khi tạo issue mới). Mỗi dự án duy trì một sequence counter để tự động tăng số thứ tự cho issue, đảm bảo mỗi issue có ID duy nhất trong phạm vi dự án. Project Manager có thể cập nhật thông tin dự án, quản lý danh sách thành viên tham gia và xóa dự án khi không còn sử dụng.

Thứ hai là quản lý sprint (Sprint Management), hỗ trợ toàn bộ vòng đời của một sprint trong Scrum. Người dùng có thể tạo sprint mới với thông tin bao gồm tên sprint, mục tiêu sprint (sprint goal), ngày bắt đầu và ngày kết thúc dự kiến. Sprint có ba trạng thái: FUTURE (đang lập kế hoạch, chưa bắt đầu), ACTIVE (đang thực hiện, chỉ có một active sprint tại một thời điểm), và CLOSED (đã hoàn thành). Khi bắt đầu sprint, hệ thống chuyển trạng thái từ FUTURE sang ACTIVE và ghi nhận thời gian bắt đầu thực tế. Khi hoàn thành sprint, hệ thống chuyển sang CLOSED, ghi nhận thời gian kết thúc và có thể tự động di chuyển các issue chưa hoàn thành về backlog hoặc sang sprint tiếp theo. Giao diện sprint hiển thị tổng quan với số lượng issue theo trạng thái (TODO, IN\_PROGRESS, DONE) giúp Project Manager dễ dàng theo dõi tiến độ.

Thứ ba là quản lý issue (Issue Management), cung cấp chức năng tạo và quản lý các công việc với nhiều loại khác nhau: Story (tính năng nghiệp vụ từ góc độ người dùng), Task (công việc kỹ thuật cụ thể), Bug (lỗi cần sửa chữa), Epic (nhóm công việc lớn bao gồm nhiều Story/Task). Mỗi issue có các thuộc tính: tiêu đề (title), mô tả chi tiết (description) hỗ trợ định dạng HTML, loại công việc (type), mức độ ưu tiên (priority: LOW, MEDIUM, HIGH, CRITICAL), trạng thái (status) tham chiếu đến custom issue status, story points để ước lượng độ phức tạp, ngày bắt đầu và ngày mục tiêu, danh sách người thực hiện (assignees) lưu dạng JSON cho phép gán nhiều người, sprint ID nếu issue thuộc một sprint, parent issue ID nếu issue là subtask của một issue khác, và sequence number để tạo issue ID duy nhất. Người dùng có thể tạo, xem, cập nhật và xóa issue, thay đổi trạng thái bằng cách kéo thả trên board hoặc backlog, sắp xếp lại thứ tự issue trong sprint hoặc backlog bằng cách cập nhật sort order.

Thứ tư là hệ thống trạng thái tùy chỉnh (Custom Issue Status), cho phép mỗi dự án định nghĩa quy trình làm việc riêng. Thay vì sử dụng một bộ trạng thái cố định, hệ thống cho phép tạo các trạng thái tùy chỉnh với tên, mô tả, màu sắc (dùng để hiển thị trên UI) và thứ tự sắp xếp trên board. Ví dụ, một dự án có thể có các trạng thái: Backlog, To Do, In Progress, Code Review, Testing, Done; trong khi dự án khác có thể có: New, Analysis, Development, QA, UAT, Closed. Project Manager có thể tạo, chỉnh sửa, xóa và sắp xếp lại thứ tự các trạng thái bằng drag-and-drop. Mỗi issue được gán một status, và khi di chuyển issue giữa các cột trên board, status được cập nhật tương ứng.

Thứ năm là giao diện bảng Kanban (Board View), cung cấp cách nhìn trực quan về tiến độ công việc. Board hiển thị các cột tương ứng với các custom status của dự án, mỗi cột chứa danh sách issue có trạng thái đó. Người dùng có thể lọc issue theo sprint (hiển thị chỉ issue của sprint đang active), kéo thả issue giữa các cột để thay đổi trạng thái nhanh chóng, sắp xếp lại thứ tự issue trong cùng một cột. Mỗi issue card trên board hiển thị thông tin quan trọng: issue ID, tiêu đề, loại công việc (icon), mức độ ưu tiên (màu sắc hoặc icon), danh sách assignees (avatar), story points, và ngày mục tiêu nếu có. Board hỗ trợ tương tác real-time, khi một người di chuyển issue thì người khác cũng thấy được ngay lập tức. Project Manager cũng có thể sắp xếp lại thứ tự các cột status bằng drag-and-drop để tùy chỉnh workflow.

Thứ sáu là giao diện quản lý backlog (Backlog View), cho phép tổ chức và lập kế hoạch công việc theo sprint. Backlog View chia thành nhiều section: các active sprint (hiển thị sprint đang chạy với danh sách issue), các future sprint (sprint đã tạo nhưng chưa bắt đầu), và backlog (danh sách issue chưa được gán vào sprint nào). Mỗi section có thể mở rộng hoặc thu gọn, hiển thị tổng hợp số lượng issue theo trạng thái (TODO, IN\_PROGRESS, DONE) và tổng story points. Người dùng có thể kéo thả issue giữa các sprint và backlog để sắp xếp lại kế hoạch, tạo sprint mới và bắt đầu/hoàn thành sprint trực tiếp từ giao diện này. Backlog View tích hợp một panel chi tiết issue ở bên phải (resizable từ 360px đến 720px), cho phép xem và chỉnh sửa thông tin issue mà không cần chuyển trang.

Thứ bảy là hệ thống bình luận và theo dõi hoạt động (Comments \& Activity Log). Mỗi issue có một danh sách comments cho phép thành viên thảo luận, đặt câu hỏi, báo cáo tiến độ hoặc chia sẻ thông tin liên quan. Comments hỗ trợ HTML để hiển thị định dạng phong phú, lưu thông tin người tạo và thời gian tạo. Người tạo comment có thể chỉnh sửa hoặc xóa comment của mình. Bên cạnh đó, hệ thống tự động ghi nhận mọi thay đổi trên issue (title, description, status, priority, assignees, story points, dates, parent, sprint) vào activity log, lưu trữ cả giá trị trước và sau khi thay đổi, người thực hiện thay đổi và thời gian. Activity log tạo audit trail đầy đủ giúp theo dõi lịch sử phát triển của issue, hữu ích cho việc đánh giá tiến độ, phát hiện vấn đề và học hỏi kinh nghiệm.

Thứ tám là module phân tích và thống kê (Analytics), cung cấp các biểu đồ và báo cáo giúp đánh giá hiệu suất dự án. Hệ thống hiển thị biểu đồ Created vs Resolved theo thời gian (line chart hoặc bar chart) để theo dõi số lượng issue được tạo và hoàn thành qua các ngày/tuần/tháng, giúp phát hiện xu hướng và điều chỉnh kế hoạch. Analytics cũng cung cấp thống kê phân bố issue theo loại (Story, Task, Bug, Epic), theo mức độ ưu tiên (Low, Medium, High, Critical), và theo trạng thái hiện tại, giúp Project Manager nắm bắt bức tranh tổng thể về công việc đang và sẽ thực hiện. Các thống kê này có thể lọc theo sprint hoặc khoảng thời gian cụ thể để đánh giá hiệu suất của từng sprint.

Thứ chín là tích hợp AI vào quy trình làm việc, bao gồm ba tính năng chính. (1) AI Issue Enhancement: khi tạo issue mới, người dùng có thể sử dụng AI để tự động cải thiện mô tả công việc, đề xuất acceptance criteria (các tiêu chí chấp nhận để xác định khi nào issue được coi là hoàn thành), và kiểm tra xem thông tin trong issue đã đầy đủ chưa. Tính năng này giúp team viết issue rõ ràng hơn, giảm thiểu hiểu lầm và tăng chất lượng công việc. (2) Smart Story Point Estimation: phân tích nội dung issue (title, description, type, priority) và tham khảo dữ liệu lịch sử các issue tương tự đã hoàn thành trong dự án, đề xuất story points phù hợp. Tính năng này giúp team ước lượng chính xác hơn, đặc biệt hữu ích cho các thành viên mới hoặc khi gặp loại công việc chưa quen thuộc. (3) Sprint Summary: khi hoàn thành sprint, hệ thống tự động tạo báo cáo tổng kết sprint bao gồm các achievements chính (issue đã hoàn thành, tính năng mới, bug đã fix), các metrics quan trọng (tổng số issue, số issue hoàn thành, velocity, burndown), và release notes (danh sách thay đổi để thông báo cho stakeholder hoặc khách hàng). Tính năng này giúp tiết kiệm thời gian viết báo cáo và đảm bảo documentation được cập nhật đầy đủ. Tất cả các tính năng AI đều tích hợp với Large Language Model API (OpenAI GPT hoặc Claude API), sử dụng kỹ thuật prompt engineering để tối ưu hóa chất lượng đầu ra, và có cơ chế fallback khi AI service gặp sự cố.

Cuối cùng là giao diện người dùng (User Interface), xây dựng bằng Next.js 14 với App Router, sử dụng MobX để quản lý state phản ứng, tích hợp thư viện Pragmatic Drag and Drop của Atlassian cho tính năng kéo thả issue, và sử dụng design system tùy chỉnh để đảm bảo trải nghiệm người dùng nhất quán. Giao diện hỗ trợ responsive design, tối ưu hóa cho cả desktop và tablet, cung cấp các phím tắt (keyboard shortcuts) để tăng tốc độ làm việc, và tích hợp React Query để quản lý cache và đồng bộ dữ liệu với server một cách hiệu quả.

Về kiến trúc hệ thống, phân hệ quản lý dự án được thiết kế theo kiến trúc microservice với sự phân tách rõ ràng giữa frontend và backend. Backend Service (PM Service) được xây dựng bằng NestJS, cung cấp các RESTful API cho tất cả các chức năng quản lý dự án (projects, sprints, issues, issue statuses, comments, activity, analytics). PM Service sử dụng Prisma ORM để tương tác với cơ sở dữ liệu PostgreSQL, đảm bảo type safety và dễ dàng quản lý schema thông qua migration. Database schema được thiết kế với các bảng chính: Project (lưu thông tin dự án), Sprint (quản lý sprint lifecycle), IssueStatus (định nghĩa custom statuses), Issue (lưu trữ toàn bộ thông tin issue với assignees dạng JSON), IssueComment (quản lý bình luận), IssueActivity (ghi nhận lịch sử thay đổi với before/after values). Các bảng có quan hệ rõ ràng với foreign keys và cascading deletes để đảm bảo tính toàn vẹn dữ liệu. PM Service expose Swagger/OpenAPI documentation cho tất cả các endpoints, giúp dễ dàng tích hợp và testing.

Frontend Application (PM Web) được xây dựng bằng Next.js 14 với App Router, tận dụng server components và client components một cách hợp lý để tối ưu hiệu năng. Ứng dụng sử dụng MobX cho state management, tạo các store (IssueStore, SprintStore, ProjectStore, IssueStatusStore) để quản lý dữ liệu và đồng bộ với server. Khi có thay đổi từ API, store tự động cập nhật và trigger re-render các component liên quan. Ứng dụng tích hợp TanStack React Query và SWR để fetch data, cache và revalidate, giảm số lượng API calls không cần thiết. Giao diện sử dụng custom design system với các components tái sử dụng (Button, Input, Modal, Dropdown), tích hợp Pragmatic Drag and Drop của Atlassian cho các thao tác kéo thả issue trên board và backlog, hỗ trợ accessibility (ARIA attributes, keyboard navigation). Ứng dụng cũng sử dụng React Hook Form để quản lý form (create/edit issue, create sprint), validation với class-validator, và Recharts để hiển thị biểu đồ analytics.

PM Service và PM Web giao tiếp qua HTTP/REST API, với authentication được xử lý bởi phân hệ nền tảng (platform services). Khi người dùng đăng nhập, họ nhận được JWT token từ account service, token này được gửi kèm trong mỗi request đến PM Service qua Authorization header. PM Service xác thực token và trích xuất thông tin organization (workspace) từ token hoặc từ signed header (HMAC) do edge service cung cấp, đảm bảo mỗi request chỉ truy cập được dữ liệu của organization tương ứng (tenant isolation). Đối với các tính năng AI, PM Service gọi đến AI Service (một service riêng trong nền tảng) hoặc trực tiếp gọi OpenAI/Claude API với API key được quản lý an toàn trong environment variables. PM Service implement retry logic và error handling cho các API call bên ngoài, đảm bảo hệ thống vẫn hoạt động được ngay cả khi AI service tạm thời không khả dụng.

Tóm lại, bài toán cần giải quyết là xây dựng một phân hệ quản lý dự án đầy đủ, hỗ trợ toàn bộ quy trình Scrum từ lập kế hoạch sprint, quản lý backlog, theo dõi công việc trên board Kanban, đến phân tích hiệu suất và tạo báo cáo. Hệ thống cần có giao diện trực quan dễ sử dụng, hiệu năng cao, hỗ trợ tùy chỉnh linh hoạt (custom statuses, custom workflows), và đặc biệt tích hợp các tính năng AI thông minh (AI Issue Enhancement, Smart Story Point Estimation, Sprint Summary) để tăng năng suất làm việc và giảm thiểu các tác vụ lặp đi lặp lại. Phân hệ được thiết kế với kiến trúc rõ ràng, có khả năng mở rộng, dễ bảo trì và có thể tích hợp với các phân hệ khác trong nền tảng SaaS tổng thể.
