% ==================== NHÓM ACTIVITY TRACKING ====================

\subsubsection{Chức năng xem Activity Feed}

\textbf{Mục đích:} Cho phép Project Members xem timeline của tất cả activities trong project với filters theo action type, user, date range để track project progress và collaboration.

\textbf{Dữ liệu được dùng:}

\begin{longtblr}[
  caption = {Bảng dữ liệu được dùng cho chức năng xem Activity Feed},
  label = {tab:func_view_activity}
]{
  width=\linewidth, rowhead=2, hlines, vlines,
  colspec={X[0.5,c]X[2,l]X[1,c]X[1,c]X[1,c]X[1,c]},
  rows={m},
  row{1-2}={font=\bfseries, c, bg=gray9}
}
\SetCell[r=2]{} STT & \SetCell[r=2]{} Tên bảng & \SetCell[c=4]{} Phương thức & & & \\
& & Thêm & Sửa & Xóa & Truy vấn \\
1 & Activity & & & & X \\
2 & Project & & & & X \\
3 & ProjectMember & & & & X \\
\end{longtblr}

\textbf{Xử lý:}

\begin{figure}[H]
\centering
\includegraphics[width=0.6\textwidth]{images/act_view_activity.png}
\caption{Sơ đồ hoạt động của chức năng xem Activity Feed}
\label{fig:act_view_activity}
\end{figure}

\subsubsection{Chức năng ghi Activity Log}

\textbf{Mục đích:} Hệ thống tự động ghi lại activities khi có actions xảy ra trong project: issue\_created, status\_changed, sprint\_started, comment\_added, issue\_assigned, sprint\_completed với before/after changes tracking.

\textbf{Dữ liệu được dùng:}

\begin{longtblr}[
  caption = {Bảng dữ liệu được dùng cho chức năng ghi Activity Log},
  label = {tab:func_log_activity}
]{
  width=\linewidth, rowhead=2, hlines, vlines,
  colspec={X[0.5,c]X[2,l]X[1,c]X[1,c]X[1,c]X[1,c]},
  rows={m},
  row{1-2}={font=\bfseries, c, bg=gray9}
}
\SetCell[r=2]{} STT & \SetCell[r=2]{} Tên bảng & \SetCell[c=4]{} Phương thức & & & \\
& & Thêm & Sửa & Xóa & Truy vấn \\
1 & Activity & X & & & \\
2 & Project & & & & X \\
3 & Issue & & & & X \\
4 & Sprint & & & & X \\
\end{longtblr}

\textbf{Xử lý:}

\begin{figure}[H]
\centering
\includegraphics[width=0.8\textwidth]{images/act_log_activity.png}
\caption{Sơ đồ hoạt động của chức năng ghi Activity Log}
\label{fig:act_log_activity}
\end{figure}

\subsubsection{Chức năng WebSocket Real-time Broadcast}

\textbf{Mục đích:} Hệ thống broadcast activities real-time đến tất cả project members qua WebSocket connections trong project room để update Board view, Activity Feed ngay lập tức khi có changes.

\textbf{Dữ liệu được dùng:}

\begin{longtblr}[
  caption = {Bảng dữ liệu được dùng cho chức năng WebSocket Broadcast},
  label = {tab:func_websocket_broadcast}
]{
  width=\linewidth, rowhead=2, hlines, vlines,
  colspec={X[0.5,c]X[2,l]X[1,c]X[1,c]X[1,c]X[1,c]},
  rows={m},
  row{1-2}={font=\bfseries, c, bg=gray9}
}
\SetCell[r=2]{} STT & \SetCell[r=2]{} Tên bảng & \SetCell[c=4]{} Phương thức & & & \\
& & Thêm & Sửa & Xóa & Truy vấn \\
1 & Activity & & & & X \\
2 & ProjectMember & & & & X \\
3 & Project & & & & X \\
\end{longtblr}

\textbf{Xử lý:}

\begin{figure}[H]
\centering
\includegraphics[width=1\textwidth]{images/act_websocket_broadcast.png}
\caption{Sơ đồ hoạt động của chức năng WebSocket Real-time Broadcast}
\label{fig:act_websocket_broadcast}
\end{figure}
