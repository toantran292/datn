% ==================== NHÓM ISSUE MANAGEMENT ====================

\subsubsection{Chức năng tạo Issue}

\textbf{Mục đích:} Cho phép Team Members tạo issue mới (STORY, TASK, BUG, EPIC) trong project với title, description, priority, assignee và optional AI-generated description. Khi tạo issue, Team Member điền các thông tin bắt buộc như title và type, sau đó có thể chọn thêm các thông tin optional như description (hoặc dùng AI generate), priority level (LOW, MEDIUM, HIGH, CRITICAL), assignee từ danh sách project members, và story points estimate cho planning. Mỗi issue được tự động assign một unique key theo format PROJECT\_KEY-NUMBER (ví dụ: PROJ-123) để dễ dàng reference trong discussions và commits. Issue description hỗ trợ Markdown format với real-time preview, cho phép Team Members format text với headers, lists, code blocks, và links. Sau khi tạo, issue được đặt ở trạng thái mặc định (thường là TODO hoặc BACKLOG tùy theo custom status configuration) và xuất hiện trên Board view hoặc Backlog. Hệ thống tự động ghi activity log để track việc tạo issue mới, và gửi notification cho relevant members nếu có assignee hoặc watchers. Team Members cũng có thể attach files, link related issues, và set due date ngay khi tạo để cung cấp đầy đủ context cho team.

\textbf{Dữ liệu được dùng:}

\begin{longtblr}[
  caption = {Bảng dữ liệu được dùng cho chức năng tạo Issue},
  label = {tab:func_create_issue}
]{
  width=\linewidth, rowhead=2, hlines, vlines,
  colspec={X[0.5,c]X[2,l]X[1,c]X[1,c]X[1,c]X[1,c]},
  rows={m},
  row{1-2}={font=\bfseries, c, bg=gray9}
}
\SetCell[r=2]{} STT & \SetCell[r=2]{} Tên bảng & \SetCell[c=4]{} Phương thức & & & \\
& & Thêm & Sửa & Xóa & Truy vấn \\
1 & Issue & X & & & X \\
2 & Project & & & & X \\
3 & CustomStatus & & & & X \\
4 & ProjectMember & & & & X \\
5 & Activity & X & & & \\
6 & AuditLog & X & & & \\
\end{longtblr}

\textbf{Xử lý:}

\begin{figure}[H]
\centering
\includegraphics[width=0.7\textwidth]{images/act_create_issue.png}
\caption{Sơ đồ hoạt động của chức năng tạo Issue}
\label{fig:act_create_issue}
\end{figure}

\subsubsection{Chức năng Update Issue Status (Drag-and-drop)}

\textbf{Mục đích:} Cho phép Team Members drag-and-drop issues giữa các custom status columns trên Board view, update status và sortOrder real-time với optimistic UI updates và WebSocket broadcast.

\textbf{Dữ liệu được dùng:}

\begin{longtblr}[
  caption = {Bảng dữ liệu được dùng cho chức năng Update Issue Status},
  label = {tab:func_update_issue_status}
]{
  width=\linewidth, rowhead=2, hlines, vlines,
  colspec={X[0.5,c]X[2,l]X[1,c]X[1,c]X[1,c]X[1,c]},
  rows={m},
  row{1-2}={font=\bfseries, c, bg=gray9}
}
\SetCell[r=2]{} STT & \SetCell[r=2]{} Tên bảng & \SetCell[c=4]{} Phương thức & & & \\
& & Thêm & Sửa & Xóa & Truy vấn \\
1 & Issue & & X & & X \\
2 & CustomStatus & & & & X \\
3 & ProjectMember & & & & X \\
4 & Activity & X & & & \\
5 & AuditLog & X & & & \\
\end{longtblr}

\textbf{Xử lý:}

\begin{figure}[H]
\centering
\includegraphics[width=0.6\textwidth]{images/act_drag_drop_issue.png}
\caption{Sơ đồ hoạt động của chức năng Update Issue Status (Drag-and-drop)}
\label{fig:act_drag_drop_issue}
\end{figure}

\subsubsection{Chức năng Generate AI Issue Description}

\textbf{Mục đích:} Cho phép Team Members request AI-generated description từ issue title, LLM Provider analyze title và generate detailed description trong Markdown format với acceptance criteria và technical notes.

\textbf{Dữ liệu được dùng:}

\begin{longtblr}[
  caption = {Bảng dữ liệu được dùng cho chức năng Generate AI Issue Description},
  label = {tab:func_generate_issue_description}
]{
  width=\linewidth, rowhead=2, hlines, vlines,
  colspec={X[0.5,c]X[2,l]X[1,c]X[1,c]X[1,c]X[1,c]},
  rows={m},
  row{1-2}={font=\bfseries, c, bg=gray9}
}
\SetCell[r=2]{} STT & \SetCell[r=2]{} Tên bảng & \SetCell[c=4]{} Phương thức & & & \\
& & Thêm & Sửa & Xóa & Truy vấn \\
1 & Issue & & X & & X \\
2 & Project & & & & X \\
\end{longtblr}

\textbf{Xử lý:}

\begin{figure}[H]
\centering
\includegraphics[width=0.65\textwidth]{images/act_generate_issue_description.png}
\caption{Sơ đồ hoạt động của chức năng Generate AI Issue Description}
\label{fig:act_generate_issue_description}
\end{figure}

\subsubsection{Chức năng Break Down EPIC (Task Breakdown)}

\textbf{Mục đích:} Cho phép Project Lead request AI-powered task breakdown cho EPIC issues, LLM analyze EPIC description và generate sub-tasks với titles, descriptions, story points estimates.

\textbf{Dữ liệu được dùng:}

\begin{longtblr}[
  caption = {Bảng dữ liệu được dùng cho chức năng Break Down EPIC},
  label = {tab:func_breakdown_epic}
]{
  width=\linewidth, rowhead=2, hlines, vlines,
  colspec={X[0.5,c]X[2,l]X[1,c]X[1,c]X[1,c]X[1,c]},
  rows={m},
  row{1-2}={font=\bfseries, c, bg=gray9}
}
\SetCell[r=2]{} STT & \SetCell[r=2]{} Tên bảng & \SetCell[c=4]{} Phương thức & & & \\
& & Thêm & Sửa & Xóa & Truy vấn \\
1 & Issue & X & & & X \\
2 & Project & & & & X \\
3 & Activity & X & & & \\
\end{longtblr}

\textbf{Xử lý:}

\begin{figure}[H]
\centering
\includegraphics[width=0.55\textwidth]{images/act_breakdown_epic.png}
\caption{Sơ đồ hoạt động của chức năng Break Down EPIC}
\label{fig:act_breakdown_epic}
\end{figure}
