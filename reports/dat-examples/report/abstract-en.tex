\thispagestyle{empty}
\setsection{ABSTRACT}

\sloppy

\noindent Context: In Agile/Scrum software development environments, effective project management requires the ability to track sprints, issues, and work progress visually. Current project management tools often lack flexibility in customizing workflows to meet specific team needs and have not yet leveraged the power of artificial intelligence (AI) to automate time-consuming tasks such as sprint retrospective summaries, detailed issue descriptions generation, and EPIC decomposition into sub-tasks. Therefore, an AI-integrated SaaS project management platform is needed to help teams work more efficiently with custom workflows and AI-assisted features.

\noindent Objective: This thesis builds an AI-integrated Agile project management SaaS platform with core functionalities: project management with custom status workflows, sprint lifecycle management (FUTURE-ACTIVE-CLOSED), Board/Backlog views with drag-and-drop, issue tracking (STORY/TASK/BUG/EPIC), project-level access control (Project Lead/Team Member/Viewer), and three main AI features: (1) AI Sprint Summary that automatically generates retrospectives upon sprint completion, (2) AI Issue Description Generation that creates detailed descriptions from titles, and (3) AI Task Breakdown that decomposes EPIC issues into sub-tasks with story points estimates. The system supports real-time collaboration via WebSocket and integrates multiple LLM providers (OpenAI GPT-4, Anthropic Claude, Google Gemini).

\noindent Methodology: The system is built on a microservices architecture with an Account Service (Spring Boot) handling authentication via JWT and Google OAuth 2.0, and a PM Service (NestJS + Prisma ORM) managing projects, sprints, issues, custom statuses with WebSocket Gateway for real-time updates. The frontend uses Next.js 14 with App Router, MobX for state management, and DnD Kit for drag-and-drop functionality on the Board view. PostgreSQL database with composite indexes (project\_id, sprint\_id, status\_id) optimizes Board queries, while Redis caches frequently accessed data and supports WebSocket Pub/Sub. AI integration is implemented through LangChain.js with structured output using Zod schemas. Issue attachments are stored on MinIO (development) and S3 (production) with path format org-id/project-id/issue-id/filename. Nginx API Gateway handles SSL termination, load balancing, and request routing.

\noindent Results: The completed product enables Organization Owners to create and manage multiple projects, while Project Leads manage sprints with FUTURE → ACTIVE → CLOSED lifecycle (only one ACTIVE sprint per project), configure custom statuses (TODO/IN\_PROGRESS/DONE categories), and manage project members with roles. Team Members can create issues with types (STORY, TASK, BUG, EPIC), drag-and-drop issues between custom status columns on the Board view with real-time synchronization, assign issues, add comments with mentions support, and upload attachments (screenshots, documents, logs). Three AI features have been successfully integrated: AI Sprint Summary generates structured retrospectives (overview, achievements, issues found, recommendations, metrics) when Project Leads complete sprints; AI Issue Description Generation creates detailed Markdown descriptions with acceptance criteria from issue titles; AI Task Breakdown decomposes EPICs into actionable sub-tasks with story points estimates. Board view with drag-and-drop and Backlog view help teams effectively visualize workflow progress and prioritize issues.

\noindent Conclusion: The AI-integrated Agile project management SaaS platform has been successfully built with comprehensive core functionalities and three prominent AI features. The system helps software development teams, product teams, startups, and SMEs adopt Agile/Scrum effectively with custom workflow flexibility, real-time collaboration, and AI automation for time-consuming tasks. With a scalable microservices architecture and multi-tenant design, the platform can serve multiple organizations with complete data isolation. Future development directions include: Git repository integration (GitHub, GitLab, Bitbucket) to link commits with issues, advanced Analytics dashboard with burndown charts and velocity trends, mobile apps (React Native/Flutter) for on-the-go management, integration with Slack/Teams/Discord for notifications, and expanded AI capabilities with Code Review Integration, Risk Prediction, and Story Points Estimation.

\fussy
