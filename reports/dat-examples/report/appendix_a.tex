\phantomsection
\setsection{Phụ lục A. Hướng dẫn cài đặt và sử dụng phần mềm}

\newcounter{appendixstep}
\setcounter{appendixstep}{0}
\renewcommand{\theappendixstep}{\arabic{appendixstep}}

\refstepcounter{appendixstep}
\subsubsection*{\theappendixstep. Yêu cầu hệ thống}

\textbf{Phần cứng tối thiểu:}
\begin{itemize}
    \item Vi xử lý: Intel Core i5 hoặc tương đương
    \item RAM: 8GB (khuyến nghị 16GB cho môi trường development)
    \item Ổ cứng: SSD 256GB trở lên (ít nhất 15GB dung lượng trống)
    \item Kết nối mạng: Băng thông ổn định để tải dependencies và Docker images
\end{itemize}

\textbf{Phần mềm yêu cầu:}
\begin{itemize}
    \item Hệ điều hành: Windows 10/11, macOS 10.15+, hoặc Linux (Ubuntu 20.04+)
    \item Node.js: phiên bản 20.x (theo file \texttt{.nvmrc})
    \item pnpm: phiên bản 9.x+ (package manager)
    \item Docker: phiên bản 20.10+ và Docker Compose v2
    \item Git: phiên bản 2.30+
    \item Trình duyệt web: Chrome, Firefox, hoặc Edge (phiên bản mới nhất)
\end{itemize}

\refstepcounter{appendixstep}
\subsubsection*{\theappendixstep. Cài đặt các công cụ cần thiết}

\textbf{1. Cài đặt Node.js}

Hệ thống sử dụng Node.js phiên bản 20.x. Khuyến nghị sử dụng Node Version Manager (nvm) để quản lý phiên bản Node.js.

\textbf{Trên macOS/Linux:}
\begin{lstlisting}
# Cai dat nvm
curl -o- https://raw.githubusercontent.com/nvm-sh/nvm/v0.39.0/install.sh | bash
# Khoi dong lai terminal, sau do:
nvm install 20 && nvm use 20
\end{lstlisting}

\textbf{Trên Windows:} Tải và cài đặt nvm-windows từ: \url{https://github.com/coreybutler/nvm-windows}, sau đó chạy lệnh tương tự.

\textbf{2. Cài đặt pnpm và Docker}
\begin{lstlisting}
# Cai dat pnpm
npm install -g pnpm@9
\end{lstlisting}
\newpage
Docker: Tải Docker Desktop từ \url{https://www.docker.com/products/docker-desktop} (Windows/macOS) hoặc cài qua apt trên Ubuntu:
\begin{lstlisting}
sudo apt-get update && sudo apt-get install docker.io docker-compose-v2
sudo usermod -aG docker $USER
\end{lstlisting}

\refstepcounter{appendixstep}
\subsubsection*{\theappendixstep. Clone và cấu hình môi trường}

\begin{lstlisting}
# Clone repository
git clone <repository-url> pm-system && cd pm-system

# Cai dat dependencies
pnpm install
\end{lstlisting}

\textbf{Cấu hình biến môi trường Docker} - Tạo file \texttt{.env} cho infrastructure services:
\begin{lstlisting}
# PostgreSQL
POSTGRES_USER=pm_user
POSTGRES_PASSWORD=<mat-khau-manh>
POSTGRES_DB=pm_database
POSTGRES_PORT=5432

# Redis
REDIS_PORT=6379
REDIS_PASSWORD=<mat-khau-redis>

# MinIO S3 Storage
MINIO_ROOT_USER=minioadmin
MINIO_ROOT_PASSWORD=<mat-khau-manh>
MINIO_API_PORT=9000
MINIO_CONSOLE_PORT=9001
\end{lstlisting}

\textbf{Cấu hình PM Service} - Tạo file \texttt{.env} cho PM Service (NestJS):
\begin{lstlisting}
# Database
DATABASE_URL=postgresql://pm_user:<password>@localhost:5432/pm_database

# Redis
REDIS_HOST=localhost
REDIS_PORT=6379
REDIS_PASSWORD=<redis-password>

# MinIO/S3
S3_ENDPOINT=http://localhost:9000
S3_ACCESS_KEY=minioadmin
S3_SECRET_KEY=<minio-password>
S3_BUCKET=pm-attachments

# LLM Providers
OPENAI_API_KEY=<your-openai-key>
ANTHROPIC_API_KEY=<your-anthropic-key>
GOOGLE_AI_API_KEY=<your-google-key>

# JWT Secret
JWT_SECRET=<your-jwt-secret-key>
\end{lstlisting}

\refstepcounter{appendixstep}
\subsubsection*{\theappendixstep. Khởi động Infrastructure Services}

\begin{lstlisting}
# Khoi dong PostgreSQL, Redis, MinIO
docker compose up -d

# Kiem tra trang thai
docker compose ps

# Xem logs
docker compose logs -f
\end{lstlisting}

\textbf{Kiểm tra health:}
\begin{lstlisting}
docker exec pm_postgres pg_isready -U pm_user   # PostgreSQL
docker exec pm_redis redis-cli PING             # Redis
curl http://localhost:9000/minio/health/live    # MinIO
\end{lstlisting}

\refstepcounter{appendixstep}
\subsubsection*{\theappendixstep. Khởi tạo Database và chạy ứng dụng}

\begin{lstlisting}
# Chay Prisma migrations cho PM Service
cd services/pm
pnpm prisma generate
pnpm prisma migrate dev

# (Tuy chon) Seed du lieu mau
pnpm prisma db seed
\end{lstlisting}

\textbf{Chạy Development:}
\begin{lstlisting}
# Chay PM Service (NestJS backend)
cd services/pm
pnpm start:dev

# Trong terminal khac, chay pm-web (Next.js frontend)
cd apps/pm-web
pnpm dev
\end{lstlisting}

\textbf{Các địa chỉ truy cập:}
\begin{itemize}
    \item PM Web (Next.js): \url{http://localhost:3000}
    \item PM Service API: \url{http://localhost:4000}
    \item PM Service API Docs: \url{http://localhost:4000/api}
    \item MinIO Console: \url{http://localhost:9001}
    \item PostgreSQL: \texttt{localhost:5432}
    \item Redis: \texttt{localhost:6379}
\end{itemize}

\refstepcounter{appendixstep}
\subsubsection*{\theappendixstep. Sử dụng các chức năng chính}

\textbf{1. Tạo Project mới}
\begin{itemize}
    \item Đăng nhập vào hệ thống
    \item Click "Create Project" trên trang Projects
    \item Nhập tên project, key (ví dụ: PROJ), và mô tả
    \item Chọn Project Lead và default assignee
    \item Click "Create" - Bạn sẽ được redirect đến Board view
\end{itemize}

\textbf{2. Cấu hình Custom Status Workflow}
\begin{itemize}
    \item Vào Project Settings > Issue Statuses
    \item Click "Add Status" để tạo status mới (ví dụ: To Do, In Progress, Code Review, Done)
    \item Drag-and-drop để reorder status columns
    \item Mỗi status có thể chọn color và category (TODO/IN\_PROGRESS/DONE)
\end{itemize}

\textbf{3. Quản lý Sprint}
\begin{itemize}
    \item Vào Backlog view
    \item Click "Create Sprint" để tạo sprint mới
    \item Drag-and-drop issues từ backlog vào sprint
    \item Click "Start Sprint" để bắt đầu (sprint chuyển sang ACTIVE)
    \item Sau khi hoàn thành công việc, click "Complete Sprint"
    \item Chọn LLM provider và enable "Generate AI Summary" để nhận retrospective
\end{itemize}

\textbf{4. Quản lý Issues trên Board}
\begin{itemize}
    \item Click "Create Issue" để tạo issue mới (STORY/TASK/BUG/EPIC)
    \item Drag-and-drop issues giữa các status columns
    \item Click vào issue để xem chi tiết, thêm comments, upload attachments
    \item Sử dụng "Refine Description with AI" để improve mô tả
    \item Sử dụng "Auto-generate Tasks" để AI tạo sub-tasks cho EPIC
\end{itemize}

\textbf{5. Sử dụng AI Features}
\begin{itemize}
    \item \textbf{AI Sprint Summary}: Tự động generate khi complete sprint
    \item \textbf{AI Refine Description}: Click nút "sparkle" trong issue editor
    \item \textbf{AI Task Breakdown}: Nhập feature description, AI tạo danh sách tasks
\end{itemize}

\refstepcounter{appendixstep}
\subsubsection*{\theappendixstep. Build Production}

\begin{lstlisting}
# Build PM Service (NestJS)
cd services/pm
pnpm build
pnpm start:prod

# Build pm-web (Next.js)
cd apps/pm-web
pnpm build
pnpm start
\end{lstlisting}

\refstepcounter{appendixstep}
\subsubsection*{\theappendixstep. Testing và Debugging}

\begin{lstlisting}
# Unit tests cho PM Service
cd services/pm
pnpm test
pnpm test:cov    # Voi coverage

# E2E tests
pnpm test:e2e

# Lint va TypeScript check
pnpm lint
pnpm typecheck

# Kiem tra Prisma schema
pnpm prisma validate
pnpm prisma format
\end{lstlisting}

\refstepcounter{appendixstep}
\subsubsection*{\theappendixstep. Troubleshooting}

\textbf{1. Port đã được sử dụng:}
\begin{lstlisting}
# MacOS/Linux
lsof -i :3000        # Kiem tra port 3000
lsof -i :4000        # Kiem tra port 4000
kill -9 <PID>        # Kill process

# Windows
netstat -ano | findstr :3000
taskkill /PID <PID> /F
\end{lstlisting}

\textbf{2. Docker container không healthy:}
\begin{lstlisting}
docker compose logs pm_postgres    # Xem logs PostgreSQL
docker compose restart pm_postgres # Restart container
docker compose down && docker compose up -d  # Restart tat ca
\end{lstlisting}

\textbf{3. Prisma connection errors:}
\begin{lstlisting}
# Kiem tra DATABASE_URL trong .env
# Reset database
pnpm prisma migrate reset
pnpm prisma generate
pnpm prisma migrate dev
\end{lstlisting}

\textbf{4. Module not found sau khi install:}
\begin{lstlisting}
rm -rf node_modules pnpm-lock.yaml
pnpm install
pnpm prisma generate
\end{lstlisting}

\textbf{5. WebSocket connection failed:}
\begin{itemize}
    \item Kiểm tra PM Service đang chạy trên port 4000
    \item Kiểm tra CORS configuration trong PM Service
    \item Kiểm tra browser console cho WebSocket errors
\end{itemize}

\textbf{6. AI features không hoạt động:}
\begin{itemize}
    \item Kiểm tra API keys trong \texttt{.env}: OPENAI\_API\_KEY, ANTHROPIC\_API\_KEY
    \item Kiểm tra network logs để xem API requests đến LLM providers
    \item Kiểm tra PM Service logs cho errors từ LangChain.js
\end{itemize}

\refstepcounter{appendixstep}
\subsubsection*{\theappendixstep. Clean Up}

\begin{lstlisting}
# Dung services
docker compose down

# Xoa volumes (MAT DU LIEU!)
docker compose down -v

# Xoa node_modules va build artifacts
pnpm clean
rm -rf node_modules
\end{lstlisting}

\refstepcounter{appendixstep}
\subsubsection*{\theappendixstep. Cấu trúc thư mục PM subsystem}

\begin{lstlisting}
pm-system/
|-- apps/
|   |-- pm-web/              # PM Frontend (Next.js 14)
|       |-- src/
|           |-- components/  # Board, Backlog, IssueCard components
|           |-- stores/      # MobX stores
|           |-- hooks/       # useDragDrop, useRealtime
|           |-- services/    # API clients
|-- services/
|   |-- pm/                  # PM Service (NestJS)
|       |-- src/
|           |-- modules/
|               |-- project/      # ProjectModule
|               |-- sprint/       # SprintModule
|               |-- issue/        # IssueModule
|               |-- issue-status/ # IssueStatusModule
|               |-- comment/      # CommentModule
|               |-- activity/     # ActivityModule
|               |-- ai/           # AIModule (LangChain.js)
|           |-- prisma/
|               |-- schema.prisma # Database schema
|               |-- migrations/   # DB migrations
|-- docker-compose.yml       # Infrastructure services
|-- package.json
|-- pnpm-workspace.yaml
\end{lstlisting}

\refstepcounter{appendixstep}
\subsubsection*{\theappendixstep. Kết luận}

Hướng dẫn này cung cấp các bước cơ bản để cài đặt và chạy phân hệ quản lý dự án ở môi trường development. Đối với môi trường production, cần có thêm:
\begin{itemize}
    \item SSL/TLS certificates với Let's Encrypt hoặc certificate provider
    \item Reverse proxy với Nginx hoặc Traefik
    \item Monitoring với Prometheus + Grafana cho metrics
    \item Logging centralized với ELK Stack hoặc Loki
    \item Database backup automation và disaster recovery plan
    \item CI/CD pipeline với GitHub Actions hoặc GitLab CI
    \item Container orchestration với Kubernetes cho scalability
    \item Rate limiting và DDoS protection
\end{itemize}
