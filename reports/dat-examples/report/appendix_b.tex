% Biến kiểm soát
\newif\iflandscapepage
\landscapepagefalse

\setlength{\TPHorizModule}{1cm}
\setlength{\TPVertModule}{1cm}

\fancypagestyle{lscape}{%
    \fancyhf{}
    \renewcommand{\headrulewidth}{0pt}
    \renewcommand{\footrulewidth}{0pt}
    \fancyhead[C]{%
        % Header bên trái trang (trên của landscape) - xoay 90 độ
        \begin{textblock}{1}(1,1)
            \rotatebox{90}{%
                \parbox{27.5cm}{%
                    \fontsize{10pt}{12pt}\selectfont\itshape
                    Xây dựng Nền tảng SaaS tích hợp AI nhằm Thống nhất Quản lý Dự án Agile - Phân hệ quản lý dự án
                    \vspace{2pt}
                    \hrule
                }%
            }
        \end{textblock}
        % Footer bên phải trang (dưới của landscape) - xoay 90 độ  
        \begin{textblock}{1}(20,1)
            \rotatebox{90}{%
                \parbox{27.5cm}{%
                    \hrule
                    \vspace{2pt}
                    Nguyễn Tuấn Đạt - B2203499 \hfill Trang \thepage
                }%
            }
        \end{textblock}
    }
}

\begin{landscape}
\pagestyle{lscape}

\phantomsection
\setsection{Phụ lục B: Từ điển dữ liệu}

Phần này trình bày chi tiết mô tả dữ liệu cho các bảng của sơ đồ lớp của phân hệ quản lý dự án được trình bày ở Phần II - Chương III - Thiết kế cơ sở dữ liệu - Từ điển dữ liệu.

\textbf{Bảng projects: Lưu trữ thông tin dự án trong hệ thống}

\renewcommand{\arraystretch}{1.5}
\begin{longtable}{|p{3cm}|p{2.5cm}|p{1.8cm}|p{1.8cm}|p{1.8cm}|p{1.8cm}|p{3cm}|p{5cm}|}
\hline
\rowcolor{gray!20}
\textbf{Tên trường} & \textbf{Kiểu dữ liệu} & \textbf{Khóa chính} & \textbf{Khóa ngoại} & \textbf{Duy nhất} & \textbf{Bắt buộc} & \textbf{Bảng tham chiếu} & \textbf{Mô tả} \\
\hline
\endfirsthead

\hline
\rowcolor{gray!20}
\textbf{Tên trường} & \textbf{Kiểu dữ liệu} & \textbf{Khóa chính} & \textbf{Khóa ngoại} & \textbf{Duy nhất} & \textbf{Bắt buộc} & \textbf{Bảng tham chiếu} & \textbf{Mô tả} \\
\hline
\endhead

\hline
\endfoot
project\_id & UUID & X & & X & X & & Mã định danh dự án. \\
\hline
org\_id & UUID & & X & & X & organizations & Tổ chức sở hữu dự án. \\
\hline
identifier & String & & & & X & & Mã định danh ngắn gọn (e.g. PROJ). \\
\hline
name & String & & & & X & & Tên dự án. \\
\hline
description & Text & & & & & & Mô tả chi tiết dự án. \\
\hline
project\_lead\_id & UUID & & X & & & users & User ID của người dẫn dự án. \\
\hline
default\_assignee\_id & UUID & & X & & & users & User ID mặc định được gán issue. \\
\hline
created\_by & UUID & & X & & X & users & User tạo dự án. \\
\hline
created\_at & Timestamp & & & & X & & Thời điểm tạo dự án. \\
\hline
updated\_at & Timestamp & & & & X & & Thời điểm cập nhật gần nhất. \\
\hline
\end{longtable}

\textbf{Bảng sprints: Lưu trữ thông tin sprint của dự án}

\renewcommand{\arraystretch}{1.5}
\begin{longtable}{|p{3cm}|p{2.5cm}|p{1.8cm}|p{1.8cm}|p{1.8cm}|p{1.8cm}|p{3cm}|p{5cm}|}
\hline
\rowcolor{gray!20}
\textbf{Tên trường} & \textbf{Kiểu dữ liệu} & \textbf{Khóa chính} & \textbf{Khóa ngoại} & \textbf{Duy nhất} & \textbf{Bắt buộc} & \textbf{Bảng tham chiếu} & \textbf{Mô tả} \\
\hline
\endfirsthead

\hline
\rowcolor{gray!20}
\textbf{Tên trường} & \textbf{Kiểu dữ liệu} & \textbf{Khóa chính} & \textbf{Khóa ngoại} & \textbf{Duy nhất} & \textbf{Bắt buộc} & \textbf{Bảng tham chiếu} & \textbf{Mô tả} \\
\hline
\endhead

\hline
\endfoot
sprint\_id & UUID & X & & X & X & & Mã định danh sprint. \\
\hline
project\_id & UUID & & X & & X & projects & Dự án chứa sprint. \\
\hline
name & String & & & & X & & Tên sprint. \\
\hline
goal & Text & & & & & & Mục tiêu của sprint. \\
\hline
start\_date & Date & & & & & & Ngày bắt đầu sprint. \\
\hline
end\_date & Date & & & & & & Ngày kết thúc sprint. \\
\hline
status & Enum & & & & X & & Trạng thái (FUTURE, ACTIVE, CLOSED). \\
\hline
created\_by & UUID & & X & & X & users & User tạo sprint. \\
\hline
created\_at & Timestamp & & & & X & & Thời điểm tạo sprint. \\
\hline
updated\_at & Timestamp & & & & X & & Thời điểm cập nhật gần nhất. \\
\hline
\end{longtable}

\textbf{Bảng issues: Lưu trữ thông tin công việc/nhiệm vụ trong dự án}

\renewcommand{\arraystretch}{1.5}
\begin{longtable}{|p{3cm}|p{2.5cm}|p{1.8cm}|p{1.8cm}|p{1.8cm}|p{1.8cm}|p{3cm}|p{5cm}|}
\hline
\rowcolor{gray!20}
\textbf{Tên trường} & \textbf{Kiểu dữ liệu} & \textbf{Khóa chính} & \textbf{Khóa ngoại} & \textbf{Duy nhất} & \textbf{Bắt buộc} & \textbf{Bảng tham chiếu} & \textbf{Mô tả} \\
\hline
\endfirsthead

\hline
\rowcolor{gray!20}
\textbf{Tên trường} & \textbf{Kiểu dữ liệu} & \textbf{Khóa chính} & \textbf{Khóa ngoại} & \textbf{Duy nhất} & \textbf{Bắt buộc} & \textbf{Bảng tham chiếu} & \textbf{Mô tả} \\
\hline
\endhead

\hline
\endfoot
issue\_id & UUID & X & & X & X & & Mã định danh issue. \\
\hline
project\_id & UUID & & X & & X & projects & Dự án chứa issue. \\
\hline
sprint\_id & UUID & & X & & & sprints & Sprint chứa issue (nullable). \\
\hline
parent\_id & UUID & & X & & & issues & Issue cha (cho sub-task). \\
\hline
seq\_id & Integer & & & & X & & ID tự tăng trong project. \\
\hline
title & String & & & & X & & Tiêu đề issue. \\
\hline
description & Text & & & & & & Mô tả chi tiết (HTML). \\
\hline
type & Enum & & & & X & & Loại (STORY, TASK, BUG, EPIC). \\
\hline
status\_id & UUID & & X & & X & issue\_statuses & Trạng thái hiện tại. \\
\hline
priority & Enum & & & & X & & Độ ưu tiên (LOW, MEDIUM, HIGH, CRITICAL). \\
\hline
story\_points & Decimal & & & & & & Điểm ước tính công việc. \\
\hline
start\_date & Date & & & & & & Ngày bắt đầu. \\
\hline
target\_date & Date & & & & & & Ngày mục tiêu hoàn thành. \\
\hline
sort\_order & Decimal & & & & X & & Thứ tự sắp xếp (fractional). \\
\hline
created\_by & UUID & & X & & X & users & User tạo issue. \\
\hline
created\_at & Timestamp & & & & X & & Thời điểm tạo. \\
\hline
updated\_at & Timestamp & & & & X & & Thời điểm cập nhật gần nhất. \\
\hline
\end{longtable}

\textbf{Bảng issue\_statuses: Lưu trữ các trạng thái tùy chỉnh của issue}

\renewcommand{\arraystretch}{1.5}
\begin{longtable}{|p{3cm}|p{2.5cm}|p{1.8cm}|p{1.8cm}|p{1.8cm}|p{1.8cm}|p{3cm}|p{5cm}|}
\hline
\rowcolor{gray!20}
\textbf{Tên trường} & \textbf{Kiểu dữ liệu} & \textbf{Khóa chính} & \textbf{Khóa ngoại} & \textbf{Duy nhất} & \textbf{Bắt buộc} & \textbf{Bảng tham chiếu} & \textbf{Mô tả} \\
\hline
\endfirsthead

\hline
\rowcolor{gray!20}
\textbf{Tên trường} & \textbf{Kiểu dữ liệu} & \textbf{Khóa chính} & \textbf{Khóa ngoại} & \textbf{Duy nhất} & \textbf{Bắt buộc} & \textbf{Bảng tham chiếu} & \textbf{Mô tả} \\
\hline
\endhead

\hline
\endfoot
status\_id & UUID & X & & X & X & & Mã định danh trạng thái. \\
\hline
project\_id & UUID & & X & & X & projects & Dự án chứa trạng thái. \\
\hline
name & String & & & & X & & Tên trạng thái. \\
\hline
description & Text & & & & & & Mô tả trạng thái. \\
\hline
color & String & & & & X & & Mã màu hex cho UI. \\
\hline
order & Integer & & & & X & & Thứ tự hiển thị. \\
\hline
created\_at & Timestamp & & & & X & & Thời điểm tạo. \\
\hline
updated\_at & Timestamp & & & & X & & Thời điểm cập nhật gần nhất. \\
\hline
\end{longtable}

\textbf{Bảng issue\_assignees: Lưu trữ quan hệ gán issue cho thành viên}

\renewcommand{\arraystretch}{1.5}
\begin{longtable}{|p{3cm}|p{2.5cm}|p{1.8cm}|p{1.8cm}|p{1.8cm}|p{1.8cm}|p{3cm}|p{5cm}|}
\hline
\rowcolor{gray!20}
\textbf{Tên trường} & \textbf{Kiểu dữ liệu} & \textbf{Khóa chính} & \textbf{Khóa ngoại} & \textbf{Duy nhất} & \textbf{Bắt buộc} & \textbf{Bảng tham chiếu} & \textbf{Mô tả} \\
\hline
\endfirsthead

\hline
\rowcolor{gray!20}
\textbf{Tên trường} & \textbf{Kiểu dữ liệu} & \textbf{Khóa chính} & \textbf{Khóa ngoại} & \textbf{Duy nhất} & \textbf{Bắt buộc} & \textbf{Bảng tham chiếu} & \textbf{Mô tả} \\
\hline
\endhead

\hline
\endfoot
issue\_id & UUID & X & X & & X & issues & Issue được gán. \\
\hline
user\_id & UUID & X & X & & X & users & User được gán. \\
\hline
created\_at & Timestamp & & & & X & & Thời điểm gán. \\
\hline
\end{longtable}

\textbf{Bảng issue\_comments: Lưu trữ bình luận trên issue}

\renewcommand{\arraystretch}{1.5}
\begin{longtable}{|p{3cm}|p{2.5cm}|p{1.8cm}|p{1.8cm}|p{1.8cm}|p{1.8cm}|p{3cm}|p{5cm}|}
\hline
\rowcolor{gray!20}
\textbf{Tên trường} & \textbf{Kiểu dữ liệu} & \textbf{Khóa chính} & \textbf{Khóa ngoại} & \textbf{Duy nhất} & \textbf{Bắt buộc} & \textbf{Bảng tham chiếu} & \textbf{Mô tả} \\
\hline
\endfirsthead

\hline
\rowcolor{gray!20}
\textbf{Tên trường} & \textbf{Kiểu dữ liệu} & \textbf{Khóa chính} & \textbf{Khóa ngoại} & \textbf{Duy nhất} & \textbf{Bắt buộc} & \textbf{Bảng tham chiếu} & \textbf{Mô tả} \\
\hline
\endhead

\hline
\endfoot
comment\_id & UUID & X & & X & X & & Mã định danh comment. \\
\hline
issue\_id & UUID & & X & & X & issues & Issue chứa comment. \\
\hline
user\_id & UUID & & X & & X & users & User tạo comment. \\
\hline
content & Text & & & & X & & Nội dung comment (HTML). \\
\hline
created\_at & Timestamp & & & & X & & Thời điểm tạo. \\
\hline
updated\_at & Timestamp & & & & X & & Thời điểm cập nhật gần nhất. \\
\hline
\end{longtable}

\textbf{Bảng issue\_activities: Lưu trữ lịch sử thay đổi của issue}

\renewcommand{\arraystretch}{1.5}
\begin{longtable}{|p{3cm}|p{2.5cm}|p{1.8cm}|p{1.8cm}|p{1.8cm}|p{1.8cm}|p{3cm}|p{5cm}|}
\hline
\rowcolor{gray!20}
\textbf{Tên trường} & \textbf{Kiểu dữ liệu} & \textbf{Khóa chính} & \textbf{Khóa ngoại} & \textbf{Duy nhất} & \textbf{Bắt buộc} & \textbf{Bảng tham chiếu} & \textbf{Mô tả} \\
\hline
\endfirsthead

\hline
\rowcolor{gray!20}
\textbf{Tên trường} & \textbf{Kiểu dữ liệu} & \textbf{Khóa chính} & \textbf{Khóa ngoại} & \textbf{Duy nhất} & \textbf{Bắt buộc} & \textbf{Bảng tham chiếu} & \textbf{Mô tả} \\
\hline
\endhead

\hline
\endfoot
activity\_id & UUID & X & & X & X & & Mã định danh hoạt động. \\
\hline
issue\_id & UUID & & X & & X & issues & Issue bị thay đổi. \\
\hline
user\_id & UUID & & X & & X & users & User thực hiện thay đổi. \\
\hline
field\_name & String & & & & X & & Tên trường bị thay đổi. \\
\hline
old\_value & Text & & & & & & Giá trị cũ. \\
\hline
new\_value & Text & & & & & & Giá trị mới. \\
\hline
created\_at & Timestamp & & & & X & & Thời điểm thay đổi. \\
\hline
\end{longtable}

\textbf{Bảng project\_analytics: Lưu trữ số liệu thống kê dự án}

\renewcommand{\arraystretch}{1.5}
\begin{longtable}{|p{3cm}|p{2.5cm}|p{1.8cm}|p{1.8cm}|p{1.8cm}|p{1.8cm}|p{3cm}|p{5cm}|}
\hline
\rowcolor{gray!20}
\textbf{Tên trường} & \textbf{Kiểu dữ liệu} & \textbf{Khóa chính} & \textbf{Khóa ngoại} & \textbf{Duy nhất} & \textbf{Bắt buộc} & \textbf{Bảng tham chiếu} & \textbf{Mô tả} \\
\hline
\endfirsthead

\hline
\rowcolor{gray!20}
\textbf{Tên trường} & \textbf{Kiểu dữ liệu} & \textbf{Khóa chính} & \textbf{Khóa ngoại} & \textbf{Duy nhất} & \textbf{Bắt buộc} & \textbf{Bảng tham chiếu} & \textbf{Mô tả} \\
\hline
\endhead

\hline
\endfoot
analytics\_id & UUID & X & & X & X & & Mã định danh analytics. \\
\hline
project\_id & UUID & & X & & X & projects & Dự án được phân tích. \\
\hline
date & Date & & & & X & & Ngày thu thập số liệu. \\
\hline
issues\_created & Integer & & & & X & & Số issue tạo mới. \\
\hline
issues\_resolved & Integer & & & & X & & Số issue đã hoàn thành. \\
\hline
issues\_in\_\allowbreak progress & Integer & & & & X & & Số issue đang thực hiện. \\
\hline
created\_at & Timestamp & & & & X & & Thời điểm tạo record. \\
\hline
\end{longtable}

\end{landscape}
\pagestyle{fancy}

\phantomsection
\subsection*{Các yêu cầu ràng buộc cần thiết khi xử lý dữ liệu}

Phần này mô tả các ràng buộc dữ liệu cần được đảm bảo khi xử lý và lưu trữ thông tin trong phân hệ quản lý dự án.

\subsubsection*{Ràng buộc dữ liệu dự án (Projects)}

\begin{itemize}
    \item Thuộc tính \texttt{identifier} phải khác rỗng, duy nhất trong organization, có độ dài từ 2 đến 10 ký tự viết hoa.
    \item Thuộc tính \texttt{name} phải khác rỗng và có độ dài tối thiểu 3 ký tự, tối đa 100 ký tự.
    \item Thuộc tính \texttt{org\_id} phải là UUID hợp lệ và tham chiếu đến một organization tồn tại trong hệ thống.
    \item Thuộc tính \texttt{project\_lead\_id} và \texttt{default\_assignee\_id} phải là UUID hợp lệ tham chiếu đến users.
    \item Cặp (\texttt{org\_id}, \texttt{identifier}) phải duy nhất để tránh trùng lặp mã dự án trong organization.
\end{itemize}

\subsubsection*{Ràng buộc dữ liệu sprint (Sprints)}

\begin{itemize}
    \item Thuộc tính \texttt{name} phải khác rỗng và có độ dài tối thiểu 3 ký tự, tối đa 100 ký tự.
    \item Thuộc tính \texttt{project\_id} phải là UUID hợp lệ và tham chiếu đến một project tồn tại trong hệ thống.
    \item Thuộc tính \texttt{status} phải có một trong các giá trị: FUTURE, ACTIVE, CLOSED.
    \item Thuộc tính \texttt{start\_date} và \texttt{end\_date} phải là ngày hợp lệ, với \texttt{end\_date} sau \texttt{start\_date}.
    \item Chỉ được phép có tối đa một sprint với status ACTIVE trong mỗi project tại một thời điểm.
    \item Thuộc tính \texttt{goal} có thể null, nhưng khuyến khích điền để rõ mục tiêu sprint.
\end{itemize}

\subsubsection*{Ràng buộc dữ liệu issue (Issues)}

\begin{itemize}
    \item Thuộc tính \texttt{title} phải khác rỗng và có độ dài tối thiểu 3 ký tự, tối đa 255 ký tự.
    \item Thuộc tính \texttt{project\_id} phải là UUID hợp lệ và tham chiếu đến một project tồn tại.
    \item Thuộc tính \texttt{type} phải có một trong các giá trị: STORY, TASK, BUG, EPIC.
    \item Thuộc tính \texttt{priority} phải có một trong các giá trị: LOW, MEDIUM, HIGH, CRITICAL.
    \item Thuộc tính \texttt{status\_id} phải tham chiếu đến một status hợp lệ thuộc cùng project.
    \item Thuộc tính \texttt{seq\_id} phải là số nguyên dương tự tăng duy nhất trong project.
    \item Thuộc tính \texttt{sprint\_id} có thể null (issue trong backlog) hoặc tham chiếu đến sprint hợp lệ.
    \item Thuộc tính \texttt{parent\_id} nếu có phải tham chiếu đến issue khác trong cùng project (để tạo sub-task).
    \item Thuộc tính \texttt{story\_points} nếu có phải là số thập phân dương.
    \item Thuộc tính \texttt{sort\_order} phải là số thập phân để hỗ trợ fractional ordering.
\end{itemize}

\subsubsection*{Ràng buộc dữ liệu trạng thái (Issue Statuses)}

\begin{itemize}
    \item Thuộc tính \texttt{name} phải khác rỗng và có độ dài tối thiểu 1 ký tự, tối đa 50 ký tự.
    \item Thuộc tính \texttt{project\_id} phải là UUID hợp lệ tham chiếu đến project.
    \item Cặp (\texttt{project\_id}, \texttt{name}) phải duy nhất để tránh trùng lặp tên status trong project.
    \item Thuộc tính \texttt{color} phải là mã màu hex hợp lệ (ví dụ: \#FF0000).
    \item Thuộc tính \texttt{order} phải là số nguyên dương để sắp xếp các status.
\end{itemize}

\subsubsection*{Ràng buộc dữ liệu bình luận (Issue Comments)}

\begin{itemize}
    \item Thuộc tính \texttt{content} phải khác rỗng và có độ dài tối thiểu 1 ký tự.
    \item Thuộc tính \texttt{issue\_id} phải là UUID hợp lệ tham chiếu đến issue.
    \item Thuộc tính \texttt{user\_id} phải là UUID hợp lệ tham chiếu đến user tạo comment.
    \item Content có thể chứa HTML để hỗ trợ rich text formatting.
\end{itemize}

\subsubsection*{Ràng buộc dữ liệu lịch sử (Issue Activities)}

\begin{itemize}
    \item Thuộc tính \texttt{field\_name} phải khác rỗng và là tên trường hợp lệ của issue (ví dụ: status, assignee, priority).
    \item Thuộc tính \texttt{issue\_id} phải tham chiếu đến issue bị thay đổi.
    \item Thuộc tính \texttt{user\_id} phải tham chiếu đến user thực hiện thay đổi.
    \item Thuộc tính \texttt{old\_value} và \texttt{new\_value} lưu trữ giá trị trước và sau thay đổi dạng text.
\end{itemize}

\subsubsection*{Ràng buộc dữ liệu phân tích (Project Analytics)}

\begin{itemize}
    \item Thuộc tính \texttt{project\_id} phải là UUID hợp lệ tham chiếu đến project.
    \item Thuộc tính \texttt{date} phải là ngày hợp lệ, duy nhất cho mỗi project.
    \item Các thuộc tính \texttt{issues\_created}, \texttt{issues\_resolved}, \texttt{issues\_in\_progress} phải là số nguyên không âm.
\end{itemize}

\subsubsection*{Ràng buộc chung cho phân hệ quản lý dự án}

\begin{itemize}
    \item Mọi bảng phải có primary key là UUID để tránh enumeration attacks và đảm bảo tính duy nhất.
    \item Mọi bảng phải có \texttt{created\_at} và \texttt{updated\_at} timestamps để tracking thời gian tạo và cập nhật.
    \item Foreign key constraints phải được enforce với ON DELETE CASCADE cho các quan hệ phụ thuộc (ví dụ: xóa project sẽ xóa toàn bộ issues, sprints).
    \item Các giá trị enum (type, priority, status) phải được kiểm tra strict để đảm bảo chỉ chứa các giá trị được định nghĩa.
    \item Tất cả các trường bắt buộc (NOT NULL) phải được validate ở application layer trước khi insert/update.
    \item Dữ liệu phải được tổ chức theo organization để đảm bảo data isolation giữa các tổ chức.
    \item Mọi thay đổi quan trọng trên issue phải được ghi lại trong bảng \texttt{issue\_activities} để đảm bảo audit trail đầy đủ.
    \item Thứ tự sắp xếp (sort\_order) sử dụng fractional indexing để cho phép re-ordering linh hoạt mà không cần update nhiều records.
    \item Mỗi project chỉ được phép có tối đa một sprint ở trạng thái ACTIVE tại một thời điểm.
\end{itemize}