% !TEX root = main.tex
\phantomsection
\addcontentsline{toc}{section}{CÁC TỪ VÀ THUẬT NGỮ VIẾT TẮT}

\begin{center}
\textbf{\Large CÁC TỪ VÀ THUẬT NGỮ VIẾT TẮT}
\end{center}

\begin{longtblr}[
  label = none,
  entry = none,
]{
  colspec = {|c|c|X[l]|},
  width = \linewidth,
  rowhead = 1,
  row{1} = {font=\bfseries, bg=gray9, c, m},
  cell{1}{3} = {c},
  column{1} = {c, m},
  column{2} = {c, m},
  column{3} = {l, m},
  hlines,
  vlines,
}
STT & Thuật ngữ / Từ viết tắt & Định nghĩa / giải thích \\

1 & Agile & Phương pháp phát triển phần mềm linh hoạt, lặp đi lặp lại, tập trung vào phản hồi nhanh và thích ứng với thay đổi \\

2 & AI & Artificial Intelligence: trí tuệ nhân tạo \\

3 & API & Application Programming Interface: giao diện lập trình ứng dụng \\

4 & Backlog & Danh sách các công việc (issues) chưa được phân bổ vào sprint, được sắp xếp theo độ ưu tiên \\

5 & Board & Bảng Kanban hiển thị công việc theo các cột trạng thái, hỗ trợ drag-and-drop để cập nhật trạng thái \\

6 & Bug & Loại công việc đại diện cho lỗi phần mềm cần được sửa chữa \\

7 & CI/CD & Continuous Integration/Continuous Deployment: tích hợp liên tục và triển khai liên tục \\

8 & DnD Kit & Drag-and-Drop Kit: thư viện JavaScript hỗ trợ chức năng kéo thả trên giao diện web \\

9 & Epic & Loại công việc lớn có thể được phân rã thành nhiều Story hoặc Task nhỏ hơn \\

10 & ERD & Entity-Relationship Diagram: sơ đồ thực thể liên kết \\

11 & HTTP/HTTPS & HyperText Transfer Protocol (Secure): giao thức truyền tải siêu văn bản \\

12 & JSON & JavaScript Object Notation: định dạng dữ liệu dạng đối tượng JavaScript \\

13 & JWT & JSON Web Token: mã thông báo web dạng JSON dùng cho xác thực \\

14 & Kanban & Phương pháp quản lý công việc trực quan sử dụng bảng với các cột trạng thái \\

15 & LangChain.js & Framework JavaScript để xây dựng ứng dụng tích hợp LLM với structured output \\

16 & LLM & Large Language Model: mô hình ngôn ngữ lớn (ví dụ: GPT-4, Claude, Gemini) \\

17 & MinIO & Hệ thống lưu trữ đối tượng tương thích S3, dùng cho môi trường development \\

18 & MobX & Thư viện quản lý state cho React với reactive programming \\

19 & NestJS & Framework Node.js để xây dựng API backend hiệu quả, có khả năng mở rộng \\

20 & Next.js & Framework React cho phát triển ứng dụng web với App Router và server-side rendering \\

21 & OAuth 2.0 & Open Authorization 2.0: tiêu chuẩn ủy quyền mở phiên bản 2.0 (ví dụ: Google OAuth) \\

22 & ORM & Object-Relational Mapping: kỹ thuật ánh xạ đối tượng với cơ sở dữ liệu quan hệ \\

23 & PM & Project Management: quản lý dự án \\

24 & PostgreSQL & Hệ quản trị cơ sở dữ liệu quan hệ mã nguồn mở, hỗ trợ SQL và JSON \\

25 & Prisma & ORM hiện đại cho Node.js và TypeScript, cung cấp type-safe database access \\

26 & RBAC & Role-Based Access Control: phân quyền dựa trên vai trò (Project Lead, Team Member, Viewer) \\

27 & Redis & Hệ thống lưu trữ key-value in-memory, dùng cho caching và WebSocket Pub/Sub \\

28 & REST/RESTful & Representational State Transfer: kiến trúc truyền tải trạng thái đại diện \\

29 & S3 & Simple Storage Service: dịch vụ lưu trữ đối tượng của Amazon AWS \\

30 & SaaS & Software as a Service: phần mềm dưới dạng dịch vụ \\

31 & Scrum & Framework Agile với các sprint có độ dài cố định, daily standup, sprint planning và retrospective \\

32 & Sprint & Chu kỳ phát triển ngắn (thường 1-4 tuần) trong Scrum, có lifecycle: FUTURE → ACTIVE → CLOSED \\

33 & SQL & Structured Query Language: ngôn ngữ truy vấn có cấu trúc \\

34 & SSL/TLS & Secure Sockets Layer / Transport Layer Security: giao thức bảo mật tầng truyền tải \\

35 & Story & User Story: loại công việc mô tả tính năng từ góc nhìn người dùng cuối \\

36 & Task & Loại công việc kỹ thuật cần thực hiện, thường là phần của Story hoặc Epic \\

37 & UI/UX & User Interface / User Experience: giao diện người dùng và trải nghiệm người dùng \\

38 & WebSocket & Giao thức truyền thông hai chiều qua kết nối TCP, hỗ trợ real-time collaboration \\

39 & Zod & Thư viện TypeScript-first schema validation, dùng để validate LLM structured output \\
\end{longtblr}
