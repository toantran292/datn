\phantomsection
\subsection*{1. Đặt vấn đề}
\addcontentsline{toc}{subsection}{1. Đặt vấn đề}
\setcounter{subsection}{1}
\setcounter{subsubsection}{0}
\setcounter{figure}{0}

Trong bối cảnh chuyển đổi số diễn ra mạnh mẽ, ngày càng nhiều doanh nghiệp và nhóm phát triển phần mềm lựa chọn phương pháp Agile nhằm rút ngắn vòng đời phát triển, tăng khả năng thích ứng với thay đổi và nâng cao chất lượng sản phẩm. Các báo cáo tổng hợp gần đây cho thấy mức độ áp dụng Agile trong các nhóm phát triển phần mềm đã tăng rất nhanh: theo báo cáo State of Agile lần thứ 15, tỷ lệ đội ngũ phát triển phần mềm áp dụng Agile đã tăng từ 37\% năm 2020 lên 86\% năm 2021, phản ánh việc Agile đã trở thành phương pháp chủ đạo trong nhiều tổ chức thay vì chỉ là một xu hướng thử nghiệm~\cite{digitalai_state_of_agile_15}. Song song đó, thị trường phần mềm quản lý dự án cũng tăng trưởng mạnh mẽ; một số nghiên cứu thị trường ước tính quy mô thị trường phần mềm quản lý dự án toàn cầu đạt khoảng 9{,}76 tỷ USD vào năm 2025 và có thể tăng lên hơn 20 tỷ USD vào năm 2030 với tốc độ tăng trưởng trung bình trên 15\% mỗi năm~\cite{mordor_pm_software_market_2025}. Những con số này cho thấy việc sử dụng các nền tảng SaaS để hỗ trợ quản lý dự án và cộng tác đã trở thành xu thế tất yếu trong ngành công nghệ.

Tuy nhiên, sự phát triển nhanh của các công cụ lại kéo theo hiện tượng ``bùng nổ ứng dụng'' trong môi trường làm việc. Một nhóm Agile điển hình thường phải đồng thời sử dụng công cụ quản lý công việc (như Jira, Linear), công cụ ghi chú và tài liệu (Notion, Confluence), nền tảng nhắn tin (Slack, Microsoft Teams) và các hệ thống họp trực tuyến (Google Meet, Zoom). Các nghiên cứu về năng suất lao động cho thấy việc liên tục chuyển đổi giữa các ứng dụng này gây tổn thất đáng kể về thời gian và sự tập trung. Một khảo sát do RingCentral công bố cho thấy có tới 69\% người lao động lãng phí đến 60 phút mỗi ngày chỉ để điều hướng giữa các ứng dụng làm việc khác nhau, tương đương khoảng 32 ngày làm việc mỗi năm~\cite{ringcentral_connected_workplace}. Một nghiên cứu khác, dựa trên dữ liệu từ Asana, cho biết nhân viên văn phòng phải sử dụng 10 ứng dụng hoặc hơn mỗi ngày và có thể mất trung bình khoảng 3{,}6 giờ mỗi tuần do bị gián đoạn bởi việc chuyển đổi qua lại giữa các công cụ~\cite{asana_context_switching_basicops}. Thời gian bị bào mòn bởi việc chuyển ngữ cảnh không chỉ làm giảm năng suất mà còn gia tăng gánh nặng tinh thần cho người sử dụng.

Trong thực tế triển khai các dự án Agile, hoạt động trao đổi thông tin giữa các thành viên diễn ra liên tục và tạo ra khối lượng dữ liệu rất lớn. Các cuộc thảo luận về yêu cầu, phản hồi về thiết kế, cập nhật tiến độ, biên bản họp và tài liệu kỹ thuật thường bị phân tán trên nhiều nền tảng khác nhau như Slack, Discord, Microsoft Teams hay email. Sự phân mảnh này khiến lịch sử liên lạc, các quyết định quan trọng và tài liệu đính kèm khó được theo dõi tập trung. Khi một thành viên mới tham gia dự án hoặc khi cần tra cứu lại một quyết định đã đưa ra trước đó, họ phải tìm kiếm qua nhiều kênh, nhiều ứng dụng và đọc lại hàng trăm tin nhắn để nắm bắt ngữ cảnh. Việc thiếu một hệ thống truyền thông tập trung không chỉ gây tốn thời gian mà còn làm tăng nguy cơ bỏ sót thông tin quan trọng, ảnh hưởng đến chất lượng phối hợp và tiến độ dự án.

Song song với sự phát triển của Agile và các nền tảng SaaS, trí tuệ nhân tạo (AI) cũng đang được tích hợp ngày càng sâu vào quy trình phát triển phần mềm và quản lý dự án. Các báo cáo gần đây về State of Agile cho thấy AI đang chuyển từ vai trò công cụ hỗ trợ sang vị trí một thành phần trung tâm trong chu trình cung cấp phần mềm, với tỷ lệ đội ngũ sử dụng AI cho các hoạt động liên quan đến Agile tăng từ khoảng 64\% lên 84\% chỉ trong một năm~\cite{digitalai_state_of_agile_18_ai_itpro}. Trong lĩnh vực truyền thông nội bộ, AI được kỳ vọng có thể hỗ trợ tóm tắt các đoạn hội thoại dài, trích xuất ý chính và danh sách công việc từ cuộc thảo luận, phân tích nội dung tài liệu đính kèm và trả lời câu hỏi dựa trên ngữ cảnh trao đổi. Tuy nhiên, phần lớn các tính năng AI hiện tại vẫn gắn với từng công cụ nhắn tin riêng lẻ và chỉ ``nhìn thấy'' phần dữ liệu thuộc về công cụ đó. Khi thông tin trao đổi bị phân tán trên nhiều nền tảng, AI khó có thể hình thành được bức tranh toàn diện về ngữ cảnh dự án để đưa ra các tóm tắt và gợi ý có chiều sâu.

Thực tế trên đặt ra nhu cầu về một phân hệ \textit{truyền thông} tập trung trong hệ sinh thái SaaS tích hợp AI phục vụ quản lý dự án Agile. Phân hệ này cần đảm nhiệm đồng thời nhiều chức năng: cho phép người dùng trao đổi tin nhắn theo kênh và theo dự án trong một giao diện thống nhất; hỗ trợ gửi, lưu trữ và quản lý các tệp đính kèm trực tiếp trong cuộc trò chuyện; cung cấp khả năng tìm kiếm và tra cứu lịch sử hội thoại một cách dễ dàng; đồng thời tích hợp AI để tự động tóm tắt nội dung hội thoại, phân tích các tệp được tải lên và cung cấp câu trả lời hoặc gợi ý dựa trên toàn bộ ngữ cảnh trao đổi. Khi phân hệ truyền thông được thiết kế tốt và tích hợp chặt chẽ với các phân hệ khác trong nền tảng, người dùng không cần đọc lại toàn bộ lịch sử tin nhắn hay mở từng tệp riêng lẻ mà vẫn nắm được nội dung chính, qua đó nâng cao hiệu quả phối hợp và hỗ trợ ra quyết định nhanh hơn.

Với ý nghĩa đó, đề tài \textit{``Xây dựng Nền tảng SaaS tích hợp AI nhằm Thống nhất Quản lý Dự án Agile -- Phân hệ truyền thông''} được đề xuất và thực hiện, nhằm góp phần giải quyết bài toán tập trung hóa hoạt động trao đổi thông tin, giảm thiểu tình trạng phân mảnh kênh liên lạc và xây dựng một hệ thống nhắn tin thông minh tích hợp AI cho môi trường quản lý dự án Agile hiện đại.