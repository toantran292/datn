\phantomsection
\subsection*{6. Những đóng góp chính của đề tài}
\addcontentsline{toc}{subsection}{6. Những đóng góp chính của đề tài}
\setcounter{subsection}{6}
\setcounter{subsubsection}{0}
\setcounter{figure}{0}

\begin{itemize}
    \item \textbf{Về mặt kinh tế -- tổ chức:}

    Đề tài xây dựng một phân hệ truyền thông cho hệ thống SaaS quản lý dự án Agile, giúp các doanh nghiệp giảm bớt sự phụ thuộc vào quá nhiều công cụ rời rạc (quản lý dự án, chat, họp, lưu trữ file…). Việc tập trung hóa quản lý tài khoản, workspace, tệp và thông báo trên một lớp nền tảng chung giúp đơn giản hóa vận hành, giảm chi phí tích hợp, nâng cao khả năng kiểm soát và theo dõi hoạt động dự án. Khi được triển khai thực tế, hệ thống có thể góp phần tăng hiệu quả sử dụng nguồn lực, rút ngắn thời gian nắm bắt tình hình dự án và hỗ trợ ra quyết định nhanh hơn.

    \item \textbf{Về mặt xã hội -- cộng tác:}

    Đề tài cung cấp một mô hình nền tảng phục vụ làm việc nhóm trong bối cảnh làm việc từ xa ngày càng phổ biến. Phân hệ truyền thông giúp kết nối các kênh làm việc khác nhau (công việc, trao đổi, họp trực tuyến, tệp tài liệu) thông qua một lớp quản lý chung, đồng thời tận dụng AI để tóm tắt, tổng hợp hoạt động dự án. Điều này giúp các thành viên trong nhóm dễ dàng nắm bắt bức tranh chung mà không cần phải tự tìm kiếm thông tin ở nhiều nơi, góp phần giảm quá tải thông tin, tăng tính minh bạch và khả năng phối hợp giữa các bộ phận.

    \item \textbf{Về mặt giáo dục -- nghiên cứu:}

    Đề tài là một ví dụ cụ thể về việc áp dụng các công nghệ và mô hình kiến trúc hiện đại trong phát triển hệ thống web: kiến trúc microservice, lớp edge sử dụng Nginx, dịch vụ tài khoản bằng Spring, các dịch vụ nền tảng bằng NestJS, giao diện quản trị bằng Next.js và tích hợp AI ở tầng nền tảng. Sản phẩm và quá trình thực hiện giúp sinh viên hiểu sâu hơn về cách phân tách hệ thống thành nhiều dịch vụ, cách thiết kế API, quản lý dữ liệu multi-tenant, cũng như cách đưa các mô hình AI vào một sản phẩm thực tế ở vai trò công cụ hỗ trợ phân tích thông tin. Ngoài ra, kết quả đề tài có thể được sử dụng làm tài liệu tham khảo cho các khóa sau khi nghiên cứu về kiến trúc SaaS và tích hợp AI trong quản lý dự án.
\end{itemize}
