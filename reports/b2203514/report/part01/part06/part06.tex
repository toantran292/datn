\phantomsection
\subsection*{6. Những đóng góp chính của đề tài}
\addcontentsline{toc}{subsection}{6. Những đóng góp chính của đề tài}
\setcounter{subsection}{6}
\setcounter{subsubsection}{0}
\setcounter{figure}{0}

\begin{itemize}
    \item \textbf{Về mặt kinh tế -- tổ chức:}

    Đề tài xây dựng một phân hệ truyền thông cho hệ thống SaaS quản lý dự án Agile, giúp các doanh nghiệp giảm bớt sự phụ thuộc vào nhiều nền tảng nhắn tin rời rạc (Slack, Discord, Microsoft Teams, email…). Việc tập trung hóa các cuộc trao đổi theo kênh và dự án, kết hợp với khả năng gửi tệp đính kèm và tìm kiếm lịch sử hội thoại trên một hệ thống duy nhất giúp đơn giản hóa quy trình giao tiếp, giảm thời gian chuyển đổi giữa các công cụ và giảm nguy cơ bỏ sót thông tin quan trọng. Đặc biệt, việc tích hợp AI với kiến trúc RAG giúp người dùng nắm bắt nội dung chính của các cuộc thảo luận dài mà không cần đọc lại toàn bộ tin nhắn, qua đó tiết kiệm thời gian và hỗ trợ ra quyết định nhanh hơn.

    \item \textbf{Về mặt xã hội -- cộng tác:}

    Đề tài cung cấp một hệ thống nhắn tin tập trung phục vụ làm việc nhóm trong bối cảnh làm việc từ xa ngày càng phổ biến. Phân hệ truyền thông cho phép các thành viên dự án trao đổi thông tin theo kênh, chia sẻ tài liệu trực tiếp trong cuộc trò chuyện và theo dõi các thread thảo luận một cách có tổ chức. Việc tích hợp AI để tóm tắt hội thoại, trích xuất ý chính và trả lời câu hỏi dựa trên ngữ cảnh giúp các thành viên mới tham gia dự án nhanh chóng nắm bắt tình hình mà không cần đọc hàng trăm tin nhắn cũ, đồng thời giúp cả nhóm dễ dàng tra cứu lại các quyết định đã đưa ra trước đó. Điều này góp phần giảm quá tải thông tin, tăng tính minh bạch và nâng cao hiệu quả phối hợp giữa các thành viên.

    \item \textbf{Về mặt giáo dục -- nghiên cứu:}

    Đề tài là một ví dụ cụ thể về việc áp dụng các công nghệ và mô hình kiến trúc hiện đại trong phát triển hệ thống nhắn tin thời gian thực: kiến trúc microservice, WebSocket Gateway với NestJS và Socket.io, giao diện chat với React/Next.js, và đặc biệt là kiến trúc RAG (Retrieval-Augmented Generation) kết hợp vector database (pgvector) với mô hình ngôn ngữ lớn (LLM). Sản phẩm và quá trình thực hiện giúp sinh viên hiểu sâu hơn về cách xây dựng hệ thống real-time có khả năng mở rộng, cách thiết kế pipeline xử lý tài liệu (document processing), cách tạo và truy vấn embedding với vector database, cũng như cách tích hợp AI vào sản phẩm thực tế để hỗ trợ tóm tắt và hỏi đáp dựa trên ngữ cảnh. Ngoài ra, kết quả đề tài có thể được sử dụng làm tài liệu tham khảo cho các khóa sau khi nghiên cứu về hệ thống nhắn tin doanh nghiệp và ứng dụng RAG trong xử lý hội thoại.
\end{itemize}