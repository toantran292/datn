\phantomsection
\subsection*{7. Bố cục quyển báo cáo}
\addcontentsline{toc}{subsection}{7. Bố cục quyển báo cáo}
\setcounter{subsection}{7}
\setcounter{subsubsection}{0}
\setcounter{figure}{0}

Bố cục luận văn dự kiến gồm các phần: mục lục, danh mục hình, danh mục bảng, danh mục từ viết tắt, tóm tắt, abstract và ba phần chính: giới thiệu, nội dung, kết luận.

\begin{itemize}
    \item Phần \textbf{Giới thiệu} trình bày bối cảnh, lý do chọn đề tài, tổng quan các ứng dụng và nền tảng liên quan đang tồn tại trong và ngoài nước, xác định khoảng trống nghiên cứu, mục tiêu, đối tượng -- phạm vi nghiên cứu, quy trình nghiên cứu, công nghệ sử dụng và những đóng góp dự kiến của đề tài.

    \item Phần \textbf{Nội dung} là trọng tâm của luận văn, mô tả chi tiết bài toán và yêu cầu hệ thống; phân tích và đánh giá các giải pháp hiện có; đề xuất mô hình kiến trúc hệ thống và cơ sở dữ liệu; thiết kế phân hệ truyền thông; phân tích và hiện thực các chức năng chính (edge, account, notification, file-storage, AI, giao diện admin…); xây dựng các kịch bản tích hợp và kiểm thử; đánh giá kết quả đạt được về mặt kỹ thuật và mức độ đáp ứng yêu cầu quản lý dự án Agile.

    \item Phần \textbf{Kết luận} tổng hợp các kết quả chính của đề tài, nêu rõ những ưu điểm, hạn chế, khó khăn gặp phải trong quá trình thực hiện, đồng thời đề xuất một số hướng phát triển tiếp theo như: mở rộng thêm các phân hệ chức năng khác, cải thiện khả năng phân tích dữ liệu của AI, tối ưu hiệu năng và bảo mật khi triển khai trên môi trường thực tế.
\end{itemize}

Ngoài ra, luận văn còn có phần tài liệu tham khảo liệt kê các nguồn tài liệu đã sử dụng và phụ lục trình bày một số nội dung hỗ trợ như: mô hình dữ liệu chi tiết, một số đoạn mã quan trọng, cấu hình triển khai, giao diện minh họa và các kết quả kiểm thử tiêu biểu.
