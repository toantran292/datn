\begin{landscape}
\pagestyle{lscape}

\phantomsection
\setsection{Phụ lục C. TÀI LIỆU KIỂM THỬ}
\setcounter{section}{7}
\setcounter{table}{0}
\renewcommand{\thetable}{7.\arabic{table}}

\subsection*{Các trường hợp kiểm thử chi tiết}

Chi tiết các trường hợp kiểm thử như sau:

% TS_REG - Đăng ký

\begin{longtblr}[
  caption = {Kiểm tra chức năng đăng ký tài khoản},
  label = {tab:tc_register}
]{
  width=\linewidth, hlines, vlines,
  colspec={X[1,c]X[1,c]X[1.5,l]X[1.5,l]X[2,l]X[1.5,l]X[1.5,l]X[0.7,c]},
  rows={m},
  row{1}={font=\bfseries, c, bg=gray9}
}
Mã TC & Mã kịch bản & Mô tả & Dữ liệu kiểm thử & Các bước thực hiện & Kết quả mong đợi & Kết quả thực tế & Kết quả \\
TC\_\-REG\_\-01 & TS\_\-REG & Kiểm thử đăng ký với thông tin hợp lệ & Họ tên: Nguyễn Văn A\newline Email: nguyenvana\-@gmail.com\newline Mật khẩu: Password123! & 1. Nhập họ tên\newline 2. Nhập email\newline 3. Nhập mật khẩu\newline 4. Xác nhận mật khẩu\newline 5. Nhấn Đăng ký & Hiển thị thông báo: "Đăng ký thành công!" & Hiển thị thông báo: "Đăng ký thành công!" & Pass \\
TC\_\-REG\_\-02 & TS\_\-REG & Kiểm thử đăng ký với email đã tồn tại & Họ tên: Trần Thị B\newline Email: nguyenvana\-@gmail.com\newline Mật khẩu: Password123! & 1. Nhập họ tên\newline 2. Nhập email đã tồn tại\newline 3. Nhập mật khẩu\newline 4. Xác nhận mật khẩu\newline 5. Nhấn Đăng ký & Hiển thị thông báo: "Email đã được sử dụng!" & Hiển thị thông báo: "Email đã được sử dụng!" & Pass \\
TC\_\-REG\_\-03 & TS\_\-REG & Kiểm thử đăng ký với email không hợp lệ & Họ tên: Lê Văn C\newline Email: invalid-email\newline Mật khẩu: Password123! & 1. Nhập họ tên\newline 2. Nhập email không hợp lệ\newline 3. Nhập mật khẩu\newline 4. Nhấn Đăng ký & Hiển thị thông báo: "Email không hợp lệ!" & Hiển thị thông báo: "Email không hợp lệ!" & Pass \\
TC\_\-REG\_\-04 & TS\_\-REG & Kiểm thử đăng ký với mật khẩu yếu & Họ tên: Phạm Văn D\newline Email: phamvand\-@gmail.com\newline Mật khẩu: 123 & 1. Nhập họ tên\newline 2. Nhập email\newline 3. Nhập mật khẩu yếu\newline 4. Nhấn Đăng ký & Hiển thị thông báo: "Mật khẩu phải có ít nhất 8 ký tự!" & Hiển thị thông báo: "Mật khẩu phải có ít nhất 8 ký tự!" & Pass \\
TC\_\-REG\_\-05 & TS\_\-REG & Kiểm thử đăng ký với mật khẩu không khớp & Họ tên: Hoàng Văn E\newline Email: hoangvane\-@gmail.com\newline Mật khẩu: Password123!\newline Xác nhận: Password456! & 1. Nhập họ tên\newline 2. Nhập email\newline 3. Nhập mật khẩu\newline 4. Nhập xác nhận khác\newline 5. Nhấn Đăng ký & Hiển thị thông báo: "Mật khẩu xác nhận không khớp!" & Hiển thị thông báo: "Mật khẩu xác nhận không khớp!" & Pass \\
\end{longtblr}

\begin{longtblr}[
  caption = {Kiểm tra chức năng đăng nhập},
  label = {tab:tc_login}
]{
  width=\linewidth, hlines, vlines,
  colspec={X[1,c]X[1,c]X[1.5,l]X[1.5,l]X[2,l]X[1.5,l]X[1.5,l]X[0.7,c]},
  rows={m},
  row{1}={font=\bfseries, c, bg=gray9}
}
Mã TC & Mã kịch bản & Mô tả & Dữ liệu kiểm thử & Các bước thực hiện & Kết quả mong đợi & Kết quả thực tế & Kết quả \\
TC\_\-LOGIN\_\-01 & TS\_\-LOGIN & Kiểm thử đăng nhập với thông tin hợp lệ & Email: nguyenvana\-@gmail.com\newline Mật khẩu: Password123! & 1. Nhập email\newline 2. Nhập mật khẩu\newline 3. Nhấn Đăng nhập & Chuyển đến trang Dashboard & Chuyển đến trang Dashboard & Pass \\
TC\_\-LOGIN\_\-02 & TS\_\-LOGIN & Kiểm thử đăng nhập với email đúng, mật khẩu sai & Email: nguyenvana\-@gmail.com\newline Mật khẩu: WrongPass123! & 1. Nhập email\newline 2. Nhập mật khẩu sai\newline 3. Nhấn Đăng nhập & Hiển thị thông báo: "Email hoặc mật khẩu không đúng!" & Hiển thị thông báo: "Email hoặc mật khẩu không đúng!" & Pass \\
TC\_\-LOGIN\_\-03 & TS\_\-LOGIN & Kiểm thử đăng nhập với email không tồn tại & Email: notexist\-@gmail.com\newline Mật khẩu: Password123! & 1. Nhập email không tồn tại\newline 2. Nhập mật khẩu\newline 3. Nhấn Đăng nhập & Hiển thị thông báo: "Email hoặc mật khẩu không đúng!" & Hiển thị thông báo: "Email hoặc mật khẩu không đúng!" & Pass \\
TC\_\-LOGIN\_\-04 & TS\_\-LOGIN & Kiểm thử đăng nhập với tài khoản bị khóa & Email: locked\-@gmail.com\newline Mật khẩu: Password123! & 1. Nhập email tài khoản bị khóa\newline 2. Nhập mật khẩu\newline 3. Nhấn Đăng nhập & Hiển thị thông báo: "Tài khoản của bạn đã bị khóa!" & Hiển thị thông báo: "Tài khoản của bạn đã bị khóa!" & Pass \\
\end{longtblr}

\begin{longtblr}[
  caption = {Kiểm tra chức năng đăng xuất},
  label = {tab:tc_logout}
]{
  width=\linewidth, hlines, vlines,
  colspec={X[1,c]X[1,c]X[1.5,l]X[1.5,l]X[2,l]X[1.5,l]X[1.5,l]X[0.7,c]},
  rows={m},
  row{1}={font=\bfseries, c, bg=gray9}
}
Mã TC & Mã kịch bản & Mô tả & Dữ liệu kiểm thử & Các bước thực hiện & Kết quả mong đợi & Kết quả thực tế & Kết quả \\
TC\_\-LOGOUT\_\-01 & TS\_\-LOGOUT & Kiểm thử đăng xuất thành công & User đã đăng nhập & 1. Nhấn vào avatar\newline 2. Chọn "Đăng xuất"\newline 3. Xác nhận đăng xuất & Chuyển đến trang đăng nhập & Chuyển đến trang đăng nhập & Pass \\
TC\_\-LOGOUT\_\-02 & TS\_\-LOGOUT & Kiểm thử truy cập trang bảo vệ sau khi đăng xuất & User đã đăng xuất & 1. Đăng xuất\newline 2. Truy cập URL dashboard & Chuyển hướng về trang đăng nhập & Chuyển hướng về trang đăng nhập & Pass \\
\end{longtblr}

\begin{longtblr}[
  caption = {Kiểm tra chức năng tạo Workspace},
  label = {tab:tc_create_workspace}
]{
  width=\linewidth, hlines, vlines,
  colspec={X[1,c]X[1,c]X[1.5,l]X[1.5,l]X[2,l]X[1.5,l]X[1.5,l]X[0.7,c]},
  rows={m},
  row{1}={font=\bfseries, c, bg=gray9}
}
Mã TC & Mã kịch bản & Mô tả & Dữ liệu kiểm thử & Các bước thực hiện & Kết quả mong đợi & Kết quả thực tế & Kết quả \\
TC\_\-CWS\_\-01 & TS\_\-CWS & Kiểm thử tạo workspace với thông tin hợp lệ & Tên: Marketing Team\newline Mô tả: Workspace cho team marketing & 1. Nhấn "Tạo Workspace"\newline 2. Nhập tên\newline 3. Nhập mô tả\newline 4. Nhấn Tạo & Hiển thị thông báo: "Tạo workspace thành công!" & Hiển thị thông báo: "Tạo workspace thành công!" & Pass \\
TC\_\-CWS\_\-02 & TS\_\-CWS & Kiểm thử tạo workspace với tên trống & Tên: (trống)\newline Mô tả: Test description & 1. Nhấn "Tạo Workspace"\newline 2. Để trống tên\newline 3. Nhấn Tạo & Hiển thị thông báo: "Tên workspace không được để trống!" & Hiển thị thông báo: "Tên workspace không được để trống!" & Pass \\
TC\_\-CWS\_\-03 & TS\_\-CWS & Kiểm thử tạo workspace với tên quá dài & Tên: (255+ ký tự)\newline Mô tả: Test & 1. Nhấn "Tạo Workspace"\newline 2. Nhập tên quá dài\newline 3. Nhấn Tạo & Hiển thị thông báo: "Tên không được vượt quá 255 ký tự!" & Hiển thị thông báo: "Tên không được vượt quá 255 ký tự!" & Pass \\
TC\_\-CWS\_\-04 & TS\_\-CWS & Kiểm thử người tạo được gán làm Owner & Tên: Dev Team\newline Mô tả: Development workspace & 1. Tạo workspace mới\newline 2. Vào trang Members\newline 3. Kiểm tra vai trò & Người tạo có vai trò "Owner" & Người tạo có vai trò "Owner" & Pass \\
\end{longtblr}

% TS_INVITE - Mời thành viên
\begin{longtblr}[
  caption = {Kiểm tra chức năng mời thành viên},
  label = {tab:tc_invite}
]{
  width=\linewidth, hlines, vlines,
  colspec={X[1,c]X[1,c]X[1.5,l]X[1.5,l]X[2,l]X[1.5,l]X[1.5,l]X[0.7,c]},
  rows={m},
  row{1}={font=\bfseries, c, bg=gray9}
}
Mã TC & Mã kịch bản & Mô tả & Dữ liệu kiểm thử & Các bước thực hiện & Kết quả mong đợi & Kết quả thực tế & Kết quả \\
TC\_\-INVITE\_\-01 & TS\_\-INVITE & Kiểm thử mời thành viên với email hợp lệ & Email: member\-@gmail.com\newline Vai trò: Member & 1. Nhấn "Mời thành viên"\newline 2. Nhập email\newline 3. Chọn vai trò\newline 4. Nhấn Gửi & Hiển thị thông báo: "Đã gửi lời mời thành công!" & Hiển thị thông báo: "Đã gửi lời mời thành công!" & Pass \\
TC\_\-INVITE\_\-02 & TS\_\-INVITE & Kiểm thử mời thành viên đã là member & Email: nguyenvana\-@gmail.com\newline Vai trò: Member & 1. Nhấn "Mời thành viên"\newline 2. Nhập email đã là member\newline 3. Nhấn Gửi & Hiển thị thông báo: "Người này đã là thành viên!" & Hiển thị thông báo: "Người này đã là thành viên!" & Pass \\
TC\_\-INVITE\_\-03 & TS\_\-INVITE & Kiểm thử mời với email không hợp lệ & Email: invalid-email\newline Vai trò: Member & 1. Nhấn "Mời thành viên"\newline 2. Nhập email không hợp lệ\newline 3. Nhấn Gửi & Hiển thị thông báo: "Email không hợp lệ!" & Hiển thị thông báo: "Email không hợp lệ!" & Pass \\
TC\_\-INVITE\_\-04 & TS\_\-INVITE & Kiểm thử Member không có quyền mời & User có vai trò Member & 1. Đăng nhập với tài khoản Member\newline 2. Vào trang Members\newline 3. Kiểm tra nút "Mời" & Không hiển thị nút "Mời thành viên" & Không hiển thị nút "Mời thành viên" & Pass \\
\end{longtblr}

% TS_ROLE - Phân quyền
\begin{longtblr}[
  caption = {Kiểm tra chức năng phân quyền thành viên},
  label = {tab:tc_role}
]{
  width=\linewidth, hlines, vlines,
  colspec={X[1,c]X[1,c]X[1.5,l]X[1.5,l]X[2,l]X[1.5,l]X[1.5,l]X[0.7,c]},
  rows={m},
  row{1}={font=\bfseries, c, bg=gray9}
}
Mã TC & Mã kịch bản & Mô tả & Dữ liệu kiểm thử & Các bước thực hiện & Kết quả mong đợi & Kết quả thực tế & Kết quả \\
TC\_\-ROLE\_\-01 & TS\_\-ROLE & Kiểm thử Owner đổi vai trò Member thành Admin & Member: Trần Thị B\newline Vai trò mới: Admin & 1. Chọn thành viên\newline 2. Nhấn "Chỉnh sửa"\newline 3. Đổi vai trò\newline 4. Lưu & Hiển thị thông báo: "Cập nhật vai trò thành công!" & Hiển thị thông báo: "Cập nhật vai trò thành công!" & Pass \\
TC\_\-ROLE\_\-02 & TS\_\-ROLE & Kiểm thử không thể đổi vai trò Owner & Owner: Nguyễn Văn A & 1. Chọn Owner\newline 2. Kiểm tra nút chỉnh sửa vai trò & Không hiển thị tùy chọn đổi vai trò Owner & Không hiển thị tùy chọn đổi vai trò Owner & Pass \\
TC\_\-ROLE\_\-03 & TS\_\-ROLE & Kiểm thử Admin không thể đổi vai trò Admin khác & User: Admin\newline Target: Admin khác & 1. Đăng nhập Admin\newline 2. Chọn Admin khác\newline 3. Kiểm tra quyền chỉnh sửa & Không cho phép chỉnh sửa vai trò Admin khác & Không cho phép chỉnh sửa vai trò Admin khác & Pass \\
\end{longtblr}

% TS_UPLOAD - Upload file
\begin{longtblr}[
  caption = {Kiểm tra chức năng upload tệp tin},
  label = {tab:tc_upload}
]{
  width=\linewidth, hlines, vlines,
  colspec={X[1,c]X[1,c]X[1.5,l]X[1.5,l]X[2,l]X[1.5,l]X[1.5,l]X[0.7,c]},
  rows={m},
  row{1}={font=\bfseries, c, bg=gray9}
}
Mã TC & Mã kịch bản & Mô tả & Dữ liệu kiểm thử & Các bước thực hiện & Kết quả mong đợi & Kết quả thực tế & Kết quả \\
TC\_\-UPLOAD\_\-01 & TS\_\-UPLOAD & Kiểm thử upload file hợp lệ & File: report.pdf\newline Kích thước: 2MB & 1. Nhấn "Upload"\newline 2. Chọn file\newline 3. Xác nhận upload & Hiển thị thông báo: "Upload thành công!" & Hiển thị thông báo: "Upload thành công!" & Pass \\
TC\_\-UPLOAD\_\-02 & TS\_\-UPLOAD & Kiểm thử upload file quá lớn & File: bigfile.zip\newline Kích thước: 150MB & 1. Nhấn "Upload"\newline 2. Chọn file lớn\newline 3. Xác nhận upload & Hiển thị thông báo: "File không được vượt quá 100MB!" & Hiển thị thông báo: "File không được vượt quá 100MB!" & Pass \\
TC\_\-UPLOAD\_\-03 & TS\_\-UPLOAD & Kiểm thử upload file không được hỗ trợ & File: malware.exe\newline Kích thước: 1MB & 1. Nhấn "Upload"\newline 2. Chọn file .exe\newline 3. Xác nhận upload & Hiển thị thông báo: "Loại file không được hỗ trợ!" & Hiển thị thông báo: "Loại file không được hỗ trợ!" & Pass \\
TC\_\-UPLOAD\_\-04 & TS\_\-UPLOAD & Kiểm thử Viewer không có quyền upload & User có vai trò Viewer & 1. Đăng nhập với Viewer\newline 2. Vào trang Files\newline 3. Kiểm tra nút Upload & Không hiển thị nút "Upload" & Không hiển thị nút "Upload" & Pass \\
\end{longtblr}

% TS_DOWNLOAD - Download file
\begin{longtblr}[
  caption = {Kiểm tra chức năng download tệp tin},
  label = {tab:tc_download}
]{
  width=\linewidth, hlines, vlines,
  colspec={X[1,c]X[1,c]X[1.5,l]X[1.5,l]X[2,l]X[1.5,l]X[1.5,l]X[0.7,c]},
  rows={m},
  row{1}={font=\bfseries, c, bg=gray9}
}
Mã TC & Mã kịch bản & Mô tả & Dữ liệu kiểm thử & Các bước thực hiện & Kết quả mong đợi & Kết quả thực tế & Kết quả \\
TC\_\-DOWNLOAD\_\-01 & TS\_\-DOWNLOAD & Kiểm thử download file thành công & File: report.pdf & 1. Chọn file\newline 2. Nhấn nút Download & File được tải xuống & File được tải xuống & Pass \\
TC\_\-DOWNLOAD\_\-02 & TS\_\-DOWNLOAD & Kiểm thử download file đã bị xóa & File đã bị xóa & 1. Truy cập URL file đã xóa & Hiển thị thông báo: "File không tồn tại!" & Hiển thị thông báo: "File không tồn tại!" & Pass \\
TC\_\-DOWNLOAD\_\-03 & TS\_\-DOWNLOAD & Kiểm thử download không có quyền & User không phải member của workspace & 1. Truy cập URL file workspace khác & Hiển thị thông báo: "Bạn không có quyền truy cập!" & Hiển thị thông báo: "Bạn không có quyền truy cập!" & Pass \\
\end{longtblr}

% TS_AIREPORT - Tạo báo cáo AI
\begin{longtblr}[
  caption = {Kiểm tra chức năng tạo báo cáo AI},
  label = {tab:tc_aireport}
]{
  width=\linewidth, hlines, vlines,
  colspec={X[1,c]X[1,c]X[1.5,l]X[1.5,l]X[2,l]X[1.5,l]X[1.5,l]X[0.7,c]},
  rows={m},
  row{1}={font=\bfseries, c, bg=gray9}
}
Mã TC & Mã kịch bản & Mô tả & Dữ liệu kiểm thử & Các bước thực hiện & Kết quả mong đợi & Kết quả thực tế & Kết quả \\
TC\_\-AIREPORT\_\-01 & TS\_\-AIREPORT & Kiểm thử tạo báo cáo AI thành công & Loại: Weekly Digest\newline Provider: OpenAI GPT-4 & 1. Nhấn "Tạo báo cáo"\newline 2. Chọn loại báo cáo\newline 3. Chọn provider\newline 4. Nhấn Tạo & Hiển thị thông báo: "Đang tạo báo cáo..." và sau đó hiển thị báo cáo & Hiển thị thông báo: "Đang tạo báo cáo..." và sau đó hiển thị báo cáo & Pass \\
TC\_\-AIREPORT\_\-02 & TS\_\-AIREPORT & Kiểm thử tạo báo cáo với prompt tùy chỉnh & Loại: Custom\newline Prompt: "Tóm tắt hoạt động tuần này" & 1. Nhấn "Tạo báo cáo"\newline 2. Chọn Custom\newline 3. Nhập prompt\newline 4. Nhấn Tạo & Báo cáo được tạo theo prompt & Báo cáo được tạo theo prompt & Pass \\
TC\_\-AIREPORT\_\-03 & TS\_\-AIREPORT & Kiểm thử Member không có quyền tạo báo cáo & User có vai trò Member & 1. Đăng nhập Member\newline 2. Vào trang AI Reports\newline 3. Kiểm tra nút Tạo & Không hiển thị nút "Tạo báo cáo" & Không hiển thị nút "Tạo báo cáo" & Pass \\
TC\_\-AIREPORT\_\-04 & TS\_\-AIREPORT & Kiểm thử xuất báo cáo ra PDF & Báo cáo đã tạo & 1. Chọn báo cáo\newline 2. Nhấn Export\newline 3. Chọn PDF & File PDF được tải xuống & File PDF được tải xuống & Pass \\
\end{longtblr}

% TS_NOTIF - Thông báo
\begin{longtblr}[
  caption = {Kiểm tra chức năng xem thông báo},
  label = {tab:tc_notification}
]{
  width=\linewidth, hlines, vlines,
  colspec={X[1,c]X[1,c]X[1.5,l]X[1.5,l]X[2,l]X[1.5,l]X[1.5,l]X[0.7,c]},
  rows={m},
  row{1}={font=\bfseries, c, bg=gray9}
}
Mã TC & Mã kịch bản & Mô tả & Dữ liệu kiểm thử & Các bước thực hiện & Kết quả mong đợi & Kết quả thực tế & Kết quả \\
TC\_\-NOTIF\_\-01 & TS\_\-NOTIF & Kiểm thử hiển thị danh sách thông báo & User có thông báo & 1. Nhấn vào icon thông báo & Hiển thị dropdown danh sách thông báo & Hiển thị dropdown danh sách thông báo & Pass \\
TC\_\-NOTIF\_\-02 & TS\_\-NOTIF & Kiểm thử đánh dấu đã đọc & Thông báo chưa đọc & 1. Nhấn vào thông báo chưa đọc & Thông báo được đánh dấu đã đọc & Thông báo được đánh dấu đã đọc & Pass \\
TC\_\-NOTIF\_\-03 & TS\_\-NOTIF & Kiểm thử đánh dấu tất cả đã đọc & Nhiều thông báo chưa đọc & 1. Nhấn "Đánh dấu tất cả đã đọc" & Tất cả thông báo được đánh dấu đã đọc, badge về 0 & Tất cả thông báo được đánh dấu đã đọc, badge về 0 & Pass \\
\end{longtblr}

% TS_PROFILE - Cập nhật profile
\begin{longtblr}[
  caption = {Kiểm tra chức năng cập nhật thông tin cá nhân},
  label = {tab:tc_profile}
]{
  width=\linewidth, hlines, vlines,
  colspec={X[1,c]X[1,c]X[1.5,l]X[1.5,l]X[2,l]X[1.5,l]X[1.5,l]X[0.7,c]},
  rows={m},
  row{1}={font=\bfseries, c, bg=gray9}
}
Mã TC & Mã kịch bản & Mô tả & Dữ liệu kiểm thử & Các bước thực hiện & Kết quả mong đợi & Kết quả thực tế & Kết quả \\
TC\_\-PROFILE\_\-01 & TS\_\-PROFILE & Kiểm thử cập nhật họ tên thành công & Họ tên mới: Nguyễn Văn B & 1. Vào trang Profile\newline 2. Nhấn Chỉnh sửa\newline 3. Đổi họ tên\newline 4. Lưu & Hiển thị thông báo: "Cập nhật thành công!" & Hiển thị thông báo: "Cập nhật thành công!" & Pass \\
TC\_\-PROFILE\_\-02 & TS\_\-PROFILE & Kiểm thử upload avatar & File: avatar.jpg\newline Kích thước: 500KB & 1. Vào trang Profile\newline 2. Nhấn đổi avatar\newline 3. Chọn file\newline 4. Lưu & Avatar được cập nhật & Avatar được cập nhật & Pass \\
TC\_\-PROFILE\_\-03 & TS\_\-PROFILE & Kiểm thử cập nhật số điện thoại không hợp lệ & SĐT: abc123 & 1. Vào trang Profile\newline 2. Nhập SĐT không hợp lệ\newline 3. Lưu & Hiển thị thông báo: "Số điện thoại không hợp lệ!" & Hiển thị thông báo: "Số điện thoại không hợp lệ!" & Pass \\
\end{longtblr}

% TS_ADMIN - Admin functions
\begin{longtblr}[
  caption = {Kiểm tra chức năng Admin khóa/mở khóa workspace},
  label = {tab:tc_admin}
]{
  width=\linewidth, hlines, vlines,
  colspec={X[1,c]X[1,c]X[1.5,l]X[1.5,l]X[2,l]X[1.5,l]X[1.5,l]X[0.7,c]},
  rows={m},
  row{1}={font=\bfseries, c, bg=gray9}
}
Mã TC & Mã kịch bản & Mô tả & Dữ liệu kiểm thử & Các bước thực hiện & Kết quả mong đợi & Kết quả thực tế & Kết quả \\
TC\_\-ADMIN\_\-01 & TS\_\-ADMIN & Kiểm thử Super Admin khóa workspace & Workspace: Marketing Team\newline Lý do: Vi phạm chính sách & 1. Đăng nhập Super Admin\newline 2. Vào Admin Panel\newline 3. Chọn workspace\newline 4. Nhấn Khóa\newline 5. Nhập lý do & Hiển thị thông báo: "Đã khóa workspace!" & Hiển thị thông báo: "Đã khóa workspace!" & Pass \\
TC\_\-ADMIN\_\-02 & TS\_\-ADMIN & Kiểm thử Super Admin mở khóa workspace & Workspace đang bị khóa & 1. Đăng nhập Super Admin\newline 2. Vào Admin Panel\newline 3. Chọn workspace bị khóa\newline 4. Nhấn Mở khóa & Hiển thị thông báo: "Đã mở khóa workspace!" & Hiển thị thông báo: "Đã mở khóa workspace!" & Pass \\
TC\_\-ADMIN\_\-03 & TS\_\-ADMIN & Kiểm thử user thường không truy cập được Admin Panel & User thường & 1. Đăng nhập user thường\newline 2. Truy cập URL /admin & Chuyển hướng về trang 403 hoặc Dashboard & Chuyển hướng về trang 403 hoặc Dashboard & Pass \\
\end{longtblr}

% ==================== PHÂN HỆ TRUYỀN THÔNG ====================

\subsection*{Các trường hợp kiểm thử phân hệ Truyền thông}

% TS_CREATE_CH - Tạo kênh
\begin{longtblr}[
  caption = {Kiểm tra chức năng tạo kênh mới},
  label = {tab:tc_create_channel}
]{
  width=\linewidth, hlines, vlines,
  colspec={X[1,c]X[1,c]X[1.5,l]X[1.5,l]X[2,l]X[1.5,l]X[1.5,l]X[0.7,c]},
  rows={m},
  row{1}={font=\bfseries, c, bg=gray9}
}
Mã TC & Mã kịch bản & Mô tả & Dữ liệu kiểm thử & Các bước thực hiện & Kết quả mong đợi & Kết quả thực tế & Kết quả \\
TC\_\-CH\_\-01 & TS\_\-CREATE\_\-CH & Kiểm thử tạo kênh public với thông tin hợp lệ & Tên: general\newline Loại: Public\newline Mô tả: Kênh chung & 1. Nhấn "Tạo kênh"\newline 2. Nhập tên kênh\newline 3. Chọn loại Public\newline 4. Nhập mô tả\newline 5. Nhấn Tạo & Hiển thị thông báo: "Tạo kênh thành công!" & Hiển thị thông báo: "Tạo kênh thành công!" & Pass \\
TC\_\-CH\_\-02 & TS\_\-CREATE\_\-CH & Kiểm thử tạo kênh private & Tên: team-lead\newline Loại: Private & 1. Nhấn "Tạo kênh"\newline 2. Nhập tên kênh\newline 3. Chọn loại Private\newline 4. Nhấn Tạo & Kênh được tạo với biểu tượng khóa & Kênh được tạo với biểu tượng khóa & Pass \\
TC\_\-CH\_\-03 & TS\_\-CREATE\_\-CH & Kiểm thử tạo kênh với tên trống & Tên: (trống)\newline Loại: Public & 1. Nhấn "Tạo kênh"\newline 2. Để trống tên\newline 3. Nhấn Tạo & Hiển thị thông báo: "Tên kênh không được để trống!" & Hiển thị thông báo: "Tên kênh không được để trống!" & Pass \\
TC\_\-CH\_\-04 & TS\_\-CREATE\_\-CH & Kiểm thử tạo kênh trùng tên & Tên: general (đã tồn tại) & 1. Nhấn "Tạo kênh"\newline 2. Nhập tên kênh đã tồn tại\newline 3. Nhấn Tạo & Hiển thị thông báo: "Tên kênh đã tồn tại!" & Hiển thị thông báo: "Tên kênh đã tồn tại!" & Pass \\
TC\_\-CH\_\-05 & TS\_\-CREATE\_\-CH & Kiểm thử Member không có quyền tạo kênh & User có vai trò Member & 1. Đăng nhập với Member\newline 2. Kiểm tra nút "Tạo kênh" & Không hiển thị nút "Tạo kênh" & Không hiển thị nút "Tạo kênh" & Pass \\
\end{longtblr}

% TS_JOIN_CH - Tham gia kênh
\begin{longtblr}[
  caption = {Kiểm tra chức năng tham gia kênh},
  label = {tab:tc_join_channel}
]{
  width=\linewidth, hlines, vlines,
  colspec={X[1,c]X[1,c]X[1.5,l]X[1.5,l]X[2,l]X[1.5,l]X[1.5,l]X[0.7,c]},
  rows={m},
  row{1}={font=\bfseries, c, bg=gray9}
}
Mã TC & Mã kịch bản & Mô tả & Dữ liệu kiểm thử & Các bước thực hiện & Kết quả mong đợi & Kết quả thực tế & Kết quả \\
TC\_\-JOIN\_\-01 & TS\_\-JOIN\_\-CH & Kiểm thử tham gia kênh public & Kênh: general (public) & 1. Mở danh sách kênh\newline 2. Chọn kênh public\newline 3. Nhấn "Tham gia" & User được thêm vào kênh & User được thêm vào kênh & Pass \\
TC\_\-JOIN\_\-02 & TS\_\-JOIN\_\-CH & Kiểm thử không thể tham gia kênh private chưa được mời & Kênh: team-lead (private) & 1. Truy cập kênh private\newline 2. Không có lời mời & Hiển thị thông báo: "Bạn không có quyền truy cập!" & Hiển thị thông báo: "Bạn không có quyền truy cập!" & Pass \\
TC\_\-JOIN\_\-03 & TS\_\-JOIN\_\-CH & Kiểm thử tham gia kênh đã là thành viên & Kênh đã tham gia & 1. Mở kênh đã tham gia\newline 2. Kiểm tra nút Tham gia & Không hiển thị nút "Tham gia" & Không hiển thị nút "Tham gia" & Pass \\
TC\_\-JOIN\_\-04 & TS\_\-JOIN\_\-CH & Kiểm thử tham gia kênh qua lời mời & Lời mời vào kênh private & 1. Nhận lời mời\newline 2. Nhấn "Chấp nhận" & User được thêm vào kênh private & User được thêm vào kênh private & Pass \\
\end{longtblr}

% TS_SEND_MSG - Gửi tin nhắn
\begin{longtblr}[
  caption = {Kiểm tra chức năng gửi tin nhắn},
  label = {tab:tc_send_message}
]{
  width=\linewidth, hlines, vlines,
  colspec={X[1,c]X[1,c]X[1.5,l]X[1.5,l]X[2,l]X[1.5,l]X[1.5,l]X[0.7,c]},
  rows={m},
  row{1}={font=\bfseries, c, bg=gray9}
}
Mã TC & Mã kịch bản & Mô tả & Dữ liệu kiểm thử & Các bước thực hiện & Kết quả mong đợi & Kết quả thực tế & Kết quả \\
TC\_\-MSG\_\-01 & TS\_\-SEND\_\-MSG & Kiểm thử gửi tin nhắn văn bản & Nội dung: "Xin chào mọi người!" & 1. Mở kênh\newline 2. Nhập tin nhắn\newline 3. Nhấn Gửi & Tin nhắn hiển thị trong kênh real-time (< 500ms) & Tin nhắn hiển thị trong kênh real-time (< 500ms) & Pass \\
TC\_\-MSG\_\-02 & TS\_\-SEND\_\-MSG & Kiểm thử gửi tin nhắn với file đính kèm & File: document.pdf\newline Nội dung: "Đây là tài liệu" & 1. Nhấn icon đính kèm\newline 2. Chọn file\newline 3. Nhập tin nhắn\newline 4. Nhấn Gửi & Tin nhắn và file được hiển thị & Tin nhắn và file được hiển thị & Pass \\
TC\_\-MSG\_\-03 & TS\_\-SEND\_\-MSG & Kiểm thử mention người dùng & Nội dung: "@john cần review" & 1. Nhập @\newline 2. Chọn user từ danh sách\newline 3. Hoàn thành tin nhắn\newline 4. Nhấn Gửi & Tin nhắn với mention được highlight, user nhận thông báo & Tin nhắn với mention được highlight, user nhận thông báo & Pass \\
TC\_\-MSG\_\-04 & TS\_\-SEND\_\-MSG & Kiểm thử gửi tin nhắn trống & Nội dung: (trống) & 1. Không nhập gì\newline 2. Nhấn Gửi & Nút Gửi bị vô hiệu hóa & Nút Gửi bị vô hiệu hóa & Pass \\
TC\_\-MSG\_\-05 & TS\_\-SEND\_\-MSG & Kiểm thử real-time đồng bộ & 2 users cùng kênh & 1. User A gửi tin\newline 2. Kiểm tra User B & User B thấy tin nhắn ngay lập tức & User B thấy tin nhắn ngay lập tức & Pass \\
TC\_\-MSG\_\-06 & TS\_\-SEND\_\-MSG & Kiểm thử gửi tin với emoji & Nội dung: "Tuyệt vời!" kèm emoji & 1. Nhập tin với emoji\newline 2. Nhấn Gửi & Emoji hiển thị đúng & Emoji hiển thị đúng & Pass \\
\end{longtblr}

% TS_CREATE_THR - Tạo thread
\begin{longtblr}[
  caption = {Kiểm tra chức năng tạo thread thảo luận},
  label = {tab:tc_create_thread}
]{
  width=\linewidth, hlines, vlines,
  colspec={X[1,c]X[1,c]X[1.5,l]X[1.5,l]X[2,l]X[1.5,l]X[1.5,l]X[0.7,c]},
  rows={m},
  row{1}={font=\bfseries, c, bg=gray9}
}
Mã TC & Mã kịch bản & Mô tả & Dữ liệu kiểm thử & Các bước thực hiện & Kết quả mong đợi & Kết quả thực tế & Kết quả \\
TC\_\-THR\_\-01 & TS\_\-CREATE\_\-THR & Kiểm thử tạo thread từ tin nhắn & Tin nhắn gốc đã có & 1. Hover vào tin nhắn\newline 2. Nhấn "Trả lời trong thread"\newline 3. Nhập nội dung\newline 4. Gửi & Thread được tạo, hiển thị panel bên phải & Thread được tạo, hiển thị panel bên phải & Pass \\
TC\_\-THR\_\-02 & TS\_\-CREATE\_\-THR & Kiểm thử thread hiển thị số replies & Thread có 3 replies & 1. Xem tin nhắn có thread\newline 2. Kiểm tra badge & Hiển thị "3 replies" dưới tin nhắn & Hiển thị "3 replies" dưới tin nhắn & Pass \\
TC\_\-THR\_\-03 & TS\_\-CREATE\_\-THR & Kiểm thử đóng/mở panel thread & Thread đang mở & 1. Nhấn X để đóng\newline 2. Nhấn vào thread badge để mở & Panel đóng/mở đúng & Panel đóng/mở đúng & Pass \\
TC\_\-THR\_\-04 & TS\_\-CREATE\_\-THR & Kiểm thử thread không làm loãng kênh chính & Tin nhắn trong thread & 1. Trả lời trong thread\newline 2. Kiểm tra kênh chính & Tin nhắn thread không xuất hiện trong kênh chính & Tin nhắn thread không xuất hiện trong kênh chính & Pass \\
\end{longtblr}

% TS_SEARCH_MSG - Tìm kiếm tin nhắn
\begin{longtblr}[
  caption = {Kiểm tra chức năng tìm kiếm tin nhắn},
  label = {tab:tc_search_message}
]{
  width=\linewidth, hlines, vlines,
  colspec={X[1,c]X[1,c]X[1.5,l]X[1.5,l]X[2,l]X[1.5,l]X[1.5,l]X[0.7,c]},
  rows={m},
  row{1}={font=\bfseries, c, bg=gray9}
}
Mã TC & Mã kịch bản & Mô tả & Dữ liệu kiểm thử & Các bước thực hiện & Kết quả mong đợi & Kết quả thực tế & Kết quả \\
TC\_\-SEARCH\_\-01 & TS\_\-SEARCH\_\-MSG & Kiểm thử tìm kiếm full-text & Từ khóa: "báo cáo" & 1. Nhấn icon tìm kiếm\newline 2. Nhập từ khóa\newline 3. Nhấn Enter & Hiển thị danh sách tin nhắn chứa "báo cáo" & Hiển thị danh sách tin nhắn chứa "báo cáo" & Pass \\
TC\_\-SEARCH\_\-02 & TS\_\-SEARCH\_\-MSG & Kiểm thử tìm kiếm semantic & Câu hỏi: "deadline dự án" & 1. Chọn chế độ Semantic\newline 2. Nhập câu hỏi\newline 3. Nhấn Enter & Hiển thị tin nhắn liên quan đến deadline, kể cả không chứa từ khóa chính xác & Hiển thị tin nhắn liên quan đến deadline, kể cả không chứa từ khóa chính xác & Pass \\
TC\_\-SEARCH\_\-03 & TS\_\-SEARCH\_\-MSG & Kiểm thử tìm kiếm không có kết quả & Từ khóa: "xyz123abc" & 1. Nhập từ khóa không tồn tại\newline 2. Nhấn Enter & Hiển thị: "Không tìm thấy kết quả" & Hiển thị: "Không tìm thấy kết quả" & Pass \\
TC\_\-SEARCH\_\-04 & TS\_\-SEARCH\_\-MSG & Kiểm thử filter theo kênh & Từ khóa: "meeting"\newline Filter: general & 1. Nhập từ khóa\newline 2. Chọn filter kênh\newline 3. Tìm kiếm & Chỉ hiển thị kết quả từ kênh general & Chỉ hiển thị kết quả từ kênh general & Pass \\
TC\_\-SEARCH\_\-05 & TS\_\-SEARCH\_\-MSG & Kiểm thử click vào kết quả để đến tin nhắn & Kết quả tìm kiếm & 1. Tìm kiếm\newline 2. Click vào kết quả & Chuyển đến kênh và highlight tin nhắn & Chuyển đến kênh và highlight tin nhắn & Pass \\
\end{longtblr}

% TS_AI_QA - Hỏi đáp AI (RAG)
\begin{longtblr}[
  caption = {Kiểm tra chức năng hỏi đáp AI (RAG)},
  label = {tab:tc_ai_qa}
]{
  width=\linewidth, hlines, vlines,
  colspec={X[1,c]X[1,c]X[1.5,l]X[1.5,l]X[2,l]X[1.5,l]X[1.5,l]X[0.7,c]},
  rows={m},
  row{1}={font=\bfseries, c, bg=gray9}
}
Mã TC & Mã kịch bản & Mô tả & Dữ liệu kiểm thử & Các bước thực hiện & Kết quả mong đợi & Kết quả thực tế & Kết quả \\
TC\_\-AI\_\-QA\_\-01 & TS\_\-AI\_\-QA & Kiểm thử hỏi AI câu hỏi về nội dung kênh & Câu hỏi: "Deadline của sprint 5 là khi nào?" & 1. Mở AI Assistant\newline 2. Nhập câu hỏi\newline 3. Nhấn Gửi & AI trả lời dựa trên context kênh, có trích dẫn nguồn & AI trả lời dựa trên context kênh, có trích dẫn nguồn & Pass \\
TC\_\-AI\_\-QA\_\-02 & TS\_\-AI\_\-QA & Kiểm thử AI trích dẫn nguồn chính xác & Câu hỏi có context rõ ràng & 1. Hỏi AI\newline 2. Kiểm tra nguồn trích dẫn & Hiển thị link đến tin nhắn gốc & Hiển thị link đến tin nhắn gốc & Pass \\
TC\_\-AI\_\-QA\_\-03 & TS\_\-AI\_\-QA & Kiểm thử AI khi không có context phù hợp & Câu hỏi: "Thời tiết hôm nay thế nào?" & 1. Hỏi câu không liên quan\newline 2. Kiểm tra response & AI thông báo không tìm thấy thông tin liên quan & AI thông báo không tìm thấy thông tin liên quan & Pass \\
TC\_\-AI\_\-QA\_\-04 & TS\_\-AI\_\-QA & Kiểm thử hiệu năng response & Câu hỏi thông thường & 1. Hỏi AI\newline 2. Đo thời gian response & Response trong vòng 5 giây & Response trong vòng 5 giây & Pass \\
TC\_\-AI\_\-QA\_\-05 & TS\_\-AI\_\-QA & Kiểm thử hỏi follow-up & Câu hỏi tiếp theo liên quan & 1. Hỏi câu đầu\newline 2. Hỏi câu follow-up & AI hiểu context từ câu hỏi trước & AI hiểu context từ câu hỏi trước & Pass \\
\end{longtblr}

% TS_AI_SUMMARY - Tóm tắt hội thoại
\begin{longtblr}[
  caption = {Kiểm tra chức năng tóm tắt hội thoại},
  label = {tab:tc_ai_summary}
]{
  width=\linewidth, hlines, vlines,
  colspec={X[1,c]X[1,c]X[1.5,l]X[1.5,l]X[2,l]X[1.5,l]X[1.5,l]X[0.7,c]},
  rows={m},
  row{1}={font=\bfseries, c, bg=gray9}
}
Mã TC & Mã kịch bản & Mô tả & Dữ liệu kiểm thử & Các bước thực hiện & Kết quả mong đợi & Kết quả thực tế & Kết quả \\
TC\_\-SUM\_\-01 & TS\_\-AI\_\-SUM\-MARY & Kiểm thử tóm tắt hội thoại từ lần truy cập cuối & Kênh có 50+ tin nhắn mới & 1. Vào kênh có tin nhắn mới\newline 2. Nhấn "Tóm tắt" & Hiển thị bản tóm tắt các điểm chính & Hiển thị bản tóm tắt các điểm chính & Pass \\
TC\_\-SUM\_\-02 & TS\_\-AI\_\-SUM\-MARY & Kiểm thử tóm tắt thread dài & Thread có 20+ replies & 1. Mở thread dài\newline 2. Nhấn "Tóm tắt thread" & Hiển thị tóm tắt nội dung thread & Hiển thị tóm tắt nội dung thread & Pass \\
TC\_\-SUM\_\-03 & TS\_\-AI\_\-SUM\-MARY & Kiểm thử tóm tắt kênh không có tin mới & Kênh đã đọc hết & 1. Vào kênh đã đọc\newline 2. Nhấn "Tóm tắt" & Thông báo: "Không có tin nhắn mới để tóm tắt" & Thông báo: "Không có tin nhắn mới để tóm tắt" & Pass \\
TC\_\-SUM\_\-04 & TS\_\-AI\_\-SUM\-MARY & Kiểm thử format tóm tắt & Hội thoại phức tạp & 1. Tóm tắt hội thoại\newline 2. Kiểm tra format & Tóm tắt có cấu trúc rõ ràng với bullet points & Tóm tắt có cấu trúc rõ ràng với bullet points & Pass \\
\end{longtblr}

% TS_EXTRACT_ACT - Trích xuất action items
\begin{longtblr}[
  caption = {Kiểm tra chức năng trích xuất action items},
  label = {tab:tc_extract_actions}
]{
  width=\linewidth, hlines, vlines,
  colspec={X[1,c]X[1,c]X[1.5,l]X[1.5,l]X[2,l]X[1.5,l]X[1.5,l]X[0.7,c]},
  rows={m},
  row{1}={font=\bfseries, c, bg=gray9}
}
Mã TC & Mã kịch bản & Mô tả & Dữ liệu kiểm thử & Các bước thực hiện & Kết quả mong đợi & Kết quả thực tế & Kết quả \\
TC\_\-ACT\_\-01 & TS\_\-EX\-TRACT\_\-ACT & Kiểm thử trích xuất action items từ hội thoại & Hội thoại chứa "@john cần hoàn thành báo cáo" & 1. Nhấn "Trích xuất Actions"\newline 2. Chọn khoảng thời gian & Hiển thị danh sách action items với người phụ trách & Hiển thị danh sách action items với người phụ trách & Pass \\
TC\_\-ACT\_\-02 & TS\_\-EX\-TRACT\_\-ACT & Kiểm thử trích xuất với mention người dùng & Tin nhắn: "@mary review PR \#123 trước thứ 6" & 1. Trích xuất actions\newline 2. Kiểm tra kết quả & Action item gán cho mary, có deadline & Action item gán cho mary, có deadline & Pass \\
TC\_\-ACT\_\-03 & TS\_\-EX\-TRACT\_\-ACT & Kiểm thử hội thoại không có action items & Hội thoại casual không có task & 1. Trích xuất actions\newline 2. Kiểm tra kết quả & Thông báo: "Không tìm thấy action items" & Thông báo: "Không tìm thấy action items" & Pass \\
\end{longtblr}

% TS_SUMMARIZE_DOC - Tóm tắt tài liệu
\begin{longtblr}[
  caption = {Kiểm tra chức năng tóm tắt tài liệu đính kèm},
  label = {tab:tc_summarize_doc}
]{
  width=\linewidth, hlines, vlines,
  colspec={X[1,c]X[1,c]X[1.5,l]X[1.5,l]X[2,l]X[1.5,l]X[1.5,l]X[0.7,c]},
  rows={m},
  row{1}={font=\bfseries, c, bg=gray9}
}
Mã TC & Mã kịch bản & Mô tả & Dữ liệu kiểm thử & Các bước thực hiện & Kết quả mong đợi & Kết quả thực tế & Kết quả \\
TC\_\-DOC\_\-01 & TS\_\-SUM\-MA\-RIZE\_\-DOC & Kiểm thử tóm tắt file PDF & File: report.pdf (10 trang) & 1. Upload PDF\newline 2. Nhấn "Tóm tắt"\newline 3. Chờ xử lý & Hiển thị bản tóm tắt nội dung PDF & Hiển thị bản tóm tắt nội dung PDF & Pass \\
TC\_\-DOC\_\-02 & TS\_\-SUM\-MA\-RIZE\_\-DOC & Kiểm thử tóm tắt file DOCX & File: proposal.docx & 1. Upload DOCX\newline 2. Nhấn "Tóm tắt" & Hiển thị bản tóm tắt nội dung DOCX & Hiển thị bản tóm tắt nội dung DOCX & Pass \\
TC\_\-DOC\_\-03 & TS\_\-SUM\-MA\-RIZE\_\-DOC & Kiểm thử tóm tắt file không hỗ trợ & File: image.png & 1. Upload file ảnh\newline 2. Kiểm tra nút tóm tắt & Không hiển thị nút "Tóm tắt" cho file ảnh & Không hiển thị nút "Tóm tắt" cho file ảnh & Pass \\
TC\_\-DOC\_\-04 & TS\_\-SUM\-MA\-RIZE\_\-DOC & Kiểm thử tài liệu đã được index cho RAG & File PDF đã tóm tắt & 1. Tóm tắt tài liệu\newline 2. Hỏi AI về nội dung & AI trả lời dựa trên nội dung tài liệu & AI trả lời dựa trên nội dung tài liệu & Pass \\
\end{longtblr}

\end{landscape}