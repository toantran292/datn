\thispagestyle{empty}
\setsection{ABSTRACT}

\sloppy

\noindent Context: In Agile software development environments, project teams frequently exchange information through messages, documents, and various file types. However, these communications are often scattered across multiple platforms (Slack, Discord, email...), making it difficult to track conversation history and documents centrally. With the large volume of content generated daily, reading through all messages to capture information is time-consuming. AI is increasingly recognized as a tool for summarizing conversations and answering questions based on context.

\noindent Objective: Build a communication subsystem for an Agile project management SaaS platform, focusing on messaging and AI integration. The subsystem enables message exchange by channel/project, attachment management, while leveraging AI to summarize conversations, analyze files, and provide context-based responses.

\noindent Methodology: The system is designed with microservice architecture, where the communication subsystem is an independent service managing channels, messages, attachments, and AI interactions. Backend uses NodeJS/NestJS to build RESTful APIs. Frontend uses React/Next.js with real-time chat interface. AI service is integrated via external APIs for conversation summarization, information extraction, and context-aware question answering.

\noindent Results: The completed subsystem allows users to create and manage project-linked communication channels, send messages and attachments directly in conversations. AI automatically summarizes long conversations, extracts key points, analyzes uploaded files, and answers questions within the chat interface. Users can grasp important information without reading through entire conversation histories.

\noindent Conclusion: The centralized communication subsystem with AI integration significantly reduces the burden of manual information reading and filtering, enhancing context awareness and decision-making support for Agile teams. The microservice architecture enables easy scalability and integration with other subsystems in the future.

\fussy
