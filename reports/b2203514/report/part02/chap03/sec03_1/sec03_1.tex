\subsection{Tổng quan hệ thống}

Trong bối cảnh phát triển phần mềm Agile hiện nay, các nhóm dự án thường sử dụng nhiều công cụ giao tiếp khác nhau như Slack, Discord, Microsoft Teams, email, và các ứng dụng nhắn tin khác. Sự phân tán này dẫn đến việc thông tin bị rải rác, lịch sử thảo luận khó theo dõi, và các thành viên mới khó nắm bắt ngữ cảnh dự án. Khi một cuộc họp quan trọng diễn ra hoặc các quyết định được đưa ra trong chat, việc tìm lại thông tin trở nên phức tạp và tốn thời gian.

Đáp ứng nhu cầu đó, "Phân hệ truyền thông" được phát triển như một thành phần của nền tảng SaaS quản lý dự án Agile, cung cấp hệ thống nhắn tin tập trung theo dự án với khả năng tích hợp AI. Phân hệ cho phép các nhóm tạo kênh trò chuyện theo dự án hoặc chủ đề, gửi nhận tin nhắn real-time, đính kèm tệp tài liệu, và quan trọng nhất là sử dụng AI để tóm tắt hội thoại, trích xuất công việc, và trả lời câu hỏi dựa trên lịch sử trao đổi.

Điểm nổi bật của phân hệ so với các ứng dụng nhắn tin thông thường là khả năng tích hợp AI với kiến trúc RAG (Retrieval-Augmented Generation). Thay vì phải đọc lại hàng trăm tin nhắn để nắm bắt context, thành viên có thể yêu cầu AI tóm tắt cuộc thảo luận từ lần truy cập cuối, liệt kê các action items được đề cập, hoặc đặt câu hỏi như "Chúng ta đã quyết định gì về API design?" và nhận câu trả lời chính xác với nguồn tham chiếu đến tin nhắn gốc. AI cũng có thể tóm tắt nội dung các tài liệu đính kèm, giúp tiết kiệm thời gian đọc tài liệu dài.

Về mặt kỹ thuật, phân hệ được xây dựng theo kiến trúc Microservices với các thành phần chính: Next.js và React cho frontend với giao diện chat real-time, NestJS cho backend services (Messaging Service, WebSocket Gateway, Document Processing Service, AI Service), Socket.io cho giao tiếp WebSocket real-time, PostgreSQL với pgvector extension để lưu trữ dữ liệu và vector embeddings, Redis cho caching và Socket.io adapter hỗ trợ horizontal scaling, BullMQ cho xử lý background jobs, và OpenAI API cho embeddings và LLM completions. Phân hệ tích hợp với File Service và Account Service từ phân hệ nền tảng để lưu trữ tệp và xác thực người dùng.

Việc xây dựng một phân hệ truyền thông tích hợp AI đòi hỏi thiết kế chuyên biệt từ giao diện chat trực quan, kiến trúc WebSocket cho real-time messaging, pipeline RAG cho AI capabilities, đến khả năng mở rộng khi số lượng người dùng và tin nhắn tăng. Sự chuyên biệt hóa này tạo nên giá trị cho ứng dụng, giúp các nhóm phát triển Agile giao tiếp hiệu quả hơn và giảm gánh nặng nắm bắt thông tin trong môi trường làm việc nhịp độ cao.