% ==================== NHÓM QUẢN LÝ KÊNH ====================

% 3.5.1 Tạo kênh mới
\subsubsection{Chức năng tạo kênh mới}

Mục đích: Cho phép người dùng tạo kênh truyền thông mới trong work\-space để tổ chức các cuộc thảo luận theo chủ đề, dự án hoặc nhóm công việc. Hỗ trợ ba loại kênh: public (công khai), private (riêng tư) và project (theo dự án).

Giao diện:

\begin{figure}[H]
\centering
\includegraphics[width=0.65\textwidth]{part02/chap03/sec03_5/ui_create_channel_numbered.png}
\caption{Giao diện chức năng tạo kênh mới}
\label{fig:ui_create_channel}
\end{figure}

Các thành phần trong giao diện:

\begin{table}[H]
\centering
\begin{tabular}{|c|l|l|p{5.5cm}|}
\hline
\textbf{STT} & \textbf{Loại điều khiển} & \textbf{Tên điều khiển} & \textbf{Nội dung thực hiện} \\
\hline
1 & Radio Button & Channel Type & Chọn loại kênh (Public/Private/Project) \\
\hline
2 & Textbox & Channel Name & Nhập tên kênh \\
\hline
3 & Textarea & Description & Nhập mô tả cho kênh \\
\hline
4 & Toggle & AI Q\&A & Bật/tắt tính năng hỏi đáp AI \\
\hline
5 & Toggle & Auto Summary & Bật/tắt tính năng tóm tắt tự động \\
\hline
6 & Button & Cancel & Hủy bỏ thao tác tạo kênh \\
\hline
7 & Button & Create Channel & Xác nhận tạo kênh mới \\
\hline
\end{tabular}
\caption{Bảng các thành phần trong giao diện tạo kênh mới}
\label{tab:ui_create_channel}
\end{table}

Dữ liệu được dùng:

\begin{table}[H]
\centering
\begin{tabular}{|c|l|c|c|c|c|}
\hline
\multirow{2}{*}{\textbf{STT}} & \multirow{2}{*}{\textbf{Tên bảng}} & \multicolumn{4}{c|}{\textbf{Phương thức}} \\
\cline{3-6}
& & \textbf{Thêm} & \textbf{Sửa} & \textbf{Xóa} & \textbf{Truy vấn} \\
\hline
1 & channels & X & & & X \\
\hline
2 & channel\_members & X & & & \\
\hline
\end{tabular}
\caption{Bảng dữ liệu được dùng cho chức năng tạo kênh mới}
\label{tab:func_create_channel}
\end{table}

Xử lý:

\begin{figure}[H]
\centering
\includegraphics[width=0.75\textwidth]{part02/chap03/sec03_5/act_create_channel.png}
\caption{Sơ đồ hoạt động của chức năng tạo kênh mới}
\label{fig:act_create_channel}
\end{figure}

% 3.5.2 Tham gia kênh
\subsubsection{Chức năng tham gia kênh}

Mục đích: Cho phép người dùng tham gia vào các kênh public có sẵn hoặc các kênh private/project khi được mời, để có thể xem và tham gia thảo luận trong kênh đó.

Giao diện:

\begin{figure}[H]
\centering
\includegraphics[width=0.85\textwidth]{part02/chap03/sec03_5/ui_join_channel_numbered.png}
\caption{Giao diện chức năng tham gia kênh}
\label{fig:ui_join_channel}
\end{figure}

Các thành phần trong giao diện:

\begin{table}[H]
\centering
\begin{tabular}{|c|l|l|p{5.5cm}|}
\hline
\textbf{STT} & \textbf{Loại điều khiển} & \textbf{Tên điều khiển} & \textbf{Nội dung thực hiện} \\
\hline
1 & Textbox & Search Channels & Tìm kiếm kênh theo tên \\
\hline
2 & List & Channel List & Hiển thị danh sách kênh có thể tham gia \\
\hline
3 & Label & Channel Info & Hiển thị thông tin kênh (tên, mô tả, số thành viên) \\
\hline
4 & Button & Join & Nhấn để tham gia kênh được chọn \\
\hline
\end{tabular}
\caption{Bảng các thành phần trong giao diện tham gia kênh}
\label{tab:ui_join_channel}
\end{table}

Dữ liệu được dùng:

\begin{table}[H]
\centering
\begin{tabular}{|c|l|c|c|c|c|}
\hline
\multirow{2}{*}{\textbf{STT}} & \multirow{2}{*}{\textbf{Tên bảng}} & \multicolumn{4}{c|}{\textbf{Phương thức}} \\
\cline{3-6}
& & \textbf{Thêm} & \textbf{Sửa} & \textbf{Xóa} & \textbf{Truy vấn} \\
\hline
1 & channels & & & & X \\
\hline
2 & channel\_members & X & & & X \\
\hline
\end{tabular}
\caption{Bảng dữ liệu được dùng cho chức năng tham gia kênh}
\label{tab:func_join_channel}
\end{table}

Xử lý:

\begin{figure}[H]
\centering
\includegraphics[width=0.85\textwidth]{part02/chap03/sec03_5/act_join_channel.png}
\caption{Sơ đồ hoạt động của chức năng tham gia kênh}
\label{fig:act_join_channel}
\end{figure}

% 3.5.3 Rời kênh
\subsubsection{Chức năng rời kênh}

Mục đích: Cho phép người dùng rời khỏi kênh mà họ không còn muốn tham gia. Lưu ý: Owner duy nhất của kênh cần chuyển quyền trước khi rời.

Giao diện:

\begin{figure}[H]
\centering
\includegraphics[width=0.85\textwidth]{part02/chap03/sec03_5/ui_leave_channel_numbered.png}
\caption{Giao diện chức năng rời kênh}
\label{fig:ui_leave_channel}
\end{figure}

Các thành phần trong giao diện:

\begin{table}[H]
\centering
\begin{tabular}{|c|l|l|p{5.5cm}|}
\hline
\textbf{STT} & \textbf{Loại điều khiển} & \textbf{Tên điều khiển} & \textbf{Nội dung thực hiện} \\
\hline
1 & Label & Channel Name & Hiển thị tên kênh sẽ rời \\
\hline
2 & Label & Warning Message & Hiển thị cảnh báo về việc rời kênh \\
\hline
3 & Button & Cancel & Hủy bỏ thao tác rời kênh \\
\hline
4 & Button & Leave Channel & Xác nhận rời khỏi kênh \\
\hline
\end{tabular}
\caption{Bảng các thành phần trong giao diện rời kênh}
\label{tab:ui_leave_channel}
\end{table}

Dữ liệu được dùng:

\begin{table}[H]
\centering
\begin{tabular}{|c|l|c|c|c|c|}
\hline
\multirow{2}{*}{\textbf{STT}} & \multirow{2}{*}{\textbf{Tên bảng}} & \multicolumn{4}{c|}{\textbf{Phương thức}} \\
\cline{3-6}
& & \textbf{Thêm} & \textbf{Sửa} & \textbf{Xóa} & \textbf{Truy vấn} \\
\hline
1 & channel\_members & & & X & X \\
\hline
2 & channels & & X & & X \\
\hline
\end{tabular}
\caption{Bảng dữ liệu được dùng cho chức năng rời kênh}
\label{tab:func_leave_channel}
\end{table}

Xử lý:

\begin{figure}[H]
\centering
\includegraphics[width=0.85\textwidth]{part02/chap03/sec03_5/act_leave_channel.png}
\caption{Sơ đồ hoạt động của chức năng rời kênh}
\label{fig:act_leave_channel}
\end{figure}

% 3.5.4 Mời thành viên vào kênh
\subsubsection{Chức năng mời thành viên vào kênh}

Mục đích: Cho phép Admin hoặc Owner của kênh mời thêm thành viên mới vào kênh private hoặc project, đồng thời có thể chỉ định vai trò cho thành viên được mời.

Giao diện:

\begin{figure}[H]
\centering
\includegraphics[width=0.85\textwidth]{part02/chap03/sec03_5/ui_channel_members_numbered.png}
\caption{Giao diện chức năng mời thành viên vào kênh}
\label{fig:ui_invite_member}
\end{figure}

Các thành phần trong giao diện:

\begin{table}[H]
\centering
\begin{tabular}{|c|l|l|p{5.5cm}|}
\hline
\textbf{STT} & \textbf{Loại điều khiển} & \textbf{Tên điều khiển} & \textbf{Nội dung thực hiện} \\
\hline
1 & Textbox & Search Members & Tìm kiếm thành viên trong work\-space \\
\hline
2 & Dropdown & User Select & Chọn người dùng để mời \\
\hline
3 & Dropdown & Role Select & Chọn vai trò (Admin/Member) \\
\hline
4 & Button & Invite & Gửi lời mời tham gia kênh \\
\hline
5 & List & Pending Invites & Hiển thị danh sách lời mời đang chờ \\
\hline
6 & List & Member List & Hiển thị danh sách thành viên hiện tại \\
\hline
\end{tabular}
\caption{Bảng các thành phần trong giao diện mời thành viên}
\label{tab:ui_invite_member}
\end{table}

Dữ liệu được dùng:

\begin{table}[H]
\centering
\begin{tabular}{|c|l|c|c|c|c|}
\hline
\multirow{2}{*}{\textbf{STT}} & \multirow{2}{*}{\textbf{Tên bảng}} & \multicolumn{4}{c|}{\textbf{Phương thức}} \\
\cline{3-6}
& & \textbf{Thêm} & \textbf{Sửa} & \textbf{Xóa} & \textbf{Truy vấn} \\
\hline
1 & channel\_members & X & & & X \\
\hline
2 & channels & & & & X \\
\hline
3 & users & & & & X \\
\hline
\end{tabular}
\caption{Bảng dữ liệu được dùng cho chức năng mời thành viên vào kênh}
\label{tab:func_invite_member}
\end{table}

Xử lý:

\begin{figure}[H]
\centering
\includegraphics[width=0.85\textwidth]{part02/chap03/sec03_5/act_invite_member.png}
\caption{Sơ đồ hoạt động của chức năng mời thành viên vào kênh}
\label{fig:act_invite_member}
\end{figure}

% 3.5.5 Cài đặt kênh
\subsubsection{Chức năng cài đặt kênh}

Mục đích: Cho phép Admin hoặc Owner cập nhật thông tin kênh như tên, mô tả, loại kênh và cấu hình các tính năng AI (Q\&A, Auto Summary, Action Items, Document Summary).

Giao diện:

\begin{figure}[H]
\centering
\includegraphics[width=0.85\textwidth]{part02/chap03/sec03_5/ui_channel_settings_numbered.png}
\caption{Giao diện chức năng cài đặt kênh}
\label{fig:ui_channel_settings}
\end{figure}

Các thành phần trong giao diện:

\begin{table}[H]
\centering
\begin{tabular}{|c|l|l|p{5.5cm}|}
\hline
\textbf{STT} & \textbf{Loại điều khiển} & \textbf{Tên điều khiển} & \textbf{Nội dung thực hiện} \\
\hline
1 & Tab & Settings Tabs & Chuyển đổi giữa các tab (Over\-view/Mem\-bers/AI/Noti\-fi\-ca\-tions) \\
\hline
2 & Textbox & Channel Name & Chỉnh sửa tên kênh \\
\hline
3 & Textarea & Description & Chỉnh sửa mô tả kênh \\
\hline
4 & Toggle & AI Q\&A & Bật/tắt tính năng hỏi đáp AI \\
\hline
5 & Toggle & Auto Summary & Bật/tắt tính năng tóm tắt tự động \\
\hline
6 & Toggle & Action Items & Bật/tắt tính năng trích xuất action items \\
\hline
7 & Toggle & Document Summary & Bật/tắt tính năng tóm tắt tài liệu \\
\hline
8 & Button & Save Changes & Lưu các thay đổi cài đặt \\
\hline
\end{tabular}
\caption{Bảng các thành phần trong giao diện cài đặt kênh}
\label{tab:ui_channel_settings}
\end{table}

Dữ liệu được dùng:

\begin{table}[H]
\centering
\begin{tabular}{|c|l|c|c|c|c|}
\hline
\multirow{2}{*}{\textbf{STT}} & \multirow{2}{*}{\textbf{Tên bảng}} & \multicolumn{4}{c|}{\textbf{Phương thức}} \\
\cline{3-6}
& & \textbf{Thêm} & \textbf{Sửa} & \textbf{Xóa} & \textbf{Truy vấn} \\
\hline
1 & channels & & X & & X \\
\hline
\end{tabular}
\caption{Bảng dữ liệu được dùng cho chức năng cài đặt kênh}
\label{tab:func_channel_settings}
\end{table}

Xử lý:

\begin{figure}[H]
\centering
\includegraphics[width=0.85\textwidth]{part02/chap03/sec03_5/act_channel_settings.png}
\caption{Sơ đồ hoạt động của chức năng cài đặt kênh}
\label{fig:act_channel_settings}
\end{figure}
