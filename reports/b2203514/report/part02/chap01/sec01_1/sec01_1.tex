\subsection{Mô tả chi tiết bài toán}

Trong bối cảnh phát triển phần mềm hiện đại, Agile trở thành phương pháp quản lý dự án phổ biến nhờ khả năng linh hoạt, phản hồi nhanh với thay đổi và tăng cường cộng tác giữa các thành viên. Tuy nhiên, khi triển khai Agile trên thực tế, hoạt động trao đổi thông tin giữa các thành viên diễn ra liên tục và tạo ra khối lượng dữ liệu rất lớn. Các cuộc thảo luận về yêu cầu, phản hồi về thiết kế, cập nhật tiến độ, biên bản họp và tài liệu kỹ thuật thường bị phân tán trên nhiều nền tảng khác nhau như Slack, Discord, Microsoft Teams hay email. Sự phân mảnh này khiến lịch sử liên lạc, các quyết định quan trọng và tài liệu đính kèm khó được theo dõi tập trung. Khi một thành viên mới tham gia dự án hoặc khi cần tra cứu lại một quyết định đã đưa ra trước đó, họ phải tìm kiếm qua nhiều kênh, nhiều ứng dụng và đọc lại hàng trăm tin nhắn để nắm bắt ngữ cảnh.

Để giải quyết vấn đề này, cần xây dựng một phân hệ truyền thông tập trung, cho phép các nhóm phát triển Agile trao đổi thông tin trong một môi trường duy nhất, gắn liền với dự án và workspace. Tuy nhiên, thách thức không chỉ nằm ở việc cung cấp chức năng nhắn tin cơ bản, mà còn phải đảm bảo hệ thống hỗ trợ truyền thông thời gian thực (real-time), quản lý tệp đính kèm hiệu quả và đặc biệt là có khả năng giúp người dùng nắm bắt thông tin nhanh chóng mà không cần đọc lại toàn bộ lịch sử hội thoại. Với sự bùng nổ thông tin trong các kênh truyền thông dự án Agile (thảo luận về user stories, phản hồi code review, câu hỏi kỹ thuật, chia sẻ tài liệu...), cần có một cơ chế thông minh giúp tóm tắt nội dung, trích xuất ý chính và trả lời câu hỏi dựa trên ngữ cảnh. Đây chính là lý do cần tích hợp các công nghệ AI với kiến trúc RAG (Retrieval-Augmented Generation) vào phân hệ truyền thông, nhằm tự động hóa việc xử lý và khai thác thông tin từ các cuộc hội thoại và tài liệu đính kèm.

Trong phạm vi đề tài này, tác giả tập trung vào việc xây dựng \textbf{phân hệ truyền thông} cho hệ thống SaaS quản lý dự án Agile tích hợp AI. Phân hệ này đóng vai trò là hệ thống nhắn tin tập trung, cung cấp các dịch vụ cốt lõi bao gồm: quản lý kênh trò chuyện (channel management), nhắn tin thời gian thực (real-time messaging), quản lý tệp đính kèm trong chat (tích hợp với File Service từ phân hệ nền tảng và thông tin), tìm kiếm lịch sử hội thoại và tích hợp AI với kiến trúc RAG để tóm tắt, phân tích và trả lời câu hỏi dựa trên nội dung trao đổi. Các phân hệ khác như nền tảng (platform), quản lý dự án (project management) và họp trực tuyến (meeting) sẽ được phát triển độc lập hoặc tích hợp từ các hệ thống bên ngoài, nhưng đều có thể kết nối với phân hệ truyền thông để chia sẻ thông tin và ngữ cảnh.

Về đối tượng người dùng, phân hệ truyền thông phục vụ hai nhóm chính trong workspace. Nhóm đầu tiên là Workspace Owner hoặc Channel Admin (Người quản lý kênh), có quyền tạo và cấu hình các kênh trò chuyện theo dự án hoặc theo chủ đề, mời thành viên vào kênh, phân quyền truy cập (kênh công khai, kênh riêng tư), quản lý các thiết lập của kênh và xem các thống kê hoạt động. Workspace Owner cũng có thể cấu hình các tính năng AI như cho phép tóm tắt hội thoại, bật/tắt AI assistant trong từng kênh. Nhóm thứ hai là Member (Thành viên), người tham gia vào một hoặc nhiều kênh trò chuyện trong workspace. Member có thể gửi và nhận tin nhắn trong thời gian thực, tạo thread để thảo luận chi tiết về một chủ đề cụ thể, gửi tệp đính kèm (tài liệu, hình ảnh, file biên bản), tìm kiếm lịch sử tin nhắn, và sử dụng AI assistant để tóm tắt đoạn hội thoại dài hoặc đặt câu hỏi về nội dung đã trao đổi trong kênh. Tùy theo vai trò trong dự án (Project Manager, Developer, Tester, Viewer), Member có thể có quyền hạn khác nhau trong việc tạo kênh mới hoặc mời thành viên.

Về chức năng, phân hệ truyền thông cung cấp các dịch vụ cốt lõi sau đây. Thứ nhất là quản lý kênh trò chuyện (Channel Management), cho phép tạo các kênh theo dự án (project channel), kênh theo chủ đề (topic channel) hoặc kênh riêng tư (private channel) cho các cuộc thảo luận nhạy cảm. Mỗi kênh có thể được gắn với một hoặc nhiều dự án trong workspace, giúp tổ chức các cuộc trao đổi theo ngữ cảnh công việc. Hệ thống hỗ trợ cấu hình quyền truy cập kênh, cho phép Workspace Owner quyết định ai có thể xem, gửi tin nhắn hoặc quản lý kênh.

Thứ hai là nhắn tin thời gian thực (Real-time Messaging), sử dụng WebSocket để cho phép người dùng gửi và nhận tin nhắn tức thì. Hệ thống hỗ trợ các tính năng nhắn tin hiện đại bao gồm: typing indicator (hiển thị khi ai đó đang nhập), trạng thái đã đọc/chưa đọc, phản hồi tin nhắn (reaction với emoji), trả lời tin nhắn cụ thể (reply), và tạo thread để thảo luận chi tiết mà không làm gián đoạn luồng hội thoại chính. Tin nhắn được đồng bộ giữa các thiết bị, đảm bảo người dùng có thể tiếp tục cuộc trò chuyện từ bất kỳ đâu.

Thứ ba là quản lý tệp đính kèm (Attachment Management), tích hợp với File Service từ phân hệ nền tảng để cho phép người dùng gửi tệp đính kèm trực tiếp trong cuộc trò chuyện. Hệ thống hỗ trợ nhiều định dạng file bao gồm tài liệu văn bản (PDF, DOCX), hình ảnh (PNG, JPG), file biên bản họp và các định dạng phổ biến khác. Người dùng có thể xem trước nội dung file (preview) ngay trong khung chat mà không cần tải về. Metadata của tệp đính kèm được lưu trữ và liên kết với tin nhắn tương ứng, cho phép tìm kiếm file theo tên, loại hoặc người gửi.

Thứ tư là tìm kiếm lịch sử hội thoại (Conversation Search), cung cấp khả năng tìm kiếm mạnh mẽ trên toàn bộ nội dung tin nhắn trong các kênh mà người dùng có quyền truy cập. Hệ thống hỗ trợ tìm kiếm hybrid kết hợp full-text search (tìm theo từ khóa chính xác) và semantic search (tìm theo ý nghĩa, sử dụng vector embedding). Người dùng có thể lọc kết quả theo kênh, khoảng thời gian, người gửi hoặc loại nội dung (tin nhắn, file). Kết quả tìm kiếm được hiển thị với ngữ cảnh xung quanh, giúp người dùng nhanh chóng định vị thông tin cần tìm.

Thứ năm là tích hợp AI với kiến trúc RAG (Retrieval-Augmented Generation), xây dựng AI Service có khả năng xử lý và khai thác thông tin từ các cuộc hội thoại và tài liệu đính kèm. Hệ thống bao gồm Document Processing Pipeline để trích xuất văn bản từ các định dạng file khác nhau (PDF, DOCX, hình ảnh với OCR), chia nhỏ văn bản (chunking) và tạo embedding. Vector database (sử dụng pgvector) lưu trữ embedding của tin nhắn và nội dung tài liệu, cho phép truy xuất ngữ cảnh liên quan khi người dùng đặt câu hỏi. AI Service kết hợp kết quả truy xuất với mô hình ngôn ngữ lớn (LLM) để cung cấp các tính năng: tóm tắt đoạn hội thoại dài, trích xuất ý chính và danh sách công việc từ cuộc thảo luận, trả lời câu hỏi dựa trên nội dung đã trao đổi trong kênh và tài liệu đính kèm, tạo bản tóm tắt tài liệu ngay trong khung chat.

Cuối cùng là giao diện chat (Chat UI), xây dựng giao diện nhắn tin hiện đại bằng React/Next.js, bao gồm các thành phần chính: danh sách kênh (channel list) hiển thị các kênh theo workspace và dự án, khung trò chuyện (chat panel) với khả năng cuộn lịch sử và tải tin nhắn theo trang, ô nhập tin nhắn với hỗ trợ định dạng văn bản và đính kèm file, khu vực hiển thị tệp đính kèm với preview, và panel AI assistant cho phép người dùng tương tác với AI để tóm tắt hoặc đặt câu hỏi. Giao diện được thiết kế responsive, hỗ trợ cả desktop và mobile, với trải nghiệm người dùng trực quan và mượt mà.

Về kiến trúc hệ thống, phân hệ truyền thông được thiết kế theo kiến trúc microservice, bao gồm các service độc lập giao tiếp qua API RESTful và WebSocket. Messaging Service (NestJS) đảm nhận quản lý kênh, tin nhắn, thread và liên kết tệp đính kèm. WebSocket Gateway (NestJS + Socket.io) xử lý truyền thông thời gian thực, quản lý kết nối và broadcast tin nhắn. Document Processing Service xử lý việc trích xuất văn bản từ tài liệu, chunking và tạo embedding. AI Service (NestJS) tích hợp vector database và LLM để thực hiện các tác vụ tóm tắt và hỏi đáp theo kiến trúc RAG. Chat UI (Next.js) cung cấp giao diện nhắn tin cho người dùng. Hệ thống sử dụng PostgreSQL làm cơ sở dữ liệu chính với pgvector extension cho vector storage, Redis cho caching và làm adapter cho Socket.io để hỗ trợ horizontal scaling. Phân hệ truyền thông tích hợp với File Service từ phân hệ nền tảng để lưu trữ và truy xuất tệp đính kèm, đồng thời có thể nhận sự kiện từ các phân hệ khác (quản lý dự án, họp trực tuyến) để hiển thị thông báo hoặc tự động tạo tin nhắn trong kênh liên quan.

Tóm lại, bài toán cần giải quyết là xây dựng một phân hệ truyền thông hoàn chỉnh với khả năng nhắn tin thời gian thực, quản lý kênh theo dự án, tích hợp tệp đính kèm và đặc biệt là tận dụng AI với kiến trúc RAG để tự động hóa việc tóm tắt, trích xuất thông tin và trả lời câu hỏi dựa trên ngữ cảnh hội thoại, giúp các nhóm phát triển Agile giao tiếp và nắm bắt thông tin hiệu quả hơn.