\subsection{Mô tả chi tiết bài toán}

Trong bối cảnh phát triển phần mềm hiện đại, Agile trở thành phương pháp quản lý dự án phổ biến nhờ khả năng linh hoạt, phản hồi nhanh với thay đổi và tăng cường cộng tác giữa các thành viên. Tuy nhiên, khi triển khai Agile trên thực tế, các nhóm phát triển thường phải sử dụng một loạt công cụ khác nhau để đáp ứng các nhu cầu đa dạng: quản lý công việc (task management), trao đổi thông tin (communication), họp trực tuyến (meeting), lưu trữ tài liệu (file storage), tích hợp CI/CD và giám sát hệ thống. Việc phải chuyển đổi giữa nhiều nền tảng riêng lẻ không chỉ gây phân mảnh thông tin, làm tốn thời gian tìm kiếm và đối chiếu dữ liệu, mà còn khó khăn trong việc duy trì tính minh bạch, theo dõi tiến độ tổng thể và tổng hợp báo cáo toàn diện về dự án.

Để giải quyết vấn đề này, cần xây dựng một nền tảng SaaS tích hợp, cho phép các tổ chức quản lý toàn bộ quy trình phát triển phần mềm Agile trong một môi trường duy nhất. Tuy nhiên, thách thức không chỉ nằm ở việc gộp các chức năng lại với nhau, mà còn phải đảm bảo hệ thống có khả năng mở rộng, dễ bảo trì và có thể tích hợp thêm các dịch vụ mới trong tương lai. Hơn nữa, với sự bùng nổ thông tin trong dự án Agile (sprint backlogs, user stories, cuộc họp hằng ngày, hội thoại trong kênh truyền thông, tài liệu kỹ thuật, biên bản họp...), cần có một cơ chế thông minh giúp người quản lý nhanh chóng nắm bắt bức tranh tổng thể, phát hiện vấn đề tiềm ẩn và đưa ra quyết định kịp thời. Đây chính là lý do cần tích hợp các công nghệ AI vào nền tảng, nhằm tự động hóa việc tổng hợp, phân tích và tóm tắt thông tin đa nguồn, từ đó hỗ trợ người dùng làm việc hiệu quả hơn.

Trong phạm vi đề tài này, tác giả tập trung vào việc xây dựng \textbf{phân hệ truyền thông} cho hệ thống SaaS quản lý dự án Agile tích hợp AI. Phân hệ này đóng vai trò là lớp xương sống, cung cấp các dịch vụ cốt lõi bao gồm: quản lý tài khoản (authentication \& authorization), quản lý workspace và multi-tenancy, quản lý file (file-storage), hệ thống thông báo (notification), API Gateway (edge service) và tích hợp AI ở tầng nền tảng để tổng hợp, phân tích thông tin. Các phân hệ chức năng khác như quản lý dự án (project management), truyền thông (communication) và họp trực tuyến (meeting) sẽ được phát triển sau hoặc tích hợp từ các hệ thống bên ngoài, nhưng đều dựa trên nền tảng chung này.

Về đối tượng người dùng, hệ thống phục vụ ba nhóm chính. Nhóm đầu tiên là Super Admin (Quản trị viên hệ thống), người quản lý toàn bộ nền tảng SaaS với quyền tạo/xóa workspace, quản lý người dùng cấp cao, cấu hình hệ thống, giám sát hoạt động tổng thể và xử lý các vấn đề kỹ thuật. Super Admin không tham gia trực tiếp vào các dự án cụ thể của workspace mà chỉ đảm bảo hệ thống hoạt động ổn định và bảo mật. Nhóm thứ hai là Workspace Owner (Chủ workspace), người đại diện cho một tổ chức hoặc nhóm phát triển, có quyền quản lý thành viên trong workspace, tạo và cấu hình các dự án, phân quyền cho các thành viên, quản lý file và tài nguyên của workspace, cũng như xem các báo cáo và dashboard tổng hợp do hệ thống cung cấp. Workspace Owner có thể mời thêm thành viên, gán vai trò (như Project Manager, Developer, Tester, Viewer…) và theo dõi tiến độ tổng thể của các dự án. Nhóm thứ ba là Member (Thành viên), người tham gia vào một hoặc nhiều dự án trong workspace, có quyền truy cập vào các tài nguyên dự án theo vai trò được phân quyền. Member có thể xem và cập nhật công việc được giao, tham gia vào các kênh truyền thông, tải lên và tải xuống file liên quan đến dự án, nhận thông báo về các sự kiện quan trọng. Tùy theo vai trò, Member có thể có quyền hạn khác nhau: Project Manager có thể tạo task và gán công việc, Developer có thể cập nhật trạng thái task, Viewer chỉ có quyền xem thông tin mà không được chỉnh sửa.

Về chức năng, phân hệ truyền thông cung cấp các dịch vụ cốt lõi sau đây. Thứ nhất là quản lý tài khoản và xác thực (Authentication \& Authorization), cung cấp dịch vụ đăng ký, đăng nhập, quản lý phiên làm việc (session/token), tích hợp xác thực đa yếu tố (2FA) nếu cần. Hệ thống sử dụng OAuth2 và JWT để bảo mật API, đảm bảo chỉ người dùng hợp lệ mới có quyền truy cập vào tài nguyên. Dịch vụ account (sử dụng Java Spring Boot) quản lý thông tin người dùng, workspace, vai trò (role) và quyền hạn (permission), hỗ trợ phân quyền dựa trên vai trò (RBAC) hoặc dựa trên chính sách (ABAC). Thứ hai là quản lý workspace và multi-tenancy, cho phép một hệ thống phục vụ nhiều tổ chức (tenant) khác nhau một cách độc lập và an toàn. Mỗi workspace có cơ sở dữ liệu logic riêng hoặc được phân vùng rõ ràng trong database chung, đảm bảo dữ liệu của workspace này không bị truy cập trái phép bởi workspace khác. Workspace Owner có thể tạo nhiều dự án (project) trong workspace, mời thêm thành viên, gán vai trò và cấu hình quyền truy cập theo từng dự án hoặc từng tài nguyên.

Thứ ba là quản lý file (File-Storage), cung cấp dịch vụ lưu trữ file tập trung, cho phép người dùng upload/download tài liệu dự án, tệp đính kèm trong kênh truyền thông, biên bản họp, hình ảnh, video. Dịch vụ file-storage (sử dụng NestJS) quản lý metadata của file (tên, kích thước, loại, workspace, project, người upload, thời gian tải lên), kiểm soát quyền truy cập theo workspace và project, hỗ trợ tìm kiếm file theo tên, loại, hoặc tag. Hệ thống có thể lưu file trên local storage hoặc tích hợp với các dịch vụ lưu trữ đám mây (S3-compatible storage) để tăng khả năng mở rộng. Thứ tư là hệ thống thông báo (Notification), tập trung hóa việc gửi và nhận thông báo từ các phân hệ khác nhau (task được giao, trạng thái dự án thay đổi, cuộc họp sắp diễn ra, tin nhắn mới trong kênh truyền thông). Dịch vụ notification (sử dụng NestJS) lưu trữ thông báo, quản lý trạng thái đã đọc/chưa đọc, ưu tiên thông báo quan trọng, và hỗ trợ gửi thông báo qua nhiều kênh (trên web, email, hoặc push notification nếu có ứng dụng di động). Người dùng có thể cấu hình loại thông báo muốn nhận và kênh nhận thông báo ưa thích.

Thứ năm là API Gateway và Edge Service, sử dụng Nginx làm API Gateway và reverse proxy, đóng vai trò điểm truy cập duy nhất vào hệ thống. Edge service thực hiện các nhiệm vụ: định tuyến request đến các service phía sau (account, notification, file-storage, AI), cân bằng tải giữa các instance của service, bảo mật bằng SSL/TLS, xác thực người dùng (kiểm tra JWT token) và ký kết thông tin xác thực bằng HMAC (Hash-based Message Authentication Code) trước khi chuyển tiếp request đến các service nội bộ. Nhờ đó, các service nội bộ không cần tự thực hiện xác thực phức tạp mà chỉ cần xác minh chữ ký HMAC từ edge service, giảm tải cho các service và tăng tính bảo mật. Thứ sáu là tích hợp AI ở tầng nền tảng, xây dựng AI service (sử dụng NestJS) có khả năng thu thập dữ liệu từ các phân hệ chức năng (project management, communication, meeting) và cung cấp các API để tổng hợp, tóm tắt, phân tích thông tin. Ví dụ: tổng hợp hoạt động gần đây của một dự án, tóm tắt nội dung chính từ các cuộc họp và hội thoại liên quan, phát hiện các rủi ro hoặc vấn đề tiềm ẩn (task bị trễ, thành viên quá tải), đề xuất hành động tiếp theo. AI service tích hợp với các API bên ngoài như OpenAI API, Gemini API hoặc các mô hình nguồn mở, xử lý dữ liệu đầu vào và trả về kết quả dạng JSON cho các phân hệ khác sử dụng. Cuối cùng là giao diện quản trị (Admin UI), xây dựng giao diện trang quản trị cho Workspace Owner bằng React/NextJS, cho phép quản lý thành viên (thêm, xóa, cập nhật vai trò, khóa tài khoản), cấu hình workspace (tên, mô tả, cài đặt bảo mật, tích hợp dịch vụ bên ngoài), xem các thống kê và dashboard tổng hợp (số lượng dự án đang chạy, số lượng thành viên, dung lượng file sử dụng, hoạt động gần đây) do phân hệ nền tảng cung cấp. Giao diện admin tích hợp với API của account service và các service khác, hỗ trợ tìm kiếm, lọc, phân trang dữ liệu, và cung cấp trải nghiệm người dùng trực quan, dễ sử dụng.

Về kiến trúc hệ thống, phân hệ truyền thông được thiết kế theo kiến trúc microservice, bao gồm các service độc lập giao tiếp qua API RESTful. Edge Service (Nginx) đóng vai trò API Gateway và reverse proxy, xử lý định tuyến, cân bằng tải, bảo mật và ký kết HMAC. Account Service (Java Spring Boot) đảm nhận quản lý tài khoản, workspace, xác thực và phân quyền. Notification Service (NestJS) chịu trách nhiệm quản lý và phân phối thông báo. File-Storage Service (NestJS) quản lý file và metadata. AI Service (NestJS) tích hợp AI để tổng hợp và phân tích dữ liệu. Admin UI (Next.js) cung cấp giao diện quản trị cho Workspace Owner. Mỗi service quản lý cơ sở dữ liệu riêng (PostgreSQL), đảm bảo tính độc lập và dễ mở rộng. Các service giao tiếp qua HTTP/REST, sử dụng JWT để xác thực và HMAC để bảo mật giao tiếp nội bộ. Khi cần đồng bộ dữ liệu liên service, hệ thống sử dụng message queue (RabbitMQ hoặc Redis Pub/Sub) hoặc gọi API trực tiếp với cơ chế retry và fallback.

Tóm lại, bài toán cần giải quyết là xây dựng một phân hệ truyền thông đầy đủ, có khả năng mở rộng, bảo mật và dễ tích hợp, đóng vai trò nền tảng cho các phân hệ chức năng khác, đồng thời tận dụng AI để tự động hóa việc tổng hợp và phân tích thông tin, giúp người dùng quản lý dự án Agile hiệu quả hơn.
