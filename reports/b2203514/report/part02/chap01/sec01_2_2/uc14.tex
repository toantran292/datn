\paragraph{UC14: Tóm tắt tài liệu}
\mbox{}

\begin{figure}[H]
    \centering
    \includegraphics[width=0.3\textwidth]{images/UC14.png}
    \caption{Sơ đồ use case UC14 - Tóm tắt tài liệu}
    \label{fig:uc14}
\end{figure}

\begin{adjustwidth}{-1.5cm}{-0.5cm}
\begin{longtblr}[
    caption = {Đặc tả use case UC14 - Tóm tắt tài liệu},
    label = {tab:uc14}
]{
    colspec={|l|p{.7\linewidth}|}
}
\hline
\textbf{Tên chức năng} & \textbf{Tóm tắt tài liệu} \\\hline
ID & UC14 \\\hline
Người sử dụng & Member \\\hline
Mức độ cần thiết & Quan trọng \\\hline
Phân loại & Phức tạp \\\hline
Các thành phần tham gia &
\begin{minipage}{\linewidth}
    \vskip 4pt
    + \textbf{Member:} Muốn hiểu nhanh nội dung của tài liệu đính kèm mà không cần đọc toàn bộ. \\
    + \textbf{AI Provider:} Xử lý tài liệu và sinh bản tóm tắt. \\
    + \textbf{File Service:} Cung cấp nội dung tài liệu.
    \vskip 1pt
\end{minipage}
\\\hline
Mô tả tóm tắt & Cho phép Member yêu cầu AI tóm tắt nội dung của một tài liệu đính kèm trong chat (PDF, DOCX, v.v.) ngay trong khung chat. \\\hline
Trigger & Member chọn một tài liệu đính kèm và yêu cầu tóm tắt. \\\hline
Kiểu sự kiện & External. \\\hline
Luồng xử lý bình thường &
\begin{minipage}{\linewidth}
    \vskip 4pt
    \begin{enumerate}
        \item Member click vào tài liệu đính kèm trong tin nhắn.
        \item Member chọn "Tóm tắt với AI".
        \item Hệ thống lấy nội dung tài liệu từ File Service.
        \item Hệ thống trích xuất văn bản từ tài liệu (PDF parser, DOCX parser).
        \item Hệ thống gửi nội dung đến AI Service.
        \item AI Service sinh bản tóm tắt.
        \item Hệ thống hiển thị bản tóm tắt trong panel AI Assistant.
    \end{enumerate}
    \vskip 1pt
\end{minipage}
\\\hline
Các luồng sự kiện con & N/A \\\hline
Luồng thay thế/ngoại lệ &
\begin{minipage}{\linewidth}
    \vskip 4pt
    \textbf{\textcolor{red}{E1}} -- Không thể trích xuất văn bản: Hệ thống thông báo "Không thể đọc nội dung tài liệu này".

    \textbf{\textcolor{red}{A1}} -- Tài liệu quá dài: Hệ thống chia nhỏ, tóm tắt từng phần và tổng hợp.
    \vskip 1pt
\end{minipage}
\\\hline
Kết quả & Bản tóm tắt tài liệu được hiển thị, giúp Member nắm bắt nội dung nhanh chóng. \\\hline
\end{longtblr}
\end{adjustwidth}
