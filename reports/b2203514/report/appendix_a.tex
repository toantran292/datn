\phantomsection
\setsection{Phụ lục A. Hướng dẫn cài đặt và sử dụng phần mềm}

\newcounter{appendixstep}
\setcounter{appendixstep}{0}
\renewcommand{\theappendixstep}{\arabic{appendixstep}}

\refstepcounter{appendixstep}
\subsubsection*{\theappendixstep. Yêu cầu hệ thống}

\textbf{Phần cứng tối thiểu:}
\begin{itemize}
    \item Vi xử lý: Intel Core i5 hoặc tương đương
    \item RAM: 8GB (khuyến nghị 16GB cho môi trường development)
    \item Ổ cứng: SSD 256GB trở lên (ít nhất 20GB dung lượng trống)
    \item Kết nối mạng: Băng thông ổn định để tải dependencies và Docker images
\end{itemize}

\textbf{Phần mềm yêu cầu:}
\begin{itemize}
    \item Hệ điều hành: Windows 10/11, macOS 10.15+, hoặc Linux (Ubuntu 20.04+)
    \item Node.js: phiên bản 20.19.0 (theo file \texttt{.nvmrc})
    \item pnpm: phiên bản 10.14.0+ (package manager)
    \item Docker: phiên bản 20.10+ và Docker Compose v2
    \item Git: phiên bản 2.30+
    \item OpenSSL: để sinh JWT keys cho Identity service
\end{itemize}

\refstepcounter{appendixstep}
\subsubsection*{\theappendixstep. Cài đặt các công cụ cần thiết}

\textbf{1. Cài đặt Node.js}

Hệ thống sử dụng Node.js phiên bản 20.19.0. Khuyến nghị sử dụng Node Version Manager (nvm) để quản lý phiên bản Node.js.

\textbf{Trên macOS/Linux:}
\begin{lstlisting}
# Cai dat nvm
curl -o- https://raw.githubusercontent.com/nvm-sh/nvm/v0.39.0/install.sh | bash
# Khoi dong lai terminal, sau do:
nvm install 20.19.0 && nvm use 20.19.0
\end{lstlisting}

\textbf{Trên Windows:} Tải và cài đặt nvm-windows từ: \url{https://github.com/coreybutler/nvm-windows}, sau đó chạy lệnh tương tự.

\textbf{2. Cài đặt pnpm và Docker}
\begin{lstlisting}
npm install -g pnpm@10.14.0
\end{lstlisting}
\newpage
Docker: Tải Docker Desktop từ \url{https://www.docker.com/products/docker-desktop} (Windows/macOS) hoặc cài qua apt trên Ubuntu:
\begin{lstlisting}
sudo apt-get update && sudo apt-get install docker.io docker-compose-v2
sudo usermod -aG docker $USER
\end{lstlisting}

\refstepcounter{appendixstep}
\subsubsection*{\theappendixstep. Clone và cấu hình môi trường}

\begin{lstlisting}
# Clone repository
git clone https://github.com/toantran292/datn.git unified-teamspace && cd unified-teamspace

# Chay setup script (tao .env.dev va sinh JWT keys)
make dev.setup
\end{lstlisting}

\textbf{Cấu hình biến môi trường Docker} - Chỉnh sửa file \texttt{infra/docker/.env.dev}:
\begin{lstlisting}
# PostgreSQL
PG_USER=postgres
PG_PASSWORD=<mat-khau-manh>
PG_DB=uts_db
PG_PORT=5432

# Redis
REDIS_PORT=6379

# MinIO S3 Storage
MINIO_ROOT_USER=minioadmin
MINIO_ROOT_PASSWORD=<mat-khau-manh>
MINIO_API_PORT=41200
MINIO_CONSOLE_PORT=41201

# OpenSearch
OPENSEARCH_PORT=9200
OS_DASHBOARDS_PORT=5601
\end{lstlisting}

\textbf{Cấu hình services} - Tạo file \texttt{.env} từ template cho mỗi service:
\begin{lstlisting}
cd services/tenant-bff && cp .env.example .env
# Chinh sua .env voi cac gia tri phu hop
\end{lstlisting}

\refstepcounter{appendixstep}
\subsubsection*{\theappendixstep. Cài đặt Dependencies và Build}

\begin{lstlisting}
# Cai dat dependencies cho toan bo monorepo
pnpm install

# Build cac shared packages
pnpm build:packages
\end{lstlisting}

\refstepcounter{appendixstep}
\subsubsection*{\theappendixstep. Khởi động Infrastructure Services}

\begin{lstlisting}
# Khoi dong tat ca infrastructure (PostgreSQL, Redis, MinIO, OpenSearch)
make dev.up

# Kiem tra trang thai
make dev.ps

# Xem logs
make dev.logs
\end{lstlisting}

\textbf{Kiểm tra health:}
\begin{lstlisting}
docker exec uts_pg pg_isready -U postgres     # PostgreSQL
docker exec uts_redis redis-cli PING          # Redis
curl http://localhost:41200/minio/health/live # MinIO
curl http://localhost:9200                    # OpenSearch
\end{lstlisting}

\refstepcounter{appendixstep}
\subsubsection*{\theappendixstep. Khởi tạo Database và chạy ứng dụng}

\begin{lstlisting}
# Chay migrations
cd services/tenant-bff
pnpm run migration:run
pnpm run seed  # (tuy chon) Seed du lieu mau
\end{lstlisting}

\textbf{Chạy Development:}
\begin{lstlisting}
# Chay tat ca frontend apps (su dung Turbo)
pnpm dev

# Hoac chay tung app rieng le
pnpm marketing:dev
pnpm --filter @uts/tenant-web dev

# Chay backend services (moi service trong terminal rieng)
cd services/tenant-bff && pnpm start:dev
cd services/chat && pnpm start:dev
\end{lstlisting}

\textbf{Các địa chỉ truy cập:}
\begin{itemize}
    \item Marketing Web: \url{http://localhost:5173}
    \item Tenant Web: \url{http://localhost:5174}
    \item Tenant BFF API: \url{http://localhost:8085}
    \item MinIO Console: \url{http://localhost:41201}
    \item OpenSearch Dashboards: \url{http://localhost:5601}
\end{itemize}

\refstepcounter{appendixstep}
\subsubsection*{\theappendixstep. Build Production}

\begin{lstlisting}
# Build tat ca
pnpm build

# Build backend service
cd services/tenant-bff && pnpm build && pnpm start:prod
\end{lstlisting}

\refstepcounter{appendixstep}
\subsubsection*{\theappendixstep. Các lệnh Makefile hữu ích}

\begin{lstlisting}
make dev.logs.postgres    # Xem logs PostgreSQL
make dev.logs.redis       # Xem logs Redis
make dev.restart.postgres # Restart PostgreSQL
make dev.build.postgres   # Build lai (no-cache)
make dev.down             # Dung tat ca services
make dev.clean            # Dung va xoa volumes (mat du lieu)
\end{lstlisting}

\refstepcounter{appendixstep}
\subsubsection*{\theappendixstep. Testing và Linting}

\begin{lstlisting}
pnpm test                 # Chay tat ca tests
pnpm test:cov             # Test voi coverage
pnpm lint                 # Chay linter
pnpm typecheck            # Kiem tra TypeScript types

# E2E tests
cd services/tenant-bff && pnpm test:e2e
\end{lstlisting}

\refstepcounter{appendixstep}
\subsubsection*{\theappendixstep. Troubleshooting}

\textbf{1. Port đã được sử dụng:}
\begin{lstlisting}
lsof -i :5432    # Kiem tra port
kill -9 <PID>    # Kill process
\end{lstlisting}

\textbf{2. Docker container không healthy:}
\begin{lstlisting}
docker logs uts_pg        # Xem logs
docker restart uts_pg     # Restart container
\end{lstlisting}

\textbf{3. Module not found:}
\begin{lstlisting}
pnpm clean && pnpm install && pnpm build:packages
\end{lstlisting}

\textbf{4. Permission errors (Linux):}
\begin{lstlisting}
sudo chown -R $USER:$USER .
sudo usermod -aG docker $USER && newgrp docker
\end{lstlisting}

\textbf{5. JWT keys không tồn tại:}
\begin{lstlisting}
cd services/identity
openssl genpkey -algorithm RSA -out private.pem -pkeyopt rsa_keygen_bits:2048
openssl rsa -pubout -in private.pem -out public.pem
\end{lstlisting}

\refstepcounter{appendixstep}
\subsubsection*{\theappendixstep. Clean Up}

\begin{lstlisting}
make dev.down    # Dung containers
make dev.clean   # Xoa volumes (mat du lieu)
pnpm clean       # Xoa node_modules va build artifacts
\end{lstlisting}

\refstepcounter{appendixstep}
\subsubsection*{\theappendixstep. Cấu trúc thư mục dự án}

\begin{lstlisting}
unified-teamspace/
+-- apps/                  # Frontend applications
|   +-- marketing-web/     # Marketing website (Vite + React)
|   +-- tenant-web/        # Tenant dashboard
|   +-- pm-web/            # Project Management web
|   +-- chat-web/          # Chat interface
|   +-- auth-web/          # Authentication UI
+-- services/              # Backend microservices (NestJS)
|   +-- tenant-bff/        # Tenant Backend-for-Frontend
|   +-- pm/                # Project Management service
|   +-- chat/              # Chat service
|   +-- notification/      # Notification service
|   +-- file-storage/      # File storage service
+-- packages/              # Shared packages
|   +-- design-system/     # UI component library
|   +-- types/             # Shared TypeScript types
+-- infra/docker/          # Docker Compose files
+-- Makefile               # Automation scripts
+-- turbo.json             # Turbo monorepo config
+-- package.json           # Root package.json
\end{lstlisting}

\refstepcounter{appendixstep}
\subsubsection*{\theappendixstep. Kết luận}

Hướng dẫn này cung cấp các bước cơ bản để cài đặt và chạy hệ thống Unified Teamspace ở môi trường development. Đối với môi trường production, cần có thêm:
\begin{itemize}
    \item Cấu hình SSL/TLS certificates và reverse proxy (Nginx, Traefik)
    \item Monitoring và logging (Prometheus, Grafana, ELK Stack)
    \item Database backup và recovery strategies
    \item CI/CD pipeline với GitHub Actions hoặc GitLab CI
    \item Container orchestration với Kubernetes hoặc Docker Swarm
\end{itemize}
