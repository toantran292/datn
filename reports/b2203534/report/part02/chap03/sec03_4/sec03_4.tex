\subsection{Thiết kế giao diện}

\subsubsection{Cơ sở thiết kế giao diện}

Giao diện người dùng của hệ thống nền tảng phần mềm dạng dịch vụ tích hợp trí tuệ nhân tạo được thiết kế hướng đến tính trực quan, thẩm mỹ, hiện đại và dễ sử dụng, đảm bảo trải nghiệm người dùng tốt nhất và đồng bộ với nhận diện thương hiệu doanh nghiệp. Thiết kế lấy cảm hứng từ các nền tảng phần mềm dạng dịch vụ hiện đại như Linear và Stripe Dashboard.

Về màu sắc chủ đạo, ứng dụng sử dụng màu cam (màu chính - \#FF8800) làm màu chủ đạo để tạo cảm giác năng động, sáng tạo và hiện đại. Màu xanh ngọc (màu phụ - \#00B894) được sử dụng cho các điểm nhấn phụ. Màu nền chủ yếu là xám rất nhạt (\#F8F9FA) và trắng, kết hợp với chữ màu đen đậm (\#1A1A1A) để tạo sự tương phản rõ ràng.

Các thành phần nổi bật như nút hành động chính sử dụng gradient màu cam để thu hút sự chú ý. Hệ thống sử dụng bóng đổ mềm, góc bo tròn (10-20px), và gradient tinh tế để tạo chiều sâu.

Ở thanh điều hướng bên, được đặt cố định ở phía trái của trang với màu nền trắng, hiển thị logo không gian làm việc, tên và gói dịch vụ. Thanh điều hướng bên giúp người dùng dễ dàng truy cập các chức năng chính như Tổng quan, Thành viên, Tập tin, Báo cáo trí tuệ nhân tạo, Nhật ký kiểm tra và Thiết lập.

Về thiết kế các nút chức năng: Nút chính sử dụng gradient cam (\#FF8800) cho hành động chính. Nút phụ sử dụng màu trắng với viền cho hành động phụ. Nút cảnh báo dùng màu đỏ nhạt cho các hành động nguy hiểm.

Bố cục tổng thể được chia thành: thanh điều hướng bên (200px), thanh trên với tìm kiếm và thông tin người dùng, và vùng nội dung. Sử dụng phông chữ hệ thống (-apple-system, BlinkMacSystemFont, Segoe UI, Roboto). Thiết kế đáp ứng đảm bảo hiển thị tốt trên các thiết bị.

\subsubsection{Phác thảo thiết kế giao diện}

\textbf{Giao diện đăng ký và đăng nhập:}

Giao diện chức năng đăng ký cho phép người dùng tạo tài khoản để đăng nhập vào hệ thống. Để đăng ký, người dùng cần nhập họ tên, thư điện tử, mật khẩu và xác nhận mật khẩu. Ngoài ra, người dùng có thể đăng ký nhanh bằng tài khoản Google (Google OAuth 2.0).

Giao diện chức năng đăng nhập cho phép người dùng đăng nhập vào hệ thống. Để đăng nhập, người dùng cần nhập thư điện tử, mật khẩu. Nếu quên mật khẩu, người dùng có thể nhấn vào liên kết "Quên mật khẩu?" để được hướng dẫn đặt lại.

\begin{figure}[H]
\centering
\includegraphics[width=0.5\textwidth]{images/uc01_02_auth_ui.png}
\caption{Giao diện phác thảo chức năng đăng ký và chức năng đăng nhập}
\label{fig:ui_auth}
\end{figure}

\textbf{Giao diện quản lý mật khẩu:}

Giao diện quản lý mật khẩu bao gồm hai phần: quên mật khẩu và đổi mật khẩu. Chức năng quên mật khẩu cho phép người dùng nhập thư điện tử để nhận liên kết đặt lại. Chức năng đổi mật khẩu yêu cầu nhập mật khẩu hiện tại, mật khẩu mới và xác nhận mật khẩu mới.

\begin{figure}[H]
\centering
\includegraphics[width=0.6\textwidth]{images/uc04_password_ui.png}
\caption{Giao diện phác thảo chức năng quản lý mật khẩu}
\label{fig:ui_password}
\end{figure}

\newpage

\textbf{Giao diện trang cá nhân của người dùng:}

Giao diện trang cá nhân của người dùng hiển thị thông tin tài khoản bao gồm: họ tên, thư điện tử, số điện thoại, vai trò và trạng thái xác thực. Thanh điều hướng bên cho phép chuyển đổi giữa các tab: Thông tin cá nhân, Mật khẩu, Thông báo và Phiên đăng nhập. Phần Tài khoản liên kết hiển thị các tài khoản liên kết như Google OAuth.

\begin{figure}[H]
\centering
\includegraphics[width=0.55\textwidth]{images/uc05_profile_ui.png}
\caption{Giao diện phác thảo trang cá nhân của người dùng}
\label{fig:ui_profile}
\end{figure}

\textbf{Giao diện danh sách không gian làm việc:}

Giao diện danh sách không gian làm việc hiển thị tất cả các không gian làm việc mà người dùng là thành viên. Mỗi không gian làm việc được hiển thị dưới dạng thẻ bao gồm: logo, tên không gian làm việc, số lượng thành viên và vai trò của người dùng (Chủ sở hữu, Quản trị viên, Thành viên). Thanh tìm kiếm cho phép lọc không gian làm việc theo tên. Nút "Tạo không gian làm việc" cho phép tạo không gian làm việc mới.

\begin{figure}[H]
\centering
\includegraphics[width=0.55\textwidth]{images/ui_workspaces.png}
\caption{Giao diện phác thảo danh sách không gian làm việc}
\label{fig:ui_workspaces}
\end{figure}

\textbf{Giao diện cài đặt không gian làm việc:}

Giao diện cài đặt không gian làm việc cho phép Chủ sở hữu/Quản trị viên cấu hình thông tin không gian làm việc. Các tab bao gồm: Chung (thông tin cơ bản như logo, tên, đường dẫn rút gọn), Thiết lập trí tuệ nhân tạo (nhà cung cấp mô hình ngôn ngữ lớn mặc định: OpenAI, Anthropic, Google), Quyền và Vùng nguy hiểm. Phần Thông tin cơ bản cho phép thay đổi logo và tên không gian làm việc.

\begin{figure}[H]
\centering
\includegraphics[width=0.35\textwidth]{images/uc07_settings_ui.png}
\caption{Giao diện phác thảo chức năng cài đặt không gian làm việc}
\label{fig:ui_settings}
\end{figure}

\textbf{Giao diện bảng điều khiển quản trị:}

Giao diện bảng điều khiển quản trị (dành cho Quản trị viên hệ thống) hiển thị tổng quan toàn bộ hệ thống. Phần trên cùng gồm các thẻ thống kê: Tổng số người dùng, Không gian làm việc, Báo cáo trí tuệ nhân tạo và Lưu trữ. Bảng Quản lý không gian làm việc hiển thị danh sách tất cả không gian làm việc với thông tin: tên, số thành viên, trạng thái (Hoạt động/Bị khóa) và các nút hành động Khóa/Mở khóa.

\begin{figure}[H]
\centering
\includegraphics[width=0.5\textwidth]{images/uc08_admin_ui.png}
\caption{Giao diện phác thảo bảng điều khiển quản trị}
\label{fig:ui_admin}
\end{figure}

\textbf{Giao diện bảng điều khiển của không gian làm việc:}

Giao diện bảng điều khiển của không gian làm việc hiển thị tổng quan về hoạt động trong không gian làm việc. Phần trên cùng gồm các thẻ thống kê hiển thị: Thành viên, Tập tin, Báo cáo trí tuệ nhân tạo và Hoạt động kèm theo phần trăm thay đổi. Biểu đồ Tổng quan sử dụng thể hiện xu hướng hoạt động theo thời gian. Danh sách Hoạt động gần đây hiển thị các hoạt động gần đây.

\begin{figure}[H]
\centering
\includegraphics[width=0.6\textwidth]{images/uc09_dashboard_ui.png}
\caption{Giao diện phác thảo bảng điều khiển của không gian làm việc}
\label{fig:ui_dashboard}
\end{figure}

\textbf{Giao diện nhật ký kiểm tra:}

Giao diện nhật ký kiểm tra cho phép Chủ sở hữu/Quản trị viên theo dõi tất cả hoạt động trong không gian làm việc. Các bộ lọc cho phép lọc theo loại hành động (Tải lên tập tin, Xóa tập tin, Thêm thành viên...) và theo người thực hiện. Bảng hiển thị: Hành động (loại hành động và chi tiết), Người dùng (người thực hiện), Thời gian và Địa chỉ IP.

\begin{figure}[H]
\centering
\includegraphics[width=0.6\textwidth]{images/uc10_audit_ui.png}
\caption{Giao diện phác thảo chức năng xem nhật ký kiểm tra}
\label{fig:ui_audit}
\end{figure}

\textbf{Giao diện quản lý thành viên:}

Giao diện quản lý thành viên hiển thị danh sách tất cả thành viên trong không gian làm việc. Bảng thông tin bao gồm: Thành viên (ảnh đại diện, họ tên, thư điện tử), Vai trò (Chủ sở hữu, Quản trị viên, Thành viên), Trạng thái (Hoạt động, Đang chờ), Ngày tham gia và Hành động (Chỉnh sửa, Xóa). Thanh tìm kiếm cho phép lọc theo tên hoặc thư điện tử.

\begin{figure}[H]
\centering
\includegraphics[width=0.6\textwidth]{images/uc11_members_ui.png}
\caption{Giao diện phác thảo chức năng quản lý thành viên}
\label{fig:ui_members}
\end{figure}

\textbf{Giao diện biểu mẫu mời thành viên:}

Giao diện biểu mẫu mời thành viên (hộp thoại) cho phép Chủ sở hữu/Quản trị viên mời người mới vào không gian làm việc. Các thành phần bao gồm: vùng nhập danh sách thư điện tử (mỗi thư điện tử một dòng), danh sách thả xuống chọn vai trò (Thành viên hoặc Quản trị viên), vùng nhập tin nhắn đính kèm và nút Gửi lời mời để gửi lời mời.

\begin{figure}[H]
\centering
\includegraphics[width=0.35\textwidth]{images/uc11_invite_ui.png}
\caption{Giao diện phác thảo biểu mẫu mời thành viên}
\label{fig:ui_invite}
\end{figure}

\textbf{Giao diện quản lý tập tin:}

Giao diện quản lý tập tin hiển thị danh sách tất cả tập tin đã tải lên trong không gian làm việc. Bảng thông tin bao gồm: Tên (biểu tượng loại tập tin và tên), Kích thước (dung lượng), Người tải lên (người tải lên), Ngày (ngày tải lên) và Hành động (Tải xuống, Xem trước, Xóa). Nút Tải lên cho phép tải lên tập tin mới.

\begin{figure}[H]
\centering
\includegraphics[width=0.6\textwidth]{images/uc13_14_files_ui.png}
\caption{Giao diện phác thảo chức năng quản lý tập tin}
\label{fig:ui_files}
\end{figure}

\textbf{Giao diện thông báo:}

Giao diện thông báo hiển thị danh sách tất cả thông báo của người dùng. Mỗi thông báo bao gồm: biểu tượng loại thông báo, tiêu đề, mô tả chi tiết và thời gian. Các loại thông báo: Lời mời không gian làm việc, Tập tin mới được tải lên, Báo cáo trí tuệ nhân tạo sẵn sàng, Thiết lập đã cập nhật. Liên kết "Đánh dấu tất cả đã đọc" cho phép đánh dấu tất cả đã đọc. Thông báo chưa đọc có chỉ báo màu cam.

\begin{figure}[H]
\centering
\includegraphics[width=0.35\textwidth]{images/uc15_notifications_ui.png}
\caption{Giao diện phác thảo trang thông báo}
\label{fig:ui_notifications}
\end{figure}

\textbf{Giao diện báo cáo trí tuệ nhân tạo:}

Giao diện báo cáo trí tuệ nhân tạo hiển thị danh sách các báo cáo đã được tạo bởi trí tuệ nhân tạo. Báo cáo nổi bật hiển thị ở trên cùng với nhãn loại báo cáo, tiêu đề, mô tả và nhà cung cấp mô hình ngôn ngữ lớn đã sử dụng. Danh sách báo cáo hiển thị các báo cáo khác với các loại: Tóm tắt hàng ngày, Tổng hợp hàng tuần, Tùy chỉnh. Mỗi báo cáo có nhãn màu tương ứng với loại.

\begin{figure}[H]
\centering
\includegraphics[width=0.6\textwidth]{images/uc16_17_reports_ui.png}
\caption{Giao diện phác thảo chức năng xem báo cáo trí tuệ nhân tạo}
\label{fig:ui_ai_reports}
\end{figure}
