\subsection{Tổng quan hệ thống}

Trong bối cảnh chuyển đổi số hiện nay, nhu cầu về các nền tảng quản lý công việc và cộng tác trực tuyến ngày càng tăng cao, đặc biệt đối với các doanh nghiệp và tổ chức muốn tối ưu hóa quy trình làm việc. Người dùng không chỉ cần các công cụ quản lý cơ bản mà còn mong muốn có khả năng tích hợp trí tuệ nhân tạo để tự động hóa và tạo báo cáo thông minh.

Đáp ứng xu hướng đó, hệ thống "Nền tảng phần mềm dạng dịch vụ tích hợp trí tuệ nhân tạo cho quản lý không gian làm việc" được phát triển nhằm cung cấp một nền tảng toàn diện cho việc quản lý không gian làm việc, thành viên, tập tin và tạo báo cáo tổng hợp bằng trí tuệ nhân tạo. Hệ thống sử dụng nền tảng phần mềm dạng dịch vụ, hỗ trợ đa không gian làm việc, phân quyền linh hoạt và tích hợp với các nhà cung cấp mô hình ngôn ngữ lớn như OpenAI, Anthropic Claude để tạo thông tin chi tiết từ dữ liệu.

Điểm nổi bật của hệ thống so với các ứng dụng quản lý không gian làm việc khác là khả năng tích hợp trí tuệ nhân tạo để tự động tổng hợp thông tin, tạo báo cáo thông minh và cung cấp thông tin chi tiết hữu ích cho người dùng. Ngoài ra, hệ thống còn hỗ trợ đầy đủ các chức năng như đăng nhập bằng Google OAuth, quản lý không gian làm việc với nhiều cấp độ phân quyền, tải lên và quản lý tập tin, thông báo thời gian thực.

Về mặt kỹ thuật, hệ thống được xây dựng theo kiến trúc vi dịch vụ với Next.js cho giao diện người dùng, Spring Boot cho dịch vụ tài khoản (xác thực), NestJS cho các dịch vụ còn lại, PostgreSQL để lưu trữ dữ liệu, Redis cho bộ nhớ đệm và quản lý phiên, MinIO/S3 cho lưu trữ đối tượng.

Việc xây dựng một nền tảng phần mềm dạng dịch vụ quản lý không gian làm việc yêu cầu một hệ thống được thiết kế chuyên biệt, từ giao diện người dùng, kiến trúc vi dịch vụ, cách tích hợp trí tuệ nhân tạo, cho đến khả năng mở rộng. Sự chuyên biệt hóa này là yếu tố then chốt tạo nên giá trị cho ứng dụng, đồng thời đáp ứng đúng nhu cầu của nhóm khách hàng doanh nghiệp muốn số hóa và tự động hóa quy trình quản lý công việc.
