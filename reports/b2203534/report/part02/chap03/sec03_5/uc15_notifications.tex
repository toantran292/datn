% UC15: Quản lý thông báo
\subsubsection{Chức năng quản lý thông báo (UC15)}

\begin{itemize}
    \item \textbf{Mục đích:} Cho phép người dùng xem, đánh dấu đã đọc và cấu hình thông báo.
\end{itemize}

\begin{itemize}
    \item \textbf{Giao diện:}
\end{itemize}

\begin{figure}[H]
    \centering
    \includegraphics[width=0.55\textwidth]{part02/chap03/sec03_5/uc15_notifications_ui.png}
    \caption{Giao diện quản lý thông báo}
    \label{fig:ui_uc15_notifications}
\end{figure}

\begin{itemize}
    \item \textbf{Các thành phần trong giao diện:}
\end{itemize}

\begin{longtblr}[
    caption = {Các thành phần trong giao diện quản lý thông báo},
    label = {tab:ui_uc15}
]{
    colspec = {|c|l|l|X|},
    rowhead = 1,
    hlines,
    row{1} = {font=\bfseries},
}
STT & Loại & Tên thành phần & Nội dung thực hiện \\
1 & Link & Mark all read & Đánh dấu tất cả thông báo là đã đọc \\
2 & List Item & Notification Item & Hiển thị: icon, tiêu đề, mô tả, thời gian, trạng thái đọc \\
3 & Link & View all notifications & Xem tất cả thông báo (phân trang) \\
\end{longtblr}

\begin{itemize}
    \item \textbf{Dữ liệu được dùng:}
\end{itemize}

\begin{longtblr}[
    caption = {Dữ liệu cho chức năng quản lý thông báo},
    label = {tab:data_uc15}
]{
    colspec = {|c|X|c|c|c|c|},
    rowhead = 1,
    hlines,
    row{1} = {font=\bfseries},
}
STT & Tên bảng & Thêm & Sửa & Xóa & Truy vấn \\
1 & Notification & & X & & X \\
2 & Notification\-Settings & & X & & X \\
\end{longtblr}

\begin{itemize}
    \item \textbf{Xử lý:}
\end{itemize}

\begin{figure}[H]
    \centering
    \includegraphics[width=0.95\textwidth]{part02/chap03/sec03_5/uc15_notifications_activity.png}
    \caption{Sơ đồ hoạt động chức năng quản lý thông báo}
    \label{fig:act_uc15_notifications}
\end{figure}
