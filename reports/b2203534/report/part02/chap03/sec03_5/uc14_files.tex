% UC14: Quản lý tệp tin
\subsubsection{Chức năng quản lý tệp tin (UC14)}

\begin{itemize}
    \item \textbf{Mục đích:} Cho phép người dùng xem, tải xuống, xóa và tìm kiếm tệp tin trong work\-space.
\end{itemize}

\begin{itemize}
    \item \textbf{Giao diện:}
\end{itemize}

Sử dụng cùng giao diện với UC13 (Hình \ref{fig:ui_uc13_files}).

\begin{itemize}
    \item \textbf{Các thành phần trong giao diện:}
\end{itemize}

\begin{longtblr}[
    caption = {Các thành phần trong giao diện quản lý tệp tin},
    label = {tab:ui_uc14}
]{
    colspec = {|c|l|l|X|},
    rowhead = 1,
    hlines,
    row{1} = {font=\bfseries},
}
STT & Loại & Tên thành phần & Nội dung thực hiện \\
1 & Button & Download & Tải file về máy tính \\
2 & Button & Preview & Xem trước nội dung file (nếu hỗ trợ) \\
3 & Button & Delete & Xóa file khỏi work\-space \\
4 & Textbox & Search & Tìm kiếm file theo tên \\
\end{longtblr}

\begin{itemize}
    \item \textbf{Dữ liệu được dùng:}
\end{itemize}

\begin{longtblr}[
    caption = {Dữ liệu cho chức năng quản lý tệp tin},
    label = {tab:data_uc14}
]{
    colspec = {|c|X|c|c|c|c|},
    rowhead = 1,
    hlines,
    row{1} = {font=\bfseries},
}
STT & Tên bảng & Thêm & Sửa & Xóa & Truy vấn \\
1 & File & & & X & X \\
2 & AuditLog & X & & & \\
\end{longtblr}

\begin{itemize}
    \item \textbf{Xử lý:}
\end{itemize}

\begin{figure}[H]
    \centering
    \includegraphics[width=1\textwidth]{part02/chap03/sec03_5/uc14_files_activity.png}
    \caption{Sơ đồ hoạt động chức năng quản lý tệp tin}
    \label{fig:act_uc14_files}
\end{figure}
