\subsection{Giới thiệu về công nghệ phía máy chủ}

\subsubsection{Khái quát}

Hệ thống phía máy chủ được xây dựng theo kiến trúc vi dịch vụ, trong đó mỗi dịch vụ chịu trách nhiệm cho một phạm vi nghiệp vụ cụ thể và giao tiếp với nhau thông qua giao diện lập trình hoặc hàng đợi tin nhắn. Kiến trúc này mang lại nhiều lợi ích: mỗi dịch vụ có thể triển khai độc lập mà không ảnh hưởng đến các dịch vụ khác, linh hoạt trong việc chọn bộ công nghệ phù hợp nhất cho từng dịch vụ, dễ dàng mở rộng từng dịch vụ riêng biệt theo nhu cầu, và cách ly lỗi đảm bảo lỗi ở một dịch vụ không làm sập toàn hệ thống.

Hệ thống sử dụng hai khung phát triển phía máy chủ chính: Spring Boot cho dịch vụ tài khoản do yêu cầu bảo mật cao và tích hợp xác thực, và NestJS cho các dịch vụ còn lại nhờ khả năng phát triển nhanh và hệ sinh thái phong phú. Cả hai khung phát triển đều tuân thủ nguyên tắc thiết kế giao diện lập trình, cung cấp tài liệu giao diện lập trình, và được đóng gói bằng công cụ chứa để dễ dàng triển khai và mở rộng.

\subsubsection{Spring Boot và các thành phần}

Spring Boot là một phần mở rộng của Spring Framework giúp đơn giản hóa việc tạo và thiết lập ứng dụng. Khung phát triển này cung cấp tự động thiết lập ứng dụng dựa trên các thư viện phụ thuộc, máy chủ tích hợp sẵn không cần triển khai riêng, các gói thư viện liên quan, và các tính năng sẵn sàng cho môi trường thực tế như kiểm tra sức khỏe, đo lường, thiết lập từ bên ngoài. Dịch vụ tài khoản được xây dựng bằng Spring Boot do yêu cầu bảo mật cao và khả năng tích hợp xác thực mạnh mẽ.

Spring Security là khung bảo mật toàn diện được sử dụng cho xác thực và phân quyền trong dịch vụ tài khoản. Khung này hỗ trợ xác thực với cặp mã thông báo truy cập và mã thông báo làm mới để quản lý phiên an toàn, đăng nhập qua tài khoản bên thứ ba, và mã hóa mật khẩu để bảo vệ thông tin nhạy cảm. Spring Data JPA cung cấp tầng truy cập dữ liệu với PostgreSQL, sử dụng mẫu kho lưu trữ với các phương thức truy vấn tự động tạo từ tên phương thức, giúp giảm thiểu mã lặp lại. Spring MVC xử lý giao diện lập trình, cung cấp các tính năng như ánh xạ yêu cầu, thương lượng nội dung, và xử lý ngoại lệ.

\subsubsection{NestJS và kiến trúc mô-đun}

NestJS là một khung phát triển Node.js tiên tiến để xây dựng các ứng dụng phía máy chủ hiệu quả và có khả năng mở rộng. Khung này được thiết kế với kiến trúc mô-đun cho phép tổ chức mã nguồn thành các mô-đun độc lập dễ bảo trì và kiểm thử, có tích hợp sẵn bộ quản lý phụ thuộc giúp quản lý các thành phần hiệu quả, có hỗ trợ TypeScript với tính an toàn về kiểu dữ liệu, và hỗ trợ sẵn cho các công cụ truy cập dữ liệu và vi dịch vụ. NestJS được sử dụng cho các dịch vụ còn lại trong hệ thống do khả năng phát triển nhanh và hệ sinh thái phong phú.

Kiến trúc của NestJS được xây dựng theo các tầng rõ ràng: bộ điều khiển xử lý các yêu cầu đến và trả về phản hồi, dịch vụ chứa logic nghiệp vụ và được tiêm vào bộ điều khiển thông qua cơ chế tiêm phụ thuộc, mô-đun tổ chức và đóng gói các thành phần liên quan với nhau, bộ bảo vệ xử lý logic xác thực và phân quyền, bộ chặn chuyển đổi dữ liệu trước hoặc sau khi xử lý yêu cầu, và đường ống thực hiện kiểm tra cũng như chuyển đổi dữ liệu đầu vào. Kiến trúc này đảm bảo phân tách trách nhiệm và giúp mã nguồn dễ kiểm thử.

\subsubsection{Nguyên tắc thiết kế giao diện lập trình}

REST (Chuyển trạng thái đại diện) là một kiến trúc thiết kế giao diện lập trình sử dụng các phương thức giao thức truyền tải siêu văn bản để thực hiện các thao tác trên tài nguyên. Giao diện lập trình theo kiểu REST tuân theo các nguyên tắc quan trọng: đường dẫn đại diện cho tài nguyên (danh từ như /users, /workspaces) chứ không phải hành động (động từ), các phương thức được sử dụng đúng mục đích (GET để lấy dữ liệu, POST để tạo mới, PUT/PATCH để cập nhật, DELETE để xóa), thiết kế không trạng thái trong đó mỗi yêu cầu chứa đầy đủ thông tin cần thiết và máy chủ không lưu trạng thái của người dùng, mã trạng thái phù hợp, và phiên bản giao diện lập trình để duy trì khả năng tương thích ngược khi có thay đổi lớn.

\subsubsection{Vận dụng vào đề tài}

Phía máy chủ của hệ thống được chia thành 5 vi dịch vụ với 2 bộ công nghệ khác nhau. Dịch vụ tài khoản được xây dựng bằng Spring Boot với Spring Security và Spring Data JPA, chịu trách nhiệm xác thực, phân quyền và quản lý người dùng. Dịch vụ này cung cấp xác thực với cặp mã thông báo truy cập và làm mới, tích hợp đăng nhập qua tài khoản bên thứ ba, các thao tác quản lý người dùng, đặt lại mật khẩu, xác nhận thư điện tử, và quản lý phiên. Việc chọn Spring Boot cho dịch vụ tài khoản đảm bảo bảo mật cao và khả năng tích hợp xác thực mạnh mẽ.

Các dịch vụ còn lại được xây dựng bằng NestJS với Prisma: dịch vụ không gian làm việc quản lý không gian làm việc, thiết lập, và bảng điều khiển; dịch vụ thành viên quản lý thành viên, lời mời và quyền hạn theo mô hình phân quyền dựa trên vai trò; dịch vụ tập tin xử lý tải lên, tải xuống và quản lý tập tin với dịch vụ lưu trữ; dịch vụ thông báo gửi thông báo thời gian thực và thông báo qua thư điện tử; và dịch vụ báo cáo tích hợp mô hình ngôn ngữ lớn để tạo báo cáo bằng trí tuệ nhân tạo. Tất cả dịch vụ đều tuân thủ thiết kế giao diện lập trình với tài liệu, kiểm tra dữ liệu, xử lý lỗi với bộ xử lý ngoại lệ toàn cục, ghi nhật ký tập trung và giám sát để đảm bảo khả năng quan sát.
