\subsection{Giới thiệu về công nghệ giao diện người dùng}

\subsubsection{Khái quát}

React.js là một thư viện mã nguồn mở được sử dụng rộng rãi để xây dựng giao diện người dùng cho các ứng dụng web một trang và ứng dụng di động. React sử dụng mô hình dựa trên thành phần, cho phép lập trình viên chia nhỏ giao diện thành các thành phần độc lập và tái sử dụng được. Mỗi thành phần quản lý trạng thái riêng và có thể được kết hợp để tạo thành giao diện phức tạp, giúp mã nguồn dễ bảo trì và mở rộng.

Next.js là một khung phát triển React cung cấp các tính năng bổ sung như kết xuất phía máy chủ, tạo trang tĩnh, và nhiều tối ưu hóa khác nhằm cải thiện hiệu suất và trải nghiệm người dùng. TypeScript là một phiên bản mở rộng của JavaScript, bổ sung hệ thống kiểm tra kiểu dữ liệu tĩnh giúp phát hiện lỗi sớm và cải thiện chất lượng mã nguồn. Sự kết hợp giữa React, Next.js và TypeScript đã trở thành bộ công nghệ giao diện phổ biến nhất trong các dự án web hiện đại.

\subsubsection{Nguyên lý hoạt động}

React hoạt động dựa trên các nguyên lý cốt lõi giúp tối ưu hóa hiệu suất và trải nghiệm phát triển. Cú pháp mở rộng cho phép viết cấu trúc giao diện trong mã JavaScript, giúp mã nguồn dễ đọc và bảo trì hơn. React sử dụng mô hình đối tượng tài liệu ảo để tối ưu hiệu suất hiển thị: khi trạng thái thay đổi, React so sánh phiên bản ảo với phiên bản thật và chỉ cập nhật những phần thay đổi, giảm thiểu các thao tác tốn kém.

Trong React hiện đại, các thành phần dạng hàm kết hợp với các móc nối đã trở thành cách tiếp cận ưu tiên. Thuộc tính và trạng thái là hai khái niệm quan trọng: thuộc tính là dữ liệu được truyền từ thành phần cha xuống thành phần con (không thể thay đổi), trong khi trạng thái là dữ liệu nội bộ của thành phần, có thể thay đổi và kích hoạt hiển thị lại. Các móc nối cho phép sử dụng trạng thái và các tính năng React trong các thành phần dạng hàm một cách dễ dàng và linh hoạt.

Next.js mở rộng khả năng của React với các chiến lược hiển thị khác nhau. Kết xuất phía máy chủ hiển thị trang trên máy chủ trước khi gửi đến trình duyệt, cải thiện khả năng tìm kiếm và thời gian tải trang đầu tiên. Tạo trang tĩnh chuẩn bị sẵn các trang tại thời điểm xây dựng, phù hợp cho nội dung ít thay đổi. Next.js cũng cung cấp định tuyến dựa trên tệp tin tự động tạo đường dẫn dựa trên cấu trúc thư mục, và tối ưu hóa hình ảnh tự động.

TypeScript bổ sung tính an toàn về kiểu dữ liệu cho JavaScript, phát hiện lỗi tại thời điểm biên dịch thay vì khi chạy, giúp giảm lỗi trong môi trường thực tế. Hệ thống kiểm tra kiểu cung cấp gợi ý thông minh và công cụ tái cấu trúc mạnh mẽ trong môi trường phát triển. Chú thích kiểu giúp mã nguồn tự giải thích, dễ hiểu hơn cho nhóm phát triển.

\subsubsection{Ứng dụng và hệ sinh thái}

React và Next.js được sử dụng rộng rãi trong các ứng dụng web hiện đại, từ bảng điều khiển quản lý, nền tảng thương mại điện tử, mạng xã hội, đến các ứng dụng phần mềm dạng dịch vụ phức tạp. Hệ sinh thái React cung cấp nhiều thư viện và công cụ hỗ trợ cho việc tạo kiểu nhanh và đáp ứng, quản lý trạng thái máy chủ và bộ nhớ đệm, quản lý trạng thái phía người dùng, xử lý biểu mẫu với kiểm tra dữ liệu, và các thư viện thành phần giúp xây dựng giao diện nhanh chóng với chất lượng cao.

\subsubsection{Vận dụng vào đề tài}

Giao diện người dùng của hệ thống được xây dựng bằng Next.js với bộ định tuyến mới, sử dụng TypeScript để đảm bảo tính an toàn về kiểu dữ liệu cho toàn bộ mã nguồn. Tailwind CSS được áp dụng làm khung kiểu dáng cho việc tạo kiểu nhanh và đáp ứng trên mọi thiết bị. React Query quản lý trạng thái máy chủ, bộ nhớ đệm và lấy dữ liệu với các tính năng như tự động làm mới, cập nhật lạc quan, và phân trang. Zustand được chọn làm công cụ quản lý trạng thái nhẹ và đơn giản cho trạng thái phía người dùng như trạng thái giao diện, tùy chọn người dùng. React Hook Form xử lý kiểm tra biểu mẫu với hiệu suất cao. Shadcn/ui cung cấp các thành phần giao diện, đảm bảo khả năng tiếp cận và tùy biến cao. Toàn bộ giao diện được tổ chức theo kiến trúc mô-đun hóa với các tầng rõ ràng: thành phần giao diện, móc nối, dịch vụ, và tiện ích, giúp mã nguồn dễ bảo trì và mở rộng.
