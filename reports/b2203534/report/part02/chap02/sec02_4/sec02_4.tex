\subsection{Giới thiệu về cơ sở dữ liệu}

\subsubsection{Khái quát}

PostgreSQL là một hệ quản trị cơ sở dữ liệu quan hệ mã nguồn mở mạnh mẽ với hơn 35 năm phát triển, nổi tiếng với độ tin cậy, tính toàn vẹn dữ liệu, và bộ tính năng phong phú. PostgreSQL đảm bảo tuân thủ các tính chất giao dịch (tính nguyên tử, tính nhất quán, tính cô lập, tính bền vững), có khả năng mở rộng cao với việc tạo các hàm tùy chỉnh, kiểu dữ liệu, toán tử và phần mở rộng, hỗ trợ định dạng dữ liệu bán cấu trúc để lưu trữ và truy vấn dữ liệu hiệu quả kết hợp ưu điểm của cả cơ sở dữ liệu quan hệ và phi quan hệ, cung cấp khả năng tìm kiếm toàn văn tích hợp sẵn, hỗ trợ lập chỉ mục nâng cao với nhiều loại chỉ mục khác nhau, và sử dụng cơ chế kiểm soát đồng thời đa phiên bản cho phép đọc và ghi đồng thời mà không khóa dữ liệu.

Redis (Máy chủ từ điển từ xa) là một kho lưu trữ cấu trúc dữ liệu trong bộ nhớ được sử dụng như cơ sở dữ liệu, bộ nhớ đệm, và trung gian tin nhắn. Redis lưu trữ dữ liệu trong bộ nhớ truy cập ngẫu nhiên cho phép đọc/ghi cực nhanh với độ trễ dưới mili giây, hỗ trợ nhiều cấu trúc dữ liệu như chuỗi ký tự, danh sách, tập hợp, tập hợp có thứ tự, bảng băm, và luồng dữ liệu. Sự kết hợp giữa PostgreSQL và Redis trong một hệ thống tạo nên kiến trúc cơ sở dữ liệu lai mạnh mẽ: PostgreSQL làm kho lưu trữ bền vững với đảm bảo giao dịch, Redis làm tầng bộ nhớ đệm và xử lý dữ liệu thời gian thực.

\subsubsection{Ứng dụng và trường hợp sử dụng}

Redis được sử dụng rộng rãi trong các ứng dụng web hiện đại cho nhiều mục đích khác nhau: lưu trữ phiên làm việc của người dùng với truy cập nhanh, bộ nhớ đệm cho kết quả truy vấn cơ sở dữ liệu và phản hồi giao diện lập trình giúp giảm tải lên cơ sở dữ liệu chính, giới hạn tốc độ yêu cầu để kiểm soát số lượng yêu cầu từ máy khách, bảng xếp hạng thời gian thực với tập hợp có thứ tự, truyền thông điệp theo mô hình xuất bản/đăng ký cho giao tiếp thời gian thực giữa các dịch vụ, và hàng đợi công việc để xử lý các tác vụ nền. Trong kiến trúc phần mềm dạng dịch vụ, Redis đóng vai trò quan trọng trong việc cải thiện hiệu suất và khả năng mở rộng của hệ thống.

\subsubsection{Ánh xạ đối tượng quan hệ và tầng truy cập dữ liệu}

Ánh xạ đối tượng quan hệ là kỹ thuật ánh xạ giữa các mô hình lập trình hướng đối tượng và các bảng cơ sở dữ liệu quan hệ, giúp lập trình viên làm việc với cơ sở dữ liệu thông qua các đối tượng thay vì các câu truy vấn thô. Hệ thống sử dụng hai giải pháp ánh xạ đối tượng quan hệ khác nhau phù hợp với từng bộ công nghệ.

Spring Data JPA kết hợp với Hibernate được sử dụng trong dịch vụ tài khoản (Spring Boot). JPA là đặc tả cho ánh xạ đối tượng quan hệ trong Java, trong khi Hibernate là triển khai phổ biến nhất của JPA. Thực thể là lớp Java được đánh dấu để ánh xạ với bảng cơ sở dữ liệu, giao diện kho lưu trữ mở rộng JpaRepository và Spring tự động triển khai các thao tác tạo, đọc, cập nhật, xóa. Các phương thức truy vấn tự động tạo câu truy vấn từ tên phương thức giúp giảm mã lặp lại, và có thể viết các câu truy vấn tùy chỉnh khi cần.

Prisma là công cụ ánh xạ đối tượng quan hệ hiện đại được sử dụng trong các dịch vụ NestJS, dành cho Node.js và TypeScript với máy khách cơ sở dữ liệu an toàn về kiểu. Tệp định nghĩa lược đồ Prisma mô tả các mô hình dữ liệu và kết nối cơ sở dữ liệu bằng cú pháp khai báo, máy khách Prisma được tự động tạo với an toàn về kiểu, Prisma Migrate là công cụ di chuyển cơ sở dữ liệu tự động tạo và áp dụng các thay đổi, Prisma Studio cung cấp giao diện đồ họa để duyệt và chỉnh sửa dữ liệu, và tính năng an toàn về kiểu với hỗ trợ TypeScript đầy đủ đảm bảo kiểm tra lỗi tại thời điểm biên dịch cho các thao tác cơ sở dữ liệu.

\subsubsection{Vận dụng vào đề tài}

Hệ thống cơ sở dữ liệu được thiết kế theo kiến trúc lai kết hợp PostgreSQL và Redis để tối ưu cả về hiệu suất và tính toàn vẹn dữ liệu. PostgreSQL đóng vai trò kho lưu trữ bền vững chính lưu trữ toàn bộ dữ liệu quan trọng như người dùng, không gian làm việc, thành viên, thông tin mô tả tập tin, quyền hạn, thông báo, báo cáo, và nhật ký kiểm tra. Dữ liệu được tổ chức theo lược đồ quan hệ với khóa ngoại để đảm bảo tính toàn vẹn tham chiếu, sử dụng cột định dạng bán cấu trúc cho dữ liệu linh hoạt như thiết lập không gian làm việc và nội dung báo cáo.

Redis được sử dụng làm tầng bộ nhớ đệm và phiên làm việc với nhiều mục đích: lưu trữ phiên cho xác thực người dùng với mã thông báo làm mới, bộ nhớ đệm cho dữ liệu truy cập thường xuyên như thông tin không gian làm việc và quyền người dùng để giảm truy vấn cơ sở dữ liệu, giới hạn tốc độ cho yêu cầu giao diện lập trình sử dụng bộ đếm Redis với thời gian sống, và truyền thông điệp theo mô hình xuất bản/đăng ký cho thông báo thời gian thực giữa dịch vụ thông báo và máy khách. Tầng ánh xạ đối tượng quan hệ được triển khai khác nhau cho từng dịch vụ: dịch vụ tài khoản sử dụng Spring Data JPA và Hibernate với Flyway để quản lý thay đổi lược đồ cơ sở dữ liệu, trong khi các dịch vụ NestJS sử dụng Prisma với Prisma Migrate. Cả hai công cụ ánh xạ đối tượng quan hệ đều cung cấp truy cập cơ sở dữ liệu an toàn về kiểu và quản lý di chuyển tự động, đảm bảo tính toàn vẹn dữ liệu và giảm thiểu lỗi khi chạy.
