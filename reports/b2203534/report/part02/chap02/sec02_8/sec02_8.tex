\subsection{Giới thiệu về WebSocket và giao tiếp thời gian thực}

\subsubsection{Khái quát}

WebSocket là một giao thức cung cấp kênh giao tiếp hai chiều đầy đủ qua một kết nối TCP, khác với mẫu yêu cầu-phản hồi truyền thống của giao thức truyền tải siêu văn bản cho phép máy chủ đẩy dữ liệu đến máy khách bất kỳ lúc nào mà không cần máy khách yêu cầu. WebSocket có kết nối liên tục được giữ mở liên tục không cần kết nối lại cho mỗi thông điệp, giao tiếp hai chiều cho phép cả máy khách và máy chủ gửi thông điệp đến nhau một cách độc lập, độ trễ thấp do không có chi phí phụ của giao thức truyền tải siêu văn bản và bắt tay cho mỗi thông điệp, và sử dụng giao thức ws:// cho kết nối không mã hóa hoặc wss:// cho kết nối mã hóa TLS trong môi trường sản xuất. WebSocket đặc biệt phù hợp cho các ứng dụng yêu cầu cập nhật thời gian thực như ứng dụng trò chuyện, thông báo trực tiếp, chỉnh sửa cộng tác, trò chơi, và nền tảng giao dịch tài chính.

\subsubsection{Socket.io và tích hợp NestJS}

Socket.io là một thư viện hỗ trợ giao tiếp thời gian thực, hai chiều được xây dựng trên WebSocket với nhiều tính năng bổ sung. Socket.io cung cấp kết nối lại tự động khi kết nối bị mất do sự cố mạng, phòng và không gian tên để nhóm các kết nối cho phát sóng có mục tiêu giúp tổ chức máy khách, xác nhận với hàm gọi lại để xác nhận việc gửi thông điệp đảm bảo tin nhắn đáng tin cậy, hỗ trợ dữ liệu nhị phân cho phép gửi dữ liệu nhị phân như hình ảnh và tập tin, và cơ chế dự phòng tự động chuyển sang thăm dò HTTP dài khi WebSocket không khả dụng do tường lửa hoặc proxy.

NestJS cung cấp hỗ trợ tích hợp sẵn cho WebSocket với bộ điều hợp Socket.io thông qua các decorator và mẫu kiến trúc quen thuộc. Decorator @WebSocketGateway định nghĩa lớp cổng WebSocket, decorator @SubscribeMessage xử lý các thông điệp đến từ các sự kiện cụ thể, decorator @WebSocketServer tiêm bản sao máy chủ Socket.io để phát sự kiện, bộ bảo vệ và đường ống có thể tái sử dụng từ giao diện lập trình REST cho kết nối WebSocket để xác thực và chuyển đổi dữ liệu, và bộ chặn chuyển đổi dữ liệu trước hoặc sau khi xử lý thông điệp. Kiến trúc này cho phép lập trình viên áp dụng các mẫu và thực hành tốt nhất từ giao diện lập trình REST sang WebSocket, giảm đường cong học tập và duy trì tính nhất quán của mã nguồn.

\subsubsection{Vận dụng vào đề tài}

Dịch vụ thông báo được xây dựng bằng cổng WebSocket NestJS với bộ điều hợp Socket.io để cung cấp giao tiếp thời gian thực cho toàn hệ thống. Dịch vụ đẩy thông báo đến người dùng ngay lập tức khi có các sự kiện như thành viên mới tham gia không gian làm việc, tập tin được tải lên, đề cập trong bình luận, hoặc thiết lập không gian làm việc thay đổi, đảm bảo người dùng luôn cập nhật thông tin mới nhất mà không cần làm mới trang.

Kiến trúc dựa trên phòng được triển khai với mỗi không gian làm việc là một phòng Socket.io, cho phép phát sóng sự kiện đến tất cả thành viên trong không gian làm việc một cách hiệu quả mà không cần lặp qua từng người dùng. Sự kiện dành riêng cho người dùng được gửi đến người dùng cá nhân dựa trên mã định danh người dùng, phù hợp cho thông báo riêng tư như mật khẩu đã thay đổi, tài khoản bị khóa, hoặc đề cập trực tiếp. Số liệu bảng điều khiển được cập nhật thời gian thực hiển thị mức sử dụng lưu trữ, thành viên đang hoạt động, hoạt động gần đây mà không cần gọi liên tục giao diện lập trình. Theo dõi trạng thái trực tuyến sử dụng các sự kiện kết nối và ngắt kết nối để phát sóng trạng thái trực tuyến/ngoại tuyến của thành viên đến các thành viên không gian làm việc, giúp người dùng biết ai đang hoạt động. Xác thực được thực thi bằng bộ bảo vệ xác minh mã thông báo web JSON khi máy khách thiết lập kết nối WebSocket, đảm bảo chỉ người dùng đã xác thực mới có thể kết nối và nhận thông báo phù hợp với quyền của họ.
