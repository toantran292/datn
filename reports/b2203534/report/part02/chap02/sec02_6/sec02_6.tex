\subsection{Giới thiệu về xác thực và phân quyền}

\subsubsection{Khái quát}

Xác thực là quá trình xác minh danh tính của người dùng để đảm bảo rằng người dùng đúng là người mà họ tuyên bố. Trong ứng dụng web hiện đại, xác thực thường được thực hiện qua nhiều phương thức: dựa trên thông tin đăng nhập sử dụng tên người dùng/thư điện tử và mật khẩu truyền thống, dựa trên mã thông báo với mã thông báo web hoặc khóa giao diện lập trình cho xác thực không trạng thái, xác thực qua bên thứ ba đăng nhập thông qua các nhà cung cấp như Google, Facebook, GitHub giúp cải thiện trải nghiệm người dùng và bảo mật, và xác thực đa yếu tố kết hợp nhiều phương thức xác thực để tăng cường bảo mật. Phân quyền là quá trình tiếp theo sau xác thực, xác định người dùng có quyền thực hiện một hành động hay truy cập một tài nguyên cụ thể hay không.

\subsubsection{Mã thông báo web JSON}

Mã thông báo web JSON là một tiêu chuẩn mở (RFC 7519) để truyền thông tin an toàn giữa các bên dưới dạng đối tượng JSON được ký bằng khóa bí mật (HMAC) hoặc cặp khóa công khai/riêng tư (RSA, ECDSA). Cấu trúc của mã thông báo web JSON gồm 3 phần được mã hóa bằng Base64URL và nối với nhau bằng dấu chấm: phần đầu chứa loại mã thông báo và thuật toán ký, phần tải chứa các tuyên bố là các thông tin về người dùng và thông tin mô tả như mã định danh người dùng, thời gian hết hạn, thời gian phát hành, và chữ ký được tạo bằng cách ký phần đầu và phần tải với khóa bí mật để đảm bảo mã thông báo không bị giả mạo.

Mã thông báo web JSON mang lại nhiều ưu điểm cho ứng dụng web hiện đại: xác thực không trạng thái do máy chủ không cần lưu phiên làm việc mà mọi thông tin cần thiết đều nằm trong mã thông báo, khả năng mở rộng cao vì dễ dàng mở rộng theo chiều ngang mà không cần kho lưu trữ phiên chia sẻ giữa các bản sao máy chủ, khả năng hoạt động xuyên miền cho phép mã thông báo được sử dụng qua nhiều miền và dịch vụ, và phù hợp với ứng dụng di động. Tuy nhiên, mã thông báo web JSON cũng có nhược điểm như không thể thu hồi trước khi hết hạn và kích thước mã thông báo lớn hơn mã định danh phiên, do đó cần có chiến lược mã thông báo làm mới và danh sách đen mã thông báo để quản lý.

\subsubsection{OAuth 2.0 và OpenID Connect}

OAuth 2.0 là một khung phân quyền cho phép ứng dụng bên thứ ba truy cập tài nguyên của người dùng mà không cần chia sẻ thông tin đăng nhập, giải quyết vấn đề bảo mật khi người dùng muốn cho phép ứng dụng truy cập tài khoản của họ trên các nền tảng khác. OpenID Connect là một tầng định danh được xây dựng trên OAuth 2.0, bổ sung khả năng xác thực và thông tin hồ sơ người dùng chuẩn hóa.

Luồng mã xác thực OAuth 2.0 hoạt động theo các bước: người dùng nhấn "Đăng nhập với Google" trên ứng dụng, ứng dụng chuyển hướng người dùng đến máy chủ xác thực Google với mã định danh khách hàng và các phạm vi yêu cầu, người dùng đăng nhập tài khoản Google và cấp quyền cho ứng dụng, Google chuyển hướng về ứng dụng với mã xác thực, ứng dụng trao đổi mã với máy chủ xác thực Google để lấy mã thông báo truy cập và mã thông báo định danh, và cuối cùng ứng dụng sử dụng các mã thông báo để xác thực người dùng và lấy thông tin hồ sơ người dùng. Luồng này đảm bảo thông tin đăng nhập của người dùng không bao giờ được chia sẻ với ứng dụng bên thứ ba, chỉ có các mã thông báo xác thực.

\subsubsection{Phân quyền dựa trên vai trò}

Phân quyền dựa trên vai trò là mô hình phân quyền dựa trên các vai trò, được sử dụng rộng rãi trong ứng dụng doanh nghiệp để quản lý kiểm soát truy cập một cách có cấu trúc và dễ bảo trì. Mô hình phân quyền dựa trên vai trò bao gồm các thành phần chính: người dùng là các cá nhân sử dụng hệ thống, vai trò là tập hợp các quyền được gán cho người dùng (như Quản trị viên hệ thống, Chủ không gian làm việc, Thành viên), quyền là quyền thực hiện các hành động cụ thể trên tài nguyên (tạo, đọc, cập nhật, xóa), và tài nguyên là các đối tượng cần bảo vệ trong hệ thống (không gian làm việc, tập tin, thành viên). Thay vì gán quyền trực tiếp cho từng người dùng, phân quyền dựa trên vai trò gán quyền cho vai trò và sau đó gán vai trò cho người dùng, giúp đơn giản hóa quản lý kiểm soát truy cập đặc biệt khi hệ thống có nhiều người dùng và yêu cầu quyền phức tạp.

\subsubsection{Vận dụng vào đề tài}

Hệ thống xác thực và phân quyền được triển khai trong dịch vụ tài khoản với Spring Security. Xác thực mã thông báo web JSON sử dụng chiến lược mã thông báo kép với mã thông báo truy cập có thời hạn ngắn (15 phút) dùng cho yêu cầu giao diện lập trình và mã thông báo làm mới có thời hạn dài (7 ngày) lưu trong Redis dùng để lấy mã thông báo truy cập mới khi hết hạn. Mã thông báo truy cập được gửi trong tiêu đề xác thực với lược đồ Bearer, trong khi mã thông báo làm mới được lưu trong cookie chỉ HTTP để tăng cường bảo mật chống tấn công mã độc kịch bản.

Tích hợp Google OAuth 2.0 cho phép người dùng đăng nhập hoặc đăng ký bằng tài khoản Google, với xử lý xung đột khi thư điện tử đã tồn tại trong hệ thống: nếu người dùng đã có tài khoản với thư điện tử/mật khẩu thì cần liên kết tài khoản xác thực bên thứ ba, nếu chưa có thì tự động tạo tài khoản mới. Phân quyền dựa trên vai trò được triển khai với 3 vai trò chính có quyền khác nhau: Quản trị viên hệ thống quản lý toàn bộ hệ thống và có quyền khóa/mở khóa không gian làm việc, Chủ không gian làm việc có toàn quyền trong không gian làm việc của mình bao gồm quản lý thành viên, tập tin, thiết lập, Thành viên có quyền hạn tùy theo quyền được Chủ không gian làm việc gán như chỉ đọc, đóng góp, hoặc quyền tùy chỉnh. Bảo mật mật khẩu được đảm bảo với thuật toán băm Bcrypt sử dụng hệ số muối 12, đủ mạnh để chống tấn công vét cạn mà vẫn duy trì hiệu suất chấp nhận được.
