\subsection{Giới thiệu về Nginx}

\subsubsection{Khái quát}

Nginx (đọc là engine-x) là một máy chủ web mã nguồn mở hiệu suất cao được phát triển bởi Igor Sysoev vào năm 2004, ban đầu được thiết kế để giải quyết vấn đề C10K về xử lý 10.000 kết nối đồng thời và đã trở thành một trong những máy chủ web phổ biến nhất thế giới. Nginx có khả năng hoạt động với nhiều vai trò khác nhau: máy chủ web để phục vụ tập tin tĩnh như HTML, CSS, JavaScript, và hình ảnh, proxy ngược để chuyển tiếp yêu cầu từ máy khách đến máy chủ phía sau, cân bằng tải để phân phối lưu lượng qua nhiều máy chủ, bộ nhớ đệm HTTP để lưu trữ phản hồi giảm tải lên máy chủ phía sau, kết thúc SSL/TLS để xử lý mã hóa và giải mã HTTPS, và cổng giao diện lập trình cho định tuyến, giới hạn tốc độ, xác thực cho giao diện lập trình. Tính linh hoạt và hiệu suất cao này làm cho Nginx trở thành lựa chọn ưu tiên trong các kiến trúc vi dịch vụ và ứng dụng đám mây gốc.

\subsubsection{Kiến trúc và ưu điểm}

Nginx sử dụng kiến trúc hướng sự kiện, bất đồng bộ, và không chặn với tiến trình chủ đọc cấu hình, liên kết cổng và quản lý các tiến trình làm việc, các tiến trình làm việc xử lý yêu cầu thực tế với mỗi tiến trình có thể xử lý hàng nghìn kết nối đồng thời nhờ mô hình hướng sự kiện, và bộ quản lý bộ nhớ đệm quản lý dữ liệu bộ nhớ đệm trên đĩa. Kiến trúc này mang lại nhiều ưu điểm: khả năng xử lý đồng thời cao với hàng chục nghìn kết nối đồng thời với bộ nhớ tối thiểu, tiêu thụ bộ nhớ thấp vì mỗi kết nối chỉ cần vài KB bộ nhớ, hiệu suất cao do mô hình hướng sự kiện hiệu quả hơn mô hình một luồng cho mỗi kết nối truyền thống, và độ ổn định cao vì các tiến trình làm việc hoạt động độc lập nên một tiến trình bị lỗi không ảnh hưởng đến các tiến trình khác.

\subsubsection{Tính năng và khả năng}

Nginx cung cấp proxy ngược đứng trước các máy chủ phía sau nhận yêu cầu từ máy khách và chuyển tiếp đến máy chủ phía sau phù hợp, ẩn hạ tầng phía sau khỏi máy khách tăng cường bảo mật, kết thúc SSL giải mã HTTPS tại Nginx và chuyển tiếp HTTP đến máy chủ phía sau giảm tải tính toán, chỉnh sửa yêu cầu/phản hồi với tiêu đề, và hỗ trợ proxy WebSocket cho ứng dụng thời gian thực. Cân bằng tải với nhiều thuật toán: luân phiên phân phối yêu cầu tuần tự đến máy chủ (mặc định), ít kết nối nhất gửi yêu cầu đến máy chủ có ít kết nối hoạt động nhất, băm địa chỉ IP định tuyến yêu cầu từ cùng một địa chỉ IP đến cùng máy chủ cho duy trì phiên, có trọng số cho phép gán trọng số dựa trên khả năng máy chủ, và kiểm tra sức khỏe tự động loại bỏ máy chủ không khỏe mạnh khỏi nhóm.

Khả năng bộ nhớ đệm cho phép lưu trữ nội dung tĩnh và phản hồi giao diện lập trình, cấu hình hết hạn và vô hiệu hóa bộ nhớ đệm, bộ nhớ đệm proxy cho phản hồi từ máy chủ phía sau, và bộ nhớ đệm FastCGI cho nội dung PHP/động. Tính năng bảo mật bao gồm hỗ trợ SSL/TLS cho TLS 1.2/1.3 và HTTP/2 với đính kèm OCSP, giới hạn tốc độ giới hạn yêu cầu mỗi giây/phút từ một địa chỉ IP chống lạm dụng, kiểm soát truy cập cho phép/từ chối dựa trên địa chỉ IP, lọc yêu cầu chặn yêu cầu độc hại dựa trên mẫu, và giảm thiểu tấn công từ chối dịch vụ phân tán với giới hạn kết nối và giới hạn tốc độ yêu cầu.

\subsubsection{Vận dụng vào đề tài}

Nginx đóng vai trò quan trọng làm điểm truy cập duy nhất cho toàn hệ thống, hoạt động như proxy ngược và cổng giao diện lập trình định tuyến yêu cầu đến các dịch vụ phù hợp: /api/auth/* đến dịch vụ tài khoản (Spring Boot), /api/workspaces/*, /api/members/*, /api/files/*, /api/notifications/*, /api/reports/* đến các dịch vụ NestJS tương ứng, đường dẫn gốc / đến giao diện người dùng (Next.js), proxy WebSocket cho thông báo thời gian thực (/socket.io), và kết thúc SSL/TLS với chứng chỉ Let's Encrypt cho kết nối HTTPS.

Chức năng cân bằng tải phân phối lưu lượng qua nhiều bản sao của mỗi dịch vụ để đảm bảo khả năng sẵn sàng cao và khả năng mở rộng theo chiều ngang, kiểm tra sức khỏe tự động phát hiện và loại bỏ máy chủ bị lỗi khỏi vòng quay, và phiên cố định cho kết nối WebSocket đảm bảo máy khách luôn kết nối đến cùng một bản sao máy chủ duy trì trạng thái. Phục vụ tập tin tĩnh được tối ưu với Nginx phục vụ tài nguyên tĩnh Next.js như gói JavaScript, tập tin CSS, hình ảnh với tiêu đề bộ nhớ đệm phù hợp để cải thiện hiệu suất, và nén gzip cho tập tin dựa trên văn bản giảm sử dụng băng thông.

Bảo mật được thực thi ở tầng Nginx với giới hạn tốc độ giới hạn 100 yêu cầu/phút cho các điểm cuối giao diện lập trình chống lạm dụng và tấn công từ chối dịch vụ phân tán, giới hạn kích thước yêu cầu ngăn chặn tấn công tải lên tập tin lớn, tiêu đề bảo mật như X-Frame-Options, X-Content-Type-Options, Chính sách bảo mật nội dung được đưa vào phản hồi, và cấu hình chia sẻ tài nguyên xuyên nguồn cho phép truy cập giao diện lập trình được kiểm soát từ các nguồn được phép. Cấu hình Nginx được đóng gói container với Docker và quản lý bằng các công cụ hạ tầng dưới dạng mã như Terraform và Kubernetes ConfigMaps cho triển khai dễ dàng và kiểm soát phiên bản.
