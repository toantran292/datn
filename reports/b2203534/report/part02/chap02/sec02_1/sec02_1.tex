\subsection{Giới thiệu khái quát về mô hình phần mềm dạng dịch vụ}

\subsubsection{Khái quát}

Phần mềm dạng dịch vụ (SaaS) là một mô hình phân phối phần mềm trong đó ứng dụng được lưu trữ trên điện toán đám mây và cung cấp cho người dùng thông qua mạng. Thay vì cài đặt và bảo trì phần mềm trên máy tính cá nhân hoặc máy chủ riêng, người dùng có thể truy cập ứng dụng qua trình duyệt web mà không cần quan tâm đến hạ tầng bên dưới. Phần mềm dạng dịch vụ là một trong ba mô hình chính của điện toán đám mây, bên cạnh hạ tầng dạng dịch vụ và nền tảng dạng dịch vụ. Trong mô hình này, nhà cung cấp chịu trách nhiệm quản lý toàn bộ hạ tầng, bao gồm máy chủ, lưu trữ, mạng, và cả phần mềm ứng dụng.

Mô hình phần mềm dạng dịch vụ đã trở thành xu hướng chủ đạo trong ngành công nghiệp phần mềm hiện đại, đặc biệt trong các lĩnh vực quản lý doanh nghiệp, phối hợp nhóm, và tự động hóa quy trình. Các ứng dụng phần mềm dạng dịch vụ phổ biến đã chứng minh hiệu quả của mô hình này trong việc giảm chi phí vận hành và tăng khả năng tiếp cận cho người dùng.

\subsubsection{Đặc điểm và nguyên lý hoạt động}

Mô hình phần mềm dạng dịch vụ hoạt động dựa trên nguyên lý hỗ trợ nhiều tổ chức, trong đó một bản sao của ứng dụng phục vụ nhiều khách hàng khác nhau, với dữ liệu được phân tách logic hoặc vật lý để đảm bảo tính bảo mật và riêng tư. Người dùng trả phí theo mô hình đăng ký định kỳ, tức là thanh toán theo tháng hoặc năm thay vì mua giấy phép một lần, giúp giảm chi phí ban đầu và linh hoạt hơn trong việc điều chỉnh quy mô sử dụng.

Nhà cung cấp phần mềm dạng dịch vụ chịu trách nhiệm tự động cập nhật và bảo trì phần mềm mà không cần sự can thiệp của người dùng, đảm bảo ứng dụng luôn chạy với phiên bản mới nhất và được vá các lỗ hổng bảo mật kịp thời. Khả năng truy cập cao là một đặc điểm nổi bật, cho phép người dùng truy cập từ bất kỳ đâu có kết nối mạng thông qua trình duyệt web hoặc ứng dụng di động. Hệ thống cũng được thiết kế với khả năng mở rộng linh hoạt, dễ dàng điều chỉnh tài nguyên theo nhu cầu sử dụng thực tế.

\subsubsection{Ưu điểm và khả năng ứng dụng}

Mô hình phần mềm dạng dịch vụ mang lại nhiều lợi ích cho cả người dùng và doanh nghiệp. Chi phí ban đầu thấp là ưu điểm nổi bật, do không cần đầu tư vào phần cứng, hạ tầng hay giấy phép phần mềm đắt tiền. Thời gian triển khai nhanh cho phép người dùng bắt đầu sử dụng ngay sau khi đăng ký, không cần qua các bước cài đặt phức tạp hay thiết lập hệ thống. Việc bảo trì tự động do nhà cung cấp đảm nhận giúp giảm gánh nặng cho đội ngũ công nghệ nội bộ, trong khi khả năng mở rộng linh hoạt cho phép dễ dàng thêm người dùng hoặc tính năng khi cần thiết.

Hầu hết các ứng dụng phần mềm dạng dịch vụ hiện đại đều cung cấp giao diện lập trình để tích hợp với các hệ thống khác, tạo nên hệ sinh thái phần mềm liên kết và tự động hóa quy trình làm việc. Mô hình này đặc biệt phù hợp với các tổ chức muốn số hóa quy trình mà không cần đầu tư lớn vào hạ tầng công nghệ, hoặc các doanh nghiệp khởi nghiệp cần triển khai nhanh sản phẩm để thử nghiệm thị trường.

\subsubsection{Vận dụng vào đề tài}

Trong đề tài này, hệ thống được xây dựng hoàn toàn theo mô hình phần mềm dạng dịch vụ với kiến trúc hỗ trợ nhiều tổ chức, trong đó mỗi không gian làm việc đại diện cho một tổ chức độc lập với dữ liệu được phân tách hoàn toàn ở cấp cơ sở dữ liệu và logic. Người dùng truy cập hệ thống thông qua trình duyệt web mà không cần cài đặt bất kỳ phần mềm nào, đảm bảo trải nghiệm nhất quán trên mọi thiết bị và nền tảng. Hệ thống được thiết kế theo nguyên lý ưu tiên giao diện lập trình, cung cấp các giao diện lập trình để tích hợp với các dịch vụ bên ngoài như nhà cung cấp mô hình ngôn ngữ lớn, dịch vụ lưu trữ đám mây, và các công cụ quản lý khác. Kiến trúc vi dịch vụ kết hợp với đóng gói ứng dụng và điều phối cho phép mở rộng theo chiều ngang khi số lượng người dùng và không gian làm việc tăng lên, đáp ứng yêu cầu về tính sẵn sàng cao và khả năng chịu tải của một nền tảng phần mềm dạng dịch vụ chuyên nghiệp.
