\subsection{Giới thiệu về mô hình ngôn ngữ lớn}

\subsubsection{Khái quát}

Mô hình ngôn ngữ lớn là các mô hình trí tuệ nhân tạo được huấn luyện trên lượng lớn dữ liệu văn bản, có khả năng hiểu và sinh ra ngôn ngữ tự nhiên một cách linh hoạt và chính xác. Các mô hình này sử dụng kiến trúc Transformer với hàng tỷ tham số để học các mẫu phức tạp trong ngôn ngữ, từ cú pháp đơn giản đến hiểu ngữ nghĩa và suy luận ở mức cao. Khác với các mô hình trí tuệ nhân tạo truyền thống chỉ thực hiện một nhiệm vụ cụ thể, mô hình ngôn ngữ lớn có khả năng sinh văn bản mới, học với ít hoặc không có mẫu trước để thực hiện các nhiệm vụ mới, hiểu ngữ cảnh của đoạn hội thoại dài, và thực hiện nhiều loại nhiệm vụ khác nhau như tóm tắt, dịch thuật, trả lời câu hỏi, và sinh mã nguồn.

\subsubsection{Các nhà cung cấp mô hình ngôn ngữ lớn và đặc điểm}

Thị trường mô hình ngôn ngữ lớn hiện nay có ba nhà cung cấp chính với các mô hình có đặc điểm khác nhau. OpenAI GPT-4 là mô hình mạnh mẽ nhất của OpenAI với khả năng suy luận và các nhiệm vụ sáng tạo xuất sắc, hỗ trợ cửa sổ ngữ cảnh 128 nghìn mã thông báo, có phiên bản GPT-4 Turbo với giá rẻ hơn và GPT-4 Vision hỗ trợ đầu vào hình ảnh. Anthropic Claude được thiết kế với trọng tâm vào tính an toàn và hữu ích, Claude 3 có cửa sổ ngữ cảnh lên đến 200 nghìn mã thông báo cho phép xử lý tài liệu dài, với các phiên bản Opus (mạnh nhất), Sonnet (cân bằng giữa hiệu suất và chi phí), và Haiku (nhanh và rẻ cho các nhiệm vụ đơn giản). Google Gemini là mô hình đa phương thức có thể xử lý văn bản, hình ảnh, âm thanh, và video, tích hợp sâu với hệ sinh thái Google, với các phiên bản Ultra, Pro, và Nano phù hợp cho các trường hợp sử dụng khác nhau từ doanh nghiệp đến ứng dụng trên thiết bị.

\subsubsection{Kỹ thuật thiết kế câu lệnh}

Thiết kế câu lệnh là nghệ thuật và khoa học của việc thiết kế các câu lệnh hiệu quả để có được kết quả mong muốn từ mô hình ngôn ngữ lớn. Các kỹ thuật chính bao gồm: gợi ý không mẫu yêu cầu mô hình thực hiện nhiệm vụ mà không có ví dụ trước, gợi ý ít mẫu cung cấp một vài ví dụ để mô hình học mẫu và áp dụng cho nhiệm vụ chính, gợi ý chuỗi suy nghĩ yêu cầu mô hình giải thích quá trình suy luận từng bước để cải thiện độ chính xác cho các nhiệm vụ phức tạp, câu lệnh hệ thống định nghĩa vai trò và ràng buộc cho mô hình tạo ra hành vi nhất quán, và gợi ý đầu ra có cấu trúc yêu cầu mô hình trả về định dạng cụ thể như JSON hoặc XML để dễ dàng phân tích và tích hợp vào ứng dụng. Việc áp dụng đúng kỹ thuật thiết kế câu lệnh có thể cải thiện đáng kể chất lượng đầu ra và giảm các phản hồi không chính xác của mô hình ngôn ngữ lớn.

\subsubsection{Khung phát triển LangChain.js}

LangChain.js là phiên bản JavaScript/TypeScript của khung phát triển LangChain, được thiết kế để tích hợp mô hình ngôn ngữ lớn vào ứng dụng Node.js một cách có cấu trúc và dễ bảo trì. Khung phát triển này cung cấp tầng trừu tượng cho chuỗi kết hợp nhiều lệnh gọi mô hình hoặc công cụ theo trình tự, tác nhân cho phép mô hình ngôn ngữ lớn tự quyết định công cụ nào cần sử dụng để hoàn thành nhiệm vụ, bộ nhớ lưu trữ lịch sử hội thoại để duy trì ngữ cảnh qua nhiều tương tác, phân tích đầu ra có cấu trúc với lược đồ Zod để trích xuất dữ liệu có kiểu từ phản hồi của mô hình ngôn ngữ lớn, hỗ trợ nhiều nhà cung cấp (OpenAI, Anthropic, Google, và nhiều nhà cung cấp khác) với giao diện thống nhất, và khả năng truyền dữ liệu theo luồng để truyền phản hồi thời gian thực cải thiện trải nghiệm người dùng. LangChain.js giúp lập trình viên xây dựng ứng dụng sử dụng mô hình ngôn ngữ lớn một cách nhanh chóng và đáng tin cậy.

\subsubsection{Vận dụng vào đề tài}

Dịch vụ báo cáo được xây dựng bằng NestJS tích hợp mô hình ngôn ngữ lớn để tạo báo cáo trí tuệ nhân tạo tổng hợp từ dữ liệu không gian làm việc. Dịch vụ hỗ trợ cấu hình đa nhà cung cấp cho phép chủ không gian làm việc chọn giữa OpenAI GPT-4, Anthropic Claude, hoặc Google Gemini dựa trên nhu cầu và ngân sách, với khả năng chuyển đổi nhà cung cấp linh hoạt thông qua thiết lập không gian làm việc. LangChain.js được sử dụng để điều phối các lệnh gọi mô hình ngôn ngữ lớn với chuỗi, quản lý bộ nhớ hội thoại cho phản hồi nhận biết ngữ cảnh khi tạo báo cáo dài, và trích xuất dữ liệu có cấu trúc từ phản hồi của mô hình ngôn ngữ lớn với phân tích an toàn về kiểu.

Chức năng tạo báo cáo sử dụng mô hình ngôn ngữ lớn để tổng hợp dữ liệu từ nhiều nguồn như tập tin, hoạt động, và thông tin mô tả không gian làm việc thành báo cáo có thông tin chi tiết và đề xuất. Chủ không gian làm việc có thể tạo các mẫu câu lệnh tùy chỉnh cho các loại báo cáo khác nhau (tóm tắt hàng tuần, phân tích dự án, đánh giá hiệu suất), mỗi mẫu được lưu trong cơ sở dữ liệu và có thể tái sử dụng. Mô hình ngôn ngữ lớn trả về định dạng JSON có cấu trúc với các phần, phát hiện chính, và các hạng mục hành động, giúp hiển thị báo cáo có cấu trúc rõ ràng trên giao diện người dùng. Phản hồi theo luồng được triển khai để cải thiện trải nghiệm người dùng khi tạo báo cáo dài, cho phép người dùng thấy đầu ra thời gian thực thay vì chờ đợi toàn bộ phản hồi. Hệ thống cũng theo dõi số mã thông báo tiêu thụ và ước tính chi phí cho mỗi yêu cầu tạo báo cáo.
