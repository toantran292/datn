\subsection{Giới thiệu về lưu trữ đối tượng}

\subsubsection{Khái quát}

Lưu trữ đối tượng là phương thức lưu trữ dữ liệu dưới dạng các đối tượng, trong đó mỗi đối tượng bao gồm dữ liệu, thông tin mô tả, và mã định danh duy nhất. Khác với hệ thống tập tin truyền thống sử dụng cấu trúc thư mục phân cấp, lưu trữ đối tượng sử dụng cấu trúc phẳng tổ chức dữ liệu trong các thùng chứa với khóa duy nhất, phù hợp cho việc lưu trữ dữ liệu phi cấu trúc như hình ảnh, video, tài liệu, và bản sao lưu. Mỗi đối tượng có thể có thông tin mô tả tùy chỉnh giúp tổ chức và truy vấn dữ liệu hiệu quả, khả năng mở rộng cho phép dễ dàng mở rộng lên petabyte dữ liệu mà không cần thiết kế lại kiến trúc, truy cập qua giao thức truyền tải siêu văn bản thông qua giao diện lập trình REST giúp tích hợp với ứng dụng đơn giản, và độ bền cao do dữ liệu được sao chép qua nhiều nút hoặc vị trí đảm bảo dữ liệu không bị mất.

\subsubsection{Amazon S3 và đặc điểm}

Amazon S3 (Dịch vụ lưu trữ đơn giản) là dịch vụ lưu trữ đối tượng của AWS với độ bền lên đến 99,999999999\% (11 số 9), được thiết kế để lưu trữ và truy xuất bất kỳ lượng dữ liệu nào từ bất kỳ đâu trên Internet. S3 tổ chức dữ liệu trong các thùng chứa là nơi lưu trữ các đối tượng, mỗi đối tượng được xác định bằng khóa duy nhất. S3 cung cấp nhiều lớp lưu trữ như Tiêu chuẩn cho dữ liệu truy cập thường xuyên, Phân tầng thông minh tự động tối ưu chi phí, Glacier cho lưu trữ lưu trữ với chi phí thấp, phù hợp cho các mẫu truy cập khác nhau. Quản lý phiên bản cho phép lưu trữ nhiều phiên bản của một đối tượng để bảo vệ chống xóa và ghi đè nhầm. Kiểm soát truy cập được quản lý qua chính sách IAM, chính sách thùng chứa, và danh sách kiểm soát truy cập với quyền chi tiết. Đường dẫn có chữ ký tạo đường dẫn tạm thời để tải lên hoặc tải xuống đối tượng với quyền truy cập có giới hạn thời gian mà không cần tiết lộ thông tin đăng nhập. Thông báo sự kiện có thể kích hoạt hàm Lambda hoặc gửi thông điệp đến hàng đợi tin nhắn khi có thay đổi trong thùng chứa.

\subsubsection{MinIO và ứng dụng}

MinIO là một hệ thống lưu trữ đối tượng hiệu suất cao, tương thích với S3 được thiết kế cho các khối lượng công việc đám mây gốc, có thể chạy tại chỗ hoặc trên đám mây. MinIO cung cấp giao diện lập trình 100\% tương thích với Amazon S3 cho phép sử dụng các bộ công cụ phát triển S3 và công cụ hiện có mà không cần thay đổi mã nguồn, được thiết kế để đạt thông lượng cao với phần cứng thông dụng, là tệp nhị phân đơn dễ dàng triển khai với Docker hoặc Kubernetes, mã nguồn mở với giấy phép Apache 2.0 có thể tự lưu trữ miễn phí, hỗ trợ Kubernetes gốc với MinIO Operator giúp điều phối và mở rộng, và cung cấp giao diện điều khiển là giao diện quản lý dựa trên web để quản lý thùng chứa và đối tượng.

MinIO phù hợp cho nhiều trường hợp sử dụng: môi trường phát triển sử dụng lưu trữ tương thích S3 cục bộ cho phát triển và kiểm thử mà không cần kết nối đến AWS, đám mây riêng làm lưu trữ đối tượng tại chỗ cho doanh nghiệp muốn kiểm soát dữ liệu và tuân thủ yêu cầu pháp lý, tích hợp và triển khai liên tục làm kho lưu trữ sản phẩm trong quy trình tích hợp và triển khai liên tục để lưu trữ sản phẩm xây dựng và hình ảnh container, và hồ dữ liệu làm tầng lưu trữ cho các khối lượng công việc phân tích với thông lượng cao cho xử lý dữ liệu quy mô lớn.

\subsubsection{Vận dụng vào đề tài}

Hệ thống quản lý tập tin được thiết kế với chiến lược hai môi trường tận dụng giao diện lập trình tương thích S3. Môi trường phát triển sử dụng MinIO chạy trong container Docker, cung cấp giao diện lập trình tương thích S3 cho chuyển đổi mượt mà sang môi trường sản xuất, và giao diện điều khiển MinIO để quản lý thùng chứa và gỡ lỗi. Môi trường sản xuất sử dụng AWS S3 cho độ bền và khả năng sẵn sàng cao với đảm bảo độ bền 11 số 9, cùng mã nguồn với môi trường phát triển nhờ giao diện lập trình tương thích S3 không cần thay đổi mã nguồn, và chính sách vòng đời để tự động chuyển đối tượng giữa các lớp lưu trữ và quản lý chi phí lưu trữ.

Dịch vụ tập tin triển khai các tính năng chung cho cả hai môi trường: đường dẫn có chữ ký cho phép máy khách tải tập tin trực tiếp lên kho lưu trữ đối tượng mà không qua máy chủ phía sau giảm tải, và tải xuống với đường dẫn tạm thời có quyền truy cập giới hạn thời gian tăng cường bảo mật. Chiến lược thùng chứa cho mỗi không gian làm việc đảm bảo cô lập dữ liệu hoàn toàn giữa các không gian làm việc, mỗi không gian làm việc có thùng chứa riêng với chính sách truy cập riêng biệt. Xử lý tập tin tự động tạo ảnh thu nhỏ cho hình ảnh với các thư viện xử lý hình ảnh như Sharp, trích xuất thông tin mô tả từ tài liệu, và kiểm tra loại tập tin trước khi tải lên. Cơ chế xóa mềm di chuyển tập tin vào thư mục thùng rác thay vì xóa ngay, cho phép người dùng khôi phục trong vòng 30 ngày, sau đó tập tin được xóa vĩnh viễn tự động bằng chính sách vòng đời.
