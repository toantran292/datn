% UC15: Quản lý thông báo
\paragraph{Chức năng quản lý thông báo}
\mbox{}

\begin{figure}[H]
    \centering
    \includegraphics[width=0.6\textwidth]{images/uc15_notifications.png}
    \caption{Sơ đồ use case chức năng quản lý thông báo}
    \label{fig:uc15_notifications}
\end{figure}

\begin{longtblr}[
    caption = {Bảng đặc tả chức năng quản lý thông báo},
    label = {tab:uc15}
]{
    colspec = {|l|X|},
    rowhead = 1,
    hlines,
    row{1} = {font=\bfseries},
}
Tên chức năng & Quản lý thông báo \\
ID & UC15 \\
Mức độ cần thiết & Bắt buộc \\
Người sử dụng & Người dùng đã đăng nhập \\
Phân loại & Đơn giản \\
Các thành phần tham gia và mối quan tâm & \textbf{Người dùng:} Muốn xem, cấu hình và quản lý các thông báo từ hệ thống. \\
Mô tả tóm tắt & Chức năng cho phép xem danh sách thông báo, cấu hình cài đặt nhận thông báo và đánh dấu đã đọc. \\
Trigger & Người dùng nhấn vào biểu tượng thông báo hoặc truy cập cài đặt thông báo. \\
Kiểu sự kiện & External \\
Luồng xử lý bình thường &
\begin{minipage}[t]{\linewidth}
\vspace{2pt}
\begin{enumerate}[nosep, leftmargin=*]
    \item Người dùng mở trang thông báo.
    \item Hệ thống hiển thị danh sách thông báo theo thời gian.
    \item Người dùng có thể xem chi tiết, đánh dấu đã đọc hoặc cấu hình.
    \item Hệ thống cập nhật trạng thái thông báo.
\end{enumerate}
\vspace{2pt}
\end{minipage} \\
Các luồng sự kiện con &
\begin{minipage}[t]{\linewidth}
\vspace{2pt}
\begin{itemize}[nosep, leftmargin=*]
    \item Xem thông báo: Hiển thị danh sách với phân biệt đọc/chưa đọc.
    \item Cấu hình thông báo: Bật/tắt email, push, in-app notification.
    \item Đánh dấu đã đọc: Đánh dấu một hoặc tất cả thông báo.
\end{itemize}
\vspace{2pt}
\end{minipage} \\
Luồng luân phiên/đặc biệt &
\begin{minipage}[t]{\linewidth}
\vspace{2pt}
\begin{itemize}[nosep, leftmargin=*]
    \item 2a. Không có thông báo: Hiển thị thông báo trống.
\end{itemize}
\vspace{2pt}
\end{minipage} \\
Kết quả & Thông báo được xem và quản lý thành công. \\
\end{longtblr}
