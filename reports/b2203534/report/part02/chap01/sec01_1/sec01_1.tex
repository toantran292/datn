\subsection{Mô tả chi tiết bài toán}

Trong bối cảnh phát triển phần mềm hiện đại, Agile trở thành phương pháp quản lý dự án phổ biến nhờ khả năng linh hoạt, phản hồi nhanh với thay đổi và tăng cường sự phối hợp giữa các thành viên. Tuy nhiên, khi triển khai Agile trên thực tế, các nhóm phát triển thường phải sử dụng nhiều công cụ khác nhau để đáp ứng các nhu cầu đa dạng: quản lý công việc, trao đổi thông tin, họp trực tuyến, lưu trữ tài liệu, tích hợp triển khai tự động và giám sát hệ thống. Việc phải chuyển đổi giữa nhiều nền tảng riêng lẻ không chỉ gây phân mảnh thông tin, làm tốn thời gian tìm kiếm và đối chiếu dữ liệu, mà còn khó khăn trong việc duy trì tính minh bạch, theo dõi tiến độ tổng thể và tổng hợp báo cáo toàn diện về dự án.

Để giải quyết vấn đề này, cần xây dựng một nền tảng phần mềm dạng dịch vụ tích hợp, cho phép các tổ chức quản lý toàn bộ quy trình phát triển phần mềm Agile trong một môi trường duy nhất. Tuy nhiên, thách thức không chỉ nằm ở việc gộp các chức năng lại với nhau, mà còn phải đảm bảo hệ thống có khả năng mở rộng, dễ bảo trì và có thể tích hợp thêm các dịch vụ mới trong tương lai. Hơn nữa, với sự bùng nổ thông tin trong dự án Agile (danh sách công việc theo chu kỳ, câu chuyện người dùng, cuộc họp hằng ngày, hội thoại trong kênh truyền thông, tài liệu kỹ thuật, biên bản họp...), cần có một cơ chế thông minh giúp người quản lý nhanh chóng nắm bắt bức tranh tổng thể, phát hiện vấn đề tiềm ẩn và đưa ra quyết định kịp thời. Đây chính là lý do cần tích hợp các công nghệ trí tuệ nhân tạo vào nền tảng, nhằm tự động hóa việc tổng hợp, phân tích và tóm tắt thông tin từ nhiều nguồn, từ đó hỗ trợ người dùng làm việc hiệu quả hơn.

Trong phạm vi đề tài này, tác giả tập trung vào việc xây dựng \textbf{phân hệ nền tảng và thông tin} cho hệ thống phần mềm dạng dịch vụ quản lý dự án Agile tích hợp trí tuệ nhân tạo. Phân hệ này đóng vai trò là lớp xương sống, cung cấp các dịch vụ cốt lõi bao gồm: quản lý tài khoản (xác thực và phân quyền), quản lý không gian làm việc và hỗ trợ nhiều tổ chức, quản lý tập tin, hệ thống thông báo, cổng giao tiếp và tích hợp trí tuệ nhân tạo ở tầng nền tảng để tổng hợp, phân tích thông tin. Các phân hệ chức năng khác như quản lý dự án, truyền thông và họp trực tuyến sẽ được phát triển sau hoặc tích hợp từ các hệ thống bên ngoài, nhưng đều dựa trên nền tảng chung này.

Về đối tượng người dùng, hệ thống phục vụ ba nhóm chính. Nhóm đầu tiên là Quản trị viên hệ thống, người quản lý toàn bộ nền tảng với quyền tạo/xóa không gian làm việc, quản lý người dùng cấp cao, thiết lập hệ thống, giám sát hoạt động tổng thể và xử lý các vấn đề kỹ thuật. Quản trị viên hệ thống không tham gia trực tiếp vào các dự án cụ thể của không gian làm việc mà chỉ đảm bảo hệ thống hoạt động ổn định và bảo mật. Nhóm thứ hai là Chủ không gian làm việc, người đại diện cho một tổ chức hoặc nhóm phát triển, có quyền quản lý thành viên trong không gian làm việc, tạo và thiết lập các dự án, phân quyền cho các thành viên, quản lý tập tin và tài nguyên của không gian làm việc, cũng như xem các báo cáo và bảng điều khiển tổng hợp do hệ thống cung cấp. Chủ không gian làm việc có thể mời thêm thành viên, gán vai trò (như Quản lý dự án, Lập trình viên, Kiểm thử viên, Người xem…) và theo dõi tiến độ tổng thể của các dự án. Nhóm thứ ba là Thành viên, người tham gia vào một hoặc nhiều dự án trong không gian làm việc, có quyền truy cập vào các tài nguyên dự án theo vai trò được phân quyền. Thành viên có thể xem và cập nhật công việc được giao, tham gia vào các kênh truyền thông, tải lên và tải xuống tập tin liên quan đến dự án, nhận thông báo về các sự kiện quan trọng. Tùy theo vai trò, Thành viên có thể có quyền hạn khác nhau: Quản lý dự án có thể tạo công việc và gán cho người khác, Lập trình viên có thể cập nhật trạng thái công việc, Người xem chỉ có quyền xem thông tin mà không được chỉnh sửa.

Về chức năng, phân hệ nền tảng và thông tin cung cấp các dịch vụ cốt lõi sau đây. Thứ nhất là quản lý tài khoản và xác thực, cung cấp dịch vụ đăng ký, đăng nhập, quản lý phiên làm việc, tích hợp xác thực đa yếu tố nếu cần. Hệ thống sử dụng các chuẩn bảo mật phổ biến để bảo vệ giao diện lập trình, đảm bảo chỉ người dùng hợp lệ mới có quyền truy cập vào tài nguyên. Dịch vụ tài khoản quản lý thông tin người dùng, không gian làm việc, vai trò và quyền hạn, hỗ trợ phân quyền dựa trên vai trò hoặc dựa trên chính sách. Thứ hai là quản lý không gian làm việc và hỗ trợ nhiều tổ chức, cho phép một hệ thống phục vụ nhiều tổ chức khác nhau một cách độc lập và an toàn. Mỗi không gian làm việc có cơ sở dữ liệu logic riêng hoặc được phân vùng rõ ràng trong cơ sở dữ liệu chung, đảm bảo dữ liệu của không gian làm việc này không bị truy cập trái phép bởi không gian làm việc khác. Chủ không gian làm việc có thể tạo nhiều dự án trong không gian làm việc, mời thêm thành viên, gán vai trò và thiết lập quyền truy cập theo từng dự án hoặc từng tài nguyên.

Thứ ba là quản lý tập tin, cung cấp dịch vụ lưu trữ tập tin tập trung, cho phép người dùng tải lên và tải xuống tài liệu dự án, tệp đính kèm trong kênh truyền thông, biên bản họp, hình ảnh, video. Dịch vụ lưu trữ tập tin quản lý thông tin mô tả của tập tin (tên, kích thước, loại, không gian làm việc, dự án, người tải lên, thời gian tải lên), kiểm soát quyền truy cập theo không gian làm việc và dự án, hỗ trợ tìm kiếm tập tin theo tên, loại, hoặc nhãn. Hệ thống có thể lưu tập tin trên bộ nhớ cục bộ hoặc tích hợp với các dịch vụ lưu trữ đám mây để tăng khả năng mở rộng. Thứ tư là hệ thống thông báo, tập trung hóa việc gửi và nhận thông báo từ các phân hệ khác nhau (công việc được giao, trạng thái dự án thay đổi, cuộc họp sắp diễn ra, tin nhắn mới trong kênh truyền thông). Dịch vụ thông báo lưu trữ thông báo, quản lý trạng thái đã đọc/chưa đọc, ưu tiên thông báo quan trọng, và hỗ trợ gửi thông báo qua nhiều kênh (trên trang web, thư điện tử, hoặc thông báo đẩy nếu có ứng dụng di động). Người dùng có thể thiết lập loại thông báo muốn nhận và kênh nhận thông báo ưa thích.

Thứ năm là cổng giao tiếp và dịch vụ biên, đóng vai trò điểm truy cập duy nhất vào hệ thống. Dịch vụ biên thực hiện các nhiệm vụ: định tuyến yêu cầu đến các dịch vụ phía sau (tài khoản, thông báo, lưu trữ tập tin, trí tuệ nhân tạo), phân phối tải giữa các bản sao của dịch vụ, bảo mật bằng chứng chỉ số, xác thực người dùng và ký kết thông tin xác thực trước khi chuyển tiếp yêu cầu đến các dịch vụ nội bộ. Nhờ đó, các dịch vụ nội bộ không cần tự thực hiện xác thực phức tạp mà chỉ cần xác minh chữ ký từ dịch vụ biên, giảm tải cho các dịch vụ và tăng tính bảo mật. Thứ sáu là tích hợp trí tuệ nhân tạo ở tầng nền tảng, xây dựng dịch vụ trí tuệ nhân tạo có khả năng thu thập dữ liệu từ các phân hệ chức năng (quản lý dự án, truyền thông, họp trực tuyến) và cung cấp các giao diện lập trình để tổng hợp, tóm tắt, phân tích thông tin. Ví dụ: tổng hợp hoạt động gần đây của một dự án, tóm tắt nội dung chính từ các cuộc họp và hội thoại liên quan, phát hiện các rủi ro hoặc vấn đề tiềm ẩn (công việc bị trễ, thành viên quá tải), đề xuất hành động tiếp theo. Dịch vụ trí tuệ nhân tạo tích hợp với các giao diện lập trình bên ngoài hoặc các mô hình nguồn mở, xử lý dữ liệu đầu vào và trả về kết quả cho các phân hệ khác sử dụng. Cuối cùng là giao diện quản trị, xây dựng giao diện trang quản trị cho Chủ không gian làm việc, cho phép quản lý thành viên (thêm, xóa, cập nhật vai trò, khóa tài khoản), thiết lập không gian làm việc (tên, mô tả, cài đặt bảo mật, tích hợp dịch vụ bên ngoài), xem các thống kê và bảng điều khiển tổng hợp (số lượng dự án đang chạy, số lượng thành viên, dung lượng tập tin sử dụng, hoạt động gần đây) do phân hệ nền tảng cung cấp. Giao diện quản trị tích hợp với giao diện lập trình của dịch vụ tài khoản và các dịch vụ khác, hỗ trợ tìm kiếm, lọc, phân trang dữ liệu, và cung cấp trải nghiệm người dùng trực quan, dễ sử dụng.

Về kiến trúc hệ thống, phân hệ nền tảng và thông tin được thiết kế theo kiến trúc vi dịch vụ, bao gồm các dịch vụ độc lập giao tiếp qua giao diện lập trình. Dịch vụ biên đóng vai trò cổng giao tiếp, xử lý định tuyến, phân phối tải và bảo mật. Dịch vụ tài khoản đảm nhận quản lý tài khoản, không gian làm việc, xác thực và phân quyền. Dịch vụ thông báo chịu trách nhiệm quản lý và phân phối thông báo. Dịch vụ lưu trữ tập tin quản lý tập tin và thông tin mô tả. Dịch vụ trí tuệ nhân tạo tích hợp trí tuệ nhân tạo để tổng hợp và phân tích dữ liệu. Giao diện quản trị cung cấp giao diện cho Chủ không gian làm việc. Mỗi dịch vụ quản lý cơ sở dữ liệu riêng, đảm bảo tính độc lập và dễ mở rộng. Các dịch vụ giao tiếp thông qua giao thức truyền tải siêu văn bản. Khi cần đồng bộ dữ liệu giữa các dịch vụ, hệ thống sử dụng hàng đợi tin nhắn hoặc gọi giao diện lập trình trực tiếp với cơ chế thử lại và dự phòng.

Tóm lại, bài toán cần giải quyết là xây dựng một phân hệ nền tảng và thông tin đầy đủ, có khả năng mở rộng, bảo mật và dễ tích hợp, đóng vai trò nền tảng cho các phân hệ chức năng khác, đồng thời tận dụng trí tuệ nhân tạo để tự động hóa việc tổng hợp và phân tích thông tin, giúp người dùng quản lý dự án Agile hiệu quả hơn.
